\documentclass[a4paper, 12pt, oneside]{article}
\usepackage[T1]{fontenc}
\usepackage{aurical}
\usepackage{csquotes}
\usepackage{booktabs}
\usepackage{url}
\usepackage{graphicx}
\setlength{\emergencystretch}{15pt}
\graphicspath{ {./figures/} }
\usepackage[figurename=]{caption}
\usepackage{fancyhdr}
\usepackage{amssymb}
\usepackage{array}
\usepackage{float}
\usepackage{imakeidx}
\usepackage{qtree}
\usepackage{microtype}
\renewcommand{\listfigurename}{List of Plates}
\makeindex[columns=2, title=Alphabetical Index, intoc]
\usepackage{sectsty}
\usepackage[titles]{tocloft}

\allsectionsfont{\Fontauri}
\sectionfont{\Fontauri\Huge}
\subsectionfont{\Fontauri\LARGE}
\subsubsectionfont{\Fontauri\Large}

\begin{document}
\Fontauri
\renewcommand{\contentsname}{
\Fontauri{Index}
}

\renewcommand{\cftsecfont}{\Fontauri}
\renewcommand{\cftsubsecfont}{\Fontauri}
\renewcommand{\cftsubsubsecfont}{\Fontauri}

% fix toc page numbers
\let\origcftsecfont\cft
\let\origcftsecpagefont\cftsecpagefont
\let\origcftsecafterpnum\cftsecafterpnum
\renewcommand{\cftsecpagefont}{\Fontauri{\origcftsecpagefont}}
\renewcommand{\cftsecafterpnum}{\Fontauri{\origcftsecafterpnum}}
\let\origcftsubsecpagefont\cftsubsecpagefont
\let\origcftsubsecafterpnum\cftsubsecafterpnum
\renewcommand{\cftsubsecpagefont}{\Fontauri{\origcftsubsecpagefont}}
\renewcommand{\cftsubsecafterpnum}{\Fontauri{\origcftsubsecafterpnum}}
\let\origcftsubsubsecpagefont\cftsubsubsecpagefont
\let\origcftsubsubsecafterpnum\cftsubsubsecafterpnum
\renewcommand{\cftsubsubsecpagefont}{\Fontauri{\origcftsubsubsecpagefont}}
\renewcommand{\cftsubsubsecafterpnum}{\Fontauri{\origcftsubsubsecafterpnum}}

\renewcommand\thefootnote{\Fontauri{\arabic{footnote}}}
\let\oldfootnote\footnote
    \renewcommand{\footnote}[1]{\oldfootnote{\Fontauri\large#1}}
\begin{titlepage} % Suppresses headers and footers on the title page
	\centering % Centre everything on the title page
	\scshape % Use small caps for all text on the title page

	%------------------------------------------------
	%	Title
	%------------------------------------------------
	
	\rule{\textwidth}{1.6pt}\vspace*{-\baselineskip}\vspace*{2pt} % Thick horizontal rule
	\rule{\textwidth}{0.4pt} % Thin horizontal rule
	
	\vspace{0.75\baselineskip} % Whitespace above the title

        {\Huge The Nummulosphere\\ Part 2\\ the Genesis of \\the Igneous Rocks\\ and of Meteorites\\} % Title
	
	\vspace{0.75\baselineskip} % Whitespace below the title
	
	\rule{\textwidth}{0.4pt}\vspace*{-\baselineskip}\vspace{3.2pt} % Thin horizontal rule
	\rule{\textwidth}{1.6pt} % Thick horizontal rule
	
	\vspace{1\baselineskip} % Whitespace after the title block
	
	%------------------------------------------------
	%	Subtitle
	%------------------------------------------------
	
	{By \scshape\Large Randolph Kirkpatrick\\} % Subtitle or further description
	
	\vspace*{1\baselineskip} % Whitespace under the subtitle
	
	%------------------------------------------------
	%	Editor(s)
	%------------------------------------------------
	
	\vspace{1\baselineskip} % Whitespace before the editors

    %------------------------------------------------
	%	Cover photo
	%------------------------------------------------
	
	%\includegraphics[scale=1]{cover}
	
	%------------------------------------------------
	%	Publisher
	%------------------------------------------------
		
	\vspace*{\fill}% Whitespace under the publisher logo
	
	1\textsuperscript{st} Edition, London 1913 % Publication year
	
	{\small Lamley \& Co. } % Publisher

	\vspace{1\baselineskip} % Whitespace under the publisher logo

    Internet Archive Online Edition  % Publication year
	
	{\small Attribution NonCommercial ShareAlike 4.0 International } % Publisher
\end{titlepage}
\pagestyle{fancy}
\fancyhf{}
\cfoot{\Fontauri{\thepage}}
\Large
\vspace*{\fill}
\begin{quote} 
``And there was light...''

``Let the waters bring forth abundantly

the moving creatures that hath life, ...''

Genesis, Chapter 1. 
\end{quote}
\vspace{2\baselineskip}
\begin{quote} 
``Geburt und Grab,

Ein ewiges Meer,

Ein wechselnd Weben,

Ein glühend Leben,''

Goethe, ``Faust.''
\end{quote}
\vspace*{\fill}
\clearpage
\setlength{\parskip}{1mm plus1mm minus1mm}
\setcounter{tocdepth}{3}
\setcounter{secnumdepth}{3}
\tableofcontents
\clearpage
\section{Plankton}
\paragraph{}
A few months ago I published a pamphlet entitled ``The Nummulosphere,''\footnote{\Fontauri{Sold by Lamley \& Co., 1, Exhibition Road, London, S.W. Price two shillings. A third part on Stromatoporoids, chalk, igneous rocks, \emph{etc.}, with illustrations will appear shortly.}} giving an account of a discovery I had made that igneous rocks, Eozoon, and numerous limestones from Cretaceous to Precambrian were mainly composed of very small nummulite shells. Further I had found organic structure of the same kind in meteorites.

Not merely were my statements concerning igneous rocks discredited, but the book was regarded by some as the freak of a ``crank'' or of an insane person.

There was, perhaps, some justification for scepticism, for it is not always easy for a beginner to detect the nummulites in the limestones and igneous rocks. At the same time I fail to understand how a careful observer using a good x10 lens can miss seeing that the weathered surface of chalk or the roughened surface of many flints is made up of disks each with a pattern of concentric circles (really spirals) enclosing extremely fine squarish meshes.

I was told that it was impossible for any trace of organic structure to survive in rocks that had been molten. Again, if the igneous rocks had once been masses of limestone thousands of feet thick, how could they have become silicated, and where had all the lime gone to?

It is now possible to give a satisfactory answer to some of these questions.

Recently when examining a section of \emph{Eozoon canadense} under a high power I found a well-formed Radiolarian skeleton embedded in the olivine. I did not at the moment realize the full significance of this discovery, for I had become thoroughly familiar with the idea that a specimen of Eozoon was a fragment of an ancient sea-bottom, and I knew that L. Cayeux had already described and figured Radiolaria from rocks of Precambrian age in Brittany (Bull. Soc. Geol. France 1894, 3. xxii., p. 197, pl. xi.). A day or two later, I found in the same section the representatives of several genera of Radiolaria. It then occurred to me to examine sections of plutonic rocks. Almost immediately I found a considerable number of Radiolaria in Swedish, Aberdeen and Cornish granites, and a little later specimens of diatoms resembling species of \emph{Synedra} and \emph{Navicula}.

These objects are best seen under the highest powers (2 millim. apochromatic, Oc. 18. 2400 diameters), and in a dim light. A strong light ``drowns'' the outlines of the skeletons.

In sections of granite the skeletons of Radiolaria can often be seen melting away, so to speak, into the homogeneous quartz, which is very probably wholly composed of the siliceous skeletons of unicellular organisms that have rained down on to the nummulitic reefs.

Recent researches of petrologists, especially of Hatschek, and of Liesegang (Geologische Diffusionen) have shown how mineral substances diffuse out from some centre into the surrounding rock, the diffusion and chemical change being much aided by heat.

In granite (or the granular rock) the little masses of siliceous skeletons which had been peppered down on to the reefs became centres whence the silex diffused itself around. The silex became silicated to the extent to which bases where available for union with the feeble silicic ``acid,'' the bases usually present being compounds of metals of the alkalis (K, Na.) of the alkaline earths (Ca, Mg.) and of the earths (Aluminium). Thus feldspar, mica, hornblende, \emph{etc.}, were formed, the residual silex remaining as quartz. Various compounds of iron are very common, the pure metal only rarely being found, as in the Ovifak specimens and in siderites. The iron in these cases has almost certainly been reduced from iron salts which had infiltrated the silicated nummulitic reefs, for indubitable traces of organic structure still persist both in the Ovifak specimens and in siderites.

Generally all the carbonate of lime disappeared probably partly by solution in carbonic acid and water, though some of the calcium was retained in certain members of the pyroxene group of minerals.

The extent to which vast accumulations of lime can disappear is seen in the oceanic deposit known as Barbados Earth. This formation attains a thickness of 130 feet in some places. It was once a series of nummulitic limestone reefs permeated with skeletons of Radiolaria. The whole of the carbonate of lime has disappeared, even from this non-igneous rock, and the nummulite shells have become silicified.

If the igneous rocks are changed nummulitic reefs peppered with siliceous plankton organisms, it may be asked how is the presence of magnesium, aluminium, gold, iron, radium, \emph{etc.}, to be accounted for.

It is no more difficult to explain the existence of magnesium in igneous rocks, Eozoon, \emph{etc.}, than in dolomitic limestones. In reply to my enquiry, Sir John Murray kindly informed me that the percentage of magnesium in the older parts of huge living Tridacna-shells was greater than in the younger parts. Here, at any rate, the dolomitization must have been due to the magnesium salts in the sea. The aluminium in igneous rocks is not so easily accounted for. Either it must have diffused out from a really azoic zone composed of salts of the metals, or, like magnesium, it must have come from the sea. There seems to be no other reasonable alternative. In my opinion the sea was the source. Some salts of aluminium, the sulphate for example, are so soluble that it is difficult to get crystals of them. When the ocean became precipitated on to the scarcely cooled crust of the planet, it must have taken up vast quantities of various salts. The present scarcity or total lack of aluminium\footnote{\Fontauri{The aluminium in clays, shales and certain minerals (e.g., Bauxite) has probably been derived either directly or remotely from decomposed igneous rocks.}} in sea water may simply be due to the fact of the salts of the metal having been used up to form the feldspars, mica, \emph{etc.}, now found in the metamorphosed paleonummulitic or igneous rocks.

Fundamentally there is but little difference between chalk and granite. Both are formed of layer upon layer of nummulitic reefs, including an admixture of plankton\footnote{\Fontauri{Plankton --- organisms living at the surface; benthos --- organisms living at the bottom of the sea.}} organisms, \emph{viz.}, Radiolaria, coccoliths, rhabdoliths, and diatoms. In the upper chalk the silex, which is usually in bands, is derived mainly from sponge spicules (see papers by Sollas, Hinde and Zittel). The flint is usually in well-marked layers because the sponges grew on the fairly level surfaces of reefs. In the case of granite the silex is derived mainly from skeletons of unicellular organisms which rained down continually. Chalk has not been subject to igneous action, consequently the silex has diffused out unchanged, only exceptionally becoming a silicate, as in the slender glauconite core found in potstones or paramoudras. In granite, on the other hand, heat caused the silex to become a silicate and, perhaps, hastened the solution and removal of the lime.

In some very interesting examples of ``Eozoon'' and ``salit-skarn'' from Finland, very kindly sent to me by Professor Otto Trüstedt\footnote{\Fontauri{Prof. Trüstedt tells me that the fragments of Eozoon formed part of the block figured by Liesegang in Geologische Diffusionen, p. 135, Fig. 19. See also Otto Trüstedt. Bull. comm. Geol. Finlande. Die Erzlagerstätten von Pitkäranta, pp. 235, 246-249.}} I can see zones rich in Radiolaria alternating with calcareous zones not so rich. In comparison with Canadian examples, the Finnish specimens are only half-baked.

A so-called specimen of ``Eozoon'' is nothing but a piece of altered nummulitic limestone reef permeated with skeletons of Radiolaria and Diatoms. In some examples the silica united with magnesium salts and iron to form olivine. In others the silicated bands are formed of loganite (hydrous silicate of iron, magnesia and alumina). It is interesting to find silicate of alumina forming in a non-igneous rock, thereby affording evidence that the feldspar, mica, \emph{etc.}, of igneous rocks had not taken up all the aluminium.

The size and shape of the specimen of Eozoon would depend on the extent and direction of the diffusing silica, just as in the case of a flint in the chalk. For a flint also is generally nothing else than a mass of silicified nummulite shells, these being easily seen on rough weathered surfaces from which apparently transparent colloidal silica has disappeared.

Petrologists are continually seeking for a natural classification of igneous rocks. I think that one of the factors to be taken into account will be the relative proportion of benthos nummulites and plankton siliceous organisms. The varying degrees of acidity and basicity would, I believe, depend partly on the varying proportion of the skeletal remains of the Eozoic fauna and flora.

Returning to the plankton organisms which with the benthos nummulitic reefs constituted an important part of the plutonic fauna and flora, it should be mentioned that probably the plutonic Radiolaria like those of the present day, contained symbiotic chlorophyll-containing zooxanthellae. Linnaeus\footnote{\Fontauri{``Sic rupes saxei non primaevi, sed temporis filiae.'' Systema naturae, Ed. VI (2nd impression) 1748. Observations in Regnum Lapideum, p. 219.}} wrote of the primeval rocks being ``the daughters of time.'' I would venture to revise that genealogy and call them the daughters of the sun. For the ruler of our system, acting from a centre over ninety millions of miles away, was primarily the creator and architect of the igneous rocks, the masons being the tiny specks of protoplasm which yielded up their bodies as food for the nummulitic ``Bathybius''\footnote{\Fontauri{This term was originally used to describe a supposed living jelly covering the ocean floor. The renewed use of the expression in connection with the former apparently universal carpet of small separate or colony-forming nummulites here seems justifiable.}} and their skeletons for the building up of the planetary crust.
\clearpage
\section{Benthos}
\paragraph{}
In the preceding paragraph the plankton organisms have been chiefly referred to. A few brief remarks will now be made on the benthos element, about which I hope to write more fully in Part 3 of ``The Nummulosphere.'' I have found that the igneous rocks, meteorites, many limestones, \emph{Eozoon}, the whole of the Archaeocyathidae, Receptaculitidae, Spongiostromids, \emph{Girvanella}, \emph{Mitcheldeania}, \emph{Solenopora}, many supposed Palaeozoic calcareous algae (\emph{Cyclocrinus}, \emph{Palaeoporella},\footnote{\Fontauri{I am much indebted to Prof. E. Stolley for sending me pieces of Palaeoporella- and Vermiporella-Kalkstein from Schleswig-Holstein.}} \emph{Vermiporella}, \emph{etc.}), the whole of the Stromatoporoids, \emph{Loftusia}, \emph{Parkeria}, and \emph{Syringosphaera} are fundamentally identical. \emph{Plus ça change, plus c'est la même chose}.

Strictly speaking, the supposed fossils in the above list are not true fossils at all but lumps mainly of nummulitic limestone more or less altered, with plankton organisms intermixed. The nummulites apparently all belong to one species, \emph{Stromatopora concentrica} Goldfuss. The ``fossils'' are really variously mineralized pseudomorphs (but not mineral pseudomorphs).

In the typical Devonian \emph{Stromatopora concentrica} the nummulitic mass has been acted upon by various metamorphosing and dissolving agencies, such as heat, pressure, water, carbonic acid. Thus lumps have become differentiated from the common mass of the reef. The concentric laminae, astrorhizae, vertical and horizontal pillars are simply sculpturings and etchings, infinitely varied owing to the varying conditions. There is the difference between opaque and less opaque or crystalline, and also between hard and soft, so that curiously shaped masses sometimes get etched out bodily from a common ground mass, and branching pseudomorphs may arise (\emph{Stachyodes}, \emph{Amphipora}, \emph{Idiostroma}). It is not surpising that the Stromatoporoids have been, to use Nicholson's expression, ``the opprobria of palaeontology.'' Again, \emph{Receptaculites} has always been a mystery. The eminent palaeontologist, Prof. H. Rauff,\footnote{\Fontauri{Untersuchungen über die organisation und systematische Stellung der Receptaculitiden. Abhand. Akad. Wiss. München, 1892, Bd. xvii. cl. ii. p. 647. Taf. 1-7.}} concludes his monograph on the Receptaculitidae with the statement, ``Ihre systematische Stellung bleibt noch ganz zweifelhaft.'' There must be some very special reason for this. The explanation is that these objects strictly speaking have no systematic position.

A section of \emph{Receptaculites} may be compared to a study in sepia. It is a case of opaque and less opaque in a uniform groundwork of nearly obliterated nummulite shells.

The segregation of these masses shaped like watchcases, cups, pears, \emph{etc.}, is due, again, to effects of mineralization, to differentiation of hard and less hard, just as in the case of flint or marcasite nodule in chalk. A careful search will reveal the plankton organisms scattered about indifferently in the ``walls'' or ``spaces'' of any section of \emph{Receptaculites}. Similarly in \emph{Cyclocrinus}. It is unlikely that this supposed Siphonaceous alga indulged in carnivorous habits in early times.

Very wonderful are the radial, concentric and lace-like sculpturings and etchings in \emph{Parkeria} and many of the eighty species of Cambrian Archaeocyathidae.

Prof. L. Rhumbler (Die Foraminifera. Plankton Expedition 2., 1913) places in the genus \emph{Girvanella} (modernised \emph{Agirvanella}) two recent species of Ammodiscidae, one of them an encrusting form encircling a Globigerina shell. There is, however, no affinity between these recent coiled Ammodiscid species and the peculiarly mineralized masses of nummulites to which the name \emph{Girvanella} has been given.

``Stromatoporoids'' proper were supposed to be almost confined to Ordovician, Silurian and Devonian strata. The biological foundation of them, however, abounds from plutonic rocks to upper chalk. Eozoon ``specimens'' are simply dolomitized, serpentinized lumps of pre-Cambrian limestone. Similarly, volcanic and plutonic rocks are masses of plankton and benthos skeletons. The abundance of Devonian ``Stromatoporoids'' is due to prevalence of volcanic disturbances in those times, but the mineralization has not caused silication of the nummulites by diffusion from the siliceous plankton elements.

In spite of much careful investigation, I have not been able to discover definitely whether the small nummulite shells are separate individuals, as in the case of all other Foraminifera, including the vast deposits of Cainozoic nummulites, or whether they are colony-forming organisms like corals. There are some reasons for regarding them as colonies of shells in vital union with each other. If so, the surface layer of shells may have been immersed in a common mass of protoplasm or even joined shell to shell. The characters of some specimens of \emph{Stromatopora concentrica} which I collected at Gerolstein, and also certain peculiar appearances of budding of shells seem to lend support to the ``colony'' theory. I hope to give the reasons for and against this theory in my next paper.
\clearpage
\section{Meteorites}
\paragraph{}
The authorities of the Natural History Museum granted me the great privilege of permission to examine several examples of the unique collection of meteorites. After a careful investigation --- from my point of view --- of several specimens and of numerous slides, I have found that these bodies are masses of metamorphosed skeletons of plankton and benthos organisms similar to those found in igneous rocks, Eozoon, Stromatoporoids, chalk, \emph{etc.}, \emph{etc.} Even the purely metallic siderites are merely ore-enriched fragments comparable, say, to the marcasite nodules in chalk, the outlines of the nummulites being visible more or less distinctly both in nodules and in siderites. If these masses of metal could be made transparent, probably Radiolaria and diatoms also would be detected.

Biological evidence is now so overwhelmingly in favour of the theory of the origin of meteorites from terrestrial volcanoes, that mathematicians who have asserted the impossibility of such origin will find it necessary to revise their calculations. The fact that there occur in meteorites minerals not found elsewhere on the planet, is not surprising. For Nature has performed an unusual experiment. In a few moments masses of minerals have been transferred from the fierce heat of the interior of a superlatively active volcano to a vacuum with a temperature of 200° C. below zero. New molecular arrangements would be likely to arise under such conditions. Organisms resembling the terrestrial Eozoic fauna and flora may, it is true, have been evolved on other heavenly bodies, but as Sir Robert Ball points out, it is more probable that meteorites crossing our orbit came from the earth rather than from elsewhere. If, as some astronomers state, there is a connection between meteorites, shooting stars and comets, then it follows that the two latter may in some instances be composed of lumps, lapilli and fine dust from terrestrial volcanoes. If so, what a singular apotheosis of the relics of organic beings!

The secular cooling of the planet in its medium of space 200° C. below zero, has caused contraction and crumpling of the planetary crust. The molar energy of contraction partly becomes molecular, leading to heating of the rocks. An age-long accumulation of heat unable to escape as fast as it was formed, would lead in time to partial melting of the rocks. The relief of pressure in the weak spots in the folding gives rise to eruptions which have sometimes had sufficient force to project fragments of the nummulosphere into space.
\clearpage
\section{The Ocean Floor}
\paragraph{}
I have already stated that the real floor of the ocean is formed of the oozy surface of a silicated palaeonummulitic limestone, more or less crowded with modern plankton organisms, the added ingredients varying in nature according to conditions of surface-temperature and depth of the ocean.

A renewed examination of the abyssal oozes collected by the ``Porcupine,'' ``Alert,'' and ``Challenger'' has served only to strengthen my conviction on this point.

The abyssal ocean floor and the land surfaces are apparently complementary elevations and troughs which have risen and sunk above and below a more or less uniform mean level. Darwin's theory of coral reefs apparently led to the conclusion that there had been a great subsidence over vast areas of the ocean; the nature of the true ocean floor shows that the whole area has sunk.
\clearpage
\section{Appendix}
\paragraph{}
Note 1. I take this opportunity of publishing a list of corrigenda to ``The Nummulosphere,'' Part 1. The book was, as I pointed out, of the nature of pioneer work in a vast new territory. The statements are correct so far as the main issues are concerned.
\begin{itemize}
    \item Page 6, line 11 from bottom. The supposed branching pseudopodial canals in ``Eozoon'' are purely mineral structures of the nature of infiltration- or diffusion-veinings, coursing over and through nummulite shells. There are no canals in Eozoon and never have been. A specimen of Eozoon is a mass of very small nummulite shells mixed with plankton organisms, and mineralized in a peculiar way.

    \item Page 23, line 5 from bottom. The Tudor specimen of \emph{Eozoon} had silicated zones as well as calcareous, but the latter were calcareous throughout and devoid of silicated casts of ``canals.''

    \item Page 29, line 6 from bottom. Pico Baixo is probably no more than 500 feet high. My original estimate of 900 feet was made from a boat a mile out at sea. Recently I have been on the summit of the little pico.

    \item Page 43, line 20 from bottom. The presence of iron compounds in the minute chambers of the nummulite shells of certain limestones and igneous rocks is probably wholly due to infiltration.

    \item Page 43, line 15 from bottom. For a body ejected from a volcano to be able to escape from the earth altogether, it is necessary that that body should have an initial velocity of at least six miles a second. The bombs of Krakatoa are said to have had an initial velocity of only one mile a second. If this is true, it is impossible that meteorites could have been ejected on the occasion of the great eruption.\\ \\ In Part 3. of ``The Nummulosphere'' I hope to bring forward biological evidence tending to confirm Sir Robert Ball's theory that meteorites were ejected from \emph{ancient} terrestrial volcanoes.

    \item Page 45, lines 6-12 from top. \emph{Re} a specimen showing granite intruded into gneiss. I refer to the gneiss as ``probably sedimentary.'' I do not think it was, for I find both benthos and plankton in certain gneisses, and it is doubtful whether these elements would remain in evidence to the same extent in sedimentary rocks.

    \item Page 55, lines 4-8 from top. It is unlikely that the Agua do Porto Santo owes its alkalinity to salts that had once formed part of an acid trachyte, for there is an excess of soda in the latter.

    \item Page 55, line 5 from top. For ``de'' read ``do.''

    \item Page 64, lines 10 and 21 from top. Typical Jurassic oolitic strata contain, in addition to oolitic grains, nummulitic \emph{débris} derived from the wearing down of hard chalk-like Stromatoporoid reefs. In the quarries of Portland (Dorset), the ``Curf,'' between the typically oolitic ``whit'' and ``base'' beds, and the cherty limestone beds below the ``base'' appear to be reefs of this kind, which have grown \emph{in situ}, but apparently the typical oolite formations are sedimentary.\\ \\ Flint and chert are mainly silicified Stromatoporoid structures, a fact which can easily be verified by examining the rough weathered surfaces of many flints. The silica has been derived from Sponges, Radiolaria and Diatoms.

    \item Page 72, footnote, lines 5 to 1 from bottom. The pure crystalline quartz rock referred to contains numerous and indubitable traces of Radiolaria and diatoms. The benthos nummulitic element is difficult to detect. This ultra-acid rock may be compared to a parboiled Barbados Earth in which plankton elements are intermixed with silicified nummulites.

    \item Page 88, line 14 from top. Since the Transvaal diamond rock is an altered marine formation, possibly the carbon of diamonds may have been derived from that which was fixed in the cell-walls of algae. It is, I believe, held by many that much or all of the pure carbon found in nature has probably had an organic origin.
\end{itemize}
Note 2. The cover design drawn by Mr. Highley represents benthos and plankton, the former being magnified about 10 diameters and the latter from about 500 to 2,000 diameters. The plankton organisms after a brief existence in sunlit waters sink in funereal procession to the depths.
\clearpage
\section{Explanation of Plate}
\paragraph{}
(Perey Higley, del. et lith., C. Hodges \& Son. imp.)

Fig. 1. Diatom in Cornish granite. \emph{a}, central nodule. x2,400.

Fig. 2. Diatom in Swedish granite. x600.

Fig. 3. Diatom in Cornish granite. x1,000.

Fig. 4. Diatom in Swedish granite. x600.

Fig. 4a. Portion of same. x2,400.

Fig. 5. Diatom in \emph{Eozoon canadense} from Burgess, Canada. x1,000.

Figs. 6-7. Radiolaria from Swedish granite. x600.

Figs. 8-9. Radiolaria from \emph{Eozoon canadense} from Burgess, Canada. x600.

Fig. 10. Radiolaria from Monte Somma bomb. x600.

Figs. 11-13. Radiolaria from ``Wold Cottage'' meteorite. x600.

Fig. 14. Coccoliths of \emph{Coccolithophora pelagica} (Wallich) ?, from Swedish granite. x2,400.

Fig. 15. Another coccolith (?) from same, like some figured by Murray and Blackman (Phil. Trans., 1898. Vol. 190. Ser. B. Pl. 16). x2,400.

Fig. 16. Coccoliths from Swedish granite, one being broken. Side view. x2,400.

Fig. 17. Rhabdospheres from Swedish granite. x2,400.
\clearpage
\begin{figure}[b]
\centering
\includegraphics[width=\textwidth,keepaspectratio]{table.png}
\end{figure}
\clearpage
\end{document}

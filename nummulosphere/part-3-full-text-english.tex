\documentclass[a4paper, 12pt, oneside]{article}
\usepackage[utf8]{inputenc}
\usepackage{fouriernc}
\usepackage{csquotes}
\usepackage{booktabs}
\usepackage{url}
\usepackage{graphicx}
\setlength{\emergencystretch}{15pt}
\graphicspath{ {./figures/} }
\usepackage[figurename=]{caption}
\usepackage{fancyhdr}
\usepackage{amssymb}
\usepackage{array}
\usepackage{float}
\usepackage{imakeidx}
\usepackage{qtree}
\renewcommand{\listfigurename}{List of Plates}
\makeindex[columns=2, title=Alphabetical Index, intoc]
\begin{document}
\begin{titlepage} % Suppresses headers and footers on the title page
	\centering % Centre everything on the title page
	\scshape % Use small caps for all text on the title page

	%------------------------------------------------
	%	Title
	%------------------------------------------------
	
	\rule{\textwidth}{1.6pt}\vspace*{-\baselineskip}\vspace*{2pt} % Thick horizontal rule
	\rule{\textwidth}{0.4pt} % Thin horizontal rule
	
	\vspace{0.75\baselineskip} % Whitespace above the title
	
	{\LARGE THE NUMMULOSPHERE\\ PART 3\\ THE OCEAN FLOOR OR\\ BENTHOPLANKTON\\} % Title
	
	\vspace{0.75\baselineskip} % Whitespace below the title
	
	\rule{\textwidth}{0.4pt}\vspace*{-\baselineskip}\vspace{3.2pt} % Thin horizontal rule
	\rule{\textwidth}{1.6pt} % Thick horizontal rule
	
	\vspace{1\baselineskip} % Whitespace after the title block
	
	%------------------------------------------------
	%	Subtitle
	%------------------------------------------------
	
	{By \scshape\Large Randolph Kirkpatrick\\} % Subtitle or further description
	
	\vspace*{1\baselineskip} % Whitespace under the subtitle
	
    {\small With 27 Plates} % Subtitle or further description
    
	%------------------------------------------------
	%	Editor(s)
	%------------------------------------------------
	
	\vspace{1\baselineskip} % Whitespace before the editors

    %------------------------------------------------
	%	Cover photo
	%------------------------------------------------
	
	%\includegraphics[scale=1]{cover}
	
	%------------------------------------------------
	%	Publisher
	%------------------------------------------------
		
	\vspace*{\fill}% Whitespace under the publisher logo
	
	1$^{st}$ Edition, London 1916 % Publication year
	
	{\small William Clowes and Sons, Limited } % Publisher

	\vspace{1\baselineskip} % Whitespace under the publisher logo

    Internet Archive Online Edition  % Publication year
	
	{\small Attribution NonCommercial ShareAlike 4.0 International } % Publisher
\end{titlepage}
\setlength{\parskip}{1mm plus1mm minus1mm}
\setcounter{tocdepth}{3}
\setcounter{secnumdepth}{3}
\tableofcontents
\clearpage
\listoffigures{}
\clearpage
\begin{displayquote}
``Be sure I will not swerve from the truth in aught that I say, nor deceive thee; but of all that the ancient one of the sea, whose speech is sooth, declared to me, not a word will I hide or keep from thee.'' --- \emph{Odyssey} 4, 348-51.

(Done into English Prose by S. H. Butcher and A. Lang.)
\end{displayquote}
\clearpage
\section*{Introduction}
\begin{displayquote}
``What is wanted is, in fact, the skilled eye guided by a brain. \emph{Hence the need of a long and patient training of the sense of sight}, attentive and reasoning observation of the phenomena to which it is directed, perfect sincerity, the entire abandonment of preconceived ideas, all united to a passionate love for science and truth.'' --- \emph{A Day in the Moon}, by the Abbé Th. Moreux, Director of the Bourges Observatory. [Transl.]
\end{displayquote}
\paragraph{}
The foundations on which the stately edifice of modern geological science is supported are to a certain extent unsound, though easily to be made secure.

A discovery, and one, too, of a very simple nature, has recently been made, which will inevitably bring about a revolution in our ideas concerning the nature and origin of by far the greater part of the planetary crust, and will modify our conceptions even of cosmic evolution. The discovery referred to is that of the organic origin of igneous rocks and meteorites.

Huxley was fond of saying there seemed to be no limit to the possibilities of Nature, but at the present time to many men of science it seems inconceivable that white-hot volcanic lava, a lump of granite or a stony or iron meteorite could have originated in any degree from the action of living matter. Yet, as a result of many thousands of extremely careful observations, I am wholly certain that these bodies --- at least, all which I have examined --- primarily owe their existence to the agency of life, and moreover afford evidence of that agency, exactly as a mass of fossil shells or fossil skeletons of any sort bear witness to the former existence of the living organisms that made them. Only, in the case of the igneous rocks and meteorites, the fossil skeletons, owing to their peculiar shape and structure (see Chap. 8), have lost their individual outlines, and have become completely mineralized and, in some cases, ore-enriched.

The significance of the newly found facts will best be shown by means of a brief survey of certain matters of geological history.

\centerline{*\hspace{15mm}*\hspace{15mm}*\hspace{15mm}*\hspace{15mm}*}
\bigskip

\centerline{Neptune \emph{versus} Pluto}

In many parts of the world,\footnote{See Appendix, Note F.} a traveller journeying from the coast or the plains, preferably along the course of a river, to some great range of mountains, will probably first meet with mud-flats, sands and gravels. On reaching the hills and outer zone of the mountain-massif, the rocks will perhaps be found arranged in more or less horizontal layers of shales, sandstones, conglomerates, limestones, \emph{etc.}, all of them evidently being hardened sediments or precipitates which, had formerly been laid down in water. The rocks forming the core of the mountain chain will be in steeply inclined layers of dense, finely-crystalline texture or in coarsely-crystalline unstratified masses. Lastly, active or extinct volcanoes, the sources of congealed rivers of once-molten rock, may be met with; and rocks resembling the volcanic will be seen as dykes or layers (sills) among the sedimentary rocks. Obviously the softer and looser sediments on the plains would be more recent than the more compact strata of sedimentary rocks higher up, the latter, again, being less ancient than the very dense crystalline rocks which they partly overlap on the flanks of the mountains.

Werner (1749--1817), the renowned professor of mineralogy at the mining school of Freiberg in Saxony, classed the rocks of the earth's crust under four groups, \emph{viz.}, the primitive (granite, gneiss, schists, basalt, \emph{etc.}); the secondary, or mainly sedimentary (shales, sandstones, limestones, ``newer basalt,'' \emph{etc.}); the recent, or relatively recent, diluvial deposits (clays, sands, gravels, \emph{etc.}); and the volcanic. He taught that a universal ocean once covered the globe, even above the highest mountain-tops. The crystalline primitive rocks were \emph{chemically} precipitated from the waters, just as salt or alum might be thrown down from their saturated solutions. Granite, which frequently forms the core of great mountain chains, was supposed to constitute the primitive and universal foundation of the globe.

The waters of the ocean receded, either by evaporation or by being whisked off by some comet or other heavenly body, or possibly by disappearing into an excavation\footnote{Behälter, a reservoir.} in the globe. All the primitive layers being in place, the sedimentary strata now became deposited, mainly by \emph{mechanical} but partly by chemical means, in the agitated waters.\footnote{For simplification, no mention is made above of ``transition'' rocks between primitive and secondary or Floetz.} Volcanic eruptions were merely superficial recent and local phenomena due to combustion of deposits of coal or bitumen, the lava being a water-formed rock altered by heat.

So far, then, according to the opinion prevailing at that time, the earth's crust owed its origin solely to the agency of water, from which it had been chemically and mechanically precipitated.

A totally different view had been slowly maturing, however, and presently began to make its influence felt.

Thirty years before the rise of the Wernerian doctrines, the great French investigator Guettard discovered the extinct volcanoes of Central France. In 1752, in the course of a journey in the Auvergne district, he noticed the mile-stones were made of a black stone which, from its resemblance to samples of rock from Vesuvius, he recognized as volcanic. Proceeding on his way he found even the villages were built of this stone, and presently discovered the cones, craters and lava-flows of long-extinct volcanoes. Evidently the now peaceful countryside had once been the scene of great volcanic eruptions.

The curious black columnar rock called ``Basalt'' --- the name given to it by Pliny --- was known to exist in various parts of Europe, at Giants' Causeway, and Fingal's Cave. Sometimes the rock forms the flat summit of a hill, and sometimes is sandwiched between layers of sedimentary rock. What was the origin of this mysterious black vitreous or very finely crystalline columnar rock, so different from the shales, sandstones and limestones? Certainly basalt had some resemblance to lava, yet apparently there were no traces of volcanic action where basalt was found, especially no remains of cones or craters. Lava was porous and full of cavities, and basalt homogeneous; and, further, lava was supposed not to be columnar. Guettard looked upon basalt as an aqueous rock that had been deposited from water by crystallization, and even twenty years after his discovery of the extinct volcanoes he defended this view. Geikie\footnote{\emph{The Founders of Geology}, Sir A. Geikie.} comments on the whimsical circumstance of one and the same man being the parent of two diametrically opposed schools --- the Neptunists and Vulcanists.

In 1763 Desmarest visited Auvergne, and discovered unmistakable lava-flows showing distinct columnar structure. He realized at once that he had found the answer to the riddle of the origin of basalt, \emph{viz.}, that the rock was volcanic. He unravelled with wonderful skill the complicated tangle of phenomena displayed before him. In the most recent eruptions the lava-streams were still connected with the craters whence they had flowed, and scoriae remained. In the case of older lava-flows the cones and ashes had all been washed away, and the lava cut into by deep valleys. Lava which, perhaps, had once flowed down into some ancient valley now capped high hills or plateaux. Lastly, the oldest lava was buried beneath piles of sedimentary strata.

Desmarest traced all these great changes to the ceaseless operation of the agencies of denudation. It was now easy to account for the isolated masses and patches of basalt found in localities whence all other traces of volcanic action had disappeared.

In spite of Desmarest's splendid work a fierce controversy arose concerning the origin of basalt, the upholders of the theory of aqueous origin being termed Neptunists and the believers in the volcanic origin Vulcanists. The visits of some of Werner's most distinguished disciples, for instance von Buch and d'Aubuisson, to Auvergne, and the perplexities that assailed them, recall the story of Balaam. They came, perhaps, to curse, but went away bestowing approval on the Vulcanists.

So far as basalt was concerned, the views of the Neptunists gradually became discredited; but a still greater defeat was in store for them.\footnote{In this general review of the Neptunist and Plutonist controversy, prominence is given rather to the period of prevalence of certain views than to the precise date of their origin.}

Werner ignored the existence of great subterranean forces. The mountains were simply huge heaps of matter deposited \emph{in situ} from the ocean.

James Hutton (1726--1797), whom Geikie calls the ``Father of Modern Geology,'' arrived at conclusions wholly different from the Wernerian. The folding, crumpling and uplifting of strata, which in some cases must necessarily\footnote{For how could highly-inclined strata composed of water-worn pebbles have been deposited in nearly vertical sheets?} at one time have been horizontal, filled his mind with the idea of great upheaving forces.

He clearly perceived the contest between the upheaving and the more obvious wearing-down agencies. Water was one of the tools which cut valleys in the uplifted masses and ultimately ground down the latter wholly into gravel, sand and mud. The sediments became hardened into rocks by heat and pressure, and again upheaved. He regarded granite and basalt as materials emanating from the molten interior of the earth. Great was his delight when he discovered in Glen Tilt veins from a mass of red granite penetrating the black schist and the limestone, thereby proving the granite had once been molten and that it was not ``primitive'' in its relation to the other rocks.\footnote{Hutton, \emph{Theory of the Earth}, vol. 3. p. 13.} Not only, then, was basalt related to volcanic lava, but even granite was formerly a molten rock of a similar nature, the coarsely crystalline texture being due to slow cooling under great pressure.

Huttonians were termed ``Plutonists'' and were regarded by Neptunists as a more extreme section of Vulcanists. Gradually the Plutonists, now including Vulcanists, prevailed, and granite basalt and their like universally came to be termed ``Igneous Rocks.''

Neptunists became almost extinct, a few isolated cases being regarded as ``eccentric.''

Now comes a strange event in the history of natural science,\footnote{Attempts have been made to prove on petrological grounds the purely aqueous origin of igneous rocks. I would especially call attention to \emph{Essai sur la genèse et l'évolution des roches}, 1912, by A. Vialay. I doubt if this learned treatise would ever have any effect against the Plutonian stronghold. The proof, however, that igneous rocks are fossiliferous is unanswerable, unless indeed we are to assume the existence of a Plutonic fauna living in an ``inflamèd sea'' with shores of ``burning marl'' (Milton, \emph{Paradise Lost}).} \emph{viz.}, the discovery beyond the faintest shadow of doubt that the Neptunian theory is right after all. The so-called igneous rocks are of aqueous origin. They are marine formations, so far as I have observed, and fossiliferous throughout.

During the summer of 1912, in the hope of solving a problem presented by a remarkable sponge, I was led to investigate certain fossils called Stromatoporoids and \emph{Eozoön} (see Chapters 3 and 10).

After a prolonged controversy \emph{Eozoön} had been finally accepted by the scientific world as a mineral of purely mineral origin. Dr. Carpenter, however, had regarded it as a gigantic reef-forming Foraminiferan.

One day I found in one of Dr. Carpenter's sections something which I took to be a small coiled nummulite shell situated in one of the supposed chambers of the \emph{Eozoön}. I concluded Dr. Carpenter was right in his theory of organic origin.

At this stage I went on a dredging expedition to Porto Santo Island, and in spare time examined the volcanic rocks. To my amazement I found in them traces of Foraminiferal shells. At Madeira, also, the igneous rocks revealed similar structures. On returning to London, I examined in the Natural History Museum volcanic rocks from all parts of the world, and in every instance detected the shells. Lastly, I found similar objects in granite and other Plutonic rocks, in meteorites and in abyssal oozes. I published a small book entitled \emph{The Nummulosphere: An account of the organic origin of so-called Igneous Rocks and of Abyssal Red Clays}, with two plates of photographs of the supposed shells. The book was, I believe, regarded by some as a symptom of mental derangement on the part of the author. I have no cause for complaint, for my startling statements were not supported by sufficient evidence, and I had only a partial glimpse of a great truth.

I found Stromatoporoids and \emph{Receptaculites}, which are universally regarded as the remains of definite and distinct organisms, to be formed of nummulites, and concluded the latter were associated in colonies. These fossils are palaeontologically similar to igneous rocks, and hence I regarded the latter as masses of nummulite colonies resembling coral-reefs. When I made the surprising discovery that the above-named fossils were pseudomorphs made of masses of ordinary nummulite shells, the colony-theory of igneous rocks likewise fell to the ground. All alike are simply masses and lumps of nummulitic rock. Continuing the study of rocks, I discovered in sections of \emph{Eozoön} and igneous rocks objects which I took to be Radiolaria and Diatoms. Here, apparently, was a source of part of the silica which forms so great a proportion of the planetary crust. Again I published a pamphlet on the genesis of the igneous rocks.

I now turned my attention to the nummulite shells which I had seen with the aid of a hand-lens in igneous rocks, many limestones, abyssal red clays and meteorites. I found the shells belonged to the genus \emph{Nummulites}.

At the same time I discovered a singular mistake on my part. All the objects which I had taken to be skeletons of Radiolaria Diatoms and coccoliths in igneous rocks I found to be portions of nummulite shells. The peculiar nummulitic spiral-disk structures (Chapter 8), as seen in rock-sections under high powers, closely resemble plankton skeletons. After a great expenditure of time and trouble, I had collected into four octavo plates thirty-eight photos of supposed plankton skeletons in igneous rocks. Some of the pictures are very deceptive and ``convincing.'' Space can now only be spared for a few of the smaller ones (Plate 2C). Other observers likewise have been deceived by these nummulitic structures in chalk, \emph{Eozoön}, \emph{etc.}\footnote{The supposed Precambrian Radiolaria figured by Cayeux (Bull. Soc. Geol. France, 22, plate 11. 1899), are, I believe, nummulitic structures. Ehrenberg (\emph{Mikrogeologie}, plate 30. B) describes parts of nummulites in the chalk of Rugen as crystalloids or morpholiths. Sorby (Ann. Mag. N. H., 1861, 8. p. 193) confuses Ehrenberg's crystalloids with coccoliths. Hahn (\emph{Die Urzelle}) makes several genera of algae and a worm out of the discoid and funnel-like parts of the pillars of the same Foraminiferan. The ``spheres'' of chalk have been mistaken for Radiolaria, Foraminifera, \emph{etc.} (Mem. Geol. Survey, 1903, \emph{Cretaceous}, 2. p. 501).}

In spite of very careful search, I am at present unable to produce indubitable evidence of the existence of a single plankton skeleton in igneous rocks. Why, then, do I entitle this book ``Benthoplankton''?\footnote{Benthos --- organisms living at the bottom; plankton --- those living at or below the surface of the water. A benthoplankton ooze or rock is one made of a mixture of benthos and plankton skeletons. Igneous rocks are here described as benthoplankton, even though no plankton skeletons persist, because the evidence is in favour of the theory that part of the silica of those rocks is derived from plankton organisms. Benthos, plankton, and benthoplankton are used either as adjectives or substantives.} This name is chosen because the circumstantial evidence is strongly in favour of the theory that a portion of the silica of igneous rocks comes from the skeletons of organisms, and especially of plankton organisms. See Chapters 1 and 2 on ``The Ocean'' and ``Chalk.'' The soluble silica of the skeletons had dissolved, and had silicified and silicated the nummulite shells amongst or above which they had fallen.

Looking back over the history narrated above, it is now easy to detect the fatal flaw in the ``Plutonic mail.''\footnote{Playfair's \emph{Illustrations}, p. 279. Adopting the analogy used by the illustrious Dr. Playfair, it may now be said with truth that the Neptunian scale-armour is absolutely impenetrable to the prongs of Plutonian pitchforks.}

When the fallacy of Werner's coal-combustion theory of volcanoes was detected,\footnote{For it soon became impossible to escape seeing the identity of the Auvergne lava and basalt. In this district basalt lies on granite, below which no Wernerian could imagine coal to exist. Here, then, were volcanic rocks remote from burning coalfields.} lava came to be regarded as an ``igneous'' rock of deep origin and not as a heated superficial aqueous rock. One variety of the ``igneous'' theory --- for there are several --- is clearly stated by Whitney and Wadsworth in their memoir on the igneous rocks: ``The original crust of the earth must have been azoic, if we adopt the views held by the large majority of geologists that our globe has cooled down from a former condition of igneous fluidity. The volcanic and eruptive rocks must necessarily be azoic \emph{because they have come from the heated interior of the globe, reaching the surface in a melted condition}. We do not, however, designate the eruptive and volcanic rocks as ``Azoic''; the fact that they are necessarily in this condition is assumed as something self-evident.''\footnote{Bull. Mus. Comp. Zool. Harvard, 1884, 7. p. 534. J. D. Whitney and M. E. Wadsworth, \emph{The Azoic System}.} The Neptunian theory of the aqueous origin of basalt though correct was obviously a pure guess. Neptunists, for instance, defended with incredible obstinacy the view that basalt had been deposited from water in the place where it was found (Appendix, Note A). Plutonists, on the other hand, were right concerning the ascent of basalt as a molten magma; but their denial of the aqueous origin of the rock has, for over a hundred years, falsified scientific speculation concerning the real nature of the planetary crust (Appendix, Note P).

Volcanoes and lava streams are heaps and masses of silicated nummulitic limestone, \emph{i.e.}, they are of aqueous and organic origin. All the rest of the igneous rocks are of a similar nature and origin. Seeing that these rocks and the sediments derived from them constitute the bulk of the planetary crust, it follows that the lithosphere is mainly composed of silicated nummulitic rock. It will be shown that the ocean floor is almost certainly formed of nummulitic deposits, even in areas covered with recent plankton dust.

Eocene nummulitic limestones thousands of feet thick extend across north-west Africa, Europe and Asia, from Morocco and Spain to Japan, forming the middle and upper parts of great mountain-chains. This ``nummulitic epoch'' is not an isolated event, for nearly the whole planetary crust is made of nummulitic rocks in the form of chalk and other limestones and igneous rocks. Rocks are sometimes classified as fossiliferous and unfossiliferous, but all are fossiliferous. Again, they are classed as igneous, sedimentary and precipitated; but igneous rocks \emph{are} precipitated, and sedimentary rocks are precipitated rocks ground down. Really, then, there is, broadly speaking, one rock, \emph{viz.}, benthoplankton.

The materials composing the vast bulk of the planetary crust may be classified as follows:

\Tree[.{Marine\\Benthoplankton\\deposits} [.{Submarine,\\(clay surface\\of ancient\\benthoplankton\\deposit plus\\recent\\benthoplankton)} ]
          [.{Supramarine or land\\(hardened benthoplankton\\deposits)} [.{Sediments of\\igneous rocks\\and limestones} ]
                [.{Original deposits} [.{Much altered\\(igneous rocks)} ]
                    [.{Not so much\\altered\\(marine\\limestones)} ]]]]

Remarks on the above table. --- Marine benthoplankton deposits are mainly composed of the skeletons of unicellular organisms; in the case of non-igneous rocks, Corals Mollusca Brachiopoda Echinodermata and Algae contribute largely. Remains of fresh-water benthoplankton and of land faunas and floras are, relatively to the mass of the planetary crust, almost negligible; and, moreover, they are mixed up with sediments of marine benthoplankton. I have found nummulitic remains even in coal cinders, as well as in slates, sandstones, grits, muds, sands, \emph{etc.}, from many localities.

The planetary crust, so far as it is accessible to observation, is composed of mineralized organic remains, the bulk of these being silicated nummulite shells. The lithosphere is veritably a silicated nummulosphere.

\centerline{*\hspace{15mm}*\hspace{15mm}*\hspace{15mm}*\hspace{15mm}*}
\bigskip

My statements concerning the igneous rocks have not yet been accepted. Some even declare it is not necessary to search these rocks in quest of organic remains, the theory of organic origin being manifestly absurd.

An attitude of this kind is not surprising in view of the history of the \emph{Eozoön} controversy; and countless observations have been made on rock-sections, without any trace of organic structure being detected. Yet how frequently evidences of some new truth have escaped notice till attention has been specially drawn to them!

At one time it was said no organic life could exist in the great abysses of the ocean. In these sunless depths there would be no plant life, on which animal life must, in the long run, subsist; and further, animals, even if they could exist, would have to be supported by enormous cuirasses to withstand the assumed high pressure. In fact, it was physically and biologically impossible for animals to live in great depths. Actual dredging in great depths soon dispelled all these fallacies by revealing a rich abyssal fauna. A slender zoophyte dredged from 2900 fathoms in the Central Pacific almost melted on deck before the eyes of its captors. So with the theory of the existence of organic remains in igneous rocks. \emph{Solvitur spectando}.

\centerline{*\hspace{15mm}*\hspace{15mm}*\hspace{15mm}*\hspace{15mm}*}
\bigskip

I shall now briefly reply to certain criticisms and objections:--

\begin{enumerate}
  \item That no traces of organic remains could survive in rocks that had once been molten.

\hspace*{5mm}Dr. J. F. Bottomley, of the Thermal Syndicate, Newcastle-on-Tyne, very kindly undertook, at my request, to melt a piece of Radiolarian earth in an electric furnace. At about 1700° C. the earth completely melted, forming when cooled a dark glassy bead. Sections of the latter show very distinct organic structures. Plate 2E, Fig. 21, shows nummulitic structure. The photo shows traces of nummulitic disk-structures.

\hspace*{5mm}With the kind help of Prof. W. H. Merrett and Mr. H. M. Chappie of the Royal School of Mines, pieces of the same earth were subjected to a temperature of 1600° C. in a Meker furnace. When the oxygen was turned on the earth became glowing white hot. Just at the melting point it was cooled down and sections were cut from the resulting slag. Radiolaria are so very clearly visible that it is possible to determine the species (Plate 11, Fig. 49). The temperature of lavas varies from 900° C. to 1500° C.\footnote{\emph{Fide} Prof. J. P. Iddings, \emph{Lectures on Vulcanism}, London, 1914.} The melting point of silica is 1600° C., but that of silicates depends on the composition, the basic being more easily fusible than the acid --- basalt, for example, being more tractable than trachyte or granite.

  \item That the supposed organic remains in igneous rocks may be of accidental origin.

\hspace*{5mm}The suggestion has been made that the skeletons said to be found by me in igneous rocks may be recent deposits or encrustations; or the rock when in a molten state may have licked up the organic remains.

\hspace*{5mm}If a plug of granite one hundred meters long and, say, one centimetre in area, were cut out of the De Lank granite quarry in any direction, and cut into sections 0.1 mm. thick (\emph{i.e.}, into one million sections), I am justified in stating that traces of organic remains would certainly be found in every section.

  \item That the author has not had sufficient experience to justify him in publishing opinions on igneous rocks.

\hspace*{5mm}It has more than once been hinted to me, and in no unfriendly manner, that I am dealing with matters beyond my special province. There is indeed an appearance of this, but in reality, so far as main issues are concerned, I am on my own ground. Knowledge of the chemical composition and mineral characters of every rock on earth avail --- I will not say nothing --- but very little. What is needed is some knowledge of nummulites, and practice in examining these shells with the hand-lens and compound microscope.

\hspace*{5mm}Tow-netting and dredging in the Atlantic and Indian Ocean have familiarised me with the exceeding richness of oceanic life. During the last year and a half I have travelled over England and Wales from Sunderland to Land's End and from St. David's Head to the Norfolk coast, to examine igneous rocks and limestones \emph{in situ} and to collect material. With the help of colleagues I have obtained specimens and collections of minerals from many parts of the world.

\hspace*{5mm}Lastly and chiefly, as a result of innumerable microscopic observations, I have acquired a certain degree of specialised skill. The eye has been trained to appreciate delicate structure, and to trace the various stages of degradation and alteration in nummulite shells found in igneous and other rocks from the Eocene to the Laurentian.
\end{enumerate}

\centerline{*\hspace{15mm}*\hspace{15mm}*\hspace{15mm}*\hspace{15mm}*}
\bigskip

The book is divided into four parts.

In Part 1 evidence is brought together showing the strong \emph{a priori} reasons for believing in a theory of organic origin of igneous rocks. It is pointed out that an origin of this kind is such as might well be expected to result from the operation of the laws of nature.

A chapter on the ocean is followed by one on a typical benthoplankton rock, \emph{viz.}, chalk. Next follows a chapter on a more changed rock of a similar nature, \emph{viz.}, \emph{Eozoön}, and, again, one on those still more changed benthoplankton structures commonly known as igneous rocks and meteorites. In Part 2 is given a description of the genus \emph{Nummulites}, followed by an account of the occurrence of nummulites in igneous rocks, meteorites, \emph{etc.} Part 3 refers briefly to certain speculations on the origin of life.

Part 4 contains a description of certain pseudomorphs commonly known as \emph{Stromatoporoids}, \emph{Receptaculitidae}, \emph{Archaeocyathus}, \emph{Girvanella}, \emph{Loftusia}, \emph{etc.}

\centerline{*\hspace{15mm}*\hspace{15mm}*\hspace{15mm}*\hspace{15mm}*}
\bigskip

I am deeply indebted to my colleagues in the Geological, Mineralogical, and Bibliographical Departments of the Natural History Museum for much kind help. Further, I have been granted many privileges by the Museum authorities, especially in being permitted to study valuable type collections of fossils, meteorites, \emph{etc.}

I take this opportunity of expressing my grateful acknowledgments to Mr. L. M. Lambe, Mr. R. A. Johnston, and Mr. A. T. McKinnon, officers of the Geological Survey of Canada, who sent me at my request a magnificent set of minerals collected at Côte St. Pierre, Quebec, the classical \emph{Eozoön}-ground, and transported with much difficulty to Montreal for shipment. One royal example of ``\emph{Eozoön canadense}'' weighs over a sixth of a ton.

The friendly sympathy of some who have followed my work has been a valuable aid. Geschworner G. Henriksen of Minde, Bergen, especially has endeavoured to call the attention of scientific men to the new discovery.

I have been fortunate in securing the services of Mr. P. Highley. I think no one has had more experience in drawing from the microscope. He was making drawings of \emph{Eozoön} for Dr. Carpenter forty years ago. Thanks are due also to Messrs. Raines, of Ealing, who have taken great pains to secure good photographic results, and to Mr. Butterworth for careful cliché work.

Also I would gratefully refer here to the very patient and skillful work of Messrs. William Clowes and Sons.

\centerline{\rule{100mm}{0.4pt}}

\centerline{NOTE.}

Anything of scientific value in \emph{Nummulosphere} Parts 1 and 2 is incorporated in Part 3. The first two parts, which are no longer of use, have been of the nature of stepping-stones that have helped me, in spite of much initial error, to arrive at the truths explained in the present work.

\centerline{\rule{100mm}{0.4pt}}

\emph{Postscript}. The irregularity in the numbering of the plates is due to the suppression of many of the earlier plates even after the latter had all been printed off.

The original title of Part 3 \emph{viz.}, ``Sea-Floors or Benthoplankton'' (see page-headings) has been changed to ``The Ocean Floor or Benthoplankton.'' The designation ``Nummulosphere'' refers to the fact that one of the concentric planetary layers (\emph{viz.} the earth's ``crust'') is almost wholly composed of mineralized nummulites, the other zones being atmosphere, hydrosphere and centrosphere.
\clearpage
\section{Part 1 --- The Genesis of Rocks}
\subsection{Chapter 1}
\subsubsection{The Ocean}
``Ocean, the parent of all.'' --- \emph{Iliad} 14. 246.
\paragraph{}
The words of the poet are literally true. For both the emerged land area and the submerged oceanic area of the planetary crust are almost wholly products of the ocean, and born of its substance, and life is the chief agency that has brought about this result.

In the course of a long sea-voyage, the traveller in some swift, high-decked ship, though he may frequently see dim forms of fish or medusae swimming in the depths, and also flying fish, porpoises, and other marine creatures, yet is apt to get an impression of boundless areas devoid of life.

To him, and even to naturalists acquainted with the wonderful richness of oceanic life, the idea of the birth of the earth's crust from ``the unapparent deep''\footnote{Smooth tropical seas in moderate depths often seem almost as transparent as air or crystal. In Milton's grand line, ``The birth of nature from the unapparent deep,'' the deep is the void of space.} may well seem chimerical. Nevertheless the facts relating to the surface-life of the ocean, and to the nature of the ocean-floor and of igneous rocks, all point to the oceanic and mainly organic origin of the planetary crust.

\centerline{*\hspace{15mm}*\hspace{15mm}*\hspace{15mm}*\hspace{15mm}*}
\bigskip

\subsubsection{The Surface of the Ocean}
\paragraph{}
If, with the aid of a water-telescope to banish reflections, we look down through the clear waters off some tropical island of the East Indies, say Java,\footnote{``Where seas of glass with gay reflections smile\\\hspace*{5mm}Round the green coasts of Java's palmy isle.''\\\hspace*{10mm}-- \emph{The Botanic Garden}, Erasmus Darwin.} we may see on the sea-bottom great masses of coral, gigantic Tridacnas and huge Neptune's-Cup Sponges. All these massive constructions began life as minute soft specks of living substance, which, in the course of their growth, extracted from the sea the carbonate of lime and silica of their skeletons. It is not, however, these large organisms that we must regard as world-builders, but, rather, certain kinds of minute shells. The white sand on the floor of the submarine garden will be found to be very rich in small calcareous shells of Foraminifera. In tropical latitudes, Foraminifera sometimes form reefs and banks obstructing navigation. The boring into the Funafuti coral-atoll revealed the fact that the reef was to a considerable extent built of Foraminifera.\footnote{\emph{The Age of the Earth}. \emph{Funafuti: the Study of a Coral-Atoll}. W J. Sollas.}

Nummulitic limestones many thousands of feet thick and chalk one thousand feet thick are exceedingly rich in these shells. Further, if igneous rocks are examined with sufficient care they also will be found to be built chiefly of Foraminifera in which the carbonate of lime has become replaced by silicates.

For one of the sources of the silica that forms such a large proportion of the earth's crust we must look elsewhere than to the floor of the sea.

Prof. W. K. Brooks in his memoir on \emph{The Genus Salpa}, when referring to the food of that pelagic Tunicate, describes the mid-ocean surface as ``a living broth.'' This apt comparison may serve to convey some idea of the abounding life existing in and near the ocean surface. The living creatures are very small and mostly of microscopic size. Prof. Brooks gives some striking examples of the almost incredible abundance of marine life: ``Salpae are often found swarming at the surface of the ocean in numbers beyond description.'' ``The smaller species are often so abundant that for hundreds of miles any bucketful of water dipped up at random will be found to contain hundreds of them. \emph{The food of Salpa consists of Radiolarians, Diatoms}, and other micro-organisms which float in the water. The supply of this food is unlimited.'' 

G. Chierchia (\emph{Viag. Vettor Pisani}, p. 31) writes of the Atlantic: ``The zone of equatorial calms is beyond measure rich in life. Sometimes the water looks as if it is coagulated, and this condition is apparent also to the touch.'' Concerning Copepods, G. S. Brady writes (\emph{Challenger Narrative} 1. 2. p. 843): ``The sea from the Equator to the Poles supports such vast numbers of them, that it is often coloured by wide bands for many miles.'' They may be compared with the herbivora on land, for their food consists of Diatoms. Further, Brady says the Copepods are an important item of food for whales in Arctic seas. Accordingly these minute ``sea locusts'' form a link between the smallest plants and the largest animals. 

Haeckel (\emph{Challenger Radiolaria}) states: ``Radiolaria occur in all seas of the world.'' ``The development of Radiolaria in large masses is very remarkable, and in many parts of the ocean is so great that they play an important part in the economy of marine life.'' 

In a classical passage, Sir Joseph Hooker (\emph{Flora Antarctica} 1. 2. p. 505) writes: ``The universal existence of such an invisible vegetation as that of the Antarctic Ocean is a truly wonderful fact, and the more from its not being accompanied by plants of a high order. During the years we spent there, I had been accustomed to regard the phenomena of life as differing totally from what obtains throughout all other latitudes; for everything living appeared to be of animal origin. The ocean swarmed with \emph{Mollusca}, and particularly entomostracous \emph{Crustacea}, small whales and porpoises: the sea abounded with penguins and seals, and the air with birds: the animal kingdom was ever present, the larger creatures preying on the smaller, and these again on smaller still: all seemed carnivorous. The herbivorous were not recognised, because feeding on a microscopic herbage of whose true nature I had formed an erroneous impression. It is, therefore, with no little satisfaction that I now class the \emph{Diatomaceae} with plants, probably maintaining in the South Polar Ocean that balance between the animal and vegetable kingdoms, which prevails over the surface of our globe.... The end these plants serve in the great scheme of nature is apparent, on inspecting the stomachs of many sea-animals.''

To add a final quotation, Murray and Renard (\emph{Deep-Sea Deposits, Challenger} p. 281) write: ``These siliceous Algae are met with everywhere in the surface and sub-surface waters of the ocean. At times they occur near the surface in enormous numbers, in great floating banks many miles in extent and several fathoms in depth.'' ``It is ... impossible to drag a very fine tow-net through the sea-water anywhere without capturing ... these minute organisms.'' 

Beyond the littoral zone the ocean appeared to be tenanted solely by animals which lived by preying on each other. On land the animal-world subsists in the long run on plant-life. Hooker found that this law holds in the ocean, but the individual plants are mostly invisible. The ocean is covered with floating prairies amid which browse the Protozoa, Copepods, Salpas, \emph{etc.}, these in turn forming the food of marine carnivores. Since plant-life other than Bacteria and Fungi depends on sunlight, the ocean vegetation cannot live below the limit reached by the sun's rays and must necessarily be a floating flora. 

Prof. V. H. Blackman (\emph{Science of the Sea}, p. 116) classes the oceanic phyto-plankton into six groups of which only one secretes silica, \emph{viz.} the Diatoms. These microscopic algae are found all over the ocean, but more abundantly in waters of relatively low salinity (Castracane, \emph{Challenger Diatomaceae}). Though the group is universal, yet many genera and species are restricted to certain areas. Castracane regards temperature as the chief barrier which prevents a cosmopolitan distribution of species.

Whether devoured or not, the siliceous frustules of Diatoms sink to the bottom, and in some areas, especially in the Southern Ocean, form a characteristic Diatomaceous ooze. Sometimes the frustules appear to get dissolved, for Murray and Renard (\emph{Deep-Sea Deposits}, p. 283) state: ``It seems difficult to account for the absence of Diatom remains in some deposits, except on the supposition of removal by exposure to the action of sea-water.''

It has been proved, especially by the researches of Hinde, Sollas, and Zittel, that the flint and chert of Chalk and other limestone formations is derived mainly from the skeletons of organisms. I think it is very probable that much of the silica of those hardened and crystallized Foraminiferal deposits known as igneous rocks is likewise of organic origin.

Diatoms\footnote{Deby regards some Diatoms as pluricellular.} and one-celled plants may be compared to the base of a great pyramid of oceanic life, and there is reason to suppose this relation has held good almost from the beginning of geological time.

The Diatoms themselves have persisted as simple cells undergoing division, the halves remaining physiologically independent. As Brooks points out, these simple algae have led an easy existence immersed in a nutrient fluid bathed in sunshine, whereas effort is the price of advance up the evolutionary scale.

These minute siliceous algae have probably existed from near the time when the ocean was born and the sun began to shine on it perhaps millions of centuries ago. The actual frustules have not yet been found among igneous rocks, but the circumstantial evidence that the silica and silicates of those organic deposits is partly derived from Diatoms and Radiolaria is very strong.

The proportion of soluble silica in sea-water is extremely small according to Murray and Irvine\footnote{\emph{On Silica in Seas}. Proc. Roy. Soc. Edinburgh, 18, p. 236, 1891.} one part in 200,000 to 500,000. These authors concluded that Diatoms obtained their silica from suspended clay. E. J. Allen's experiments on Diatom-culture,\footnote{Journ. Mar. Biol. Assoc. Plymouth, 10. p. 417, 1914.} apparently showing the necessity for the presence of certain vitamins in the seawater, reveal the complicated nature of the problem of metabolism in Diatoms.

There are certain other algae that indirectly aid in the extraction of silica from the sea, \emph{viz.} the zooxanthellae or ``yellow cells'' found in many Radiolaria. The symbiosis between the algae and their Radiolarian ally has been shown by experiment to be so effective, that the combination can be self-sufficing, the Radiolarian being able to live without capturing food.

The Diatom mud of the Antarctic forms a nutritious food for fishes and other animals. Probably the Diatoms from the surface formed part of the food supply of the nummulites that constitute the mass of the igneous rocks, the Foraminifera afterwards becoming silicated by material derived from the frustules.

The Diatoms and zooxanthellae depend on sunshine. Accordingly, I was led to modify Linnaeus' designation of the rocks as ``daughters of time'' and to call them ``daughters of the sun.''\footnote{\emph{The Nummulosphere}, Part 2.}

\bigskip
\centerline{Summary} 

At the present day oceanic life depends on the simple plant vegetation at the surface, and especially on the Diatoms, and there is every reason to believe this relation has held good almost from the beginning of geological time \emph{i.e.} the beginning of the formation of the earth's crust.

The surface life of the ocean has been an important source of the silica which forms a great part of the lithosphere. The planetary crust is mainly a product\footnote{From a philosophical point of view, the earth's crust might be described as a by-product of the evolutionary process, psychic development being the aim.} of the evolution of life.

\subsubsection{The Floor of the Ocean}

Until recent times the floor of the ocean beyond a few hundred fathoms was a region of profound mystery, and consequently men of science were in complete ignorance concerning the greater part of the surface of the planetary crust. During the last sixty years, however, human ingenuity has succeeded in surveying and charting the ocean floor almost as completely and surely as if it had been laid bare to ordinary observation.

The merit of this great achievement belongs especially to the \emph{Challenger} Expedition (1873-76), though great credit is due to other enterprises undertaken before and since that epoch-making voyage. As a result of these explorations it is now possible to picture to the mind the submerged mountains, valleys and plains of the ocean, the weird inhabitants of the abysses and the nature of the bottom. It is this last feature that is a matter of present concern.

Sir John Murray\footnote{Scottish Geogr. Mag., 4. p. 1, 1888, and 6. p. 265, 1890. } estimates the area of the surface of the globe at 197 millions of square miles, of which dry land occupies 54 millions and the ocean 143 millions. The ocean floor is covered with muds, ``oozes'' and clay. Murray classes these deposits under two groups, \emph{viz.} the Terrigenous, fringing the great land areas, and the Pelagic remote from land. The terrigenous deposits (littoral muds sands and gravels, coral sand and mud, volcanic sand and mud, greensand, and green red and blue muds) covering an area of 28.6 millions of square miles are more or less composed of materials derived from land.

The pelagic deposits, formed either of Red Clay or of skeletons of plankton organisms,\footnote{\emph{Nummulites} are abundantly present in \emph{Globigerina} and Diatom-oozes. In Nummulosphere 1 I unnecessarily suggested adding the designation ``palaeonummulitic'' to the names in use. The nummulites in the above oozes are probably derived from subaerial and submarine eruptions. In depths greater than 2700 fms. calcareous plankton dust is dissolved, and the clayey surface of the probably universal nummulitic deposit is left bare.} cover an area of 114.6 millions of square miles, and are distributed as follows:--
\begin{center}
\begin{tabular}{ |m{9em}|m{4em}|m{4em}|m{12em}| }
 \hline
 --- & Mean depth in fathoms & Area in millions sq. miles & Distribution \\
 \hline
 \emph{Globigerina} Ooze & 2000 & 49.5 & Atlantic chiefly, Indian Ocean, and Pacific \\
 \hline
 Diatomaceous Ooze\footnote{See Appendix, Note L.} & 1477 & 10.9 & Southern Ocean chiefly \\
 \hline
 Radiolarian Ooze & 2894 & 2.3 & Indian Ocean and Pacific \\
 \hline
 Pteropod Ooze & 1044 & 0.4 & South Atlantic \\
 \hline
 Red Clay & 2730 & 51.5 & Eastern Pacific chiefly, Atlantic, and Indian Ocean \\
 \hline
\end{tabular}
\end{center}
\paragraph{}
The results of the \emph{Challenger} dredgings so far as concerned the first three deposits in the above list did not come as a surprise, for scientific men were already familiar with \emph{Globigerina}- and Diatomaceous oozes and with fossil ``oceanic'' deposits of Radiolaria; but the ``capital discovery'' --- to use Huxley's expression --- that at a mean depth of 2730 fathoms and over an area of 51.5 millions of square miles there existed a deposit of red clay, was something wholly new and unexpected. At first the clay was thought to be the finest detritus of the land, but this hypothesis was soon found to be untenable.

A remarkable feature about the deposit was the almost entire absence of shells of pelagic Foraminifera, which were abundant in neighbouring areas of less depth, and which flourished at the surface of the ocean above the Red Clay. The absence of the shells was attributed to their being dissolved in the course of their very slow descent in the greatest depths.\footnote{Dittmar considered sea-water to be the chief dissolving agency of calcareous skeletons, the carbonic acid being apparently already held up. Murray believed carbonic acid derived from decomposing organic matter inside the shells to be an additional dissolving agency (\emph{Challenger Narrative}, 1. p. 981). Possibly, too, carbonic acid, however formed, would gravitate slowly to the deepest troughs; but if so, the amount is not sufficient to prevent or to destroy life.}

Sir Wyville Thomson\footnote{Proceedings Roy. Soc. 23. p. 45, 1874.} came to the conclusion that Red Clay was the residue or ash of the calcareous organisms left after the removal of the carbonate of lime. When samples of \emph{Globigerina} ooze were carefully washed and treated with weak acid, after the carbonate of lime had been removed there remained a small residue of reddish mud composed of silicate of alumina. Huxley, who accepted provisionally Wyville Thomson's theory, writes:\footnote{\emph{On some of the Results of the Expedition of H.M.S. Challenger}, 1875.} ``So long as the \emph{Globigerinae} collected at the surface have not been demonstrated to contain the elements of clay, the \emph{Challenger} hypothesis, as I may term it, must be accepted with reserve and provisionally, but, at present, I cannot but think that it is more probable than any other suggestion that has been made. Accepting it provisionally, we arrive at the remarkable result that all the chief known constituents of the crust of the earth may have formed part of living bodies; that they may be the ``ash'' of protoplasm, ....'' Wyville Thomson's theory was incorrect, the fallacy in it having arisen owing to the omission to wash the samples of \emph{Globigerina} ooze with \emph{sufficient} care before dissolving them in acid. There is no clay residue to be got from clean pelagic shells gathered from the surface. The residue from shells lying on the bottom is simply that which has permeated them from the surrounding clay. Murray (\emph{Deep-Sea Deposits}, footnote p. 190) states that Wyville Thomson himself gave up the idea that the calcareous shells contained silicate of alumina.

Huxley's supposition concerning the organic origin of all the rocks --- in the absence of the definite data now available --- showed prescience and insight on the part of its author. He clearly saw how defective were the various theories concerning the nature of Red Clay, for he writes (\emph{l.c.}): ``I think it probable that we shall have to wait some time for a sufficient explanation of the origin of the abyssal red clay,'' and again, ``It must be admitted that it is very difficult, at present, to frame any satisfactory explanation of the mode of origin of this singular deposit of red clay.'' Seemingly the answer to the riddle of the red clay has been found, and that answer helps to solve at the same time the problem of the planetary crust as a whole.

In 1877\footnote{\emph{On the distribution of Volcanic débris over the Floor of the Ocean}. Proc. Roy. Soc. Edin. 9. p. 247 (1876).} Sir John Murray advocated the theory that the clay in pelagic deposits was a volcanic product derived from subaerial and to a lesser degree submarine eruptions. He and M. Renard adopted the volcanic theory in the report on \emph{Challenger Deep-Sea Deposits}, M. Renard attributing a more important role to submarine eruptions.

In 1913 I examined the \emph{Challenger} deep-sea deposits and found that the samples of red clay were masses of compressed ``nummulite'' shells with sparsely scattered plankton skeletons embedded in them. Further investigation has shown that they belong to the genus \emph{Nummulites}. It is possible to distinguish parts of the shells with a hand-lens, and under higher powers the nummulitic tubulated structure so characteristic of the genus (Fig. 1). These clay shells are present in all pelagic deposits, and also in abyssal terrigenous red and blue muds, but in red clay their presence is not masked by the calcareous plankton skeletons so abundant in some of the other pelagic oozes. Naturally it is difficult to find traces of nummulites in a ninety per cent. \emph{Globigerina} ooze, but they can be detected in a half and half \emph{Globigerina} ooze such as the one from Station 129 (South Atlantic, 2150 fathoms), where the carbonate of lime and silicates are in about equal proportions. Traces of clay shells, perhaps of volcanic origin, are present even in the nearly pure Radiolarian ooze from Station 225, 4475 fathoms, the deepest \emph{Challenger} sounding.

What, then, is the origin of the clay distributed over the ocean floor? I believe there are three sources, two of which, \emph{viz.} subaerial and submarine volcanic eruptions, have already been indicated by Murray and Renard. Many of the great volcanoes border on the sea. During violent eruptions huge columns of dust and ashes are shot up miles high into the air, where they spread out over land and sea like a great pall blotting out the light for days together.\footnote{During the eruption of Katmai in Alaska in June, 1912, darkness prevailed for sixty hours at Kodiak Island 100 miles away from the volcano, and the fumes reached Vancouver Island, 1500 miles away, the whole country for thousands of square miles being covered with a layer of volcanic dust about a foot thick. During the Krakatoa eruption the pall of darkness extended 276 miles away from the center, and dust fell 1800 miles away. G. C. Martin, \emph{The recent eruption of Katmai Volcano in Alaska}. National Geographical Magazine, February, 1913.} Air currents distribute the dust over vast areas, and on land, rivers carry down great masses of pumice to the sea, where they float for long periods till they sink water-logged to the depths.

Murray and Renard give many reasons for regarding submarine eruptions as a source of abyssal clay. It would be difficult on any other theory to account for the presence of vitreous lumps as large as walnuts in pelagic deposits remote from land. For these compact masses of volcanic glass could hardly have been transported either by air or water, but must have originated in the neighbourhood where they occur.

Mr. James Chumley of the ``Challenger Office,'' Edinburgh, very kindly sent me at my request a typical piece of volcanic glass from Station 293, 2025 fathoms in the center of the Western Pacific. The nearest land is the little speck of Easter Island 800 miles away. The precious fragment, about the size of a pea, is 7 x 5 x 5 millimeters in dimensions. A black glassy part is surrounded by brown bands of palagonite. Certainly the fragment has had an organic origin, for under the microscope it is not difficult to see indubitable and abundant evidences of nummulitic structure.

If --- as Murray and Renard believe --- this fragment has been ejected from a submarine volcano, it will afford evidence concerning the deeper parts of the ocean floor itself.

The presence of air- or water-borne products of land volcanoes on the floor of the ocean is of no special interest, but momentous conclusions must be drawn if material from some deep zone of the abyssal floor is found to be of organic origin. For we should learn that the Red Clay is not a mere superficial accumulation, but the surface layer of a deposit attaining a thickness of hundreds or thousands of feet --- like the igneous rocks and nummulitic limestones on the land or emerged area of the planetary crust. Subaerial and submarine eruptions seem to me inadequate as sources of the whole of the oceanic clay. Murray and Renard record that ``the sounding-tube sometimes penetrated to a depth of nearly two feet in the Red Clay.''

Even if the deposit had only this small depth, it would be assuming a good deal to suppose that products of volcanic eruptions had been spread over an area of over 51 millions of square miles of ocean floor. The abyssal red clay areas are mostly remote from land, and probably very little air- or water-borne material would ever reach them. Even the finest land sediments would sink not very far from shore owing to the ionized condition of the seawater.\footnote{Joly, \emph{Radio-activity and Geology}, p.123.} In my opinion Red Clay is partly the decomposed surface layer\footnote{It is, however, of little importance for the present argument, whether the clay floor of the ocean is the decomposed original surface of the rock, like, for example, the surface of a Cornish moor, or whether that floor is made of formerly deep-seated erupted material of the same rock.} of a vast formation of nummulitic so-called ``igneous'' rock thousands of feet thick.

As stated above, human ingenuity has only succeeded in penetrating ``nearly two feet'' into the deposit. What are the grounds, then, for supposing that deposit to be thousands of feet thick?

New facts have recently come to light showing that the vast bulk\footnote{\emph{i.e.} igneous rocks and their sediments and most of the marine limestones.} of the emerged or land area of the planetary crust is a deposit of nummulites. Certainly one third, and almost certainly the whole, of the oceanic area of the planetary crust is formed of a clay composed of silicated nummulites (admittedly derived in small part from land volcanoes). The difference between the emerged and submerged areas is simply one of position. Essentially and fundamentally they are identical, \emph{i.e.} they are parts of a probably universal benthoplankton deposit of silicated nummulites.

Nature has not only helped us to understand the nature of the submerged ocean floor by pushing up above the water 54,000,000 of square miles of the crust of the globe for comparison, but by means of submarine eruptions she has provided us with materials, otherwise wholly unobtainable, from regions deep below the abyssal floor.

The bulk of the land area, a great part and probably the whole of the ocean floor, and materials from deep below that floor are nummulitic.

Murray and Renard write (\emph{Deep-Sea Deposits}, p. 189): ``With some doubtful exceptions, it has been impossible to recognise, in the rocks of the continents, formations identical with these pelagic deposits.'' Leaving aside the recent plankton elements of ocean deposits and the clay derived from subaerial eruptions, the whole of the rest of the clay and Red Clay seems to me to be the decomposed part of igneous rocks identical biologically and petrologically with igneous rocks on land.

The plateau of Clee Hill or a Cornish moor are fundamentally identical with the abyssal ocean floor. In all three cases there is a surface layer of decomposed nummulitic rock.

Pieces of igneous rocks from quarries show all gradations from the fresh dense crystalline rock to a thick brownish surface crust that can be powdered between the fingers.

The products of abyssal submarine eruptions are found to be of organic origin, and to afford proof that Red Clay is only the surface layer of a very thick deposit; but we are not dependent for our evidence on rare products got with great difficulty from the abysses of the central Pacific. Many of the volcanic islands scattered over the ocean rise from deep water. These \emph{emerged} submarine volcanoes are masses of nummulitic rock. A lump of rotten trachyte containing sulphur, which I took hot from the upper crater of Tenerife, is little else than a mass of fossil nummulites (Plate 21, Fig. A, B). Does it not become clear that these huge volcanic heaps of mineralized Foraminiferal deposits are simply local upheavals of a very thick universal formation?

There is one other point to mention here. Although the bathy metric range of many species of Nummulitidae is very great, yet it is wholly certain the huge deposits of nummulites never lived where their clay models are found in depths below 2,500 fathoms. Undissolved lime has only a very precarious and limited existence in such depths. It is improbable that the living surface layer of these vast deposits of shells existed in depths below 1,000 fathoms. The ocean bottom itself affords clear evidence that over vast areas it must have sunk from a lesser depth to its present position.\footnote{Certain coral reefs appear to afford evidence concerning local sinkings of the ocean floor. Assuming that the coral at the lower end of the Funafuti boring was in the position in which it grew and that it was not talus, then the Funafuti area must have sunk at least 140 fathoms.} 

\emph{Summary}. --- The ocean floor is carpeted with terrigenous deposits in the neighbourhood of land and, in abyssal areas remote from land, with skeletons of plankton organisms, excepting in certain very deep areas where a clayey deposit is exposed owing to the failure of the calcareous remains to reach the bottom.

The clay --- apart from a limited amount due to subaerial eruptions --- is derived partly from the decomposed surface-material of a submarine silicated nummulitic rock, and partly from deeper-seated erupted material of the same rock. The oceanic floor and land area are respectively the submerged and emerged parts of a probably universal deposit of nummulite shells --- the Nummulosphere. The bulk of the emerged part of this deposit has become hardened, mineralized and crystallized, and is known as igneous rock.

\centerline{*\hspace{15mm}*\hspace{15mm}*\hspace{15mm}*\hspace{15mm}*}
\bigskip
\subsubsection{On the Probable Former Existence of a Universal Ocean}

Werner believed that a universal ocean once covered the globe above the highest mountain-tops, the mountains being masses of minerals deposited from the sea \emph{in situ}. So indeed they were, but not in Werner's sense. Werner knew little of the subterranean forces of upheaval. It never occurred to him to level the mountains below the water, so he brought the water above the mountains.

The discovery of the real nature of igneous rocks strongly supports the theory of a formerly universal ocean. The land area and the oceanic area of the earth's crust are now found to be the sunk and upheaved parts of a universal deposit of mineralized nummulite shells. The abyssal ocean floor must have sunk between one and two thousand fathoms. The whole land area up to the summits of the highest mountains has been below the sea.

The mean height of the land\footnote{\emph{On the Height of the Land and the Depth of the Ocean}. Sir John Murray, Scottish Geogr. Mag., p. 1, January, 1888.} is only 375 fathoms, the mean depth of the ocean 2080 fathoms. If the solid land were levelled beneath the ocean, the surface of the earth would be covered by an ocean with a uniform depth of two miles.

At the present day five-sevenths of the area of the globe is covered with ocean. The remaining two-sevenths or land area would be submerged by a rise of between one and two thousand fathoms in the ocean bed; and it must be remembered that the land has not been stationary, but has actually been submerged.

Some astronomers believe the earth will in course of time become waterless like the moon. If this be so, as the earth is already very old, this drying up process may by this time have made some progress. C. L. Bloxam (\emph{Chemistry}, ed. 3, p. 39) writes: ``In its chemical relations water presents this very remarkable feature, that although it is an indifferent oxide'' (\emph{i.e.} neither acid nor basic) ``its combining tendencies extend over a wider range than those of any other compound.'' Many minerals hold more or less permanently in the solid state water of crystallization and of constitution.

Again, living substance has extracted a vast quantity of solids from the ``universal solvent'' to form the planetary crust. Accordingly there was more liquid and less solid at one time than there is now, the solid kernel of the globe having been smaller to the extent of the thickness of the lithosphere, and the ocean mantle deeper.

The areas of existing continents were, in primitive times, much restricted in comparison with their present size, and even those restricted areas were former sea-bottoms. The possibility, however, of existing ocean basins having been continents must not be left out of account.

The solid planetary surface with its organic crust has been heaving up and down like a troubled sea: and taking the above circumstances into account it is probable that the ocean has extended over a greater area of the globe than at present, and, during the earliest period, even over the whole area.

\centerline{*\hspace{15mm}*\hspace{15mm}*\hspace{15mm}*\hspace{15mm}*}
\bigskip

\subsubsection{Summary of Chapter 1}

An abundant siliceous micro-fauna and -flora live at the surface of the ocean, and probably have done so for aeons. The skeletons sink, in time dissolve, and permeate benthos deposits. The ocean floor is formed chiefly of ancient mineralized deposits of nummulites, the surface layer being clayey. Near land these deposits are covered with water- or air-borne land materials (themselves nummulitic); and in the open ocean, in depths less than 2700 fms., with plankton dust chiefly calcareous; in greater depths the nummulite deposits are exposed as grown \emph{in situ} or as erupted \emph{in situ}. The land is also composed of mineralized nummulites. Apparently the abyssal ocean-floor has sunk, for almost certainly nummulites did not live in depths over 1000 fms. Probably there was once an universal ocean.

\emph{Postscript}. --- My first preparations of Red Clay, \emph{Globigerina} Ooze, Diatom Ooze and sulphury trachyte from the crater of Tenerife consisted chiefly of crushed particles. Later, excellent sections were made showing nummulitic structure. The nummulites of the Diatom Ooze from the Southern Ocean (St. 157, 1950 fms.) come probably from ice-borne erratics and pumice. Yet nummulitic deposits formed \emph{in situ} are probably universal, even where covered with plankton dust. For the nummulitic masses (igneous rocks) of the Antarctic continent have emerged in an oceanic area now carpeted with Diatoms. The nummulitic mass of Tenerife has emerged in an area now covered with \emph{Globigerina} ooze (see \emph{Challenger Deep-Sea Deposits}, Charts 5, 6). The volcanic glass, dredged from 2025 fms. South Pacific (St. 293, \emph{Challenger}), occurs in \emph{Globigerina} ooze, and was very probably erupted locally. The bulk of the land area of the globe and the Red Clay (105 million square miles) bear in themselves the evidence concerning their nummulitic origin.

\emph{Note}. --- Referring to the quotation from Bloxam on p. 41. In the contest between liquid and solid, notwithstanding the great solvent powers of water, yet the balance apparently leans towards solidification. Not only have the rocks of the earth's crust been abstracted from solution, but the minerals composing them are not rarely hydrated, as in the case of the serpentines for example.
\clearpage
\subsection{Chapter 2}
\subsubsection{Chalk and Flint, and other Limestones}
\begin{displayquote}
``A great chapter of the history of the world is written in the chalk. Few passages in the history of man can be supported by such an overwhelming mass of direct and indirect evidence as that which testifies to the truth of the fragment of the history of the globe which I hope to enable you to read.... Let me add, that few chapters of human history have a more profound significance for ourselves. I weigh my words well when I assert, that the man who should know the true history of the bit of chalk which every carpenter carries about in his breeches pocket, though ignorant of all other history, is likely, if he will think his knowledge out to its ultimate results, to have a truer, and therefore a better, conception of this wonderful universe, and of man's relation to it, than the most learned student who is deep-read in the records of humanity and ignorant of those of Nature.'' --- \emph{On a Piece of Chalk}. Huxley.
\end{displayquote}
\paragraph{}
Huxley's eloquent testimony concerning the value of a knowledge of chalk would, perhaps, have been even more emphatic had he known that the history of this rock was essentially an epitome of the history of the earth's crust as a whole: and, further, that ``a piece of chalk'' was fundamentally identical in its nature and origin with granite lava and meteorites, probably with many shooting stars, and possibly with some comets. For chalk is an old sea-bottom composed of skeletons of benthos and plankton organisms, the planetary crust is a still more ancient benthoplankton sea-bottom, and meteorites are now found to be lumps of ancient benthoplankton rock.

Lastly, astronomers point out the existence of relationships between meteorites, shooting stars and certain comets.

\centerline{*\hspace{15mm}*\hspace{15mm}*\hspace{15mm}*\hspace{15mm}*}
\bigskip

Pure chalk is mainly an accumulation of calcareous and siliceous skeletons of marine organisms. The greatest recorded thickness of the deposit is 1831 feet at Kharkov in Russia. The Chalk Ocean, according to an estimate of Dr. W. Fraser Hume, covered an area of 500,000 square miles, extending across Europe in a north and south direction from Sweden to Nice, and east and west from Scotland to Uralsk.

In very high chalk cliffs marked differences can generally be seen in the character of the rock at different levels. Usually the upper part shows numerous parallel lines of flints, lower down the flints are fewer, and at the lowest part absent. These divisions were described respectively as ``Chalk with flints,'' ``Chalk with few flints,'' and ``Chalk without flints.'' This classification was soon found to be unworkable, for sometimes flints are abundant in the middle division and absent in the upper. (Appendix, Note N.)

Although the flints-classification broke down, yet more than ever as knowledge increased was the chalk found to be quite other than a mere homogeneous mass. The devoted labours of geologists, especially of Hébert and Ch. Barrois, brought to light numerous successive zones of distinct faunas, recognizable to a greater or less degree over the whole area of the Chalk Ocean floor.

No less than eleven of these zones, with subsidiary ones, are now accepted, each being named after some predominant fossil, \emph{e.g.} \emph{Holaster planus} zone, Terebratulina zone, \emph{etc.} The zones are distributed in three main divisions, known in England as Lower, Middle, and Upper Chalk, the Middle Chalk being marked off from the other two by bands of hard nodular rock (Melbourn Rock and Chalk Rock).

In 1837 Lonsdale showed that Foraminifera, small but visible to the naked eye, were very abundant in chalk.

In 1838 Ehrenberg\footnote{\emph{Über die Bildung der Kreidefelsen und des Kreidemergels durch unsichtbare Organismen}. Akad. Wiss. Berlin, 1838, p. 39.} discovered the important part taken in the formation of chalk by still smaller Foraminifera scarcely visible to unaided vision.

In addition, then, to the larger fossils, Chalk contains very minute ones, and these latter --- Foraminifera (chiefly), Radiolaria, Diatoms, Coccospheres, \emph{etc.}, make up the main mass or ``matrix'' of the rock.

\centerline{*\hspace{15mm}*\hspace{15mm}*\hspace{15mm}*\hspace{15mm}*}
\bigskip

At the beginning of these nummulosphere studies I had regarded chalk as a \emph{Globigerina} ooze somewhat similar to that now forming over vast areas of the ocean floor. On one occasion when examining a piece of British chalk with a lens I saw something which started me on journeys to chalk-formations in all the counties of England south-east of a line from the Dorset coast to Yorkshire, for the purpose of collecting samples of Lower, Middle, and Upper Chalk, Totternhoe Stone, Melbourn Rock and Chalk Rock. Further, I read all the important memoirs on the subject dating from Ehrenberg to the present time. As a result of all this reading and investigation of material a singular impression gradually arose in my mind.

The voluminous literature on chalk reminded me of a performance of Hamlet without the Prince of Denmark. All the other familiar characters of the immortal drama were present, excepting the principal one that gave the name to the play. Every now and then a mysterious figure in various guises --- in reality Hamlet, though unrecognized --- would appear on the scene and flit across the stage.

\emph{Note}. --- When making a final revision I deleted Fig. 1 illustrating the nummulitic nature of chalk, owing to the figure being partly inaccurate and wholly inadequate to show what I can now see with a lens in most pieces of this rock. The following hint may help towards a realization of the strange truth that a seemingly homogeneous lump of chalk is a mass of nummulites. Imagine a lump of Eocene nummulitic limestone with small medium or large nummulites to become as soft as clay, to be pressed, later to be hardened, and, lastly, heated under pressure. There would result earthy, hard, and marmorized limestones in which the outlines of the shells would have disappeared. Soaking in accumulations of dissolved silica has resulted in silicified shell deposits, and heating and cooling have converted the latter into crystalline silicates.

I had found the piece of British chalk referred to above to be a mass of ``nummulite'' shells. In the list of 350 species of Foraminifera of the Upper Cretaceous rocks of Great Britain given in the Survey Report there is no mention of a single genus or species belonging to the Nummulitidae, yet I found all the lithological varieties of chalk from all levels to be mainly nummulitic. The shells belong to the genus \emph{Nummulites}.

Parts of the shells can be discerned with less difficulty in samples weathered to a certain extent than in freshly broken material. The outlines are easily destroyed by being bruised or rubbed.

A very considerable degree of ``training of the sense of sight'' is required to enable the observer to distinguish the shells in sections of chalk. The nummulitic characters are more easily seen in samples poor in \emph{Globigerina}.\footnote{Munier-Chalmas and Schlumberger (Bull. Soc. Geol. France, (3), 13. 1885, p. 274) point out how greatly chalks differ in composition. Beds of Paris chalk rich in Foraminifera are stated to be ``véritables exceptions,'' Bryozoa and Corals being abundant.}

The upper chalk from the old quarry at Haling near Croydon shows well the nummulitic element. One significant feature can soon be seen in any section of chalk under a magnification of about 450 diameters, \emph{viz.}, a very finely and apparently uniformly dotted appearance visible in the finely granular matrix. The dots show up as dark points each in a little circle. Here and there some of the circles are slightly larger. Gradually as the eye acquires skill, it will be found the dots are really not uniformly scattered, but are arranged in more or less circular groups and disk-like groups of groups; also in thick sections series of disks may be detected passing down obliquely in the depth. After several days of practice and much patience outlines of layers of spiral lamina, pillars, seen transversely or lengthwise, furrows of marginal cord, septa and alar prolongations, will be made out. Where the shells as a whole are masked or almost obliterated by the abundance of plankton ingredients, even then portions of disk structure are always to be seen.
\begin{figure}[H]
\centering
\includegraphics[width=\textwidth,height=\textheight,keepaspectratio]{figures/Fig2.png}
\caption*{\small \textsc{Figure 2 --- Nearly transverse but slightly oblique section of a nummulite, showing striated spiral lamina (4 layers) and central plane.}

\centerline{\small From Totternhoe stone (middle chalk), 65x.}}
\end{figure}
There remains still another method of investigation, and a highly important one. For its use will enable students, who at present may well be skeptical concerning the revelations made in this chapter, to realize the astonishing truth that the familiar chalk is veritably a mass of nummulite shells reduced to a calcareous mud. The method referred to is that of maceration.

Ehrenberg found the smallest Foraminifera of the chalk to be embedded in a matrix of still finer particles, which in the memoir already cited he describes as little granular disks.\footnote{These objects are figured for the first time in Poggendorf's \emph{Annalen}, vol. 39, p. 101, Plate 1. 1836.} The latter, which often appear as beaded rings,\footnote{The ringed appearance is a purely optical effect due to the center of the disk remaining invisible, especially in dark-field illumination. (See Plate 22, Fig. E, F.)} may break down into their constituent granules. Ehrenberg concluded that as chalk had been subject to the influence of water and as the finest particles surround the small Foraminifera and are not sifted, therefore these granules must be derived from broken-down shells. He considered the particles to be concretionary bodies of an inorganic nature and to have been precipitated from the dissolved calcite of the Foraminiferal \emph{débris}. Therefore he regarded them as ``crystalloids'' or ``morpholiths.'' He found some chalks (as that of Rügen) to be half made up of this supposed inorganic material. He gives numerous figures of chalk powder macerated in water showing Foraminifera, and also morpholiths in the form of granular disks, rings and finest particles.
\begin{figure}[H]
\centering
\includegraphics[width=70mm,keepaspectratio]{figures/Fig3.png}
\caption*{\centerline{\small \textsc{Figure 3 --- Chalk from Rügen, powdered in water.}}

\hspace{5mm} \small Showing granulated ``disks,'' ``rings,'' and ultimate granules mixed with known Foraminifera, etc., 300x diam. After Ehrenberg. Abhand. Akad. Berlin, 1838, Plate 4 Fig. 3.}
\end{figure}
After Huxley and Wallich had discovered coccoliths and coccospheres in the abyssal mud of the Atlantic, Sorby\footnote{Ann. Mag. N.H., 8, p. 197. 1861.} concluded that Ehrenberg's crystalloids were really organic objects of the nature of coccoliths. In one respect he was right, for these crystalloids are organic structures, but Ehrenberg's Plate 30B (cited by Sorby) shows ``crystalloids'' which are certainly not coccoliths. In 1873\footnote{Akad. Wiss. Berlin, 1873, p. 361.} Ehrenberg still adhered to his view that the granular disklets were inorganic, and pointed out --- quite correctly --- how entirely they differed from coccoliths.
\begin{figure}[H]
\centering
\includegraphics[width=70mm,keepaspectratio]{figures/Fig4.png}
\caption*{\centerline{\small \textsc{Figure 4 --- Disk-like granulated ``morpholiths''}}

\centerline{\small \textsc{from chalk of Antilebanon.}}

\hspace{5mm} \small A, in different aspects. B, fragments 300x. After Ehrenberg. ``Mikrogeol.'' 1854, Plate 25 Fig. 2. B. 16. Probably figure on right in A is a coccolith.}
\end{figure}
He writes: ``According to Sorby and Huxley the main constituent of the chalk --- described by me in 1838 as oval granulated morpholithic scales and their still finer granular fragments, which, along with Foraminifera as larger elements, often constitute half the mass of the common chalk --- consists of small concave animal shells named by them Coccospheres and Coccoliths. On the contrary, I have made it clear there is absolutely no cavity for the reception of an animal body in these chalk morpholiths in the form of granulated scales, and no air-vesicle giving evidence of a space, nor are there double shells for the inclusion of an animal. I have become ever more firmly convinced that these granular scales must be regarded as inorganic morpholithic products due to the change of double-refracting calcareous shells of Foraminifera into amorphous not doubly-refractive, peculiarly disposed (sich eigenthümlich ordnende) calcareous particles,'' I quote this in full, because literally these granular scales or disks furnish the clue to the problem of the nature of the earth's crust. Although Ehrenberg misinterpreted the meaning of the disks (Scheibchen), yet for a period of nearly forty years he persistently called attention to their existence. In spite of this, his observations have been either ignored or misunderstood by other investigators.

Murray and Blackmail judiciously describe some of the bodies in chalk resembling Ehrenberg's crystalloids as spurious coccoliths.\footnote{Phil. Trans., vol. 190 (ser. B), 1898, p. 440, plate 16. fig. 4, a-b.}
\begin{figure}[H]
\centering
\includegraphics[width=\textwidth,height=\textheight,keepaspectratio]{figures/Fig5.png}
\caption*{\centerline{\small \textsc{Figure 5 --- Powdered chalk of Rügen macerated in water.}}

\hspace{5mm} \small A, granulated ``disklets'' and ``rings,'' and ``granules,'' 300x. B, the same, 1000x. After Ehrenberg (``Mikrogeologie,'' Plate 30. Fig. B).}
\end{figure}
Lohmann\footnote{\emph{Protistenkunde}, 1. p. 95, 1902.} thinks the crystalloids or morpholiths (Ehrenberg's Plate 30B) may be coccoliths, but concludes that Ehrenberg's descriptions make it difficult to arrive at a definite opinion.

Careful microscopic examination, under very high powers, of macerated chalk powder shows the granular disks to be peculiar structures --- hitherto undescribed --- which go to make up the entire shell of nummulites (Chap. 8). The two disks to the left in Fig. 4A are spiral bodies with the coils nearly in one plane and with radial loops or coils (``septa'') in a plane at right angles to the plane of the main coil. The disks, in fact, are to a considerable degree repetitions of the complete nummulite shell. The large disks in Fig. 5 are masses of smaller disks (Plate 23, Fig. F). The beaded ``rings'' are not really hollow rings but solid disk bodies, a fact which becomes evident on careful focusing (Plate 23, Fig. E, E').

That the above surprising statements are correct will soon become evident to the trained observer. For any section of chalk (but preferably one not very rich in \emph{Globigerina}) will reveal the nummulites with their coiled marginal cord, alars\footnote{Abbreviation for ``alar prolongations.''} and septa with granulated disks \emph{in situ} (Plate 22, Fig. E). The disks belong to various parts of nummulite shells.\footnote{It is significant that Ehrenbefg found the ``morpholiths'' in Tertiary nummulitic limestone (Abhand. Akad. Berlin, 1855, p. 86). He would have seen them in powder scraped from the surface of any nummulite!}

Chalk, then, viewed as a general formation, is a nummulitic limestone composed of a deposit of nummulite shells, in which all the other ingredients are embedded.

Perhaps geological history will have few stranger incidents to record than the failure, after eighty years of scientific investigation, to detect the fact that chalk is mainly a deposit of nummulite shells. This failure, that has retarded the discovery of the true nature of flint, \emph{Eozoön} and the igneous rocks, has not been due to lack of care on the part of very keen observers, but rather to the qualities of nummulites. The shells, formed as they are of successive layers of highly porous walls, are highly capillary and apt easily to become soaked in water and mineral solutions, to become soft and earthy or hard and crystalline, and in any case to lose their individual outlines. In purely nummulitic chalk, the shells may remain \emph{in situ}, but where other fossils abound the mud nummulites may be pressed into cavities, or be riddled by smaller shells.

Probably a long interval separates the Cretaceous from the Eocene nummulites. Very likely many species exist in the chalk.

The methods recommended for the study of the Foraminifera of chalk are not suitable for preserving the nummulites: more damage is done in an instant of time than has been wrought by Nature in the course of millions of years. It might have been supposed that small thin-walled delicate Foraminifera would have been less resistant than large thick-walled nummulites, but this is not the case. We are advised to grate the rock into a fine powder or to macerate lumps into a paste. I cannot imagine proceedings more fatal to the integrity of nummulite shells than to grind them to powder or convert them into mud. It should be stated that Mr. Heron-Allen's methods\footnote{\emph{Prolegomena}, 1894, p. 12.} of maceration are excellent for the smaller Foraminifera.

I have not succeeded in isolating and cutting into sections the individual nummulite shells, though I visited the Sorbonne chiefly to examine Schlumberger's beautiful sections of Foraminifera and, if possible, to learn something of his technique.\footnote{By the great kindness of Mr. C. S. Smith, British Consul-General at Barcelona, and of Mr. F. W. Abbot, I obtained a large quantity of the Cretaceous limestone of Trago di Noguera, the wonderful Foraminifera of which had been described by Schlumberger.}

The above-detailed methods for showing the nummulitic characters will not leave the student in doubt for very long; these are:--
\begin{enumerate}
\item The use of the lens, especially on weathered examples of chalk and rough flint. I can occasionally detect shell-outlines with the naked eye.

\item The study of thick and thin sections (\emph{a}) under powers of 300 to 400, showing the dotted appearance, the disks, pillars (series of disks) and perhaps the median and lateral chambers; (\emph{b}) under powers of 3000 diameters, showing disk-structures.

\item Maceration of powdered chalk in water, stained or not, under medium and high powers.
\end{enumerate}
\paragraph{}
The study not only of Ehrenberg's morpholiths but also of certain other mysterious objects abundant in chalk will throw light on the real nature of that rock.

Thin sections of the hard bands of nodular Melbourn Rock and Chalk Rock are very rich in peculiar bodies which Jukes-Browne\footnote{Geological Survey, \emph{Upper Cretaceous Rocks}, vol. 2, p. 501.} refers to as ``spheres.'' Many theories have been held to account for them. Prof. Blake thought they looked like the hollow spheres of \emph{Orbulina universa}. Messrs. Parker and Jones suggested they were the ``separate cells of \emph{Globigerina} and \emph{Dentalina}, the former predominating.'' M. Cayeux, who has noted their extreme abundance in the chalk, refers to them as monolocular Foraminifera of the genus \emph{Fissurina} (\emph{Lagena}) or \emph{Orbulina}. In the zone of \emph{Inoceramus labiatus} near Rouen, he estimates the spheres to form 95 to 98 per cent. of all the Rhizopodal fauna.

F. Chapman believed some of them to be Radiolaria. In others he saw a strong resemblance to the globate dermal spicules of certain sponges; ``but the real nature of the majority of these calcareous spheres still remains an enigma.'' These mysterious bodies vary considerably in size, shape, and structure, being discoidal or oval, 0.08 mm. in diameter, with opaque center and clear periphery, or clear throughout. Quite early in the present research I found these objects to be made up of nummulitic disk structures not only filling in the body but extending right across the ``wall'' into the surrounding matrix. I have now advanced much further, for I can see the exact position of the ``spheres'' in the anatomy of the nummulite shell. They occur most commonly serially along the convexity of the marginal cord and the edges of septa, also at the point where double-walled septa bifurcate within the concavity of the cord, also in the pillar-ends in the spiral lamina. Obliquely tilted disks give the \emph{Fissurina}-aspect. Spheres, then, are clarified areas of nummulitic structure rendered glassy by becoming more calcitic or silicic owing to greater facility of diffusion of water or silica at those points. From overflow the areas may extend as in ``\emph{Saccammina carteri}'' in Carboniferous deposits of nummulites. In wholly silicified deposits (Ordovician and Precambrian phthanites) the clear areas resemble Radiolaria, \emph{etc.} Spheres abound in hard bands of chalk (Melbourn Rock and Chalk Rock), probably owing to increased formation of calcite or to a greater degree of dissemination of silica in disturbed or emerged zones.

Barrois considered the hard bands of chalk (bancs durcis) to be due to actual emersion. Hume thought the hard nodular condition resulted from the setting up of currents owing to alteration of the level of land bordering the Chalk Sea.

Whatever the cause, the effect has been to bring about a greater degree of dissemination of soluble silex and a greater degree of silicification of Foraminiferal structure than in ordinary chalk. It is only necessary to put a little acid on a thin section of Melbourn Rock to realize that the ``spheres'' owe their transparency to the fact that they are partly silicified. In ordinary chalk small defined areas have not been made glassy and, therefore, distinct; but the dotted disk structures are better seen.

Certain beds of Carboniferous limestone are almost wholly composed of minute spheroidal or fusiform bodies about 3.2 mm. ($\frac{1}{8}$ inch) in diameter and frequently produced and tubular at each end. Often several of these bodies are joined end to end in a chain. The weathered surface of a limestone rich in these structures is coarsely warty. The spheroids usually have a smooth but granular wall surrounding a clear mass of colloidal silica or of crystals of calcite. Brady\footnote{Brady, \emph{Monograph, Carboniferous and Permian Foraminifera}. Pal. Soc., 1876.} considered these bodies to be Sandy Foraminifera related to a living deep-sea form described by Sars as \emph{Saccammina spherica}. Certainly when a chain of several little fusiform globules is isolated and mounted on a slide nothing could more closely resemble an arenaceous Foraminiferan.

Sections of typical Saccammina limestone, however, show that the Saccammina condition is due to mineralization of areas of nummulite shells, just as in the spheres of the hard chalks. Nummulitic shell-structure can be traced in continuity from the interior through the supposed walls of the spheres to the surrounding matrix. Brady figures certain ``scars'' in the form of granular concentric rings on the ``test'' of the supposed shells (Plate 12, Fig. 7, \emph{l.c.}). These structures are coiled disk-structures. Brady's drawing shows even the circles of the smaller bundles, one of the latter being equivalent to one of Ehrenberg's disklets. I can now make out in sections viewed with a lens features of the larger anatomy of the nummulites. The chain condition is found also in Jurassic oolite; and even the produced tubular structures occur, though only rarely, probably because the points have been dissolved out. Brady speaks of extensive formations of limestones being wholly made of Saccammina. The Elfhills formation, where this pseudomorph is well developed, used to be considered a concretionary limestone.

The outlines of the nummulites can be made out, but with difficulty, in lumps of the rock. The ``Saccammina'' condition is found in carboniferous and also in Ordovician limestones.

A peculiar spherular structure exists in the rock known as malmstone or gaize, which outwardly resembles an ordinary sedimentary sandstone, and which indeed contains to a greater or less degree sedimentary particles of sand, mica, \emph{etc.} The purer varieties contain, however, a large proportion of colloidal silica.

My sections of this remarkable rock are from material which I collected from a pit near Devizes, and from the formation (referred to by Barrois) situated by the railway siding.

The siliceous spherules or ``corps globuleux'' in malmstone have been taken for Radiolaria, parts of siliceous sponges, silicified oolites, \emph{etc.}

Some spherules resembling botryoidal structure, and varying from 20 to 33 $\mu$, were seen, under a power of 3000 diameters and low illumination, to be nummulitic spirodisks. The spherules usually show a well-defined equatorial band --- the outermost coil of the disk, within which there will be lesser coils. The disks can frequently be seen to be arranged in series.

Cayeux,\footnote{L. Cayeux. \emph{Contrib. à l'étude micrographique des terrains sédimentaires}. Mém. Soc. Géol. Nord, vol. 4. 2, p. 21, 1897.} in his description of a siliceous gaize of Launois, has noted therein a reticulated and radiated structure. Under a high power ``ce ciment présente un aspect spongieux très particulier. Le ciment n'est ici qu'un véritable squelette de silice.'' The appearance of spongy structure is due to the glassy layers of spiral lamina, and of radiating alars and septa of vitrified nummulites.

The colloidal silica apparently is derived partly from organisms, and partly, according to Cayeux, from decomposition of clay.

A band of flint in chalk is a layer of silicified nummulites, the silex being derived from skeletons situated in and above the once-calcareous layer.

Lapparent (\emph{Géologie} 1. p. 332, 1906) writes:-- ``l'origine de cette silice doit être incontestablement cherchée dans l'alteration de dépôts siliceux origin-airement superposés aux roches en question.''

\centerline{*\hspace{15mm}*\hspace{15mm}*\hspace{15mm}*\hspace{15mm}*}
\bigskip

In connection with this subject of ``spheres'' and siliceous ``spherules,'' I shall refer here --- rather than under ``flint and chert'' --- to an interesting discovery concerning certain supposed Radiolaria.

When I found that I had mistaken for Radiolaria and Diatoms numerous ``appearances'' in igneous rocks, it occurred to me that other observers might have fallen into the same error, and I find this to be the case. The very transparent disk-structures strongly suggest Radiolarian lattices and spines, but very careful inspection will reveal the real structure. L. Cayeux\footnote{Cayeux, Bull. Soc. Géol. France, 22. p. 197, plate 11. 1894.} described numerous genera of Radiolaria and some Foraminifera from the Pre-cambrian phthanites of Brittany. I believe all these structures to be parts of the anatomy of nummulites. During a recent visit to Brittany I collected phthanites and quartzites at Port-à-le-Duc and Ville-au-Roi-en-Maroué near Lamballe, localities whence M. Cayeux obtained his material. In my opinion, the sections show clearly that all the Radiolarian Cyrtids, Dicyrtids, \emph{etc.}, are different aspects of nummulitic structure. I had made the same mistake in \emph{Eozoön}, where I thought I had found seven genera of Radiolaria.

Dr. H. Rauff\footnote{H. Rauff. Neues Jahrb. Mineralogie 1. p. 117. 1896.} was right in refusing to believe in the Radiolarian nature of the very minute objects faithfully figured by Cayeux, but was mistaken in assuming them to be of an inorganic nature: the statement of Hinde and Fox also, ``There is no evidence to show that they were of organic origin.'' must be corrected.\footnote{Quart. Journ. Geol. Soc., vol. 51, p. 631. 1895.}

In the case of very ancient formations it seems to me improbable that Radiolaria (or any other siliceous plankton) could have escaped being dissolved. Accordingly I think the Radiolaria described by David and Howchin\footnote{Proc. Linn. Soc. N.S.W. 1896, p. 571.} from Precambrian rocks of South Australia may be of the same nature as the supposed Radiolaria in the Precambrian phthanites of Brittany, \emph{i.e.}, that they may be nummulitic structures. We find then that eminent microgeologists have seen in some of the older rocks certain appearances which they have mistaken for Radiolaria. Precisely the same appearances will be found in igneous rocks and meteorites. Dissolved silica of plankton organisms and, perhaps, of sponges has diffused itself through the benthos deposits of nummulitic shells replacing the lime. The sections of pillars, marginal cord and chamber-walls of these mineralized shells frequently present a striking resemblance to Spumellarian and Nassellarian Radiolaria and to Diatoms.\footnote{The picture on the cover of \emph{Nummulosphere}, Part 2, representing the genesis of igneous rocks, depicts plankton skeletons sinking through the water on to the heaps of nummulite shells constituting the ocean floor. The supposed plankton skeletons seen by me in igneous rocks are really parts of nummulites. Notwithstanding this error, the picture symbolizes events which have almost certainly happened. There is no doubt whatever about the nummulitic nature of igneous rocks, and in my opinion no reasonable doubt as to the source of much of the silica. It is a strange phenomenon this creation of the bulk of the earth's crust out of oceanic deposits of skeletons of benthos organisms mineralized by material derived partly from the siliceous skeletons of plankton organisms.}

Concerning the question of the relationship between Chalk and \emph{Globigerina} ooze, Cayeux\footnote{Mém. Soc. Géol. Nord, 4. p. 521. 1897.} clearly demonstrates that they are very distinct, especially with regard to the conditions of depth and the characters of the Invertebrate faunas. The common presence of \emph{Globigerina} has sometimes led to the profound dissimilarities between the two deposits being overlooked; and even in the Foraminiferal faunas there are essential differences. For Chalk is a nummulitic limestone, and at the present day the genus \emph{Nummulites} is practically extinct. The nummulites met with in \emph{Globigerina} ooze are silicated shells derived from subaerial or submarine eruptions.

\centerline{*\hspace{15mm}*\hspace{15mm}*\hspace{15mm}*\hspace{15mm}*}
\bigskip

\subsubsection{Flint}
\paragraph{}
Although it is the carbonate of lime that chiefly strikes the eye when looking at a chalk cliff, yet the silica element is almost equally important. Even where the silica is not collected into bands of tabular flint or of nodules, there is usually a good deal of silex disseminated throughout the whole rock.

Seemingly it has been clearly proved, especially by Hinde, Sollas and Zittel, that the greater part of the silex of flint and chert is derived from the siliceous skeletons of organisms, mostly Sponges in the case of chalk.

The silex of Diatoms, Radiolaria and Sponges is in the form of colloidal silica or opal, and is fairly soluble in sea-water. Accordingly this silica tends to accumulate on the sea-floor and to sink down a little way in that floor until it meets a dense stratum through which it cannot easily permeate. In course of time the silica molecules replace the carbonate of lime of calcareous skeletons and become grouped in semi-crystalline and crystalline arrangement, the silica then being less soluble.

Some geologists have denied the organic origin of the silica, believing it to be derived from highly charged sea-water rather than from organic remains accumulated on the sea-bottom. On the one hand there is no evidence to show that sea-water is highly charged with silica; indeed, the evidence is in the other direction, and on the other hand benthos and plankton siliceous skeletons are extremely abundant on sea-floors and often persist still undissolved in flint and chert.

Dr. Hinde,\footnote{G. J. Hinde, \emph{On the Organic Origin of the Chert of the Carboniferous Limestone Series of Ireland}. Geol. Mag. (3) 4. p. 435 (1887).} in a paper on the organic origin of chert, criticises the opinion of Professor Hull, who considered chert to be a pseudomorphic formerly-calcareous rock silicified by silica of inorganic origin. My own extremely careful observations on flint, chert, and Ordovician and Precambrian phthanites convince me that they are all silicified calcareous rocks, for in every case I can detect the original nummulites. On the other hand, direct and indirect evidence are strongly in favour of the theory of organic origin of the silica; in chalk-flint and Portland chert remains of siliceous skeletons still persist. (The supposed Radiolaria described by Cayeux from the earlier rocks are appearances presented by the transparent nummulitic structure.) The present theory reconciles the opposing views of Dr. Hinde and Professor Hull. For flint is not a mass of dissolved siliceous skeletons. It is, as Professor Hull surmised, a mass of silicified calcareous skeletons, amidst which, it is true, there are embedded a certain average proportion of siliceous remains perhaps no greater than in any other part of the whole deposit. The calcareous skeletons are mostly nummulites, though these have not yet been generally recognized. The silica of the flint comes from the whole mass of calcareo-siliceous deposit lying between two zones of flint and including the lower zone.

The successive layers of flints represent in a certain degree the material derived from successive ill-defined zones of organisms rather than from sharply defined beds. Sometimes flint is deposited along oblique fissures in the rock.

When rough weathered flints are examined carefully with a lens, the surface will nearly always be seen to be made of shells of \emph{Nummulites}. Further, even the most translucent sections of flint, under high powers and suitable illumination, will show the structure, especially the ``disks'' of the nummulite shells. All the light should be cut off at first, and then the faintest glimmer admitted to the field. Gradually the trained eye will see the structural features of \emph{Nummulites}. The least excess of light completely ``drowns'' the details. Flints, is formed of casts of Foraminifera (Appendix, Note B). Glauconite, which is a silicate of potash and alumina, is supposed usually to result from the breaking up of clay in presence of organic matter.

Concerning ``Barbados Earth,'' see Appendix, Note M.

\centerline{*\hspace{15mm}*\hspace{15mm}*\hspace{15mm}*\hspace{15mm}*}
\bigskip

Scattered in the chalk are nodules of iron pyrites, which often ``degenerate'' into powdery lumps of oxide of iron. Country people regard the nodules as ``thunder-bolts.'' Curiously enough these marcasite nodules actually do throw a light on the origin of iron meteorites. For both objects are ``ore-enriched'' masses of nummulitic rock.

The shells are best seen on finely granular fractured surfaces, especially if allowed to become weathered a little by exposure. The marcasite nodules have not formed in the void. The metallic salts have permeated a mass of Foraminiferal shells. Metallic Trilobites and Ammonites are familiar objects: the metallic nummulite shells on the other hand are indistinguishably massed together. I have seen the nummulitic structure in ore-deposits of iron, tin, uranium, \emph{etc.}

\centerline{*\hspace{15mm}*\hspace{15mm}*\hspace{15mm}*\hspace{15mm}*}
\bigskip

\subsubsection{On certain other Limestones}
\begin{displayquote}
``Sed calx a petrificatis.'' --- \emph{Linnaeus}.
\end{displayquote}
\paragraph{}
Calcareous formations are being built up at the present day; and right down through the ages to the Archaean Period these deposits have come into existence and have persisted as records of events in the history of Nature. In many cases we can see the mass of rock to be composed to a greater or less degree of the calcareous skeletons of organisms such as Corals, Crinoids, Mollusca, Polyzoa, the larger Foraminifera, \emph{etc.}, but often nothing is apparent excepting a fairly homogeneous or granular matrix.

Where no fossils or only a very few are visible, even under the microscope, it becomes a difficult question to account for the presence of the carbonate of lime.

The following extract from H. B. Brady's \emph{Memoir on the Carboniferous and Permian Foraminifera}, 1876, brings into relief some aspects of the problem referred to. He writes (p. 5):

``Take them as a whole, the Carboniferous Limestone beds of Great Britain cannot be regarded as a Microzoic formation in quite the sense in which the term is rightly applied to many Cretaceous rocks : indeed, as a rule, they owe their origin, so far as their organic constituents are concerned, much more to animals of higher organization and larger individual dimensions, such as Crinoids and Corals than to Microzoa. So far from owing its origin, like the true Chalk, chiefly to Foraminifera; or, indeed to go further, so far from being a deposit formed directly and exclusively by the agency of animals secreting carbonate of lime, \emph{there are considerable areas of Carboniferous Limestone in which the sea appears to have deposited its excess of mineral constituents in accordance with chemical and physical laws, without the intervention to any great extent of animal life}.''

``This has been brought about by a precipitation and the subsequent coalescence of the impalpable particles of amorphous precipitate into minute spheroids, the result being a concretional or oolitic limestone (often fossiliferous at the same time)... There need be no difficulty in the acceptance of a physical explanation of this sort, even by those who hold most firmly the theory that all limestones have primarily an organic origin.''

Recently\footnote{Carnegie Institution of Washington, Year Books Nos. 11, 12, 13, 1912-1914.} numerous observations and experiments on the precipitation of calcium carbonate in sea-water have been made by and under the direction of Dr. T. Wayland Vaughan. Off Florida and the Bahamas, banks of limestone are in process of formation which owe their origin entirely to chemical precipitations from sea-water through the agency of \emph{Pseudomonas} (\emph{Bacterium}) \emph{calcis} Drew: and, further, oolitic banks, apparently due partly to the aggregation of particles of carbonate of lime round bubbles of gas, are in process of formation.

Again, travertines, tufas, stalactites, amygdaloids, cementation of sea-floors, \emph{etc.}, all show that chemical precipitation of carbonate of lime is a very common occurrence.

G. Linck\footnote{\emph{Ueber Bildung der Oolithe und Rogensteine}. Jenaische Zeitsch. vol. 45, p. 276. 1909.} states: ``Oolite consists originally of a modification of carbonate of lime (Aragonite) in the formation of which organisms are not directly concerned. It is not a metabolism product of organisms.''

According to Van Hise,\footnote{\emph{Treatise on Metamorphism}, p. 907.} ``It is well known that it is difficult or impossible to discriminate limestones of organic or chemical origin.'' I would point out that the microscopic examination for nummulitic structure will be found to furnish an invaluable aid to diagnosis, though the test may be at times a little difficult to apply, as, for example, in the supposed non-fossiliferous matrix of Tyrol dolomites.

As a result of the examination of numerous marine limestones of all ages, I find that, just as in Chalk, a point of fundamental importance has been overlooked. Jurassic, Triassic, Permian, Carboniferous, Devonian, Silurian, Ordovician, Cambrian and Pre-Cambrian limestones are mainly nummulitic rocks. The ``matrix,'' whether homogeneous cannon-ball-like, pisolitic, oolitic, soft or hard, earthy or marmorized, dolomitized, silicified, silicated, or ore-enriched, will in most cases be found to consist of a mass of nummulite shells. The shells, though not easy to detect, can usually be made out with a hand lens, and the structure, especially the dotted or granulated disk-like structures, can be seen in sections under the microscope. Accordingly the carbonate of lime in all these rocks is mainly derived from organisms.

The oolitic structure not uncommon in calcareous material evidently arises in many different ways. The Jurassic Oolites are nummulitic limestones with oolitic structure, and the lime is certainly derived from the nummulites which constitute the mass of the rock. The shell-structure can be best seen in weathered material from ancient buildings. In these cases the oolitic structure which masks the shells becomes more or less dissolved out. (See Chapter 12)

My sections of typical hand-specimens of Schlern and Mendola dolomite from Seiss show that these lumps and presumably the huge masses from which they were detached are nummulitic limestones (See Appendix, Note C). Similarly with sections of marbles from various parts of the world kindly supplied to me by Messrs. Bingham.

\subsubsection{Summary of Chapter 2}
\paragraph{}
Chalk is a nummulitic limestone, formed mainly of a deposit of shells of \emph{Nummulites}, a genus hitherto only doubtfully known to occur before the Eocene. Many writers from Ehrenberg to the present time have seen in Chalk objects named ``morpholiths,'' ``spheres,'' \emph{etc.}, the nature of which has been doubtful, but which are now found to be portions of nummulite shells, the spheres being concretionary areas. Numerous other limestones from Jurassic to Archaean are nummulitic; but the shells have escaped detection, and the carbonate of lime often has been regarded as a chemical deposit.

Flint is composed mainly of nummulites in which the lime has been replaced by silica derived chiefly from organisms. The silicified shells are transparent in the interior of flint, but visible again on rough weathered surfaces. Silicified rocks of all ages (quartzites, jaspers, phthanites, \emph{etc.}) may show nummulitic structure, the latter sometimes having been mistaken for Radiolaria, other Foraminifera, sponge spicules, \emph{etc.} [The silicated rocks known as igneous (including also meteorites) are very ancient metamorphosed nummulitic limestones fundamentally similar to chalk.]

\emph{Postscript, Note} 1. --- I have now learned to distinguish in sections of chalk, Mendola- and Schlern-Dolomite, Jurassic oolite and Carrara marble, held up to the light and examined with a lens (3x or 10x), the larger structure of the nummulite shells, \emph{viz.} the successive coils of marginal cord and the radially disposed septa and alar prolongations. Certain parallel lines in the minute particles of calcite of Carrara marble are not due to crystallization but to the organic structure of the marginal cord.

I should mention that the simple skill in detecting the above-named structures in the mass of calcareous particles has only come to me at the end of a long period of painstaking observation and that some patience is necessary. The faint outlines of the spiral central coils and radial alars of perpendicular sections or aspects of shells will be the first structures to be seen. I now prefer to use a lens magnifying only 3 diameters.
\clearpage
\subsection{Chapter 3}
\subsubsection{``\emph{Eozoön}'' or ``The Dawn Animal''}
\centerline{\emph{The story of a double delusion.}}
\paragraph{}
The British sedimentary strata from recent to Cambrian, if piled horizontally, would at their thickest form a mass several miles high, the lower five-sixths being palaeozoic. Birds and mammals begin to appear about one-sixth of the way down from the top, reptiles and amphibia about one-third, and fishes half-way down, the lower half being invertebrates only.\footnote{These figures refer only to \emph{British} strata.} The strata may be compared to the leaves of a great folio in which is written --- though very imperfectly --- the record of the evolution of life from the lower to the higher stages. Beneath the Cambrian are Precambrian formations of unknown thickness, chiefly crystalline, stratified, foliated or schistose rocks (gneisses and schists), some sandstones and limestones, and volcanic rocks. They are exposed mostly in North Britain and the Hebrides. No undoubted fossils have been found in them. Once it was thought the Cambrian period was coeval with the beginning of life, but now the absence of organic remains in the older rocks is attributed rather to their obliteration.

Precambrian rocks are widely distributed over the world, but nowhere do they attain such enormous development as in Canada, where they cover an area of about two million square miles and attain a thickness of several tens of thousands of feet (Appendix, Note D). The accumulation of this vast series of highly metamorphosed deposits possibly required a greater time than was necessary to form the strata ranging from the Cambrian to the present. So far only a few Precambrian fossils have been recorded from the American continent, some even of these being of a doubtful nature (Appendix, Note E).

Forming the base of the Precambrian in Canada is a confused assemblage of highly crystalline igneous rocks to which Sir William Logan's term ``Laurentian''\footnote{Called after the Laurentide mountains north of the river St. Lawrence.} is now restricted: in contact with these, in the region of the Ottawa River, there are limestones, quartzites, gneisses, metamorphosed clays and sandy-clays known as the ``Grenville'' series.\footnote{J. Stansfield, \emph{Internat. Geol. Congress, Canada, 1913}. Guide Book No. 3, p. 82. Bibliogr. Index, p. 115.} These sedimentary rocks are broken into by masses and dykes of ``Ottawa gneiss''; the masses, notwithstanding their stratified structure, now being generally regarded as plutonic up-wellings (bathyliths) rather than as altered sediments.

The Laurentian and Grenville rocks belong to the most ancient era, ``the Archaean.'' They constitute the ``fundamental or basal complex.''

\centerline{*\hspace{15mm}*\hspace{15mm}*\hspace{15mm}*\hspace{15mm}*}
\bigskip

In 1858 Sir William Logan, one of the great pioneers of Canadian geology, began to suspect that certain peculiar lumps of rock found in Archaean limestone, and which had been sent in as minerals, might possibly be of organic origin. Weathered lumps showed a layered structure (Plate 2) like that of the palaeozoic stromatoporoids --- which though palaeozoic had come into existence ages later.
\begin{figure}[H]
\centering
\includegraphics[width=90mm,keepaspectratio]{figures/Fig6.png}
\caption*{\centerline{\small \textsc{Figure 6 --- \emph{Eozoön canadense}.}}

\hspace{5mm} \small Polished slab showing white and green bands. 2x. Traces of nummulitic shell structure very faintly visible in places.}
\end{figure}
Sections revealed a banded structure consisting of alternate white and green wavy bands of calcite and serpentine (Fig. 6).

Sir W. Dawson, who supported Logan's view, regarded the lumps as giant Foraminifera belonging to the nummulite family, the white zones being the walls of the chambers, and the green parts the spaces formerly filled in with living protoplasm. The white zones contained systems of branching ``canals.''

In 1862 Logan, when on a visit to London, showed the specimens to Dr. W. B. Carpenter, the highest authority on Foraminifera. Carpenter confirmed Dawson's nummulite theory, and demonstrated that in some specimens the green spaces were lined with a definite, very finely fibrillated layer, which he regarded as the proper tubulated wall of each chamber of the shell, the white bands with their branching ``canals'' being the ``supplementary'' skeleton commonly found in nummulitid Foraminifera.

In 1865, Logan, Dawson, Carpenter and Sterry Hunt\footnote{Quart. Journ. Geol. Soc., Feb. 1865, pp. 45-71, and plates.} published a joint paper on \emph{Eozoön canadense} gen. et sp. nov. Dawson. This paper started a prolonged and stubborn controversy.
\begin{figure}[H]
\centering
\includegraphics[width=65mm,keepaspectratio]{figures/Fig7.png}
\caption*{\centerline{\small \textsc{Figure 7 --- Diagram of section of \emph{Eozoön}}}

\centerline{\small \textsc{as it was supposed to be in life.}}

\hspace{5mm} \small After Dawson and Carpenter. \emph{a}, chambers of shell full of protoplasm (now green bands of serpentine); \emph{b}, finely tubulated ``true wall,'' best seen at upper part of figure where there is no additional calcite; \emph{c}, pseudopods; \emph{d}, supplementary or additional deposit of calcite forming the white bands; \emph{e}, canals in the white bands.

\hspace{5mm} \small ``\emph{Eozoön}'' was once a mass of nummulite shells like any lump of Tertiary nummulitic limestone say of Biarritz or Egypt. The living top layer shells would show what ``\emph{Eozoön}'' was like in life.}
\end{figure}
In 1866 Professors King and Rowney\footnote{Quart. Journ. Geol. Soc., Feb. 1866, pp. 185-218, and plates.} pointed out that all the structures brought forward by Carpenter as proofs of organic origin could be shown to be of a purely mineral nature and origin. The alternating zones, and the dendritic structures in the white bands were common in minerals, and the fibrillated so-called ``nummuline'' or ``proper wall'' was simply a fibrous or asbestiform variety of serpentine known as chrysotile.

In 1878 Professor Möbius of Kiel, a high authority on Foraminifera, published a well-illustrated memoir\footnote{\emph{Palaeontographica} 25. p. 175.} on \emph{Eozoön}, which latter he regarded as an object of purely mineral origin.

In 1894 Professors H. J. Johnston-Lavis and J. W. Gregory\footnote{Sci. Trans. Roy. Dublin Soc. (2) 5. p. 259. 1893.} described certain bombs ejected from Monte Somma (Vesuvius), which have a zoned arrangement of calcite and silicates practically identical with that of ``\emph{Eozoön canadense},'' excepting that the silicates are not hydrated into serpentines and there is no ``nummuline'' layer of chrysotile (fibrous serpentine). The authors considered the material of the bombs to be of Cretaceous or Jurassic age.

The problem was now considered settled. In all the chief text-books \emph{Eozoön} was put down as an object of mineral origin, and science, with evident relief, pronounced its \emph{requiescat in pace} on the controversy. Any signs of recrudescence call forth a note of alarm --- ``We are threatened with a revival...''

New evidence, however, renders it imperatively necessary to exhume the dawn-animal and hold a fresh inquest. This time the enquiry need only be brief, and beyond doubt the verdict will be final.

\centerline{*\hspace{15mm}*\hspace{15mm}*\hspace{15mm}*\hspace{15mm}*}
\bigskip

Once when examining one of Carpenter's sections under the microscope I saw embedded in serpentine a very small object 0.1 mm. in diameter, which I took to be a young nummulite shell\footnote{Described, but very incorrectly, in A.M.N.H. Sept. and Oct. 1912.} (Plate 11, Fig. 42), and presently found several similar objects.\footnote{Photographed better in \emph{Nummulosphere} 1. 1913. The photo on Plate 11 is too dark.} Further research apparently led to the discovery that a specimen of \emph{Eozoön} was a mass of small ``nummulite'' shells. At first they seemed to me to be associated in colonies. The supposed shells, on an average about four millimeters in diameter, were present both in the green and white bands, frequently also one ``shell'' being half in the green and half in the white. Next I discovered in the sections under high powers ``appearances'' closely resembling Radiolaria and Diatoms.\footnote{\emph{Nummulosphere} 2. 1913.}

Since \emph{Nummulosphere} 2 was published I have made numerous observations on varied and abundant material, \emph{viz.}, Carpenter's collection, including ``\emph{Eozoön}'' from Scotland, Ireland, and Bavaria (sent by Gümbel); a magnificent set of specimens sent to me by Mr. L. Lambe from Canada; material from Finland sent to me by Prof. J. J. Sederholm and Dr. O. Trüstedt; a typical Monte Somma bomb (Hamilton Coll. N. H. M.); also archaean limestone from the Sikhim Himalayas, very kindly obtained for me by Mr. John C. White.

I have now found that the small disks visible with a hand lens, and which had been taken by me for small nummulites, are portions of nummulite shells. Further, the supposed Radiolaria and Diatoms are found to be appearances due to transparent series and groups of nummulitic disk structures viewed under high powers and in various aspects; and, lastly, the supposed microscopic shells of which I had published photographs were also merely portions of nummulitic structure.

Professors King and Rowney\footnote{\emph{An Old Chapter in the Geological Record}, 1881.} asked believers in the organic theory to reply without question-begging irrelevancy or reticence to eleven points, foraminiferal, mineralogical, chemical and geological.

The organic theory has now taken on an almost entirely new aspect, and I think it will be amply sufficient to reply here to the first two points, which are the most important.

Point No. 1. The existence of an \emph{upper} and an \emph{under} ``proper wall'' in immediate connection with a ``chamber,'' also frequent absence of intermediate skeleton, and frequent horizontality --- instead of verticality --- of tubuli to the adjacent chamber.

Answer. The green bands are not chamber-spaces formerly containing protoplasm but now filled in with serpentine; there is no ``proper wall'' to the supposed ``spaces,'' and there is no supplementary skeleton (in the form of white bands) between ``spaces'' which never existed.

There \emph{is} a lump of limestone in which silicates have been diffused out from some central accumulation and precipitated in zones, thereby differentiating the lump from the common limestone matrix. Careful observation shows the archaean limestone to be a nummulitic deposit similar to Tertiary nummulitic limestones, but more changed owing to long subjection to metamorphosing agencies, and especially to the proximity of igneous intrusions. Even Cretaceous and Jurassic limestones give rise to silicatic Eozoönal structure when subjected to volcanic heat as in the Monte Somma bombs.

Careful observation under high powers shows the nummulitic disk structure underlying fine fibrillar pattern, just as in the striated bands of spiral lamina in Tertiary nummulites. (See Chapter 8)

Möbius denied the nummulitic tubulated nature of the fibrils or aciculae on the ground of their occasional great length and curved shape, and their varied direction and ``non-verticality.'' I have now found the reply to these reasonable objections. Very frequently in sections of \emph{Eozoön} long fibrous bands are met with having characters such as Möbius described. They are portions of concentric bands of furrowed marginal cord with septa astride. The cord is really a wavy almost polygonal structure, and the direction of series of disk structures varies greatly in adjacent parts. Again, in oblique sections of large shells across many successive layers of spiral lamina, there is sometimes a false appearance of fibrillar continuity.

In the fine collection of serpentine minerals and rocks in the Natural History Museum there are a number of asbestiform fibrous varieties (chrysolite, metaxite) of this ``protean mineral.'' Here the fibrous structure is a character of the mineral. I have found nummulitic structure persisting even in real asbestos, that structure being situated at any angle to the mineral fiber.

Point No. 2. ``The supposed `canal-systems' have no constant correlativeness.''

Answer. King and Rowney are wholly right. The ``canals'' are due to mineralization, and are not canals at all, for they have never been hollow. Plate 23, Fig. G, shows two of these ``canals.'' They are filled with nummulitic structure in continuity with similar but barely visible structures in the adjacent calcite. Solid serpentine ``casts'' of the canals etched out by acid are solid masses of nummulitic structure and not structureless infillings.

Dr. Hahn (\emph{Die Urzelle}) was opposed to the nummulite theory. He writes: ``The assertion that \emph{Eozoön} is a Foraminiferan comes to grief with the proof that it is formed of plants.'' Dr. Hahn thought the existence of fiber-like and also of cup-shaped canals fatal to the Foraminiferal theory. He made many genera of algae out of the ``canal-systems.'' (I myself at one time imagined I had found algae in \emph{Eozoön}.) Dr. Hahn figures a worm with a tripartite operculum, ``das erste Tier'' --- the ancestor of Trilobites --- ``which browsed on this succulent vegetation'' (Whitney and Wadsworth). Some figures of the worm apparently show transverse sections of spiral lamina or cord, the operculum perhaps being a pillar-end or septum. Notwithstanding his mistakes, Dr. Hahn was nearer the truth than were some of his critics.

On the weathered lumps of \emph{Eozoön} I can trace with the naked eye and lens obscure outlines of large nummulites about an inch in diameter and in various aspects. Many hours of patient observation were required at first to make out the details, but I can now make out with certainty numerous details of shell structure (Plate 13, Fig. C). Sections 3x show the shells fairly well.

Under a power of 400 diameters (16 mm., Oc. 18, a useful combination) I can now see with the utmost certainty abundance of nummulitic shell structure both in the serpentine and in the calcite. I had long studied my sections without seeing what is now so obvious. In fact it was a case of not seeing the wood for the trees. Any large transparent nummulite in horizontal aspect, and showing floor upon floor of spiral lamina with range upon range of band-like alar prolongations, series upon series of conical pillars, and the concentric many-furrowed marginal cord, considerably obliterated here and there, with mottled patches of serpentine and calcite, would naturally be a very complicated mass of structure. Under lower powers it is difficult to trace the continuity, and under high ones only a very small part of a shell comes into the field; for a shell an inch across, magnified 400 diameters, would cover 120 square yards. Under this power the structural outlines are clear and indubitable. The transverse aspect shows layers of spiral lamina and pillars seen lengthwise. Very high powers show abundantly the disks, especially in their transverse aspect.

The complicated and confused reticular aspect of the shells as viewed in transparent serpentine is due to radial and concentric layers being cut at various angles. A little training and much patience are necessary to enable the beginner to discern and piece together the nummulitic pattern. The concentric circles of furrowed cord and the radial alar prolongations are good guides to orientation in shells viewed in horizontal aspect.

\subsubsection{On the Origin of the Banded Structure in \emph{Eozoön}}
\paragraph{}
Many theories have been advocated to account for the white and green (calcareous and silicatic) zones of \emph{Eozoön}. Firstly there is the organic theory, \emph{viz.} that the calcareous zones were the skeleton of a Foraminiferan, the silicatic zones being in-fillings of the spaces of the shell. This theory may be finally dismissed.

W. J. Sollas and Cole\footnote{Sci. Proc. Dublin Soc. 7. p. 124. 1891.} suggest that calcareous and silicatic zones might arise by formation of successive layers of particles of carbonate of lime and olivine on the reef-bound shores of volcanic islands. I myself have seen on the beach of Porto Santo Island layers of calcareous grains and of dark green particles of augite sifted by gentle wave-action. In places, these sandy layers have been cemented into solid stone, of which I have specimens. The calcitic and augitic bands, however, vary in thickness irregularly, whereas the bands of \emph{Eozoön} diminish regularly and serially from a base or center.

Johnston-Lavis and Gregory regard the pyroxene lumps that often form the nucleus of the \emph{Eozoön} as portions of an igneous magma that have been separated from the main source and grouted into a soft pasty limestone. From the nuclei thus deposited silicic vapours and fluids became diffused outwards, combining with the magnesium, calcium and iron to form ferromagnesian or pyroxene compounds, these being deposited in zones gradually diminishing in width. Firstly a thick band is formed, then another not so thick, and so on, till the material is used up.

There are two questions involved here, \emph{viz.} the origin of the pyroxene and of the zonal structure associated with that material. Nothing could at first sight seem more reasonable than the views of these writers. There are always igneous masses or intrusions in the neighbourhood of \emph{Eozoön}, and the zones of \emph{Eozoön}, as viewed, actually do diminish in width from center to periphery. Notwithstanding, the derivation of \emph{Eozoön} from igneous material is very improbable; and, further, possibly the zones did not originate quite in the manner suggested by Johnston-Lavis and Gregory.

O. Trüstedt,\footnote{Bull. Comm. Geol. Finlande. No. 19, p. 214. 1907.} in a report on the ore-deposits of Pitkaranta, Finland, gives a picture (from a photo) of the limestone quarry of Hopunwaara. Fig. 8 below is a diagrammatic plan of the same, showing bands of ``sahlit-skarn'' (pyroxene), serpentine and \emph{Eozoön} as they appear in the limestone. The limestone is separated from the granite by layers of mica-schist and iron-ore. 

If the lumps and bands of pyroxene had originated from the granite, one would expect the first ten meters of the limestone in the neighbourhood of the granite to be crowded with the pyroxene, but there is scarcely a trace, and none at all in the first five meters. 
\begin{figure}[H]
\centering
\includegraphics[width=95mm,keepaspectratio]{figures/Fig8.png}
\caption*{\centerline{\small \textsc{Figure 8 --- Diagram of cliff at Hopunwaara, Finland.}}

\hspace{5mm} \small \emph{Ka}, limestone; \emph{Sa}, sahlite; \emph{Se}, \emph{Eozoön}; \emph{Fe}, iron ore; \emph{R}, Granite; \emph{Gl.}, mica-schist; N, S, north, south; \emph{m}, meters. After O. Trüstedt.}
\end{figure}
Prof. T. G. Bonney,\footnote{\emph{The Story of our Earth}. 1893, p. 388, Fig. 131; also Geol. Assoc. 1895, p. 292.} who saw \emph{Eozoön} \emph{in situ} in the limestone at Côte St. Pierre, Quebec, figures masses of pyroxene or serpentine or of both minerals surrounded by bands of \emph{Eozoön}. The conditions in the Laurentian and Hopunwaara limestones are closely similar. Prof. Sederholm, who very kindly sent me specimens of Hopunwaara limestone and of other Finland rocks, thereby supplementing a set of \emph{Eozoön}, sahlit, \emph{etc.}, previously sent by Dr. Trüstedt, writes in a letter to me:-- ``The limestone is Ladogian,'' \emph{i.e.} it is to be correlated with the Archaean.

All the evidence goes to show that the lumps and bands of pyroxene and their surrounding zones of serpentine belong to the limestone itself, just as the lumps or bands of flint in a chalk quarry belong to the chalk. In the archaean limestone, owing to the heat from neighbouring igneous magmas, the silica has combined with the magnesium, calcium, aluminium, iron, \emph{etc.}, to form serpentine, loganite, \emph{etc.}, the silicates being precipitated in bands. Both the flint and the \emph{Eozoön} are mineralized lumps or masses of nummulite shells, the silica probably being derived from organisms.

The experiments of R. Liesegang (``Geologische Diffusionen'') on precipitation, show the order of deposition of the ``rhythmical precipitations'' to be from periphery to center. If, as is probable, the same law holds for \emph{Eozoön} the fine outer bands of ferromagnesian silicates would be thrown down first. 

\centerline{*\hspace{15mm}*\hspace{15mm}*\hspace{15mm}*\hspace{15mm}*}
\bigskip

\emph{Eozoön} has been a twofold source of error. Firstly there was the mistake for which Dawson and Carpenter were chiefly responsible, that certain banded lumps of serpentine limestone were gigantic reef-like Foraminifera. After doing great service in destroying this illusion, the mineralogists and petrologists went too far and fell into error themselves. Whitney and Wadsworth\footnote{Bull. Mus. Comp. Zool. 7. p. 534. 1884.} write of ``the extraordinary delusion which has prevailed among palaeontologists with reference to the organic nature of \emph{Eozoön}.''

It was a case of denying not merely the organic nature of \emph{Eozoön} \emph{qua} \emph{Eozoön}, but also an organic origin of any sort. As these authors expressly state, their ``Azoic System'' comprised a series of the earliest rocks which had, according to them, never been anything else than azoic. I have already given reasons for my conviction (\emph{Nummulosphere} 1) that there is no ``azoic system'' accessible to investigation. Numerous later observations have confirmed this view.

King and Rowney (Q.J.G.S. \emph{l.c.} p. 216) regard \emph{Eozoön} as having ``existed at one time as hornblendic or augitic gneiss, and that it is primarily of sedimentary origin,'' \emph{i.e.} an azoic sediment derived from azoic rocks. Johnston-Lavis and Gregory (\emph{l.c.}) consider \emph{Eozoön} to be derived from igneous magmas, \emph{i.e.} to be azoic.

\emph{Eozoön} is not an organism any more than is a lump of chalk or flint, but at the same time it resembles those objects in being of organic origin. For they are all masses of Foraminiferal shells, not much altered in the chalk, solidly silicified in the flint and silicated in zones in the ``\emph{Eozoön}.'' Seeing that \emph{Eozoön} is neither a giant Foraminiferan nor an object of purely mineral origin, its story may justly be described as that of a double delusion.\footnote{One might have said ``multiple delusion,'' but it is only necessary to refer here to the main points in the historic controversy.} As in so many controversies there were truth and error on both sides. The present theory will effect a reconciliation.

\centerline{*\hspace{15mm}*\hspace{15mm}*\hspace{15mm}*\hspace{15mm}*}
\bigskip

There is a vague and indefinite era in the history of the planet when geology first enters into its inheritance and astronomy withdraws.

All theories of planetary origin postulate a nebula of some kind. The Laplacean and meteoritic theories assume that a hot gaseous nebula or one of clashing meteorites condensed into a molten globe on which a crust formed on cooling. The planetesimal theory postulates a nebula of solid and liquid particles all rotating in one direction round a solid central nucleus, the latter growing by accretion. In the first case the ocean would form rather by precipitation, in the second by exudation.\footnote{Chamberlin and Salisbury. \emph{Geology, Advanced Course}, 2. 1906.}

The part of the earth's crust that might be expected to furnish evidence concerning planetary origin would naturally be in the oldest zone, \emph{i.e.} the archaean or ``basal complex.''

In Canada Laurentian rocks constitute the lowest floor. Immediately over this lie the Grenville series in the Ottawa River region, and the Keewatin schists and volcanic rocks in Western Ontario.

This volcanic schistose and sedimentary series overlying the Laurentian igneous rocks has been a source of perplexity to plutonists. The problem, in fact, recalls the great ``Tortoise myth.'' Between the schist series and igneous rocks there is no intervening floor, the contact being ``intrusive'' and immediate.

Now sediments and lavas must rest upon or flow over a firm floor of some sort. The difficulty has been met by assuming that a formerly existing floor has been dissolved in rising floods of molten magma.

Perhaps the difficulties above referred to may be explained in the following manner. There was certainly a time when ``Laurentia'' was being built up shell by shell beneath the sea. Then came a gradual upheaval, the lower deeper-seated region of the mass having undergone the plutonic kind of crystallization. During long ages Laurentia was subject to denudation. Then came a period of partial subsidence when the Grenville limestones were deposited on the submerged Laurentian floor, along with sediments from neighbouring land. Still later there was violent disturbance which once again melted the Ottawa gneisses and injected them into the crumpled layers\footnote{See photo of contorted Grenville limestone. Canadian Geol. Survey, 12. J. 1899, plate 4. Dawson's ``oscular chimneys'' in \emph{Eozoön} are probably structures due to compression.} of the Grenville series, the heat giving rise here and there to \emph{Eozoön} structure.

The Laurentian rocks are not primeval magmas welling up from some very deep central source, but are ``daughters of time,'' in fact old sea-floors. The Laurentian rocks are themselves permeated by intrusive dykes. So there are sea-floors under sea-floors, ``deep under deep.''

Possibly the world is vastly older than has been usually believed.

Certainly the Laurentian and igneous rocks have just as much to do with nebular hypotheses as a lump of chalk has --- and just as little.

The conclusion is reached that hitherto we have been studying only a thin pellicle of organic origin and have in no wise got beyond the zone of life into the azoic.

\centerline{*\hspace{15mm}*\hspace{15mm}*\hspace{15mm}*\hspace{15mm}*}
\bigskip

King and Rowney (\emph{An Old Chapter....}, 1881) give in the course of fifty-three pages an annotated history of the \emph{Eozoön} controversy, and refer to 108 papers, memoirs, \emph{etc.} In fact, \emph{Eozoön} has a literature of its own. Dr. Hahn (\emph{Die Urzelle}, p. 11) writes:-- ``Veritably a greater riddle than \emph{Eozoön} has rarely been proposed to natural philosophy.'' In view of the prolonged controversy what now seems so strange is the simplicity of the problem. All that is required is to examine for nummulites. It is surprising to find that it is possible to see with the naked eye the outlines of the shells on the surface of weathered blocks of \emph{Eozoön}, and to trace under a weak lens a good deal of the structure. (See Plate 13, Fig. C, and Plate 14, Figs. D, E).

The discovery of the nummulites confirms the theory of the mineralogists that the banding was a purely mineralogical feature, but at the same time indicates the organic origin of the carbonate of lime and the marine origin of the silica and magnesia.

Seemingly the \emph{tria regna} have each contributed to the making of the Dawn Animal, for it is a mass of nummulites mineralized by various minerals and especially by silica probably derived partly from plants.

From the point of view of rock metamorphism, \emph{Eozoön} (limestone + silicate) is important as illustrating a transition-stage between ordinary limestone (limestone + silica) [Chap. 2] and igneous rocks (pure silicates) [Chap. 4]. There is a common nummulitic basis in all.

Perhaps it was well that the riddle of \emph{Eozoön} was not quickly solved, for in the prolonged efforts to find the answer much valuable knowledge was acquired.

\centerline{*\hspace{15mm}*\hspace{15mm}*\hspace{15mm}*\hspace{15mm}*}
\bigskip

\subsubsection{Summary of Chapter 3}
\paragraph{}
Specimens of ``\emph{Eozoön}'' are lumps of nummulitic limestone mineralized by silicates deposited in zones.
\clearpage
\subsection{Chapter 4}
\subsubsection{The Igneous Rocks}
\begin{displayquote}
``It is important to acquire a hospitable and intelligent preparedness to appreciate new light as it shall present itself.'' --- \emph{Chamberlin and Salisbury}.
\end{displayquote}
\paragraph{}
Scattered over the face of the world are mountains of a very peculiar nature. They are usually conical and with an open bowl-shaped cavity at or near the summit. From time to time in the course of years or centuries, lumps of hot rock, clouds of ashes, vapours and steam are shot up from the interior perhaps to a great height, and floods of molten rock well up and flow down the mountain side. The name ``volcano'' called after ``Vulcano'' or ``Vulcan's stithy'' in the Lipari Isles, expresses an ancient idea as to the nature of these mountains.

The volcanic mountain, which is an accumulation of ejecta round a pipe leading deep beneath the surface, is not an essential feature. Sometimes the molten rock has welled up through one or many fissures in the ground. In Iceland in 1783 a great flood of lava poured out through a fissure twenty miles long and only a few feet wide in places, and submerged the country for a distance of forty miles from the source.

Desmarest and Hutton showed basalt and granite to be rocks which had once been molten like lava, \emph{i.e.}, they were ``igneous'' rocks which had been pressed upwards into and, to a greater or less extent, through the earth's crust.

The emission of lava through the surface is merely one feature of igneous activity, the latter consisting essentially in the ascent of molten rock into the overlying crust.

With the advance of knowledge, it has become evident that by far the greater part of the planetary crust is composed of igneous rocks and of sediments mainly derived therefrom.

Igneous rocks are widely and extensively distributed over the globe, and igneous phenomena have manifested themselves in all eras from the beginning of geological time up to the present.

Sir A. Geikie enumerates six great epochs of igneous activity in the British Isles, \emph{viz.}, two Archaeozoic, two Palaeozoic and two Cainozoic, the Mesozoic era happening to fall within a period of lull.

Among the more remarkable records of volcanic activity may be mentioned the lava-floods which poured out in Tertiary times over areas extending from north-west Britain and Ireland to within the Arctic Circle. The lava now forms great plateaux of basalt in N. W. Scotland, Antrim, the Hebrides, Faroe, Iceland and Southern Greenland.

The great Pliocene lava-flood of Western America which Le Conte considered ``among the most extraordinary in the world,''\footnote{J. Le Conte, \emph{On the Great Lava-flood of the West}. American Journ. Sci., 1874, p. 167.} covers an area of about 200,000 square miles, and has a thickness of from 700 to 3700 feet. The traps of the Deccan, poured out during the Cretaceous period, cover 200,000 square miles, the average thickness being about a mile.

Again, in a map, ``South of the Amazon, Boué colours an area composed of rocks of this nature'' --- \emph{i.e.} granitic --- ``as equal to that of Spain, France, Italy, part of Germany, and the British Islands, all conjoined.'' (Darwin, \emph{Origin of Species}.)

In the north of England, the Great Whin Sill in the carboniferous limestone is estimated by Sir A. Geikie\footnote{Sir A. Geikie, \emph{Ancient Volcanoes of Britain}, 2. p. 2. 1897.} to cover an area of 1000 square miles and to be on an average about 90 feet thick.

The igneous rocks forming the base of the Archaean and said to be exposed over about one-fifth of the land area of the globe are thought to be a universal formation.

In a region included within the states of Colorado, Utah and Arizona there are curious isolated groups of mountains not connected with the ordinary orographic lines, and now known to have been elevated by the rising up of huge loaf-like masses of igneous rock which have failed to burst through the vast but plastic\footnote{``The solid crust of the earth, and the solid earth if it be solid, are as plastic \emph{in great masses} as wax is in small.'' G. K. Gilbert. \emph{Report on the Geology of the Henry Mountains}, p. 97, 1877.} pile of sedimentary strata formerly completely --- and now partly --- overlying them.\footnote{G. K. Gilbert \emph{l.c.}; and W. Cross, \emph{The Laccolitic mountain groups of Colorado, Utah, and Arizona}. U.S. Geol. Survey, 14th Ann. Rep. 1892-3, 2. p. 157.} Denudation has exposed here and there the huge igneous bosses or laccolites, thereby revealing the secret of the origin of these exceptional mountains, which may in a sense be compared to aborted volcanoes.

The ocean floor is formed chiefly of nummulitic clay derived from igneous rocks (see Chapter 1).

\centerline{*\hspace{15mm}*\hspace{15mm}*\hspace{15mm}*\hspace{15mm}*}
\bigskip

The igneous rocks are masses of compounds of silica, in the form of silicates, in unconformable\footnote{The term ``unconformable'' is used in an extended sense. Fossiliferous formations (igneous rocks) have been thrust upwards into unconformable relation with later formed strata.} relation with sedimentary strata.

What, then, are these masses of silicates which from the beginning of the Archaean era have been bursting into and between the rocks above them, and erupting through and pouring over the surface of the land or ocean floor. What is their origin? Why have they been hot and plastic at one time in their history? and why have they shifted their position?

J. D. Dana writes in 1890:\footnote{\emph{Characteristics of volcanoes}. J. D. Dana, 1890, p.24.} ``The origin of volcanic heat, the source of lava columns beneath the volcano, the cause of the ascensive force in the lava column, are subjects on which science has various opinions and no positive knowledge.'' Precisely the same statement would apply to igneous phenomena of which the volcanic are but a part.

With regard to the second of the three problems referred to by Dana, I hope to demonstrate that positive knowledge is now available concerning the source and origin of igneous material.

If that is so, it is reasonable to expect that some additional light may be thrown on the other two problems.

\subsubsection{The Origin of the Igneous Rocks}
\paragraph{}
Werner and his school believed lava, basalt and granite to be minerals precipitated from a universal ocean, the heated condition of lava being supposed to be due to burning coal-fields. Desmarest showed convincingly that basalt was simply an ancient lava, and that it got into position by flowing there as a molten rock, and not by being deposited as an aqueous sediment. Hutton proved that granite likewise was a once molten rock essentially similar to lava, but deep-seated. Accordingly lava, basalt, granite and their like are termed ``igneous'' rocks.

Some have regarded the earth as a molten globe with a solid crust, and the igneous rocks as up-wellings from the original still molten planetary magma: but according to physicists the globe must be as rigid as steel.

A view still held by many is that the original planetary material deep below the crust is at a temperature above normal melting-point but remains solid on account of pressure. When, perhaps owing to crust disturbances, pressure is relieved the rock melts and rises through the crust.

Some, again, have regarded igneous rocks as metamorphosed sediments. Gradations could, it was supposed, be traced from recent soft sediments, through less recent hard shales, sandstones, \emph{etc.}, to ancient very hard crystalline gneisses and schists, and on --- by somewhat of a jump rather than a step --- even to certain igneous rocks, ``and thus we look in vain for the original material'' (Le Conte), but it is the source of the original material that we are in search of.

The theory of organic origin held from time to time by a few ``eccentrics,'' and usually dismissed as soon as mentioned is beyond any doubt the true one.

Organisms in igneous rocks have often been described (Appendix, Note G). Sometimes the supposed igneous rock is really a fossiliferous sedimentary one changed by igneous intrusions. Again the ``organisms'' may be ``structural simulations.''\footnote{King and Rowney \emph{On the Serpentine of the Lizard}. Phil. Mag. (5) 1. 1876, p. 280.}

Dr. Carpenter, an expert on the microscope and the highest authority on Foraminifera, mistook mineral for Foraminiferal structure in a very ancient serpentine limestone --- not, it is true, an igneous rock, but containing materials akin to igneous. I, for my part, arrived at the truth almost by accident. In the course of examining certain fossils in connection with a problem in sponges, I drifted on to \emph{Eozoön}, thence to the investigation of a Monte Somma (Eozonal) bomb, then to volcanic rocks in Porto Santo, and finally to igneous rocks in general. I continually found traces of nummulite shells. Although the truth was first discovered by direct observation, the fact of organic origin might have been, or rather, at the present day, might be suspected on (1) biological, (2) geological and (3) chemical grounds.

\bigskip
\centerline{\emph{1. Biological Considerations}}

The earth is an ocean planet. Five-sevenths of its area is now covered by ocean, and the remaining two-sevenths has been submerged.

The ocean surface is rich in a simple siliceous flora and fauna, and the ocean floor over vast areas is carpeted with benthos and plankton calcareous remains. The Diatoms and the often symbiotic silica-secreting Radiolaria exist in sunshine and nutrient fluid, and have done so for aeons past.

The powers of multiplication are almost unlimited. One diatom dividing only once in twenty-four hours might have a billion descendants in a month.

It is not surprising that igneous rocks have been classified according to silica-content. The skeletons of silica sink to the bottom, dissolve\footnote{In \emph{Nummulosphere} 2 I wrote of siliceous skeletons forming centers whence silica was diffused, a residuum remaining as quartz. The supposed skeletons are nummulitic structures.} there, and become diffused through calcareous deposits. The silica replaces the lime of the little fossils in the same way as in larger fossils. (See postscript on p. 110.)

\bigskip
\centerline{\emph{2. Geological Considerations}}

Immense accumulations of nummulite shells of Eocene age extend across N.W. Africa, Europe and Asia from Morocco to Japan.

Chalk is mainly an accumulation of nummulite shells of Cretaceous age. The soluble silica derived from sponges and plankton diffuses itself down through the mass of calcareous skeletons, replacing the carbonate of lime and forming lumps and layers of ``flint'' (chiefly silicified nummulites).

Most of the marine limestones from Eocene to Archaeozoic are nummulitic, even where macrozoic skeletons abound.

In very ancient Archaeozoic limestones, owing to the heat from intrusive dykes, the silica unites with magnesia, alumina, iron, \emph{etc.}, to form silicates in the form of pyroxenes.

The masses of pyroxene are often surrounded by a zone of banded limestone and serpentine (``\emph{Eozoön}''). It is significant that these lumps and layers of silicate, which may be compared with lumps and layers of flint, have been mistaken for fragments of igneous rock projected in from intrusive magmas.\footnote{In this instance Daubenton's name is well bestowed: ``Piroxène, c'est-à-dire, hôte ou étranger dans le domaine du feu.'' Haüy, \emph{Traité de Minéralogie} 3. p. 180.} Even in Cretaceous and Jurassic limestones silica becomes silicate when heated, as in the Monte Somma bombs.

Igneous rocks are very ancient highly metamorphosed nummulitic limestones from which naturally all the original calcium carbonate has gone, although a good deal of calcium and even some imprisoned carbonic acid may remain.

Iddings writes: ``Calcium is one of the most abundant and wide-spread elements in igneous rocks, and enters into a great variety of compounds, the commonest of which are silicates.''\footnote{\emph{Igneous Rocks} 1. p. 39. 1909.} The presence of calcalkalic feldspars constitutes one of the chief characteristics of the great ``Pacific petrographical province.''\footnote{Iddings, \emph{The Origin of Igneous Rocks}. Bull. Phil. Soc., Washington, p. 183. 1895.}

\bigskip
\centerline{\emph{3. Chemical Considerations}}

A. Important constituents of sea-water.

\Tree[.{chlorides and sulphates} [.Calcium ]
          [.Magnesium ]
          [.Sodium ]
          [.Potassium ]]

\bigskip

\Tree[.{oxides and silicates} [.Silicon ]
          [.Aluminium ]
          [.Iron ]]

\bigskip

B. Important constituents of plankton and benthos organisms.

\Tree[.{in skeletons} [.Silica ]
          [.{Calcium\\carbonate} ]
          [.{Magnesium\\carbonate} ]]

\bigskip

\Tree[.{salts in\\protoplasm} [.Iron ]
          [.Sodium ]
          [.Potassium ]
          [.Magnesium ]]

\bigskip

C. Seven oxides, the chief constituents of igneous rocks (F. W. Clarke).

\begin{center}
\begin{tabular}{ | m{5em} | m{1cm} | } 
  \hline
  Silica & 59.87 \\ 
  \hline
  Alumina & 15.02 \\ 
  \hline
  Iron & 5.98 \\ 
  \hline
  Calcium & 4.79 \\ 
  \hline
  Magnesium & 4.06 \\ 
  \hline
  Sodium & 3.39 \\ 
  \hline
  Potassium & 2.93 \\ 
  \hline
  \emph{Sum} & 96.04 \\ 
  \hline
\end{tabular}
\end{center}

\emph{Remarks on the above tables}. --- A. --- Composition of sea-water. Silica occurs either uncombined or in combination with alumina, the silicate of alumina being finely suspended.

``Iron, easily detected directly'' (Dittmar).

``Aluminium: in alumina'' (Dittmar).

With the exception of gases fixed from the atmosphere there is reason to believe that everything solid comes from the sea. The igneous rocks are oceanic silicated deposits of nummulites, and the silica and silicate of alumina carried down from them by rivers return to their original source.

B. --- Constituents of plankton and benthos organisms. Silica. According to Murray and Irvine\footnote{Proc. Roy. Soc. Edinburgh, 18. p. 237, 1890-1.} the silica-secreting organisms get that material, not from silica, which is only present in minute quantities, but from silicate of alumina finely suspended. Their culture-experiments showed that Diatoms flourished when fine clay was added. Amorphous silica was useless, but silicic acid jelly nutritious. At the present day the mechanically suspended silicate of alumina appears to be a fine sediment derived from igneous rocks. There was certainly a time when there were no igneous rocks, and probably an era when there were neither land nor rivers nor sediment. Did the first silica-secreting organisms get their silica from finely suspended silicate of alumina?

Magnesium. Analysis of shells of two representatives of Nummulitidae contained 5 per cent, of magnesium carbonate (Brady). (Probably some of the magnesium in limestones is derived from calcareous algae.) (See also Appendix Note H.)

Willstätter\footnote{R. Willstätter. In Liebig's \emph{Annalen d. Chemie}, vol. 350, p. 65, 1906.} has shown that magnesium is present in chlorophyll, and not iron, as usually believed.

C. --- Constituents of igneous rocks. All the constituents enumerated are present in the sea, and all except alumina are found in benthos and plankton organisms. The alumina may have been directly deposited from the sea. [The history of the discovery of the composition of Acantharia skeletons leads me to hazard the suggestion that possibly some of the 15.02 per cent, of alumina in igneous rocks may be due to an extinct race of plankton organisms. Acantharia skeletons were first thought tobe organic (\emph{Acanthin}, Haeckel). Then Shevyakov believed them to be made of a double silicate of aluminium and calcium, and finally Bütschli discovered they were composed of sulphate of strontium. Who would have suspected these wonderful skeletons, which bear impress of the effect of aeons of evolution, to be composed of sulphate of strontium!]

\bigskip
\centerline{\emph{Metamorphism}}

According to Aristotle we understand a thing when we see the cause of it. A knowledge of the origin of the igneous rocks and of the manner of accumulation of the materials composing them will throw light on problems of metamorphism. Igneous rocks are oceanic deposits of nummulites mineralized by silica, magnesium, aluminium, \emph{etc.}

At the beginning of the chain there are sunlight, protoplasm and the minerals of the sea, and at the end a crystalline mass of silicates. What are the intervening links?

Limestones and silica (chalk and flint) may be indicated as the second link, and limestone with silicates (\emph{Eozoön}) as the third. In link No. 3, heat of intrusive magmas has caused the silica to combine with magnesium, aluminium, iron, \emph{etc.}, to produce silicates\footnote{At first sight glauconite seems to be one of the silicates outside the igneous rock group, but perhaps this mineral should be regarded as an altered product of igneous rocks.} in the form of pyroxene: the peripheral banding is a mere detail. In the final stage, if limestone had been present, the carbonic acid would have gone and the calcium would have combined with the seven other chief oxides which form the crystalline mass of silicates. Limestone may have been replaced long before the igneous stage as in flint, phthanite, \emph{etc.} The existence of silica-aluminium and ferro-magnesian groups of silicates is attributed to magmatic differentiation.

\bigskip
\centerline{\emph{Magmatic differentiation}}

It may well seem incredible to a petrologist unfamiliar with biology and palaeontology that sunshine and living matter could be the parents of such a heterogeneous progeny as the igneous rocks. Yet the origin and metamorphoses of the bentho-plankton deposits known as ``igneous rocks'' are traceable. For some reason these very ancient mineralized organic deposits have become heated even above melting point. Chemical changes have thereby become greatly facilitated. The silicic acid with its four grades of acidity\footnote{Ortho-, meta-, poly-, and di-silicic acid.} has formed various compounds with the chief bases mentioned above. When the hot solid mass of silicates (or oxides) becomes liquid, the solid containing-walls constitute a reservoir. If the magma is pressed up through the overlying crust and through the surface it will flow out as a lava, with or without explosions and the formation of a cone of solid ejecta.

A certain sequence has been discovered in the character of the lavas erupted during successive phases in the ``life-history'' of a volcano.

``The general succession is from a rock of average composition through less silicious and more silicious ones to rocks extremely low in silica and others extremely high in silica --- that is, the series commences with a mean and ends with extremes.''\footnote{Iddings. Bull. Phil. Soc. Washington, 12. p. 145. 1892.} At Porto Santo, for instance, the black heavy basalt low in silica (basic) is covered by the pale-coloured light trachyte high in silica (acid). Both rocks must have come from the same reservoir. The chief differentiating cause is said to be temperature. The magma near the walls of the reservoir and its conduits is cooler than that in the center.

``The chemical differentiation of igneous magmas, which appears to be due to so simple a cause as temperature in different parts of the magma, leads to an endless series of variations'' (Iddings \emph{l.c.} p. 164).

Igneous rocks of any particular district, however much they may differ from each other, yet generally show what Iddings terms a ``consanguinity.'' Further, groups of districts forming some great ``petrographical province'' possess certain characteristics distinguishing them from the rocks of other provinces. Finally the igneous rocks as a whole are regarded by some eminent authorities as derivatives from some great primeval common magma.

Seeing that an oceanic organic deposit is parent of the igneous rocks, it is not surprising that traces of their common origin are discernible.

Variations in rate of cooling give rise to great differences in texture. Granite, rhyolite and an obsidian may have precisely the same chemical composition, but in the case of the volcanic glass the magma has cooled and set so quickly that the molecules have become fixed in position before they could marshal themselves in crystalline order. The slow-cooling granite, on the other hand, is a crystalline mass.

\centerline{*\hspace{15mm}*\hspace{15mm}*\hspace{15mm}*\hspace{15mm}*}
\bigskip

\subsubsection{On the Source of the Heat of Molten Rocks}
\paragraph{}
The interior of the earth must be in an intensely heated condition. Hot and boiling rocks, forced up from below, can actually be seen accumulating in heaps and pouring out in floods over the surface; and further, it is found that the temperature rises on an average 1° C. for every 32 meters of vertical descent into the earth.

This heat continually flowing out from within has naturally been regarded as part of the original stock of a cooling planet. The sky is crowded with glowing-hot bodies, and perhaps the earth was once in a similar condition. Even so, it does not follow that the heat \emph{now} escaping must be accounted for in this way.

When Lord Kelvin attempted to calculate the age of the earth on the basis of its having been a once-glowing but gradually cooling mass, he arrived at an estimate of 20 to 40 million years. The geologists protested that the time was wholly inadequate to account for the vast accumulation of sedimentary strata. The biologists, too, though ready ``to set their clock'' in accordance with Lord Kelvin's estimate if compelled to do so, were nevertheless inclined to suspect some flaw in his calculation.

Within the last twenty years a source of energy wholly unsuspected, but yet omnipresent, has been brought to light, \emph{viz.}, radioactivity.

Prof. Joly writes concerning the new science of radioactivity: ``First definitely opened up in 1898 by Mme. and M. Curie, when polonium and radium were discovered, to-day we are in possession of established views in contradiction to the tenets of centuries.''\footnote{Joly. \emph{Radioactivity and Geology}, p. 1. 1909.} It has been found that uranium is the parent of a series of substances, of which radium is one. The unstable atoms of these elements are undergoing disintegration and discharging energy, part of the latter being manifested as heat. Radioactive substances are universally distributed in the crust of the earth and in the ocean, but in varying amount. They are more abundant in igneous rocks than in sedimentary, in red clay than in \emph{Globigerina} ooze, in the land than in the ocean.

According to authorities who have studied the problems relating to the effects of radioactivity in geology, this form of energy is a source of the heat of the deep-seated igneous rocks.

The scaffolding of the earth's crust was formed in the ocean and out of seawater. Accordingly the general prevalence of radioactive matter in igneous rocks and in sediments derived from them, is what might be expected. The uranium entering the ocean along with the sediments from igneous rocks has only comeback to its original source. Prof. Joly estimates that 10<sup>6</sup> tonnes of radioactive substances have been received and again precipitated on to the ocean floor in the course of geological time. He writes:\footnote{\emph{On the Radium-Content of Sea-Water}. Phil. Mag. (6) 18. p. 407. 1909.} ``Presumably all this radioactive material has been at one time in solution or suspension in the ocean, whose waters, for all that we could have anticipated, might have possessed a content of radium some fifty times greater than the figures appear to indicate.''

The discovery of the origin and history of igneous rocks will lead, I believe, to a revival of the theory of chemical activity as an important source of the heat that has melted those rocks. Lord Kelvin preferred the view that ``the earth is merely a warm chemically inert body.'' The inert masses of silicates constituting the igneous rocks, however, were at one time benthoplankton mixtures of silicic acid, alkalies, alkaline earths, oxides of iron, phosphates, sulphates, \emph{etc.} The older and deeper zones of this mixture would be buried beneath a thick pile of strata and permeated with water and mineral solutions. A spark may set a forest ablaze. Heat arising locally in the deposit would hasten chemical activities all round, leading to ever-increasing accumulations of heat. The inert silicates of igneous rocks are the end term of a long series of changes, and may be compared to a spent conflagration.

Limestone seems inert, but the heat of a magma will cause the silica to combine with magnesium, \emph{etc.}, to form silicates as in the Canadian, Finland and Vesuvian ``\emph{Eozoön}''; and the heat of a kiln will make lime quick, the latter giving out heat when slaked.

\centerline{*\hspace{15mm}*\hspace{15mm}*\hspace{15mm}*\hspace{15mm}*}
\bigskip

\subsubsection{On the Ascent of Molten Magmas}
\paragraph{}
The deep-seated rocks, though at a temperature above normal melting-point would be solid owing to high pressure. Relief of pressure would cause them to melt and move in the direction of least resistance. If, as is stated, the nucleus of the earth contract more than the crust, the latter, owing to adjustment to a smaller radius, will be thrown into folds. The ascent of molten rocks through the overlying crust is characteristic of sinking areas; (Geikie).

\centerline{*\hspace{15mm}*\hspace{15mm}*\hspace{15mm}*\hspace{15mm}*}
\bigskip

\subsubsection{On the Chronology of Igneous Rocks}
\paragraph{}
Hitherto igneous rocks have not been allotted a chronological position proper to themselves. They have been regarded as primeval and beyond chronology, but nevertheless they are daughters of time. The date affixed is that of the strata into which they have been intruded, and, further, it has been the custom to regard the invaded overlying strata as older than the intrusive. Writing of the Laurentian granites, Chamberlin and Salisbury state,\footnote{\emph{Geology, Advanced Course} 2. p. 143. 1906.} ``but it is now known that they are intrusive into the schist series. They are therefore younger than the latter.'' Now that the igneous rocks must take their place in the palaeontological series, the lower rocks will be regarded as the older and first formed.

An igneous rock has two dates, the first being its date of actual formation shell by shell in the ocean, and the second its time of uprising into the overlying crust.

\subsubsection{Summary of Chapter 4} 
\paragraph{}
The known crust of the earth is mainly a metamorphosed silicatic deposit of organic origin precipitated from the ocean. The deeper zones of this deposit are heated above their standard-pressure melting-points, but remain solid owing to high pressure. Crustal movements, due to shrinkage and adjustment have led to relief of pressure followed by melting, and uprise of the molten material along lines of least resistance.

The cause of the heating of ``igneous'' rocks is now being attributed to radioactivity. Radioactive elements are disseminated throughout the earth's crust and the ocean. Seeing that the lithosphere is a product of the ocean, the uranium family of elements diffused through the earth's crust must necessarily have had a similar source. Probably chemical activity has been an important source of the heat.

\emph{Postscript, Note} 1. --- There are no authentic records of Diatom remains before the Carboniferous (Castracane). What are the reasons for assuming that the silica of igneous rocks is probably derived in part from Diatoms? Firstly, these rocks are seen to be deposits of marine shells. If the silica of the \emph{immensely thick} deposits did not come from the sea, where did it come from? Secondly, oceanic life at present depends on unicellular plants and especially on Diatoms, and there is no reason for assuming that a different relation held in the past.

\emph{Note} 2. --- I have seen a recent shell of \emph{Lagena marginata} with Diatoms completely embedded in the wall. Here we have a picture in miniature of the great planetary deposit of silicated shells. It is doubtful whether sponges were accountable for much or any of the silica during the earliest era.
\clearpage
\subsection{Chapter 5}
\subsubsection{Meteorites}
\begin{displayquote}
``We cannot but agree with the common opinion which regards meteorites as fragments broken from larger masses, and we cannot be satisfied without trying to imagine what were the antecedents of those masses.'' --- Lord Kelvin.
\end{displayquote}
\begin{displayquote}
``For there is almost as good a trade in exposing cosmogonies as in constructing them. But no special opprobrium attaches to failure, because everybody has failed, from Laplace down, or up, as you are pleased to consider.'' --- P. Lowell.
\end{displayquote}
\paragraph{}
Every now and again a strange and startling event happens; suddenly in the midst of a great light followed by a loud detonation something falls out of the sky and strikes the earth. No wonder such objects have been regarded with awe and veneration, and that temples have been built to enshrine these visible and tangible messengers from the invisible powers. The thunderbolt was the special appanage of high Jove. A stone in the temple of Apollo at Delphi, and also ``Diana of the Ephesians'' were probably sky stones. In B.C. 204 a black stone which had long been worshipped as Cybele the mother of the gods was removed from Phrygia to Rome with great ceremony, the guardians of the Sybilline Books having decreed that its presence was necessary for the safety of Rome.

In the wall of the Kaaba at Mecca is a sacred stone which for ages has been the omphalos of the Mohammedan world. According to scientific travellers the stone is probably a meteorite. Mohammed alleged it was a gift from the Angel Gabriel.

At the present day these rare and wonderful sky stones are the special objects of pride of our great Museums.

The evidence that meteorites have fallen from the heavens is now clear, as the following narratives, selected from many, may show.
\begin{quotation}
``The oldest undoubted sky-stone still preserved is that which was long suspended by a chain from the vault of the choir of the parish church of Ensisheim in Elsass, and is now kept in the Rathhaus of that town. The following is a translated extract from a document which was preserved in the church:--
\begin{displayquote}
On the 16th of November, 1492, a singular miracle happened: for between 11 and 12 in the forenoon, with a loud crash of thunder and a prolonged noise heard afar off, there fell in the town of Ensisheim a stone weighing 260 lbs. It was seen by a child to strike the ground in a field near the canton called Gisgaud, where it made a hole more than five feet deep. It was taken to the church as being a miraculous object. The noise was heard so distinctly at Lucerne, Villing and many other places, that in each of them it was thought that some houses had fallen. King Maximilian, who was then at Ensisheim, had the stone carried to the castle: after breaking off two pieces, one for the Duke Sigismund of Austria, and the other for himself, he forbade further damage, and ordered the stone to be suspended in the parish church.''\footnote{Copied from an \emph{Introduction to the Study of Meteorites}, 10th Edition, 1908. British Museum (Natural History), pp. 19.}
\end{displayquote}
\end{quotation}
\paragraph{}
Again,
\begin{quotation}
``About three o'clock in the afternoon of December 13, 1795, a labourer working near Wold Cottage, a few miles from Scarborough, in Yorkshire, was terrified to see a stone fall about ten yards from where he was standing. The stone, weighing 56 lbs., was found to have gone through 12 inches of soil and 6 inches of solid chalk rock. No thunder, lightning, or luminous meteor accompanied the fall; but in the adjacent villages there was heard an explosion likened by the inhabitants to the firing of guns at sea, while in two of them the sounds were so distinct of something singular passing through the air towards Wold Cottage, that five or six people went to see if anything extraordinary had happened to the house or grounds. No stone presenting the same characters was known in the district. The stone is preserved in the Museum Collection.''\footnote{Copied from \emph{An Introduction to the Study of Meteorites}, 10th Edition, 1908. British Museum (Natural History), pp. 22.}
\end{quotation}
\paragraph{}
On September 13, 1902, a meteorite fell at Crumlin near Belfast. The evidence given below was gathered by Sir L. Fletcher on the spot and within about a fortnight of the occurrence.\footnote{\emph{Nature}, Oct. 9, 1902, p. 578.}
\begin{quotation}
``At 10:30 A.M. on September 13, which was a cloudy morning, W. John Adams, who is in the employment of Mr. Walker at Crosshill farm, was gathering apples from a tree on the edge of the cornfield and near the house; he was startled by a noise of such a character that he thought it was due to the bursting of the boiler at the mill, which is about a mile to the south and situated near Crumlin railway station.''

``Another loud noise like that of escaping steam, was followed by the sound as of an object striking the ground near by, and a cloud of dust immediately arose above the standing corn at a spot only twenty yards away from where he was at work. Adams ran through the corn towards the cloud of dust and found a hole in the soil; whereupon he hurried to the farmyard for a spade, and within a quarter of an hour of the fall had extracted a black, dense stone, which had penetrated the soil to the depth of one and a half feet and had been stopped by impact against a much larger terrestrial stone.''

``The black stone was hot and, according to Mr. Walker, was still warm to the touch even an hour later. There was a sulphurous odour. Two other men were working at a haystack twenty yards further away from the hole made by the stone and also heard the sounds.''

``The detonation was remarked at places five miles to the north, nine miles to the east, eleven miles to the south east, thirteen miles S.S.W. by south. Mrs. Walker said that some of the hearers had taken the sound to herald the arrival of the Day of Judgment.''

``The stone weighs 9 lbs. 5½ oz.; it is 7½ inches long, 6½ inches wide, and 3½ inches thick. Its form is irregular and distinctly fragmental.''
\end{quotation}
\centerline{*\hspace{15mm}*\hspace{15mm}*\hspace{15mm}*\hspace{15mm}*}
\bigskip

\paragraph{}
It is curious to reflect that the idea of stones falling from the sky was at one time ridiculed by the learned as a popular superstition, the objects in question being regarded by them as ordinary volcanic ejecta. When the truth became known, meteorites were regarded with special interest, but men of science still held very divergent opinions concerning the nature and origin of these bodies.

Meteorites are classed under three groups, \emph{viz.} the purely metallic (siderites) composed mainly of iron alloyed with nickel, the siderolites composed of iron and stone and the aerolites or purely stony. The three groups belong to one class, for the stony pass by gradations into the purely metallic.

Meteorites are covered with a thin usually black crust due to fusion of the outer surface during the passage through the air, the interior being lighter in colour. Stony meteorites are usually friable, and can almost be crumbled between the fingers; sections show a granular glassy structure, with, in many cases, peculiar aggregations of granules (chondrules) having a radiating or reticulate pattern. The chondrules are said to be due to rapid crystallization.

The elements present in meteorites are the same as those known on earth, but some (\emph{e.g.} phosphorus) occasionally exist in a free state, such as would be impossible under ordinary conditions on this planet.

Meteorites contain well-known terrestrial minerals, \emph{viz.} olivine, feldspar (oligoclase), \emph{etc.}; and several which are unknown on earth, \emph{viz.} troilite (proto-sulphide of iron), schreibersite (phosphide of iron and nickel), \emph{etc.} They also contain gases occluded within their substance, \emph{viz.} carbonic acid most abundant in aerolites, carbon monoxide abundant in siderites; also hydrogen, in greater volume in the former than in the latter.

The theories of origin of meteorites may roughly be grouped under three headings, \emph{viz.}:
\begin{enumerate}
    \item Accretion of particles.
    \item Fragments of a disrupted heavenly body.
    \item Ejecta from volcanoes existing (\emph{a}) on cosmical bodies outside the solar system; (\emph{b}) on bodies within the solar system (excluding the earth); and (\emph{c}) on the earth.
\end{enumerate}
\paragraph{}
\centerline{\emph{1. Accretion theories}}.
Dr. H. C. Sorby\footnote{\emph{Nature} 15. p. 497. 1877.} believed that particles of matter at the solar surface were collected together by gravitation, fused into masses, and ejected during violent disturbances. Meteorites, then, according to this theory were portions of the sun.

Arrhenius\footnote{\emph{Worlds in the Making}, 1908. \emph{The Life of the Universe}, 1909.} believes that the sun and stars by means of radiation-pressure are continually driving out small particles of cosmic ``dust'' into space and that these become mutually attracted, thereby in time forming larger or smaller aggregates in the shape of cosmic dust and meteorites.

``By the action of radiation pressure small globules (spherical drops of matter) condensed in the solar atmosphere are pushed away from the sun and wander through space with velocities perhaps nearly equal to that of light. It is not improbable that the strange messengers from other worlds, the so-called meteorites, are composed of such spherules which had been driven into space. The meteorites are distinguished by an entirely peculiar structure and composition, from all the rocks and minerals known on earth, from the so-called plutonic, which have been formed by the congelation of the liquid interior of the earth, as well as from the neptunic which have been formed upon the bottom of the sea.'' ``These little drops which are ejected and propelled by the sun will collect chiefly in the external portions of the nebulae which owe their luminescence to the electrically charged dust. In the intense cold of the nebulae the drops will condense part of these gases, in particular the hydro-carbons and carbon monoxide, upon their surface. When such masses collide with one another they will be cemented by these materials. In this way small drops of spherules will grow into meteorites, which will continue their migration through space.''

\bigskip
\centerline{\emph{2. The ``Cosmic catastrophe'' theory}}

According to P. Lowell\footnote{\emph{The Evolution of Worlds}. \emph{Mars as the Abode of Life}.} the solar system may have begun in a catastrophic approach of two dead and dark suns, not necessarily crashing into one another, but coming sufficiently close to set up tidal stresses resulting in each sun being torn into fragments.

There would be a big mass and lesser ones and tiny fragments. The large central and the lesser masses would become hot and nebulous from the mutual clashings. Consequently the very small residual fragments would represent a condition of things before the origin of the nebular phase of the solar system! ``These things --- the meteorites --- are the oldest bits of matter we may ever touch.'' ``Here the meteorites tell us of another, an earlier stage of our solar system, one that mounts back to before even the nebula arose to which we owe our birth.''

``For the large body to whose dismemberment the meteorites were due can have been no other than the one whose cataclysmic shattering produced that very nebula which was for us the origin of things.''

The meteorites ``tell us of a nebula made up of meteorites out of which our planets were by agglomeration formed, and of which material they are the last ungathered remains, and they speak of times more remote, when our nebula was a cold sun.''

Suess\footnote{\emph{The Face of the Earth} 4. p. 543. 1909.} is led to regard the meteorites as planetary fragments, possibly parts of an anonymous body once occupying the gap between Mars and Jupiter.

\bigskip
\centerline{\emph{3. Volcanic theories}}

Tschermak believed that meteorites have had a volcanic source on some celestial body or bodies.

Sir Robert Ball wrote,\footnote{\emph{The Story of the Heavens}, p. 359, and \emph{Nature}, 1879, 19. p. 493.} ``With reference to the origin of meteorites it is difficult to speak with any degree of confidence. \emph{Every theory of meteorites is in itself improbable}, so it seems the only course open to us is to choose that view of their origin which seems least improbable.'' Accordingly Ball chose the volcanic theory, and then set about to discover the most probable volcanic source. Firstly he eliminated space beyond the solar system, on account of the extreme improbability of bodies from such distances hitting such a microscopic globule as our planet. For various other reasons the sun and larger planets were excluded. The moon might have been a source, but there are no active volcanoes visible, and meteorites which would quickly traverse a distance equal to the space between us and the moon, arrive at the present time.

By a process of elimination the earth was selected as the possible source. Ball points out that the probability of ejecta from the earth being again recaptured is considerable, because, assuming meteorites travel on a closed orbit round the sun, they must necessarily cross the track of the earth's orbit, and may do so at a moment when the earth is at the same point as the meteorite.

Of 50,000 bombs ejected with a ``3-mile power'' from the planet Ceres, only one would cross the earth's orbit. Of 50,000 bombs ejected with ``6-mile power'' from the earth, 50,000 would cross the earth's orbit (Ball).

The difficulty about terrestrial origin is the need for assuming volcanic eruptions of such terrific force that they could eject a body with an initial velocity of six miles a second. Even bombs of Krakatoa were \emph{said} to have had a velocity not exceeding one mile a second. A body ejected with a force below a six-mile power would fall back on to the earth, but if at a higher the earth's attraction would not be able to draw it back, and it would probably revolve round the sun in an orbit of its own. Ball supposed that during the very early stages of our planetary history there may have been much fiercer eruptions than at later epochs.

Having found traces of organic structure in igneous rocks, I naturally proceeded to examine meteorites. I soon discovered evidence of the existence of similar structure in them also, and presently found what I took to be remains of Radiolaria and Diatoms. Just as in the igneous rocks, the Foraminiferal shells turned out to be nummulites and the supposed plankton remains to be parts of those shells.

I have been allowed the great privilege of studying the whole of the unique collection of sections of the British Museum meteorites and have now arrived at complete certainty. Not only can I see --- though with difficulty --- the nummulite shells with a lens, but under higher powers can make out spiral laminae, furrowed marginal cord, septa, alar prolongations and disks so clearly that for me at any rate any further doubt is out of the question. I can trace --- again with difficulty --- the Foraminiferal outlines even in siderites. Rust from the great Melbourne meteorite also reveals nummulitic features.

Nummulitic structure is clear enough --- to my practised sight --- in meteorite sections containing stone and iron, and it is only a step to the completely iron meteorites. Accordingly the true nature of meteorites is now definitely discovered. These bodies are lumps of mineralized and ore-enriched nummulitic rock. They are portions of benthoplankton sea-bottoms, in which the benthos deposit of nummulites has become silicated by silica in all probability derived from plankton organisms. In my opinion, meteorites have no more to do with nebular and prenebular theories than have lumps of chalk.

The evidence is now overwhelmingly in favour of a theory of terrestrial origin. There is, it is true, some adverse evidence, but rather of a negative kind, \emph{viz.}, the absence of proof that a terrestrial volcano could eject a meteorite with an initial velocity of more than six miles a second. Sir Robert Ball assumed that sufficiently powerful eruptions might have occurred in some early azoic period, but the meteorites are of organic origin, therefore they were erupted after the time when the ocean had become the abode of life. Even in historic times we know of eruptions --- \emph{e.g.}, Krakatoa, so terrific that the explosion was heard 3000 miles away. In the long course of geologic time, much greater eruptions than that of Krakatoa may have occurred. Volcanic blocks and bombs are known to be hurled miles high with great velocity. Why assume the impossibility of their being hurled to a much greater height with a much greater velocity than hitherto suspected?

Dr. Otto Hahn\footnote{\emph{Die Meteorite (Chondrite) und ihre Organismen}. 1880.} believed in the organic origin of meteorites, but he fails to produce any evidence in support of his theory. He mistook the chondrules of aerolites for Sponges, Corals, and Crinoids. These bodies are peculiar formations of a mineral nature, but at the same time they owe their characters in some measure to the structure of the underlying organic basis.

Further, he considered the Widmänstatten figures of siderites to be ``for the most part plant-cells and not crystals.'' (Appendix, Note G.)

Looking back on the various theories referred to above, we may at once dismiss the accretion theories of Sorby and Arrhenius.

It is very unlikely, too, that Lowell's catastrophic theory will bear the additional strain that will now have to be imposed upon it. For we would have to assume the dark or dead sun had an ocean floor carpeted with nummulites, and products of a plankton fauna and flora, \emph{etc.}

At the same time Lowell's ingenious theory of the catastrophic origin of the solar system is considered to be based on the laws of probability, and events such as he describes have perhaps actually happened in the universe; but the meteorites (terrestrial organic \emph{sautés}) in our museums afford no evidence in support of that theory.

When we come to the volcanic theory, we must assume that the planet whence the meteorites were shot off was the abode of life and that things were much the same as on this earth. When I am examining nummulitic structure in igneous rocks or in meteorites, I am scarcely aware of any difference, though the material, it is true, is brecciated in the case of the meteorites. The organic structure is identical in both.

Meteorites contain minerals not only not known to exist on this earth, but which could not exist under ordinary conditions where water and oxygen are present. This fact, which at first sight might seem to dispose of the terrestrial theory, is by no means antagonistic. When conditions are varied in laboratory experiments, different results are obtained. In the case of meteorites Nature has performed an unusual experiment. In the course of a few seconds masses of mineral compounds (or metals) have been suddenly transferred from an environment subjected to heat and high pressure to one of intense cold and no pressure. Is it surprising that the companies of molecules suddenly thrust out of their familiar aqueous and gaseous environment have started making new alliances? Accordingly the existence in meteorites of minerals not actually known on earth is no valid argument against the terrestrial theory.

On the other hand, the fact that aerolites are \emph{mainly} composed of pyroxene compounds, with a certain percentage of feldspar and iron, is one rather in favour of an earthly origin. Prof. Judd\footnote{\emph{Volcanoes}, p. 322.} gives a diagram illustrating the relations between terrestrial rocks and meteorites (which last he designates extra-terrestrial). The aerolites, siderolites, and siderites closely compare respectively with enclosures in basalt, the basalt of Ovifak, Greenland, and the iron masses of Ovifak.

``The Asiderites'' (aerolites without iron) ``are quite identical in composition with the ultra-basic lavas of our globe'' (Judd). Ultra-basic ``nodules'' of nearly the same composition as meteorites are found in basalt, and in the center of volcanic bombs.

Dr. G. T. Prior in a paper \emph{On the remarkable similarity in chemical and mineral composition of chondritic meteoric stones}\footnote{See analyses of Baroti and Wittekranz meteorites. G. T. Prior, Min. Mag., 17., pp. 27, 31. 1913.} has pointed out ``the practical identity in chemical composition of different meteoric stones.'' They contain 74 per cent. of silicate of magnesia and iron (Olivine and Bronzite), 10 per cent. of silicate of alumina, with sodium, calcium and a trace of potassium (soda-lime feldspar), 15 per cent. iron alloys and protosulphide (nickel-ferrous iron and troilite) and 1 per cent. Chromite. Here are all the eight familiar elements of igneous rocks, of benthoplankton organisms and sea-water, \emph{viz.}, O. Si. Fe. Al. Mg. Ca. Na. K. Lockyer (\emph{The Meteoritic Hypothesis}, p. 25) marks in italics the statement that magnesium is present in all siderites. It is also common in the sea, in limestone deposits formed in the sea, and in igneous rocks.

The sea is the common parent of benthoplankton, igneous rocks, and meteorites, these being one and the same material variously modified by time and circumstance.

\centerline{*\hspace{15mm}*\hspace{15mm}*\hspace{15mm}*\hspace{15mm}*}
\bigskip

\subsubsection{On the Ultra-basic Composition of the Meteorites}
\paragraph{}
What first induced men of science to pay attention to the popular belief that meteorites were really sky-stones was the discovery that these bodies had approximately the same composition even though found in far-apart localities. Meteorites are ultra-basic, that is to say, there is only a low percentage of silicic acid (silica) and a high one of bases, especially of magnesium. A simple explanation now, I think, offers itself.

In Chapter 4 it was suggested that \emph{Eozoön} and Monte Somma bombs might be regarded as transition-stages between ordinary limestones and igneous rocks. The silicatic element of \emph{Eozoön} and the bombs is ultra-basic. In the case of the Canadian and Finland and Vesuvian \emph{Eozoön} it is abundantly clear that heat has caused the silica to unite with bases to form silicates. It may well be that while some great volcanic outburst was developing, the gathering high temperature caused the silica of overlying limestones to unite with magnesium, \emph{etc.}, to form olivine. When the explosion came the fixed lid of the cauldron was shattered with terrific force, and masses would be torn off from the sides of the volcanic pipe. It may be, then, that meteorites had never reached the fourth or igneous-rock stage of metamorphism, and therefore had never been products of magmatic differentiation. The absence of quartz would perhaps be due to the fact that the very fierce eruptions did not happen to take place where there were overlying volcanic rocks of the nature of rhyolites, nor in the neighbourhood of richly siliceous limestones.

\centerline{*\hspace{15mm}*\hspace{15mm}*\hspace{15mm}*\hspace{15mm}*}
\bigskip

The differences in the densities of meteorites have led to the supposition that these objects may have come from different zones of some cosmic body, \emph{viz.}, the pure siderites from a very deep or nuclear zone, the siderolites from a higher, and the aerolites from a still more superficial zone.

Walterhausen (1853) regarded the earth as a body composed of concentric shells, lighter acid rocks being at the surface, heavier basic rocks deeper and metallic zones still deeper. Suess,\footnote{\emph{The Face of the Earth}, 4. p. 544. 1909.} again, assumes the existence of three zones or envelopes as determining the structure of the earth, \emph{viz.}, the deepest or barysphere of nickel and iron, the middle Silica-Magnesium zone and the surface Silica-Aluminium zone.

Dr. Leigh Fermor,\footnote{\emph{Preliminary Note on the Origin of Meteorites}. Journal and Proceedings of the Asiatic Society of Bengal, 8, 1912, p. 315. I would point out that we have not yet arrived even at a plutonic zone, let alone an infra-plutonic. I find traces of oceanic organic life in garnetiferous schists from Sikhim and Tyrol. The birthplace of siderites is in Neptune's territory. This criticism is merely one of nomenclature.} in a paper on the origin of meteorites, is led to assume the existence of a garnetiferous zone between a central metallic and a plutonic zone. Consequently he makes use of the term ``infra-plutonic'' for the garnet zone.

Whether the lithosphere is sorted into zones or not, it is only one common organic deposit. Certainly the siderites are not ``iron dug from \emph{central} gloom.'' It is by no means certain they originate from a specially deep zone. Probably all the iron in the known crust of the earth has one common primal origin --- the sea.

The deeper layers of the mineralized oceanic organic deposit have been heated, and have risen up in the form of dykes, bosses, and sills penetrating the overlying crust and even breaking through the surface. Upheaved portions of this same deposit have been ground down and leached, the soluble iron salts becoming deposited along certain planes of fissures and faults (veins). There is no evidence to show that the basal part of the nummulosphere (or lithosphere) is richer in metal than the higher levels. In the case of veins, at any rate, some practical mining engineers are wholly opposed to the ancient theory of persistence of ore deposit in depth.\footnote{\emph{Persistence of Ore in Depth}. T. A. Rickard. A paper discussed at a meeting of the Institution of Mining and Metallurgy, Nov. 1914. I have to thank Mr. G. Henriksen for this interesting reference.}

According to the Hon. R. J. Strutt, the earth's interior can hardly consist mainly of iron, for the mean density of the earth is only 5.5, while that of iron is 7.7 (P.R.S. [A] 77, p. 484. 1906).

\centerline{*\hspace{15mm}*\hspace{15mm}*\hspace{15mm}*\hspace{15mm}*}
\bigskip

\subsubsection{Note on Shooting-stars and Comets}
\paragraph{}
In the chapter on chalk I have ventured to suggest the possibility of a relationship between chalk and some comets.

If astronomers are correct in assuming there is a connection between meteorites, shooting-stars, and comets, then assuredly there is a justification for this apparently audacious suggestion. As a result of innumerable, long-continued and patient observations, I have arrived at the conviction that 100 per cent, of the magnificent collection of meteorites I have examined are bodies of organic origin.

It is not unreasonable to assume that some of the ``shooting'' or ``falling stars'' that do not happen to reach our earth may also be of similar origin.\footnote{Tschermak and Sir Robert Ball regarded the meteorites on the one hand, and the almost imponderable shooting stars and comets on the other, as wholly distinct classes of objects.}

Even the great showers of shooting-stars may be related to the meteorites captured by the earth. Perhaps it is not merely a coincidence that during one of these displays a siderite fell, \emph{viz.}, at Mazapil, Mexico.

Sir Norman Lockyer considers the difference between an ordinary meteorite and a shooting-star to be merely one of size.\footnote{\emph{The Meteoritic Hypothesis}, 1890.} A shooting-star weighing only one grain and moving thirty miles a second, would possess an energy of 55,675 foot-pounds. No wonder it makes a splash in the aerial ocean into which it plunges!

If we could imagine some terrific volcanic eruption, capable of overcoming diminished aerial resistance and gravitational pull, it would not perhaps be difficult to account for the 33-year showers of shooting stars. A violent preliminary discharge would cleave the heated rarefied column of air like a battering-ram, and a closely following rush of matter would take place in a partial vacuum. The Alaskan volcano Katmai covered the country in a thick bed of dust over thousands of square miles. If a column of such material got shot out into space it would become drawn out into a great band of dust, each particle of which would become a shooting-star when, with its acquired orbital velocity, it re-entered our atmosphere.

According to a theory held at the present day the nuclei of comets are made up of clashing meteorites, the latter not present in sufficient numbers to obscure the stars in front of which the comet may be passing. If the meteorite-theory of comets be true, then some of the meteorites maybe of the same terrestrial origin as those in our museums. It has been suggested that the Mazapil siderite is part of the broken up Biela's comet.

Probably other bodies in the solar system and in the universe eject volcanically from their mass materials which escape the gravitation-pull of the parent body, and, further, catastrophes may have broken up planets and dark suns. However that may be, it is not improbable in face of the new facts, that all terrestrial captures are of terrestrial origin.

I find the brown-coloured ``chondres'' or spherules from oceanic abyssal deposits to be organic,\footnote{I have to thank Mr. James Chumley for kind permission to make a microscopic preparation of one of these objects.} but I could discern no trace of organic structure in the black magnetic spherules --- though it is probably there.

I do not believe there is positive evidence of the existence on earth of any cosmic masses or particles that are of non-terrestrial origin.

\subsubsection{Summary of Chapter 5}
\paragraph{}
Meteorites are mineralized and often ore-enriched lumps of nummulitic rock which have been ejected from volcanoes. On palaeontological and other grounds these bodies may be assumed to be of terrestrial origin.\footnote{O. C. Farrington (Meteorites, 1915) thinks the abundance of meteorites constitutes an objection to the theory of earthly origin. Meteorites could hardly be plentiful, or the market value would not be so high.}
\clearpage
\subsection{Chapter 6}
\subsubsection{Sedimentary Rocks}
\paragraph{}
A scene in which there is a river cutting through a gap in the chalk and flint cliffs to enter the sea represents in miniature the history of the earth's crust. For the chalk is a raised sea-bottom composed of an accumulation of calcareous and siliceous skeletons of sea-creatures. The siliceous remains have mostly dissolved, and the silica, wherever it has penetrated, has replaced the carbonate of lime of the calcareous skeletons. At the mouth of the river there are shoals of mud, and at the foot of the cliffs boulders, gravel and sand. Submarine deposits of these materials will become hardened into rocks, and perhaps in course of time elevated above the sea.

Almost all the rocks which make up the crust of the earth are either accumulations of skeletons or of fragments of those skeletons.

The silicated deposits of nummulites --- \emph{i.e.} the igneous rocks --- have been upheaved and broken up into coarser or finer fragments and particles by the action of heat, frost and water.

The conclusive proof that the sedimentary rocks are mainly derived from mineralized deposits of nummulites is established simply by examining particles of sand, mud or clay under a microscope. Portions of the shell-structure almost imperishably preserved will generally be seen (Fig. 9).
\begin{figure}[H]
\centering
\includegraphics[width=75mm,keepaspectratio]{figures/Fig9.png}
\caption*{\centerline{\small \textsc{Figure 9 --- A grain of sand from Brighton, Sussex,}}

\centerline{\small showing the nummulitic disk structure. 100x.}}
\end{figure}
Dust from a London street, mud trom the Thames, flint sand from Brighton, granite sand or clay from Cornwall, earth from the fields, volcanic sands from any part of the world, all show disk-structure of nummulite shells.\footnote{I have heard of unbelievers who would as soon expect to find nummulites in their carpet, in horse-manure, or in coal-cinders as in igneous rocks. I fear none of these materials will be found to have escaped nummulitic admixture. Dirty boots, gritty fodder and poor coal would furnish plenty of nummulitic structure in each of the materials enumerated.} Similarly with slates, and with shales and sandstones from all horizons. A sandstone, without any apparent organic remains, is nevertheless fossiliferous throughout, only the fossils are in very small fragments. The known fossils in sedimentary rocks are buried in the \emph{débris} of fossils of an earlier period or era.

The crust of the earth is mainly built of minute disks, the smallest visible being 0.25 to 1 $\mu$ in diameter, the disks form the nummulite shell, and nummulite shells the scaffolding of the earth's crust.

\emph{Note on gneisses and crystalline schists}. --- ``The question of the origin and meaning of the banded, foliated, and allied structures exhibited by crystalline rocks over large tracts of the earth's surface, is one of the oldest problems of geology, and still awaits a solution.''\footnote{\emph{The Problem of the Gneissic Rocks}. Trans. Hull Geol. Soc., 1906, vol. 6, p. 24.} --- A. Harker.

Distributed over vast areas of the earth's surface there are formations possessing characters both of sedimentary and of igneous rocks: for on the one hand they are crystalline and apparently devoid of fossils, and on the other stratified and banded in a manner suggestive of sedimentation. At one time these rocks, of which the gneisses and schists are the most typical examples, were regarded as ancient sediments laid down in water, and ``metamorphosed'' into a crystalline condition by the action of heat.

There are many instances of sedimentary rocks becoming crystalline in the neighbourhood of intrusive igneous rocks (contact metamorphism), and it was assumed that ancient sediments extending over vast areas might similarly have become modified by subterranean heat (regional metamorphism).

In 1884 J. Lehmann\footnote{\emph{Untersuchungen über die Entstehung der altcrystallinischen Schiefergesteine}, 1884.} published a great work illustrated by an atlas of photographs, showing that a stratified structure was often brought about in igneous rocks as a result of pressure. Now metamorphic rocks are especially found in mountainous regions, where there has necessarily been great dislocation and pressure. Hence he concluded that metamorphic rocks might in many cases be igneous rocks modified by mechanical means.

At the present day the tendency is to judge each case on its own merits.

Sir A. Geikie wrote, in his critical appreciation of Lehmann's work, ``The question is attacked on all sides'' (\emph{Nature}, June 5, 1884, p. 121). There is, however, one side whence no attack has yet been delivered. I find as a matter of observation that the metamorphic rocks are replete with organic structure.\footnote{Nummulite structure is visible in many of Lehmann's photographs, \emph{viz.} 2. 3; 3. 4; 6. 3, 4; 7. 4; 9. 1; 16. 6; 21. 6; 23. 3; \emph{etc.}} They are not only masses of crystalline minerals, but also masses of mineralized nummulites. Just as the arrangement and condition of the minerals afford evidence concerning events in the history of the rock, so likewise does the condition of the nummulitic material.

I have not yet been able to follow this clue very far, but I believe the organic factor will yield evidence of considerable diagnostic value.

In garnetiferous schist of Tyrol, for example, the nummulitic material is in a condition somewhat resembling that of slate. A section of slate transverse to the plane of cleavage shows bands and strings of rods, here and there intertwined around flat plates; the former are compressed spiral laminae. The same appearances occur in some parts of my sections of the schist, especially where the garnets are not too abundant. Accordingly these observations seem to point to the conclusion that the schist is a metamorphosed nummulitic mud.

In the case of Ottawa banded gneiss, portions of a shell can be seen traversed by more than one band. The banding, at any rate, cannot here be due to sedimentation. On the other hand, mechanical pressure is probably not the only force that has been at work, but also thermal, aqueous and chemical agencies.

Lastly the organic structure must be considered as one of the determining factors in mineral arrangement (see Chapter 9).

Both igneous and metamorphic rocks are metamorphosed nummulitic deposits, whether primary or sedimentary. Accordingly Lyell's term ``metamorphic'' cannot logically be restricted to the gneisses and schists.

Glauconitic Deposits --- Glauconite is a granular greenish mineral now forming deposits on the ocean floor, mostly in depths of about 100 fathoms, and off bold coast lines.

The green grains are either shapeless or in the form of casts of calcareous --- mainly Foraminiferal --- shells. The mineral is composed ``of hydrous silicate of potash and of ferric oxide, containing variable quantities of alumina, ferrous oxide, magnesia and often lime.''\footnote{Murray and Renard: \emph{Challenger Deep-Sea Deposits}, 1891, pp. 378-391.} Glauconite occurs in geological formations down to the Cambrian.

The most approved explanation is that the sediments of igneous rocks, especially those with potash feldspar, permeate the shells as fine mud, and that the latter becomes decomposed by acid resulting from the action of organic matter.

A new feature of interest about this mineral is the discovery of persisting nummulitic structure. In some beautiful casts of decalcified \emph{Rotalia} and \emph{Globigerina} from the glauconitic core of Norwich potstones, I can clearly see the dotted disks. Thus we have Cretaceous (or other) Foraminifera filled with Foraminiferal \emph{débris} of the ages when the igneous rocks were formed.

A. Vialay\footnote{A. Vialay. \emph{Essai sur la Genèse et l'Évolution des Roches}, 1912, p. 94.} takes a glauconitic deposit as the starting point of his evolutionary rock series, \emph{viz.} ultra-basic, basic and acid. However logical the theory may be from the chemical point of view, yet actually the course of events has probably been different. For glauconite is the offspring, and not the progenitor, of igneous rocks. The learned author rightly postulates an aqueous origin for the latter. He believes their metamorphosis to have taken place \emph{à froid}.

\centerline{*\hspace{15mm}*\hspace{15mm}*\hspace{15mm}*\hspace{15mm}*}
\bigskip

An esteemed correspondent writes to me:-- ``I cannot agree with you when you speak of `the organic origin of igneous rocks.' I think it should be the sedimentary origin of igneous rocks. If the so-called igneous rocks are largely composed of organic remains, surely they are far from exclusively made of that material.'' A band of chalk-flint will be found to be a mass of silicified once-calcareous fossils, these being chiefly nummulites. When that flint is ground down into sand it will still be possible to see the remains of nummulites. I find the same structure in the grains of all the sandstones I have examined. Similarly granite will be found to be a mineralized deposit of nummulite shells. The granite --- sand, mud and clay also show the almost indestructible nummulitic structure. The original granite may itself be a sediment or it may be in the condition of a primary deposit. Whether the rock is primary or ground down, if I find traces of organic structure in nearly every particle --- even though that structure be mineralized by silica magnesium or iron, or by alumina, which is not known to be secreted by organisms, yet I should feel justified in speaking of the organic nature and origin of the rock.

No one would deny the organic origin of a silicified fossil sea-urchin; and if the fossil were broken up into small particles, these latter would not lose their claim to organic origin. Similarly with Tertiary nummulites, and similarly too with all the rest of the nummulites which make up nearly the whole of the earth's crust.

\centerline{*\hspace{15mm}*\hspace{15mm}*\hspace{15mm}*\hspace{15mm}*}
\bigskip

That in every pinch of sand or mud from any part of the world some or frequently all of the particles should show nummulitic structure is a fact not so surprising as might appear.

For aeons a simple universal rhizopodal ``bathybius''\footnote{The meaning of ``bathybius'' here differs slightly from that of the term as originally employed. It is not a case of a continuous plasmodium of protoplasm, but of a layer of little separate masses.} has been building a scaffolding composed almost wholly of nummulites, which have become mineralized by material derived partly from plankton. This huge deposit has been heaving up and down, and here and there elevated above the ocean surface.\footnote{F. Ratzel (\emph{Anthropo-Geographie}, Ed. 3, 1. p. 207) compares the ocean to a mantle seemingly with holes in it (\emph{i.e.} the land). Nature is always repairing the rents. According to one estimate it would require six million years to sweep away the present land areas.} Instantly Nature attacks the mass with a meteorological broom and sweeps it back again as sand, mud, \emph{etc.} The igneous rocks, their sediments, and the main mass of most of the limestones are nummulitic, the proportion of preserved skeletons of later and higher organisms entombed in the sediments of igneous rocks and in limestones being very small.

A mountain range (like all the rest of the earth's crust) with a plutonic core flanked by schists, sedimentaries, intrusives and limestones, is one thing, \emph{viz.} benthoplankton, in various states.
\clearpage
\subsection{Chapter 7}
\subsubsection{Miscellanea}
\centerline{\emph{The Age of the Earth}}
\paragraph{}
Leaving aside the astronomical phase of the earth's history, and coming to the geological period dating from the birth of the ocean, it is obvious that the discovery of the oceanic and organic origin of the lithosphere, or rather of all that we know or are likely to know of that ``sphere,'' will materially affect all future calculations of geological age.

Hitherto an estimate has been arrived at by calculating the thickness of the whole mass of ordinary sediments, and also some average rate of deposition. Sollas, calculating the earth's sediments at a thickness of 50 miles, reckons they have taken 26 million years to form. Lapparent's estimate is 75 to 80 millions, Geikie's 100 millions.

These calculations leave out of consideration the fact that the igneous rocks --- much more bulky than all the sediments and limestones put together, are themselves metamorphosed oceanic sediments --- probably both precipitated and also precipitated-ground-down or detrital. Consequently future estimates must take in the igneous rocks.

Another method of reckoning the geological age is one based on ``solvent denudation.'' A calculation is made of all the sodium in the ocean, and of the amount of sodium annually carried there by the rivers of the world. Assuming the ocean was fresh water to start with, the amount now present divided by the quantity annually added would give the number of years required to accumulate the salts. Joly considers 111 million years to be the best estimate.

The lithosphere with its sodium-content is an oceanic deposit. There is no evidence to show that the sodium now in the ocean is derived wholly from the leaching of the upheaved part of that deposit. Probably the ocean has been abundantly salt from the beginning. The sodium in the sea is in the form of chloride. The sodium base in igneous rocks is mostly combined with silicic acid, only a residual trace of sodium chloride remaining under protection in quartz crystals, \emph{etc.}

Recent estimates of duration based on the disintegration of uranium and its products yield enormous periods of time varying from 240 to 11,470 millions of years.\footnote{Quoted by Joly in \emph{The Birth-time of the World}. Science Progress, July, 1914. Seeing that radioactive substances and their products of disintegration are found together in igneous rocks, \emph{i.e.} in nummulitic deposits, well might the poet ask --- \\``What oldest star the fame can save\\Of races perishing to pave\\The planet with a floor of lime?''} Joly adds ``That we have much to learn is indeed probable.''

My sections of Uraninite from Joachimsthal and Cornwall show indubitable traces of organic structure. Accordingly the uranium very probably came from the sea. The possibility of the metal having been deposited from water ascending from sources below the nummulosphere does indeed exist, but the ocean seems the more likely source.

Reviewing the deeply interesting but somewhat conflicting results referred to above, one looks back to the non-scientific chroniclers of former times almost with a feeling of envy. Archbishop Usher, for instance, quite definitely and to the satisfaction of himself and his contemporaries --- for his results were accepted by the compilers of the authorized version of the Bible --- fixed to a moment the date and time of the creation of the world, \emph{viz.} precisely at midnight preceding the beginning of October 23rd, B.C. 4004. (Appendix, Note J.)

\centerline{*\hspace{15mm}*\hspace{15mm}*\hspace{15mm}*\hspace{15mm}*}
\bigskip

\centerline{\emph{On certain Hypotheses concerning the Origin of Life on the Earth}}

It seems something of a paradox that at the moment when the organic origin of meteorites is definitely proved, the improbability --- indeed, the absurdity, of the theory of conveyance of life to this planet by meteorites should be demonstrated. For when all the different factors are considered, the terrestrial origin ot meteorites will be generally accepted. Accordingly if these bodies were the bringers of life to a lifeless planet, they must either have picked up or spontaneously developed the germs on the way. Now, too, that the whole planetary crust --- veritably a life-worn mould --- is found to be a product of life, why should we look elsewhere for the origin of something so abundantly and universally present here. The mere fact of human agency having so far failed to create living from not-living material is no proof that Nature has failed to evolve life on this planet.

Again, Arrhenius' version of the panspermia theory, \emph{viz.} that living particles are being driven through space by radiation pressure, is not founded on evidence. It has already been shown in Chapter 5 that the theory of the formation of meteorites by accretion of similarly driven particles of matter is now obviously untenable.

\centerline{\emph{Note on Precious Stones}}

I have found traces of organic structure in several of the precious stones, \emph{viz.}, ruby, sapphire, emerald, garnet and diamond.

The persistence of such traces may well seem almost incredible, but the careful examination of sections under high powers leaves me in no doubt. Rubies and sapphires consist of nearly pure alumina (Al$_{2}$O$_{3}$) with a trace of chromium. They come mostly from the lime-feldspars of Burmah. The matrix of this rock has the usual nummulitic character.

Anything visible even under the highest powers of the microscope is of Brobdingnagian proportions in ``molecule-land.'' According to Lord Kelvin, a drop of water magnified to the size of the earth would reveal the molecules as objects smaller than cannon-balls but larger than shot. If owing to molecular change the Brobdingnagian nummulitic disks are composed of some highly intractable substance there is no reason why they should not escape destruction when subjected to very high temperature. Barbados earth, a nummulitic rock rich in Radiolaria, after boiling reveals the disk structures.

I have seen very distinct traces of nummulitic structure in diamonds (Fig. 10). With the kind permission of Mr. W. Busch of the Premier (Transvaal) Diamond Co., I was able on several occasions to examine some hundreds of rough stones. Igneous rocks being nummulitic, it is not surprising that precious stones composed of silicates and of alumina should show traces of nummulitic structure, but it seems very remarkable that carbon molecules should replace other molecular nummulitic scaffoldings. The drawings have been made from a small diamond in my own collection (see also Plate 20, Figs. D, E). Further, in some curious black diamonds\footnote{Described by D. P. McDonald in Trans. Geol. Soc. South Africa, 16. 1913, p. 156.} which Mr. P. R. Frames, the Transvaal representative of the Premier Diamond Co., very kindly brought to me, traces of nummulitic structure are visible. These stones contain three forms of carbon, \emph{viz.}, soft graphite, hard graphite, and ordinary white diamond.

In specimens of graphite mixed with Wollastonite rock (silicate of calcium) I am firmly persuaded, after prolonged and close observation, that I can see casts of nummulites with a hand lens. The Wollastonite rock itself is nummulitic. Carbon dissolved in molten magmas may separate out as graphite or as diamond. In Moissan's experiments diamonds resulted from the solidification of metal carbides (\emph{Ore Deposits}, Beyschlag, Vogt, and Krusch, 1, 1914, p. 129).
\begin{figure}[H]
\centering
\includegraphics[width=75mm,keepaspectratio]{figures/Fig10.png}
\caption*{\centerline{\small \textsc{Figure 10 --- Nummulitic structure in diamond.}}

\hspace{5mm} \small A, marginal cord and pillar. B, disk-structure. C, portion of marginal cord, and of three rows of disks in perspective. All 1300x.}
\end{figure}
\bigskip
\centerline{\emph{Infra-Nummulosphere}}

Darwin wished for ``some doubly rich millionaire who would take it into his head to have borings made in some of the Pacific and Indian atolls, and bring home cores for slicing from a depth of 500 to 600 feet.''

Thanks to the enterprise of Profs. Sollas and David a boring of this kind was actually accomplished, and with very successful results. A still greater or at least more difficult enterprise would be to bore or run a shaft through the nummulosphere down to the really plutonic or azoic zone below the domain of Neptune.

Although the accumulated pile of sedimentary strata at maximum would be many miles thick, perhaps the average depth of the earth's crust if uniformly spread would not be very great. The lowest Archaean formations --- the fundamental gneiss, granite gneiss and the gneiss of Finland, N.W. Scotland, central Europe and Canada --- are undoubtedly of organic origin. The place to choose for the boring should obviously be in the floor or sides of some depression where these most ancient rocks are exposed.

If all difficulties, such as those arising from temperature or from a possible up-welling of lava, could be overcome what might we expect to find? From the nature of its formation particle by particle on the sea-floor, the known crust of the earth is a heterogeneous mixture. Possibly the sub-crustal region would be more homogeneous. The term ``crust'' is here limited to the oceanic deposits, whatever is lower being a ``zone.'' Prof. Milne (Recent advances in Seismology. P.R.S. [A] 77, p. 369, 1906) estimates that there is an outer zone of rock thirty miles thick resembling that of the surface. The nummulosphere crust would constitute only a fraction of this zone, for the dissolving powers of the ocean are not unlimited --- an old anti-Wernerian argument.

\centerline{\emph{The Nummulosphere in Arts and Crafts}}

Apart from recent organic substances, most of the materials used in the arts have a nummulitic origin, and it is surprising to what an extent the nummulitic structure persists.

Architecture, sculpture, and painting depend on nummulitic material in the form of igneous rocks, limestones, sandstones, and clays.

The terracotta and the bricks and slates of the Natural History Museum are found by observation to be replete with nummulitic structure. The materials of the Greek marble statuary and of the basaltic monsters from Easter Island also have a common origin. Sections of most of the jewels, of jade ornaments, of soapstone images, of crockery and porcelain, and of a meerschaum pipe, have all revealed a nummulitic groundwork. Nummulitic structure can be detected without much difficulty in oil paintings, the pigments of which are volcanic clays. I made a preparation of rust from an old iron saucepan, and found the ubiquitous nummulitic structure.

``The cloud-capp'd towers, the gorgeous palaces, the solemn temples,'' and the crust of ``the great globe itself'' \emph{were} once dissolved --- in the sea.
\clearpage
\section{Part 2 --- The Direct Evidence}
\paragraph{}
In the preceding chapters, facts concerning organic life at the surface and on the floor of the ocean during present and past times have been brought forward to show that an organic origin of the planetary crust is rather what we should expect to find.

Detailed investigation now confirms this expectation by revealing the fact that the lithosphere is mainly composed of a mineralized deposit of Foraminiferal shells belonging to the genus \emph{Nummulites}, the deposit being either ``primary'' (\emph{i.e.}, as originally accumulated) or detrital.

The more recent primary deposits (Cainozoic to Archaeozoic) --- either unaltered, or silicified into flints, gaizes, phthanites, \emph{etc.} --- are composed chiefly of calcium carbonate or silica, wholly or partly of organic origin. In the more ancient deposits, commonly known as igneous rocks, the silica has entered into combination with various bases, \emph{viz.}, oxides of Al. Ca. Mg. K. Na. Fe. --- derived from the ocean partly through the agency of living matter --- to form silicates. Chemical activity has been favoured by high temperature, possibly due in greater or less degree to the storing-up of heat derived from radio-active sources.

A short account of Foraminifera and more especially of nummulites will now be given. A chapter on the generally known and fairly well-preserved nummulites of the Eocene period will be followed by one on the nummulites of igneous rocks and meteorites. Pre-Eocene nummulitic limestones from Mesozoic to basal Archaeozoic have already been referred to in Chapters 2 and 3.
\begin{figure}[H]
\centering
\includegraphics[width=\textwidth,keepaspectratio]{figures/Fig10-a.png}
\caption*{\centerline{\small \textsc{Figure 10A --- Diagrammatic figures illustrating the}}

\centerline{\small \textsc{spirodiscoid character of the minute structure of the}}

\centerline{\small \textsc{nummulite skeleton or shell.}}

\hspace{5mm} \small I. Spirodisk in horizontal, and II., in vertical aspect. Each figure shows three orders of dimensions, \emph{viz.}, \emph{a}, \emph{b}, \emph{c}, like, say, a cog-wheel with cog themselves cogged.

\hspace{5mm}If, again, \emph{c} were enlarged it also would show spirodisks with descending orders of size down to the limits of microscopic vision.}
\end{figure}
\bigskip
\clearpage
\subsection{Chapter 8}
\subsubsection{Foraminifera}
\paragraph{}
If shelly sands, especially from near low-tide mark, be examined with a hand-lens, probably some tiny little shells like lilliputian Pearly Nautiluses, Florence oil-flasks, or porcelain-white millet seeds will be detected. Twenty to fifty of them could be laid in a line an inch long (Fig. 11).
\begin{figure}[H]
\centering
\includegraphics[width=\textwidth,keepaspectratio]{figures/Fig11.png}
\caption*{\centerline{\small \textsc{Figure 11 --- Shells of foraminifera.}}

\hspace{5mm} \small A, \emph{Rotalia}. B, \emph{Nodosaria}. C, \emph{Lagena}. D, \emph{Miliolina}. E, \emph{Textularia}.

\hspace{5mm} \small A, C, D, E 30x. B 15x}
\end{figure}
Naturalists first regarded some of them as minute Nautilus shells. Fig. 12 shows a section of a coiled shell, the interior being divided by a series of ``septa'' or partitions, each with a V-shaped row of pores or ``foramina.'' The real Pearly Nautilus has similar partitions, but all are joined by a coiled tube or siphon. D'Orbigny (1826) called the very small shells Cephalopoda Foraminifera and the real Nautilus group Cephalopoda Siphonifera.
\begin{figure}[H]
\centering
\includegraphics[width=\textwidth,keepaspectratio]{figures/Fig12.png}
\caption*{\centerline{\small \textsc{Figure 12 --- \emph{Polystomella crispa}, about 50x.}}

\hspace{5mm} \small A, exterior. B, end view showing V-shaped row of pores in final septum. C, section of shell showing septa.}
\end{figure}
\paragraph{}
In 1835 Dujardin collected in the mud and on seaweeds the living creatures whose empty dead shells alone had hitherto been studied. So far from the little creatures having mouth, stomach, heart, \emph{etc.}, he found the shells to be filled merely with a granular jelly which extended itself outwards in the form of a network of trunks and threads. The animals moved about by extending the threads or ``pseudopods''\footnote{\emph{Pseudes}, false or resembling, \emph{poda} feet.} which served also as organs for the capture of food particles. He called these organisms Rhizopoda.\footnote{\emph{Rhiza}, branched, root-like, \emph{poda} feet.} Later the group, now usually called Foraminifera (after d'Orbigny), was removed to the Protozoa or simplest primitive animals.
\begin{figure}[H]
\centering
\includegraphics[width=90mm,keepaspectratio]{figures/Fig13.png}
\caption*{\centerline{\small \textsc{Figure 13 --- \emph{Polystomella crispa} with numerous pseudopods,}}

\centerline{\small \textsc{about 50x. After Lister.}}

\hspace{5mm} \small [Rows of dots on shell are incorrectly copied. They should be bars or streaks.]}
\end{figure}
\paragraph{}
There are thousands of species, most of them very small, although some --- the giants of their tribe --- are over an inch in diameter. They are nearly all marine, and mostly live on the sea-bottom at all depths from shallow water to the abysses. A few species live floating on the surface of the ocean, the commonest of these being \emph{Globigerina bulloides}. Over an area of 49 million square miles and at depths of about 2,000 fathoms the ocean floor is carpeted with a pinkish-white mud called \emph{Globigerina} ooze, chiefly composed of shells which have rained down from the surface.
\begin{figure}[H]
\centering
\includegraphics[width=90mm,keepaspectratio]{figures/Fig14.png}
\caption*{\centerline{\small \textsc{Figure 14 --- \emph{Globigerina bulloides}, 175x.}}

\centerline{\small (From photo.)}}
\end{figure}
\paragraph{}
The skeletons of Foraminifera are mostly made of carbonate of lime, but some are built of agglutinated sandy or other foreign particles. The essential feature is not so much the skeleton as the nature of the body itself with its branching network of pseudopods.

The calcareous shells are either porcellanous or vitreous. The former kind have one or a few large orifices, the rest of the shell being imperforate --- at least in the adult stage. The vitreous shells, in addition to a main opening, have minute pores all over the walls.

The great variety of shells can be grouped under a few types. The shell may consist of a single chamber (Fig. 11 C) or of several in a row (Fig. 11 B). The chambers may alternate on each side of a line (Fig. 11 E) or may form a spiral or cyclical series (Fig. 11 A and Fig. 12).

Foraminifera reproduce themselves chiefly by division of the protoplasm within or outside the shell, into creeping amoebulae or motile biflagellate spores. A new shell is secreted by an amoebula or by a cell formed by the union of a pair of spores. There is an ``alternation of generations.'' A ``megalospheric'' generation, having a shell with a large central chamber forms the biflagellate spores, pairs of which unite to form a cell which secretes a ``microspheric'' shell with a small central chamber. The protoplasm of the latter divides up into amoebulae, each one of which forms a megalospheric shell. The two kinds of shell which are nearly always found associated were once thought to belong to distinct species.

Brady divides the group into ten families. Only the tenth --- the \emph{Nummulitidae}, and only one genus, \emph{viz.} \emph{Nummulites}, concern the present work.

The important part taken by Foraminifera in the formation of the earth's crust is already well known, but it is now found that nearly the whole lithosphere is composed of mineralized deposits of nummulites.

\centerline{*\hspace{15mm}*\hspace{15mm}*\hspace{15mm}*\hspace{15mm}*}
\bigskip

\subsubsection{\emph{Nummulites}}
\begin{displayquote}
``Ce genre de débris organiques.'' --- \emph{Joly and Leymerie}.
\end{displayquote}
\paragraph{}
Nummulites furnish a powerful instrument for the solving of problems of the highest importance in geology, petrology, and meteoritology. Accordingly a short account of their structure is given below.

Nummulites are not common objects of the country in Great Britain, because the Eocene strata in which they abound as recognizable fossils are here relatively thin and difficult to find,\footnote{The \emph{Nummulina} zone at the base of the Barton Beds at Highcliff, Hants, is only 8 inches thick and not easy to locate. At Selsey, again, some of the Bracklesham strata containing nummulites can only be seen at low spring tides.} but in many parts of the old world, in North Africa, Europe and Asia there are Eocene nummulitic limestones of great thickness and vast extent.

Nummulites are mentioned by ancient writers. Strabo, for instance, refers to their presence in the sands around the pyramids of Egypt, and to the common belief that they were petrified lentils. In some countries they were believed to be pieces of money turned into stone.\footnote{The hybrid Graeco-Latin word nummulites is derived from \emph{nummus}, money, and \emph{lithos}, stone.} Leaving aside the numerous popular legends, we find that even the learned had many and various theories concerning these shells. They were regarded as corals, as ``worms with shells,'' or, again, as Cephalopods.

Some naturalists, misled by the appearance of transverse sections of shells as seen on rock surfaces, supposed them to be petrified willow leaves. According to another theory the shells were opercula of Ammonites. Ehrenberg thought they were the supporting skeleton of certain Jellyfish (\emph{Porpita}). It was only after Dujardin's discovery that nummulites were placed in the group of Foraminifera and among the Protozoa or simplest animals.

In 1848 Joly and Leymerie wrote an important historical and scientific memoir on nummulites. In 1853 a splendid foundation for systematic arrangement was laid by d'Archiac and ably extended and built upon by Dr. de la Harpe, Max de Hantken and others. Carpenter did excellent work on the minute anatomy. The mysterious problem of the reproduction of nummulites and other Foraminifera has been dealt with by Munier-Chalmas and Schlumberger, Lister, Schaudinn, and Winter.

\centerline{*\hspace{15mm}*\hspace{15mm}*\hspace{15mm}*\hspace{15mm}*}
\bigskip

Nummulites are shaped like biconvex lenses or disks, and vary in diameter from 1 to 107 mm. ($\frac{1}{25}$ to about 4½ inches).\footnote{D'Archiac describes gigantic specimens of \emph{N. complanata} from Crete, 107 x 3 mm. (\emph{Mon}. p. 88).} The surface may show radiating sinuous lines or a reticulate pattern, and may be smooth or visibly granular. To the naked eye there is no perceptible orifice.

The shells have the singular and convenient characteristic of splitting into two equal plano-convex half-disks when heated and suddenly thrown into cold water. Each flat inner face of the split shell reveals one half of a spiral cavity with few or many turns, divided up into numerous chambers by septa or partitions also in halves (Plate 2B). Fitting the halves together again will help the observer to realize the shape of the whole spiral cavity and septa.

There is a central chamber either extremely small and invisible to the naked eye, or fairly large and easily visible.

A section through the central median plane is called ``perpendicular,'' as in the case of a coin standing on edge, and cloven in half between the faces.

When a shell is broken in half like a biscuit, the section is termed ``transverse.'' A transverse section through the center reveals the so-called willow-pattern, in the form of a series of apparently separate pointed ovals one within another, the ovals being the broken edges of the coils of the continuous spiral.
\begin{figure}[H]
\centering
\includegraphics[width=0.75\textwidth,keepaspectratio]{figures/Fig15.png}
\caption*{\centerline{\small \textsc{Figure 15 --- A--E. Various species of nummulites small}}

\centerline{\small \textsc{and large, whole and in perpendicular and transverse}}

\centerline{\small \textsc{section. (After d'Archiac and de la Harpe.)}}

\hspace{5mm} \small A. \emph{N. variolaria}, nat. size. B. \emph{N. curvispira}, 4x. C. \emph{N. curvispira}, \emph{a}, \emph{b}, \emph{d}, nat. size; \emph{c}, 4x; \emph{e}, 2x; in various aspects. This species really megalospheric phase of \emph{N. gizehensis}. D. \emph{N. guettardi}, \emph{a}, \emph{b}, \emph{c}, nat. size; \emph{d}, 4x. E. \emph{N. gizehensis} in microspheric phase. \emph{a}, \emph{b}, shell nat. size. \emph{c} alar prolongations on peripheral part of surface of an interior coil of spiral lamina, 4x. F. \emph{N. (Operculina) murchisoni}, with only a few rapidly-widening turns of spira, \emph{a}, nat. size, \emph{b}, trans. sect., 2x.}
\end{figure}
\paragraph{}
What makes a nummulite so puzzling at first sight is its apparently circular outline. How can there be embracing \emph{spiral} coils in a circular disk? As a matter of fact the disk is not circular. Although it is almost impossible to see it, the apex of the V-shaped terminal opening of the last coil projects a little beyond the margin of the preceding coil, and the embracing arms of that V extend to the center.
\begin{figure}[H]
\centering
\includegraphics[width=0.75\textwidth,keepaspectratio]{figures/Fig16.png}
\caption*{\centerline{\small \textsc{Figure 16 --- Diagram of a nummulite to show perpendicular}}

\centerline{\small \textsc{and transverse sections in one view.}}

\hspace{5mm} \small \emph{a}, outer coil of shell; \emph{b}, \emph{b'}, \emph{b''}, three preceding coils; \emph{c}, septa; \emph{d}, aperture in septum: \emph{e}, alar prolongations; \emph{f}, edges of spiral lamina, \emph{f'} surface of final coil (same as \emph{a}); \emph{g}, marginal cord, \emph{g'}, \emph{g''} vessels in septa and walls; \emph{h}, central chamber.}
\end{figure}
\paragraph{}
Let there be imagined an embracing-spiral shell with only a few wide-apart coils, and with the embracing arch of the terminal opening high and well-separated from the margin of the preceding coil (Fig. 17 A). If such a shell, placed on edge and with the mouth upwards, could be unrolled, a double or bent lamina, V- or U-shaped in section, would result; the flaps would be low and close together at the beginning (small central coil), gradually becoming higher and wider apart towards the end (final coil and edge of mouth) (Fig. 18 B). The V-shaped spirally-coiled lamina is termed the Spiral Lamina, the edge at the bend being known as the Marginal Cord.
\begin{figure}[H]
\centering
\includegraphics[width=85mm,keepaspectratio]{figures/Fig17.png}
\caption*{\centerline{\small \textsc{Figure 17 --- Diagram of imaginary shell with a few}}

\centerline{\small \textsc{wide-apart coils.}}

\hspace{5mm} \small A., coiled up; B, uncoiled. To help to explain continuous embracing spiral, spiral lamina, marginal cord.}
\end{figure}
\paragraph{}
A model illustrating the spiral lamina, septa, \emph{etc.}, can be made from a clock spring by bending diamond-shaped segments of variously-coloured papers round the successive coils, and bringing the points to the center. Pieces of card shaped like a V with a broad apex can be inserted between the coils to represent septa and alar prolongations or ``alars.'' The spring itself represents the marginal cord; but in the shell the width increases gradually from the innermost to the outermost coil.
\begin{figure}[H]
\centering
\includegraphics[width=85mm,keepaspectratio]{figures/Fig18.png}
\caption*{\centerline{\small \textsc{Figure 18}}

\hspace{5mm} \small A. \emph{N. planulata} showing a furrowed ``marginal cord'' or edge of last coil but one; a septum; and slit-like orifice between lower edge of septum and marginal cord of preceding coil, 25x. B, C, E, shells 2x; D, nat. size. Shells whole and in section. After d'Archiac.}
\end{figure}
\paragraph{}
The successive coils of the spiral lamina are not in contact, but separated by a space widening from center to periphery, the wide marginal part being divided into chambers by the septa. In some species the narrow inner region of the inter-laminar spaces is divided by straight or slightly sinuous radial lines or walls continuous with the septa; in other species these lines are very sinuous, and in other again joined into a network. These septal threads (\emph{filets cloisonnaires}, d'Archiac) or ``alar prolongations'' of the septa appear on the successive surfaces of the spiral lamina as radial sinuous or reticulate lines or patterns, which serve as characters for dividing the genus into its main groups.
\begin{figure}[H]
\centering
\includegraphics[width=75mm,keepaspectratio]{figures/Fig19.png}
\caption*{\centerline{\small \textsc{Figure 19 --- Portion of outer convex surface of}}

\centerline{\small \textsc{marginal cord.}}

\centerline{\small 260x.}}
\end{figure}
\paragraph{}
The Marginal Cord is the ribbon-like edge of the spiral lamina. The surface of the cord is marked with fine furrows and ridges, and hence Carter termed it the ``spicular'' cord from a fancied resemblance to sponge spicules (Figs. 18-20).
\subsubsection{On the Minute Structure of \emph{Nummulites}}
\paragraph{}
Very little appears to be known concerning the finer structure of nummulites, and it is a strange paradox that some new light should be thrown on this subject as a result of the study of silicated nummulites of igneous rocks and meteorites rather than of calcareous shells of Eocene limestones.\footnote{The reasons for this may be partly chemical, partly optical. Carbonic acid and pure water would act more powerfully on calcareous than on silicated shells, and the nummulitic deposits constituting igneous rocks may often have become mineralized.} For it was only after finding certain structures in silicated shells that I was led to seek for and to find similar structures in the Tertiary nummulites.
\begin{figure}[H]
\centering
\includegraphics[width=85mm,keepaspectratio]{figures/Fig20.png}
\caption*{\centerline{\small \textsc{Figure 20 --- Transverse section of marginal cord}}

\centerline{\small \textsc{bounded on each side by the striated}}

\centerline{\small \textsc{inter-pillar areas.}}

\centerline{\small 260x.}}
\end{figure}
\paragraph{}
My observations on the finer microscopic structure are incomplete, but are published in the belief that they will afford some help in interpreting the photographs of sections of igneous rocks and meteorites.

Just as Tertiary nummulites are recognized when seen, so the universal pre-Tertiary shells possessing precisely the same general characters, can be detected, though they require more careful observation. Accordingly, a complete investigation of the minute anatomy though very desirable, is not essential for the recognition of nummulitic structure in igneous rocks and meteorites, for I can now recognize the larger features of shells even with a hand-lens.
\begin{figure}[H]
\centering
\includegraphics[width=85mm,keepaspectratio]{figures/Fig21.png}
\caption*{\centerline{\small \textsc{Figure 21 --- Transverse section (willow pattern) of part}}

\centerline{\small \textsc{of shell of \emph{N. laevigata}, 10x. After Carpenter.}}

\hspace{5mm} \small  \emph{a}, rows of pillars in section in spiral lamina, \emph{a'} ends of pillars forming granules on surface of successive coils of spiral lamina; \emph{b}, \emph{b'}, septa of chambers; \emph{c} marginal cord; \emph{d}, orifice in each septum; \emph{e}, canals.}
\end{figure}
\paragraph{}
A thorough investigation of the minute structure of nummulites will, I believe, throw light not only on the mutual affinities of the great groups of Foraminifera, but also on the structure of rhizopodal protoplasm.

One of the characteristics of nummulites is the ``tubulated'' structure of the shell-walls, tubuli being very fine vertical parallel passages through which the protoplasm of the interior communicates with the outside. In many ``perforate'' Foraminifera these passages are little else than circular holes or pores.

When microscopically examining nummulites or sections of igneous rocks I was continually seeing granular or dotted disk-like structures. These I took to be places of junction of ends of bundles of tubules, bundles in cross-section, groups of bundles of short tubuli, or, lastly, disk-like groups of tubuli arranged in radial horizontal rather than longitudinal direction. Very naturally, at first I was thoroughly possessed by the idea of tubules, and thought the disks must be joined together by parallel longitudinal lines of tubules, even though I failed to detect these lines.

Presently, however, I found the disks showed a spiral plan and also a radial structure. The spirals, with coils alternately and slightly above and below a median plane appeared to be replicas in miniature of the marginal cord, and the radii to be edges of disks in the position of septa and in a plane vertical to the plane of the ``parent'' spiral. Seemingly, the disks were to some extent repetitions of the structure of the shell as a whole. See Diagram, p. 147 [Fig. 10A].

The first disks definitely to be located were rather large ones nearly half a millimetre in diameter, at the outer convex surface of outer coils of the marginal cord between any pair of septa. Later I found that the ridged and laminated structure of the whole cord was made up of disks, as also were the septa and alars. Next it became apparent that the little spiral-radial disks were themselves constructed of smaller disks also on a spiral and radial plan, and these latter again of still smaller. Under a magnification of 4000 diameters I could make out disks 1 $\mu$ ($\frac{1}{25000}$ an inch) in diameter, each with a circle of points (about 0.25 $\mu$ in size), \emph{i.e.} visible and definitely arranged organic structures. Probably improved optical apparatus would resolve even these points into ``spirodisks,'' and possibly the structural repetitions may go on even until molecular dimensions are reached. These little calcareous models of proto-plasmic architecture will, I believe, throw light on the structure of the protoplasm that built them, for the granular protoplasm of Foraminifera also appears to me to show a spirodiscoid structure discernible by means of the granules.

The planetary crust is almost wholly built of ``spirodisks,'' the smallest visible unit of structure being about 0.25 $\mu$ ($\frac{1}{100000}$ an inch) in diameter.
\begin{figure}[H]
\centering
\includegraphics[width=85mm,keepaspectratio]{figures/Fig22.png}
\caption*{\centerline{\small \textsc{Figure 22 --- Transverse section of \emph{Nummulites laevigata}}}

\centerline{\small \textsc{showing light pillar and dark inter-pillar bands.}}

\centerline{\small  25x.}}
\end{figure}
\paragraph{}
The successive layers of spiral lamina seen in a transverse section of a nummulite (Figs. 22, 23) show alternate clear and striated vertical bands. The clear bands are the ``pillars,'' and the striated ones the ``tubulated'' or inter-pillar areas. Keyserling, Joly and Leymerie, and d'Archiac regarded the pillars as wide-open channels which became, rilled with calcite (\emph{remplissage}, d'Archiac) during fossilization. Carpenter described the pillars as solid non-tubulated structures serving as wedges of support for the tubulated walls. Careful examination will show the hyaline and apparently structureless pillars to be replete with disk structures. Further, a still closer scrutiny will show the tubulated areas also to be full of disks, but here the dominating tubulated-striated pattern (Plate 23, Figs. A-D) renders their detection difficult.

The conclusion reached is that the whole nummulite shell --- the septa and alars, and the spiral lamina, marginal cord, pillars and inter-pillar tubulated areas, is built of ``disks'' having spiral and radial construction.
\begin{figure}[H]
\centering
\includegraphics[width=85mm,keepaspectratio]{figures/Fig23.png}
\caption*{\centerline{\small \textsc{Figure 23 ---  Transverse sections of nummulites showing}}

\centerline{\small \textsc{layers of spiral lamina with pillar and inter-pillar}}

\centerline{\small \textsc{bands.}}

\hspace{5mm}\small A. \emph{N. scabra}, 24x. B. \emph{N. deshayesi}, 28x. \emph{p}, pillar areas. After d'Archiac.}
\end{figure}
\paragraph{}
In the inter-pillar tubulated areas the common mass of spirodiscoid structure constituting the wall is penetrated by ``tubuli.''

If the above observations are correct, the fact of a nummulite shell being built of spirodiscoid elements is not so surprising as might appear. For when a recent and related form, such as \emph{Polystomella crispa}, is about to undergo division, the protoplasm leaves the shell and divides up into numerous microscopic particles, and each one or pair of the latter proceeds to build a new spiral shell.

A nummulite has been regarded by some as a kind of colony. E. Van den Broeck considered the successive chambers or segments as units of a colony. He asks how else can we regard the successive \emph{Lagena}-like segments of a \emph{Nodosaria} (Bull. Soc. Beige Geol., 7, p. 21, 1893). Schlumberger refers to this ``ancienne hypothèse'' as ``inadmissible'' (Feuille des Jeunes Naturalists, 1896, p. 86).

An examination both of vitreous and porcellanous shells (\emph{Lagena}, \emph{Truncatulina}, \emph{Polystomella}, \emph{Miliolina}, \emph{Biloculina}, \emph{Orbitolites}, \emph{Alveolina}] shows that these also have a spirodisk structure. So far, photographs have not been sufficiently good for publication. Under very high powers (3,000 to 4,000 diam.) the shell-substance is seen to be very finely granular and the granules to have a spirodiscoid plan.

Lastly, I firmly believe I have found the finely granular protoplasm of three of the above-named shells, of \emph{Gromia}, and of the lobose Protozoan \emph{Amoeba} likewise to show the same arrangement.

The fact of the earth's crust being mainly composed of spirodiscoid skeletons of (nummulitic) protoplasm may be significant from points of view other than those of systematic zoology, and may point to the possibility of a widespread spirodiscoid construction of living matter, even though it may not be possible to detect the latter. These suggestions are based on careful but insufficient observations, and are mentioned merely for the purpose of calling attention to the matter. (See Postscript, p. 1 80.)

Reproduction. --- Students of nummulites were greatly mystified at one time at finding supposed species always in couples. \emph{N. gizehensis}, for instance, is invariably found with \emph{N. curvispira}, and so on with every well-known species (Plate 2B). Generally one of the couple is considerably smaller than the other, the smaller being much more numerous. On splitting the shells the smaller are seen to have a large central chamber (Form A) and the larger a very small one (Form B). After much controversy it was discovered by Munier-Chalmas that the two forms belonged to one and the same species. Later the ``dimorphism'' was found to be associated with ``alternation of generations.'' Study of nearly related living Foraminifera showed that the protoplasm of Form A (megalospheric) divides up into bi-flagellate spores pairs of which conjugate to form a cell which secretes a microspheric shell. The protoplasm of the latter forms amoebulae each one of which secretes a megalospheric shell. The division of the protoplasm may take place within the parent shell, in the interior of which the little shells will be seen (\emph{Miliolina}), or outside the parent shell (\emph{Polystomella}).

Sometimes shells form buds, which separate from the parent at a certain stage.\footnote{Heron-Allen. Phil. Trans., 1915, vol. 206, p. 245. In \emph{Nummulosphere} 1, I stated, under the influence of the \emph{Eozoön} delusion, that with a little poetic license the planet might be compared to a gigantic budding reef-like Rhizopod encrusting a foreign body. Really it is a case of a mass of separate Rhizopods.}

\centerline{*\hspace{15mm}*\hspace{15mm}*\hspace{15mm}*\hspace{15mm}*}
\bigskip

Food Supply. --- Whence did the nummulites get their food? The investigations ot Gwyn Jeffreys, Wyville Thomson and Carpenter point to the conclusion that deep-sea benthos organisms live mostly on decayed organic matter sunk down from above. A research of Möbius\footnote{Möbius. \emph{Whence comes the nourishment of the animals of the deep seas?} A.M.N.H., 1871, (4) 8. p. 193 (Transl.). Many references to literature.} showed that in some areas a good deal of decayed shallow-water vegetation drifted down to deeper zones. At the present day the Antarctic mud is so rich in diatomaceous protoplasm that fishes feed on it, \emph{and the Foraminifera are crammed with Diatoms}. The abundance of silica in igneous rocks is in itself suggestive of a plankton food supply for the nummulites.

\centerline{*\hspace{15mm}*\hspace{15mm}*\hspace{15mm}*\hspace{15mm}*}
\bigskip

Classification. --- The classification of nummulites is a matter of great difficulty. Carpenter was inclined to the view that in the genus \emph{Nummulites} there was really only one species, with numerous varieties. D'Archiac and de la Harpe both refer to the difficulty of arriving at specific characters. Owing to the great abundance of the material it is often possible to find gradations between very distinct forms.

D'Archiac made use of the surface markings on the spiral lamina or wall of the shell to constitute the main groups. A modification of this system was adopted by de la Harpe, who defined \emph{Nummulites} as a form with completely embracing coils. The genus was divided into two main groups (\emph{a}) the radiate, with separate straight or sinuous radial lines, and (\emph{b}) the reticulate with a network of linear markings.\footnote{Zittel adopts three of d'Archiac's divisions --- Radiate, Sinuate and Reticulate; de la Harpe merges the first two into one.} Each of these groups was again divided into granulate and non-granulate forms according to whether granules were a marked feature or not. The groups were again sub-divided into forms with a small central chamber and those with a large one.

The last feature is now known to be merely a different phase in one and the same species. Apparently the granular or non-granular character of the surface is due to variations in the degree of development of the ends of the pillars, these ends being flush with the surface in non-granular forms. (De la Harpe had adopted Carpenter's view that the pillars were solid imperforate wedges of support.)

The characters available are radial or reticulate markings, granular or non-granular character, thickness of spiral lamina, size and shape of chambers, form and size of the shell as a whole.

Hantken, who investigated the Tertiary nummulitic formations of Hungary, was the first to demonstrate that different horizons were characterised by different species of nummulites. His diagrams\footnote{\emph{The stratigraphical importance of nummulites in the early Tertiary strata of the mountains of S.W. Hungary}. Proc. Roy. Hungarian Acad. Sci., Buda-Pesth, 5., 1875, No. 6. Also de la Harpe. Mem. Soc. Pal. Suisse, 1880, 7. p. 68.} show the strata with their distinct nummulitic faunas. Probably similar zones exist in all the limestones and igneous rocks, and we may possibly be able to locate the zones whence meteorites were derived.

De la Harpe writes of the difficulty involved by the immense amount of material which, however, included only Tertiary nummulites. What would he have thought of the addition of the nummulite faunas of most of the Mesozoic, Palaeozoic and Archaeozic formations and the igneous rocks! A gigantic task awaits future nummulitologists.

\centerline{*\hspace{15mm}*\hspace{15mm}*\hspace{15mm}*\hspace{15mm}*}
\bigskip

Distribution in Space. --- During Mid-Eocene times the Nummulitic ocean and seas extended across the Old World from north-west Africa to Japan. The floor of this ocean is now elevated to form the middle and upper parts of many mountain ranges along the great belt above referred to. These ancient sea-floors form Himalayan peaks 19,000 feet above sea-level.

It is wholly certain that nummulitic deposits were formed in all eras all over the globe, and it is probable that during the early part of the Archaeozoic era they were laid down in a universal ocean wherever the undulating floor was at a suitable depth. For igneous, \emph{i.e.} nummulitic, rocks are universal.

The area of distribution (in space) must now be extended to outer space; for the meteorites now resting in museums, and possibly, therefore, some of the comets and shooting stars, are nummulitic rocks.

The bathymetric range of nummulites probably varied within considerable limits. For they are found mixed both with shallow and fairly deep-water organisms, with calcareous algae and with crinoids and glass-sponges. A moderate depth of about one hundred or of a few hundred fathoms was perhaps most suited to them. Apparently the shallow muddy waters of estuaries were not congenial, for the nummulitic strata of the Bracklesham and Barton beds are thin, and (excepting \emph{N. laevigata}) the species very small. The nummulites of deposits thousands of feet thick, as in ``The Dolomites'' of Tyrol, may, like certain coral-atolls, have been formed on sinking ground.

Further, what is now the abyssal floor of the ocean must once have been in relatively shallow water, for that floor is a deposit of nummulites. This fact will, perhaps, further discredit the theory of the permanence of ocean basins. It is certain that the highest mountain peaks,\footnote{} whether formed of igneous rocks, limestones, or sediments, were once sea-floors, and it is equally certain that the abyssal floor of the ocean was once relatively shallow --- now in one area, now in another. The ocean level has been stable, but the ocean floor continually heaving up and down. If there is now the roar of traffic where there was once ``the stillness of the central sea,'' it is possible there were continents where there are now the deepest abysses.

\centerline{*\hspace{15mm}*\hspace{15mm}*\hspace{15mm}*\hspace{15mm}*}
\bigskip

Distribution in Time. --- Nummulites have hitherto been supposed to be characteristic of the early part of the Tertiary era. There are a very few doubtful records of the occurrence of the genus (\emph{sensu stricto}) in Carboniferous, Jurassic, Cretaceous and recent times.\footnote{See list by de la Harpe. Mem. Soc. Pal. Suisse, ed. 3, p. 37, 7., 1880-1, p. 68, footnote; also Zittel, \emph{Grundzüge}, ed. 3, p. 37.} D'Archiac calls the Eocene ``the nummulitic epoch.'' Carpenter writes, ``There is no fact in Palaeontology more striking than the sudden and enormous development of the nummulitic type in the early part of the Tertiary period and its almost equally sudden diminution bordering on extinction.''

There is no longer reason to regard as singular the existence of thick and extensive belts of nummulitic limestones in the Eocene period. For the chalk ocean was similarly nummulitic, and also the ocean during the mesozoic, palaeozoic and archaeozoic eras.\footnote{At one place in England it is possible to see nummulites of several eras mingled together, or in close proximity, \emph{viz.}, at Selsey. There are Cainozoic (Eocene, Bracklesham;) Mesozoic (Cretaceous silicified nummulites, \emph{i.e.}, flints);? Palaeozoic quartzite and sandstone (ice-borne erratics); and Archaeozoic or pre-Archaeozoic igneous rocks (also ice-borne erratics).}

It is, however, a remarkable fact that the genus \emph{Nummulites} is almost extinct.\footnote{Excluding doubtful records, there are probably only two recent species, and these small, \emph{viz.}, \emph{N. planulata} (Arctic seas), and \emph{N. cumingii} (tropical seas). Williamson records recent \emph{N. radiata} from Portsmouth, but Brady excludes it from his synopsis of British recent species (Journ. Roy. Micr. Soc., 1887, part 2, p. 872).}

\centerline{*\hspace{15mm}*\hspace{15mm}*\hspace{15mm}*\hspace{15mm}*}
\bigskip

\subsubsection{On the Extinction of the Genus \emph{Nummulites}}
\paragraph{}
``What could have been the conditions which so specially favoured their production at the period in question, ... and what change in these conditions put a sudden and almost complete stop to these operations constitute most interesting subjects for physiological and geological enquiry.'' (Carpenter).

The earth's crust is mainly composed of nummulites. The genus has persisted throughout immeasurable aeons from near the beginning of geological time almost up to the middle of the present era, but is now wholly or almost wholly extinct. What have been the causes of this remarkable phenomenon? The solution of the problem must be sought by enquiry into the conditions presumably favourable and unfavourable to the existence of these organisms. Three important conditions of environment are (1) depth; (2) temperature; and (3) the presence or absence of detritus.
\begin{enumerate}
    \item Judging from the cretaceous nummulitic deposit (\emph{i.e.}, chalk), the faunas buried therein are not abyssal; some of the organisms belong to rather shallow water, others to fairly deep, but not very deep water. Of the 197 millions of square miles of nummulitic deposits, over 130 millions are now in depths below 1,000 fathoms, and, as might be expected, no living nummulites are to be found. Further, I hope it may be permissible to add, by way of balancing accounts, that the 54 millions of square miles of nummulitic deposits pushed above the ocean as land are no longer a suitable abode for living nummulites! The presence of nummulite skeletons on mountain tops and abyssal ocean floors is a clear proof that the raised and sunk areas must at one time in the course of the secular see-saw have been at levels suitable to the existence of vast masses of nummulites. The absence of the living shells over an area of 13 millions of miles still has to be accounted for.
    \item It would appear that nummulites preferred a temperature not too low. The chalk and Eocene oceans were within the temperate zone. The vast deposits of Triassic nummulites forming the Schlern and Mendola dolomites in South Tyrol were laid down in warm, temperate, or sub-tropical seas where coral reefs grew. Apparently, then, low temperature would kill off nummulites in high latitudes and in considerable depths. The former existence of these organisms within the Arctic circle and close to the South Pole may be accounted for on the reasonable assumption that the temperature in those latitudes has been higher from time to time. The presence of certain fossil plants in the south-polar regions affords evidence of a milder climate. The specimens of igneous rocks dragged back by Captain Scott and his party from the neighbourhood of the South Pole are nummulitic. A piece of diorite from the neighbourhood of Beardmore glacier, not far from the South Pole, shows nummulitic structure plainly under a hand-lens, and the volcanic rocks of Erebus and Terror are made of nummulites.
\end{enumerate}
\paragraph{}
After eliminating regions too deep, too cold, or emerged, there still remain a few million miles offering moderate depths and equable temperature. Why have the marvellously persistent nummulites failed (or almost failed) to survive even there? In the Eocene ocean and seas the shells apparently flourished best in clear water remote from land. Where the sea was shallow, and turbid owing to land detritus carried down by rivers the shell deposits were thin and the shells usually small as in the British Eocene formations.

There is reason to believe that the final extinction of \emph{Nummulites} over areas where the genus might otherwise have persisted in abundance may have been due to the ushering in of a \emph{Globigerina} epoch, \emph{i.e.} an epoch of calcareous Foraminiferal plankton.

Zittel's useful charts of distribution-in-time of various families of Foraminifera (\emph{Grundzüge}, ed. 3, 1910, p. 37) reveal a remarkable fact. Globigerina is stated to begin in the Cambrian;\footnote{Although I have not seen the phosphatic nodules of the Cambrian of New Brunswick stated by W. D. and G. F. Matthew (Trans. New York Acad. Sci., 1893, 12. p. 108; and 1895, 14., p. 101, Plate 1. figs, 1-8\emph{b}) to contain \emph{Globigerina} and \emph{Orbulina}, I am convinced there is here an error of interpretation. In the course of the present work I was at one time continually mistaking nummulitic structures for \emph{Globigerina} and \emph{Orbulina}. The same mistake has been made in the case of the ``spheres'' in chalk. Apparently the true Globigerina record begins with the Mesozoic era.} then during the remainder of the Palaeozoic era there is no record. During the Mesozoic era and the early part of the Gainozoic (Eocene and Oligocene) the genus exists but not abundantly. Suddenly, at the beginning of the Miocene and on to the present time Globigerina becomes abundantly prevalent, \emph{i.e.}, at the time when \emph{Nummulites} becomes almost extinct, \emph{Globigerina} becomes abundant.

At the present time \emph{Globigerina} flourishes over the whole of the tropical and temperate areas of the ocean and forms deposits over 49 million square miles of ocean floor. Doubtless the 51 million miles of red clay would also be covered with \emph{Globigerina} if the shells were not dissolved when sinking to depths below 2,500 fathoms. Possibly these calcareous plankton shells containing the corpses of Foraminifera were obnoxious to the heavy nummulites which were choked by the accumulating \emph{débris}\footnote{Sponges seem able to protect themselves from mud and plankton \emph{débris} by developing lids and sieves. Again, certain cup-shaped glass sponges appear to have given rise to sitz-bath-shaped forms, and finally, by loss of the cavity and lengthening and narrowing of the back of the ``bath'' to sword-shaped stems.} or succumbed to microbic invasion. \emph{Globigerina}, it is true, is found in the chalk, but not to the extent at one time believed. \emph{Nemo fuit repente turpissimus}, nor did the nummulites become utterly decadent all at once; but after enduring for immeasurable aeons, they at last all but died out in the Miocene or possibly Pliocene period.

Phylogenetic Note. --- It seems singular that nummulites, regarded as the highest type of Foraminifera, should prevail in the oldest known rocks. The shell, although apparently very complicated, is built on a simple plan, \emph{viz.}, that of an embracing spiral. The thickness of the walls is not perhaps a character denoting high organization, but the canal system in the walls, and the double-walled septa are considered to be features of a highly developed type.

It may be assumed that nummulites and other many-chambered Foraminifera are descended from one-chambered ancestors, and the latter from forms without shell. In the life cycle of nummulites, the protoplasm of the microspheric shell breaks up into amoebulae, bodies which probably exhibit the ancestral characters not only of Foraminifera but of the whole of the Protozoa.

Prof. Minchin in his great work, \emph{An Introduction to the Study of the Protozoa} (p. 465) writes, ``We may, then, regard as the most ancestral type in the Protozoa a minute amoebula-form, in structure a true cell, with nucleus and cytoplasm distinct, which moved by means of pseudopodia.'' Before the great planetary nummulitic deposits were formed, probably there existed a period during which the ocean floor was more or less covered with the shell-less amoebula-ancestors of nummulites.

Huxley's ``Bathybius theory,'' although based on an error, is now seen to be not far from the truth. He mistook gelatinous networks with calcareous particles found in bottles of preserved abyssal Atlantic ooze for specimens of a primitive form of living matter spread over the ocean floor. The jelly was found to be a precipitate of sulphate of lime in alcohol, the calcareous particles being coccoliths. It is now apparent that the whole ocean floor actually has been covered with rhizopodal lime-secreting protoplasm in the form of little separate masses.

What is the origin of this benthos protoplasm, distributed at one time or another over the whole surface of the planet, and the builder, moreover, of the original calcareous scaffolding of the lithosphere?

Did the amoebula-ancestors of nummulites originate on the sea-bottom, or were they immigrants from elsewhere?

Before attempting to answer this question it will be desirable to make a few observations on the simplest forms of life.

The simplest of all are the bacteria --- living particles which are not even cells, for they are commonly devoid of a definite nucleus. Above the bacteria come the simplest plants and animals (Protophyta and Protozoa), the simpler forms of which consist of a single nucleated cell, the presence of the nucleus marking a higher grade.

The distinction between plants and animals, so obvious in the higher types, is often ill-defined and vague in the lowest, identical groups of the latter being frequently claimed both by botanists and zoologists.

The Protozoa gain their subsistence in four ways, \emph{viz.}, the holozoic or purely animal way of capturing and feeding on other organisms; the holophytic or purely plant\footnote{The Protozoan and \emph{animal} nature of holophytic Protozoa is shown chiefly in the affinities with the holozoic forms, in the general life history, in the absence of cellulose, \emph{etc.}} way of subsisting on inorganic materials through the agency of chlorophyll in presence of sunlight; the saprophytic (or saprozoic) way of living on the products of metabolism or decomposition of organisms; and the parasitic way.

Many of the Protozoa vary their mode of nutrition either in different phases of their life-cycle or even during one and the same phase. \emph{Euglena} for example, which by reason of its chlorophyll is holophytic in sunlight, loses its green colour and becomes saprophytic in the dark.

The four modes of nutrition come under two categories, \emph{viz.}, the independent (the best-known example being the holophytic), and the dependent comprising the other three. A holophytic organism is entirely independent of the existence of any other form of life, and simply requires sunshine and the presence of suitable inorganic materials. Organisms with animal, saprophytic or parasitic characters are dependent directly or indirectly on other organisms.

The primitive (igneous) rocks are composed of nummulites mineralized by silica (combined to form silicates). There are reasons for believing that the silica is partly derived from plankton organisms.

The continuous rain of plankton material from ocean surface to ocean floor suggests a plankton origin for the primitive forms of benthos life, including the amoebula-ancestor of nummulites.

Primitive independent organisms reaching the bottom alive might continue to exist, but in the absence of sunlight they would, like \emph{Euglena}, take to a saprophytic or an animal mode of nutrition.

The discovery of the organic origin of the earth's crust may throw light on problems concerning the origin of life on this planet.

From its beginning life may have originated on a planetary scale, or indeed, in view of the importance of sunlight in relation to life, it might be said on a solar scale.

The whole animal world in the great ocean basins is ultimately dependent on minute forms of vegetable plankton, and, moreover, has been so during the long course of evolution. Many facts seem to point to the conclusion that life began in the sunlit surface waters of the ocean.

The sequence of events leading to the formation of the earth's crust might well have been universal sunshine on a universal ocean engendering universal primitive ``independent'' plankton, whence was derived primitive benthos whenever an area of the ever-undulating ocean floor came within suitable bathymetric range.

Apart from theory there \emph{is} the sunshine, the broad ocean surface, the vegetable plankton, the animal benthos, and the earth's crust made of benthos skeletons mineralized by silica very probably derived in part from plankton skeletons.

\emph{Postscript}. --- I am finding more and more the idea of spirality to be of the greatest help in tracing nummulitic structure in rock sections under the microscope. Often it is as if a plan had been given of a complicated labyrinth.

A spirodisk of any dimensions, say from 0.5 to 0.005 mm. will always have in relation with it a spiral series of spirodisks of a lesser order and in planes vertical to the larger one. Similarly each lesser spirodisk will have its primary coil with series of lesser secondary ones, and so on to the limits of microscopic vision, and, I believe, far beyond. Any primary coil has somewhat the aspect of a ``marginal cord,'' and the secondaries of that coil the aspect of ``septa'' of a nummulite or cogs of a cogwheel. Similarly the ``septa'' or ``cogs'' are themselves ``marginal cords'' with ``septa'' or ``cogs'' of a diminished order.

It will now be necessary to work out the structure of the nummulite shell in terms of spirodisks and to learn how the spiral lamina, marginal cord and septa are built up, and the relation of tubules to the spirodisks.

All calcareous Foraminiferal shells appear to be built of spirodisks. The primitive form was probably one-chambered and spheroidal, with the wall composed of spirodisks. Whether spiral shells owe their form to \emph{retention} of a shape impressed on them by inherent qualities of protoplasm, or whether the form has been moulded by forces of the environment it is hard to say: the antiquity and universality of nummulites rather suggest that the first-named cause has at least contributed. One might then suppose the assumed inherent spiral tendency to have become overpowered in non-spiral forms. A \emph{Lagena}, a \emph{Biloculina}\footnote{See Appendix. Note R.} and a nummulite certainly have two, and probably three orders of structure, of which only the first has hitherto been recognized: \emph{viz.} (1) the visible shell; (2) the microscopic spirodisk structure composing that shell; and (3) ultra-microscopic spirodisk structure (see p. 147, diagram: p. 162: and Chapter 10).
\clearpage
\subsection{Chapter 9}
\subsubsection{The Nummulitic Structure of Igneous Rocks and Meteorites}
\begin{displayquote}
``A vast thickness of solid rock... formed of little else than their remains.'' Carpenter, on the enormously thick and extensive deposits of Tertiary nummulites, his words being equally descriptive of igneous rocks.
\end{displayquote}
\paragraph{}
Nothing at first sight could appear more remote from anything of an organic nature than a lump of granite, seemingly a confused mass of quartz, feldspar and mica. Apparently, too, the rock has once been in a molten or semi-molten condition, and has slowly cooled down and become crystallized under enormous pressure deep in the interior of the earth. Accordingly, in the apparent absence of direct evidence, it is not surprising that the significant hints afforded by oceanic biology and by geology concerning the genesis of the earth's crust have not been duly appreciated. Happily, however, the igneous rocks bear in themselves clear and positive evidence of their oceanic and organic origin during some remote aeon of the past. For they are not merely masses of minerals, but mineralized masses of nummulites.

The failure to detect the prevailing nummulitic nature of the earth's crust notwithstanding the countless observations on rocks and rock-sections is indeed a strange phenomenon, and one worth enquiring into.

After d'Archiac had established a ``nummulitic epoch,''\footnote{``Et jamais le mon \emph{horizon} n'a été plus heureusement employé que pour \emph{l'ère des nummulites}, que pour ce court espace de temps qui vit naître, se développer et presque disparaître tout à fait ce genre encore si énignmatique.'' \emph{Histoire du progrès de la Géologie}, 1850, 3. p. 214.} usually any supposed nummulites found in non-Tertiary strata were either considered not to be nummulites at all, or the strata were regarded as really Tertiary.\footnote{``Although nummulites have been described as existing at periods anterior to this, it seems probable that such descriptions have been founded on the occurrence of other helicoid Foraminifera bearing an incomplete resemblance to them.'' Carpenter, Introduction, p. 276. Gümbel (Neues Jahrb. 1872, 13. p. 251), criticizing Fraas' account of three species of supposed nummulites in the Cretaceous rocks of Palestine, concludes that the traveller had mistaken the age of the beds which Gümbel regarded as probably Tertiary, ``but does not give any very clear ground for this conclusion'' (Brady).

Easily recognizable nummulites, however, are nearly always Tertiary. Even the supposed Carboniferous \emph{N. pristina} (Brady) is now said to be Tertiary, \emph{fide} a pencil note in an N.H.M. copy of Nicholson's Palaeontology referring to a paper on this matter by Van den Broeck.}

Again, it has been assumed that high temperature would obliterate all traces of organic structure in igneous rocks if anything of the kind had ever existed there; and further, any supposed organic forms have been put down either as pseudomorphic\footnote{King and Rowney, \emph{An Old Chapter of the Geological Record with a New Interpretation: or, Rock-metamorphism and its Resultant Imitations of Organisms}, 1881. Often these supposed imitations are really organic structures, and often not so.} or wholly phantasmal. Actual experiment shows,however, that organic structure is not wholly dissolved away even in boiling rocks, and that it easily survives in white hot slags. Experience and training of the sense of sight are the remedies against illusion and misinterpretation.

The omnipresent nummulites have escaped detection not so much on account of certain tendencies of the human mind, but rather owing to their own qualities.

The shell is built of a series of slightly separated layers on each side of a hollow median region, and each layer is highly porous. No apparatus could be better adapted for capillarity and permeation by fluids. The effect on the shells varies with the nature of the permeating material. Water and carbonic acid tend to soften and disintegrate them, as in chalk; or partial solution and redeposition may cause the mass to become hard and crystalline or oolitic. Colloid silica, again, may replace the more soluble carbonate of lime, so that the shells become silicified, homogeneous and transparent. Heat causes silica to enter into combination with various bases to form silicates. The heat may come from volcanoes as in the Jurassic Monte Somma bombs; from intrusive magmas as in \emph{Eozoön}; or from some not certainly known cause, as in igneous rocks. The cooled rocks become masses of crystalline silicates, the outline of the nummulites being much broken up.

Apart from chemical processes, mechanical pressure tends to mould these deposits of lens-shaped shells into homogeneous masses in which the outlines of individual nummulites become more and more obliterated.

Lastly, in the case of the main mass of most of the sedimentary rocks, the breaking down of the mineralized shells into particles so conceals the nummulitic structure that it can only be seen by examining those particles under rather high powers of the microscope.

Accordingly, what with pressure, solution, mineralization, clarification, crystallization and pulverization, the nummulosphere has remained a \emph{terra incognita} despite the fact that it confronts us on all sides in the mountains and plains and cities.

The nummulites of the ``nummulitic epoch'' are easily seen, and those of all the preceding epochs not easily seen. The explanation of the enigma of the Tertiary nummulites is simply that the individual forms of the shells of the older deposits have become more or less obliterated, or, at any rate, difficult to trace.

\subsubsection{Methods of Observation}
\paragraph{}
The study of lumps and sections of Tertiary nummulitic rocks will form a good introduction to that of the igneous rocks. The eye will become familiarized with the appearance of the shells in horizontal, oblique and transverse aspects and sections. Further the effects of the various kinds and degrees of mineralization, crystallization and degradation of nummulitic structure can be observed, for there is a considerable range of variation in such processes even in the Tertiary rocks. The nummulitic structure of igneous rocks can be detected by the naked eye, the hand lens, and the microscope.

It will be well to recapitulate briefly certain points. The shell is lens-shaped, and varies (in Tertiary species) from 1 to 107 mm. in diameter. The surface shows linear and granular markings. The bulk of the shell is composed of layers of the spiral lamina (Figs. 21, 22). A transverse section reveals a series of pointed ovals one within another (willow-pattern). The walls of the ovals show alternating opaque and crystalline bands.

A transverse section or aspect with its embracing gothic arches is easy to understand.

On the other hand, a horizontal view of a large transparent shell with its many layers of striated spiral lamina, each layer having numerous pillars expanding towards the periphery, with its spiral series of septa and chambers in the central plane, with tier upon tier of radial, sinuous or reticulate alar prolongations, and with its many-coiled spiral of the furrowed marginal cord, the whole often very variously mineralized, such a view requires long and patient observation to enable the observer to interpret and to piece together the numerous details. One might almost compare a transparent nummulite viewed in perspective under magnification to a many-roomed crystal mansion. An advance was made in this research when I found that the nummulite shells of igneous rocks and of \emph{Eozoön} were sometimes of large size, \emph{viz.} 25 mm. (i inch) or more in diameter, and that certain well-defined circular areas and patches in these rocks were not small shells but portions of marginal cord or spiral lamina of large ones.

The most persistent structure is the furrowed marginal cord.
\begin{enumerate}
    \item Detection by unaided vision. It is usually difficult to see the shells on a fresh crystalline surface of rock, but after careful preliminary study with a lens they can not infrequently be traced by the naked eye on weathered and disintegrated surfaces. Manoeuvring the rock in the light will often help the eye to distinguish the large oval and circular outlines of the shells. Much nummulitic structure can be seen in photographic negatives, the camera having wonderful discriminating powers. On a weathered block of Canadian \emph{Eozoön} about 15 square inches in area, I can now make out six or seven shell outlines; also I can distinguish the nummulites in weathered basalt and granite. Probably an untrained observer would fail to see anything of the kind, but would have less difficulty in seeing under high magnification nummulitic structure in every particle from the surfaces in question.
    \item The use of a hand-lens. A surface of rock shows shells whole or in section, and in many aspects horizontal, transverse, or oblique. A horizontal aspect (\emph{en face}) will show segments (single or concentric) of marginal cord, and series of arches of alar prolongations and septa.
    Transverse sections or fractures will show embracing gothic arches with striated walls, portions of marginal cord with septa seen in perspective as transverse bars across the cord.

\hspace{5mm}Photographic negatives enlarged two or three times are very helpful indeed. It must be remembered that a shell only 1 inch in diameter will occupy 16 square inches under a low magnification of only four diameters.

\hspace{5mm}A very careful study of some of the large photographs in Dr. Clinch's memoir on Spongiostromids will enable a trained observer to realize the effect of slight magnification, for Spongiostromids are simply deposits of Carboniferous nummulites.
    \item The study of sections under the microscope. Under low powers it is often possible to see portions of shell with various structures still in normal relation to each other. For instance a transverse section or aspect will show two or three embracing gothic arches with striated walls and with pillars in radial series; also the furrowed marginal cord with fan-like septa astride, and again the band-like edges of alar prolongations. If any doubts remained, a high power will reveal perforated disk-structures in various aspects, and it will then become impossible for any instructed observer to doubt the significance of what he sees.
\end{enumerate}
\subsubsection{1. Igneous Rocks}
\paragraph{}
The igneous rocks have once been semi-molten or molten. Those masses which have reached the surface in a heated condition, \emph{i.e.}, the volcanic or superficial igneous rocks, have cooled quickly, and consequently have formed a glassy ground-mass usually with a varying number of very minute crystals embedded in it. Dykes and sills, occupying an intermediate position, have somewhat larger crystals visible to the naked eye or with the aid of a lens. The deep-seated abyssal or plutonic rocks, owing to very slow cooling, have large crystals.\footnote{Plutonic rocks now at the surface have been uncovered owing to denudation. The macro-crystalline nature is in itself a proof of slower cooling and therefore of a former deep-seated position.}

The condition of the nummulite shells of volcanic, intermediate and plutonic rocks naturally varies correspondingly. The nummulites of volcanic rocks will have a glassy ground-mass usually with innumerable small embedded crystals which to some extent map out the various structures of the shell. A plutonic nummulite, on the other hand, may be wholly formed of quartz or feldspar, but usually of several little masses of minerals.

A short description will now be given of the nummulitic features of a few typical examples of igneous rocks.

\bigskip
\centerline{A. \emph{Basalt from Snake River, North-West America}}

This rock, covering an area of a fifth of a million square miles to an average depth of 2000 feet, forms ``the world's greatest lava flow.''

I failed to come across an example of this basalt in London, but an application to the United States Geological Survey for a hand-specimen was favourably received.

This specimen is a smooth water-worn boulder. The freshly-fractured surface is black, and with a finely granular glistening appearance. Very careful observation with a lens (10x) will show here and there defined circular areas about 1 to 2 mm. in diameter, and indications of pointed ovals of the spiral lamina. It would be well for the observer to leave aside these obscure indications for a time and turn to the transparent sections. (Probably \emph{roughly} weathered surfaces of the rock would show shell structure clearly.) Careful and detailed observation of sections with lens and microscope will lead to an astonishing revelation. At first sight nothing is to be seen excepting a confused medley of small linear crystals, granules and light and dark patches (Plates 14, A, and 17, A).

Gradually a plan becomes dimly discernible in the midst of the apparent confusion. Frequently a banded arrangement will be seen with the linear crystals in straight lines or in convergent conical groups across the width of a band. The granules will often be found in circular groups. Again a concentric plan may be observed, the concentric lines or bands being broad and furrowed and with rows of beads across the width. Both the light structureless glassy patches and the black ones break up irregularly the circular groups of granules or the linear parallel and convergent groups of linear crystals.

What we have been studying are strangely metamorphosed silicated nummulites. The broad bands are the successive layers of the spiral lamina. The furrowed concentric bands with transverse beads are portions of marginal cord and septa (Fig. 24). If after much careful observation doubts still exist, higher powers will reveal abundant disk-structure in every crystal, and in the vitreous patches.

The conclusion will force itself irresistibly upon the trained observer that the world's greatest lava flow is a mass of silicated nummulites.
\begin{figure}[H]
\centering
\includegraphics[width=85mm,keepaspectratio]{figures/Fig24.png}
\caption*{\centerline{\small \textsc{Figure 24 --- Snake River basalt.}}

\centerline{\small Fragment of marginal cord of a nummulite. 100x.}}
\end{figure}
\paragraph{}
The mineral characters of Snake River basalt resemble to some extent those of the Renazzo meteorite figured and described by Tschermak.\footnote{G. Tschermak, \emph{Die mikroskopische Beschaffenheit der Meteoriten}, 1885.} His figure (Plate 15, Fig. 4, 160x) shows linear crystals, granules and glassy and black patches just as in the basalt. He calls the augite crystals spreuformige (chaff-like), a term admirably descriptive of the augite crystals of the basalt. The figure shows well a portion of parallel-banded marginal cord with a transverse row of beads.

In both basalt and meteorite the linear crystals are augite, the granules olivine, and the dark patches magnetite.

\bigskip
\centerline{B. \emph{Syenite from the Plauenscher Grund, near Dreden}}

This rock exhibits, as is well known, a finely layered arrangement. Pressure, chemical action, and perhaps also the organic factor may have contributed to produce this structure. The rock is a mass of large nummulites compressed and mineralized. Some large sections viewed with a lens show series of bands finely striated across their breadth. Patches of dark green hornblende are situated often at fairly regular intervals between the bands. Other sections show a reticulate pattern of broad transparent bands, with the hornblende patches showing an obscurely marked concentric plan. The reticulate pattern is formed by radial alar prolongations crossing concentric marginal cords. Under low or high powers the syenite is seen to be a mass of nummulitic structure throughout (Plates 12, C, 15, C, and 19, C).

\bigskip
\centerline{C. \emph{Clee Hill Basalt}}

According to Sir A. Geikie, the summit of Clee Hill is a sill formed by the outspread of this rock which is known as an olivine-dolerite. I collected numerous specimens in all states of preservation. The weathered surfaces of long-exposed blocks viewed with a lens show the shells in horizontal and transverse aspect or section, but not immediately to the untrained eye. Sections reveal the nummulitic structure very clearly. The soil of the heath and the fine soil of rabbit burrows preserve even in the finest particles clear evidence of their origin (Plate 15, Fig. A).

\bigskip
\centerline{D. \emph{Porphyritic Rocks}}

Sections of precious porphyry (from Mons Porphyritis) show the shell structure clearly in the large crystals. There sometimes appears to be a continuity of organic structure between crystals and matrix. If so, the large crystals could not have been greatly displaced floating objects. The organic factor will furnish diagnostic evidence on these points.

A rhombic porphyry picked up on the Yorkshire coast, and forming part of an erratic from Norway, shows light-coloured ovals which, I believe, actually are in some cases central blocks of large nummulites. In a section of nummulitic limestone from Kalibagh, Scinde, some of the shells cut across transversely exhibit oval central areas homogeneous and translucent, somewhat resembling the rhomboidal patches of the porphyry. The latter are certainly masses of nummulitic structure, whatever their precise position in the shell.

\bigskip
\centerline{E. \emph{Cornish Granite from De Lank Quarry}}

The De Lank is a typical grey granite with quartz, feldspar and mica. On weathered blocks I can see outlines of large shells, especially in the rotten feldspar, in which sections of spiral lamina and pillar-ends can be made out without much difficulty.

The shell-structure is best preserved, or, rather, best seen, in the feldspar, but it is possible to detect it even in quartz and mica.
\begin{figure}[H]
\centering
\includegraphics[height=85mm,keepaspectratio]{figures/Fig25.png}
\caption*{\centerline{\small \textsc{Figure 25 --- Cornish granite.}}

\centerline{\small Portion of nummulite in transverse aspect. 65x.}}
\end{figure}
\paragraph{}
As in the case of Clee Hill dolerite, the organic structure is preserved in the smallest particles of earth or clay, and can be seen under very high magnification. (Plate 18, Figs. A, B.)

\subsubsection{2. Meteorites}
\paragraph{}
(Plates 14, Fig. B; 16, Figs. A-D; 19, Figs. A, B; 20, Figs. C, D, E.)

Meteorites are nothing else than mineralized or ore-enriched masses of nummulite shells. On the surface both of the stony and iron kinds it is possible to see outlines of shell-structure, especially the small circular pillar-ends, and pillar and inter-pillar areas of spiral lamina, and portions of marginal cord.

A section of a stony meteorite viewed with a lens shows a minutely granular structure with dark patches of nickel-iron, each surrounded by a reddish zone of diffused rust. In addition, in most of the stony meteorites, there are areas --- circular or fan-shaped as viewed in section --- the chondrules, showing a radiating or reticulate pattern.

Under a hand-lens it is not difficult to trace out larger or smaller parts of nummulite shells in horizontal or transverse aspect or section; also series of pillar-ends, marginal cord, and especially portions of willow-pattern. All these objects are easily visible to the trained eye in the Wold Cottage and Stavropol meteorites.

The finding of the organic pattern in the seemingly confused mass of granules is now amazingly simple to me, but formerly, before I had arrived at the ultimate point of the truth, even though near to it, many hours of careful search were needed to find some obscure indication of organic structure. Now it is easy to see in a few moments abundance of such structure. The use of the highest powers, again, will reveal disk-structure in almost every particle.

An interesting point now comes out, \emph{viz.}, that the chondrules are to some extent based on organic structure, even though the ``mineralizing power'' may get the upper hand, so to speak, of the organic paths and partly obliterate and overflow beyond them. The large coarse chondrules with radiating pattern of thick bars are usually thick marginal cords in transverse section. Fine and small fan-like reticulate chondrules occur in pillar areas in the spiral lamina, and appear fan-shaped or circular according to the plane of section or aspect. One thing is wholly certain, \emph{viz.}, that these chondrules are crammed with the organic structure peculiar to nummulites, \emph{viz.} the ``granulated'' disk-like structure.

Some stony meteorites are devoid of chondrules. Although nummulitic structure is clearly visible in chondrules, I was mistaken in supposing the existence and the generally known characters of these bodies to have been dependent on the nummulitic factor. Meteorites are heterogeneous masses of mineralized nummulites hurled up from a region of intense heat and pressure and plunged suddenly into an intensely cold vacuum. According to Dr. Brezina the chondrules may be the result of a ``hurried crystallization.'' There is a certain degree of mineral differentiation even in the calcareous Tertiary shells. The pillars, for example, are as transparent as crystal, but the inter-pillar areas striated. In the long history of mineralized nummulites these differentiations persist and give rise to further differentiations.

However much the purely mineral may overpower the organic it is nearly always possible to discern the constraint imposed by the latter upon the former. Even in asbestos, labradorite, obsidian mica and meteoric iron, the organic pattern still holds its own, though to a very attenuated degree. The form of the scaffolding built up by life remains here and there, but the original material has been replaced.
\begin{figure}[H]
\centering
\includegraphics[height=85mm,keepaspectratio]{figures/Fig26.png}
\caption*{\centerline{\small \textsc{Figure 26 --- Stavropol meteorite.}}

\centerline{\small Transverse aspect and section of a nummulite showing spiral lamina. 85x.}}
\end{figure}
\paragraph{}
Igneous rocks often show coarse patterns (based on nummulitic structure) only requiring to be brought into relief with a little extra temperature to make chondrules.

Photographs of sections of meteorites often show the nummulitic structure fairly well.

Tschermak figures on Plate 9, Fig. 3 of his great atlas a section (160x) of the chondritic meteorite of Seres, and describes it as follows: ``Oblique section through an olivine crystal. The crystal is forced open on the right side, and there have been pressed in from the ground mass two antler-like glass-inclusions which are arranged symmetrically on each side of a middle line. Accordingly, the crystal appears to be divided into several walls symmetrically ranged against a middle wall. The surroundings consist of olivine granules.'' Tschermak is unconsciously describing definite organic structure. The antler-shaped bodies on each side of a horizontal line belong to a segment of furrowed marginal cord with a septum or base of a septum across it. At the upper end of the crystal the cord is seen in section with striae radiating down to a defined semicircular edge. Within the right margin of the photograph a second marginal cord runs parallel to the first. In the lower left corner is a third piece of cord with a pointed-gothic septum leaning back to the right. The olivine grains are here and there in obscurely circular groups or groups of groups, these apparently being sections of pillars coming off at right angles to the marginal cords.

With or without a hand-lens I can trace organic structure, sometimes with much difficulty, but usually with great ease, in every photograph of Tschermak's great work.

\bigskip
\centerline{\emph{Purely Metallic Meteorites}}

There is a continuous series from completely stony to pure iron meteorites, \emph{i.e.} from stony masses of nummulites to completely ore-enriched masses of these shells. Patches of metal are scattered about in the walls of the shells in the former, and this metal does not fill empty gaps but is nummulitic structure composed of iron and nickel molecules.

The iron may have been reduced by the carbon monoxide often found along with the dioxide. When a siderolite returns to our planet the pure iron again combines with oxygen, and a reddish halo spreads into the surrounding minerals. I can certainly make out disk-structure in the rusty \emph{débris} of the great Melbourne meteorite, as well as in the rust of an old iron saucepan which I submitted to microscopic examination.

It is possible also to make out the disks on the surface of patches of pure nickel-iron in sections of meteorites viewed by reflected light under high powers. The dots are seen arranged in circular groups, each dot having a shining point in the center of a circular area. Lastly, I am certain I can trace nummulites even with a hand-lens. (Plate 22)

Metallic fossils (\emph{e.g.} of Trilobites) are not uncommon; marcasite nodules, again, are masses of nummulites in sulphide of iron. I have found nummulitic structure in samples of haematite, clay iron stone and other iron ores.

\centerline{*\hspace{15mm}*\hspace{15mm}*\hspace{15mm}*\hspace{15mm}*}
\bigskip

\emph{Petrological note}. --- The lithosphere being a mineralized deposit of nummulites, Petrology will learn much concerning the arrangement of minerals in rocks by studying the structure of nummulites. In the past, if anyone asserted that he saw traces of organic structure among the crystalline masses of minerals, he was told that probably he was looking at some pseudomorphic resemblance wholly mineral in origin. In future, when nummulitic structure and its gradual transformation and degradation have been duly studied, it will be seen that the mineral structure of igneous rocks is to no small degree constrained to follow the path laid down by previously existing organic structure. Many times I have traced the parallel striations of the spiral lamina gradually becoming fainter till there remained rows of straight lines in a crystal.

\emph{Postscript. Note 1}. --- I can now detect distinctly and with certainty abundant nummulitic structure in the purely metallic Mazapil siderite. The structure is visible with a hand-lens and by reflected light under medium and high powers (see photos, Plate 22, Figs. C, D). The meteorite fell during a display of shooting stars. If, as some believe, this meteorite be part of Biela's comet, the discovery of the real nature of the Mazapil iron will strengthen the theory that many shooting stars and some comets may be composed of nummulitic material.

\emph{Note 2}. --- All siderites are almost certainly products of terrestrial metallurgy. The richness of certain areas, as Mexico,\footnote{Fletcher, Min. Mag. 9. p. 91 (1890).} in supposed siderites may be due not to showers of many or dispersal of one of these bodies, but simply to volcanic activities in regions rich in ore-bearing strata. A siderite is an eject that has left a volcano with a six mile power. Masses of nickel-iron ejected with less power would quickly return to the earth probably to the west of and not far from the point of ejection.

\emph{Note 3}. --- The nummulitic nature of granite can be made out with certainty and after very little trouble, on the level but unpolished ``under'' surfaces of the little sample blocks used by stone merchants. Not only can the circular oval or willow-pattern outlines be seen, but details of the spiral lamina and chambers, the latter commonly outlined by shining dots formed by in-filling minerals such as quartz. I can see the outlines with the naked eye, but much better with a lens 3x or 10x. Some samples of Petersham and Swedish granite show the shells very well. The nummulites are more easily seen than in chalk!
\clearpage
\section{Part 3}
\subsection{Chapter 10}
\subsubsection{On Problems relating to the Origin of Life}
\begin{displayquote}
``O Thou who hast come safely into this Being's Land;\\Strange, thou thyself not knowest, how thou didst reach its\\strand.''\\-- Jelaleddin (Hastie's version).
\end{displayquote}
\paragraph{}
In this chapter an attempt will be made to indicate briefly the bearing of the nummulosphere discovery on certain biological questions.

The present research will, I believe, throw light on problems concerning the place of origin of terrestrial life, the conditions that prevailed on the earth when life began, the nature of the first formed living matter, and, lastly, the structure of protoplasm.

Certain of the efforts that have been made to seek life's origin ``from afar,'' will perhaps some day seem overstrained. The theory, for instance, that life was brought here on meteorites will now be obviously untenable, seeing that these bodies are merely mineralized masses of nummulites almost certainly cast off from this earth and recaptured. Again, there is no evidence to show that life came in the form of particles ``fleeting thro' the boundless universe'' and driven hither and thither by ``radiation pressure.'' (Arrhenius, who favours this panspermia theory, believes meteorites to have been formed by the coming together of mutually-attracted particles of matter, but this hypothesis is now clearly seen to be unsound. See Plates 16 and 19, Fig. A.)

Life is universal from pole to pole and from the summits of the mountains to the floor of the ocean abysses, and the nummulosphere research shows this universal existence to have endured back to some remote aeon. Facts of this nature, if they do not forbid, should at least discourage the adoption of far-fetched theories when a very simple one is available, and seemingly the reasonable inference to be drawn is that of the terrestrial origin of life.

\centerline{*\hspace{15mm}*\hspace{15mm}*\hspace{15mm}*\hspace{15mm}*}
\bigskip

Having found the earth's crust to be a deposit of nummulites mineralized by silica derived partly from plankton, and the benthos rhizopodal bathybius probably to have had a plankton ancestry (Chap. 8), it is necessary to go a stage further and enquire as to the origin of the primordial living matter or ``heliobius'' (sun-life) that came into existence presumably\footnote{Oceanic animal life at the present time depends on a plankton heliobius (Diatoms, Zooxanthellae, \emph{etc.}), and there is no reason to assume that this relation has not existed from the beginning, the sun-formed colloids constituting the basic food-supply of the animal world.} on the surface of the ocean.

Many now hold the belief that organic evolution follows on uninterruptedly from what Sir Norman Lockyer terms inorganic evolution. As heavenly bodies from the hottest suns to planets diminish in temperature their materials are able to form ever more complex alliances. ``As matter is allowed capacity for assuming complex forms those complex forms appear'' in accordance with what Professor B. Moore terms ``The Law of Complexity.''\footnote{\emph{The Origin and Nature of Life}, 1912.}

Very different pictures are drawn of the condition of the earth's surface when life first became possible. Professor Minchin, for instance, conjures up a vision of a hot crust formed over molten magma, the latter continually breaking through, with condensing vapours exploding into steam, and general conditions favouring a great synthesis of chemical compounds. ``It is conceivable,'' he cautiously remarks, ``that this period of chaos, of storm and stress on a gigantic scale, might have been the womb of life.''\footnote{\emph{Speculations with regard to the Simplest Forms of Life and their Origin on the Earth}. Journ. Quekett Micr. Club, vol. 11., 1912.}

Carl Snyder, on the other hand, is forced to the conclusion that the early conditions ``were probably not immeasurably different from such as obtain now.''\footnote{\emph{The Physical Conditions at the Beginning of Life}, \emph{Science Progress}, vol. 3., April, 1909.} He doubts whether life could have begun in the sea, chiefly on chemical grounds. Apparently a sufficient concentration of carbon dioxide and nitrogen compounds would be more likely to take place on volcanic land areas than in sea or air.

For me, a piece of igneous rock conjures up a picture of the sun shining on the waters of a universal ocean. For igneous rocks are silicated nummulitic deposits, the silica of which was in all reasonable probability derived partly from plankton vegetation; and there is no reason to suppose that the still more remote, possibly askeletal, ancestors of the rock-forming organisms existed under conditions differing from those that prevailed later when the skeletons were accumulating. It is doubtful whether volcanic phenomena will throw any light on problems concerning the synthesis of protoplasm, but it is certain that the slaggy and crystalline masses of marine fossils constituting existing volcanic and other igneous rocks do tell us unmistakably of the conditions under which those materials accumulated.

There is a grand simplicity about Nature's methods. The planetary crust apparently has been built out of sea-water and chiefly by protoplasm, and it is probable that protoplasm itself has been formed originally out of sun-lit sea water.

If the theory that life originated at the ocean surface is true, then the earliest form of life was probably of the ``independent'' kind, like that of green plants. Some naturalists believe, however, that animal life came first, and that its food supply consisted of simpler organic nitrogen compounds gradually evolved in the course of ages and culminating in the formation of living matter itself, or as Sir Ray Lankester tersely expresses it, the earliest forms of life fed on the products of their own antecedent evolution. There certainly appear to be strong reasons for this theory, in view of the enormous complexity of the proteins that form the basis of protoplasm. Even the simplest proteins, the protamines, have a molecular weight of over 1000, and a minimum estimate for haemoglobin is 16,600! Emil Fischer, making use of the most ingenious and laborious methods, so far has succeeded in building up a compound with a known molecular weight of 1213.

Nature has been called an expert geometrician, and a skillful mathematician, but above all she is a marvellous chemist, and beyond doubt she can accomplish rapidly and with the greatest ease syntheses at present wholly beyond the powers of man to achieve.

Carbon, oxygen, hydrogen and nitrogen (C.O.H.N.) constitute the pillars of the temple of life, and carbon may be said to be the great main central pillar.

The problem of the building up by green plants of carbohydrates out of inorganic materials seems to be fairly solved. The theory of von Baeyer that formaldehyde was an intermediate product between water-and-carbonic-acid and the starches and sugars is now definitely proved. F. L. Usher and J. H. Priestley\footnote{\emph{A Study of the Mechanism of Carbon Assimilation in Green Plants}. Proc. Roy. Soc. [B], 77, 1906, p. 369, and 78, 1907, p. 318.} have shown that in presence of sunlight and chlorophyll there is a ``photolysis'' of CO$_{2}$ resulting in the formation of formaldehyde (CH$_{2}$O) and hydrogen peroxide (H$_{2}$O$_{2}$). An enzyme splits up the latter into water and oxygen, and the formaldehyde becomes condensed into a carbohydrate in presence of living protoplasm. These processes take place so rapidly that the formaldehyde has only recently been detected in the living plant. Dr. S. B. Schryver,\footnote{\emph{Photochemical Formation of Formaldehyde in Green Plants}. Proc. Roy. Soc. [B] 82, 1910, p. 226.} by means of a new method --- the use of phenylhydrazine --- can detect formaldehyde in the proportion of one part in a million.

Both formaldehyde and hydrogen peroxide are very poisonous to plant life, but the transformation takes place so rapidly that in healthy plants no harm arises. Usher and Priestley state that starch can be detected in previously starchless filaments of \emph{Spirogyra after three minutes exposure to light}. CO$_{2}$ can be photolysed independently of chlorophyll. Usher and Priestley have obtained Formic acid from CO$_{2}$.H$_{2}$O, by using Uranium sulphate as an optical sensitizer. Again W. Löb has obtained formaldehyde by exposing a mixture of water vapour and carbon dioxide to the action of the silent electric discharge. (Zeitsch. Elektrochem. 1905-6.)

Willstäter has shown magnesium to be the essential metallic constituent of chlorophyll; and Fenton, by using a solution of carbon dioxide in water containing bars of magnesium, has observed the splitting up of CO$_{2}$ and the production of formaldehyde.\footnote{Journ. Chem. Soc. Trans., 1907, p. 691.}

Proteins are split up by digestion into simpler bodies --- the peptones, and the latter can be split (by hydrolysis) into the still simpler ``amino-acids.''\footnote{The amines are regarded as derived from ammonia by substitution of hydrocarbons for hydrogen, and the ``acids'' as derived from hydrocarbons by substitution of the carboxyl group CO.OH for hydrogen.} Most of the proteins contain twenty or more different amino-acids, all with complicated formulae. Emil Fischer has separated out many of these bodies, and, further, by synthesizing them has produced numerous substances which he terms ``polypeptides'' from their resemblance to the natural peptones. In spite of their complexity the amino-acids, the peptones and the proteins are all endless variations on one theme, \emph{viz.} NH.--CH.--CO.OH (\emph{i.e.} on C.O.H.N.).

If Nature has shown such marvellous ingenuity in the production of C.O.H. compounds from water and carbonic acid, why should it be supposed that her resourcefulness and rapidity of action stop short when the N element comes in?

Very little seems to be known concerning the mode of assimilation of the nitrogen of inorganic nitrogen compounds by green plants. As soon as links or simple chains of NH.--CH.--CO.OH bodies have arisen, it is conceivable that complicated chains would soon arise, for the simple chains have hungry links ready to grapple each other, the acid and amine links of one chain respectively with the amine and acid links of other chains. Emil Fischer has forged chains of eighteen amino-acid residues (an octo-deca peptide, \emph{viz.} leucyl-triglycyl-octoglycyl-glycin C$_{48}$H$_{50}$O$_{19}$N$_{18}$), and it is probable that bodies as complicated as the proteins will be produced artificially.

At the present day the warm ocean surface flooded with sunshine is at times like a living jelly. There is now a good deal of evidence in favour of the view that life was born in the sun-lit seawater\footnote{Plain water is now believed to be a highly complicated material formed of hydrone polyhydrones and hydronols, and is no longer represented by the simple formula H$_{2}$O, but by a bewildering series of symbols filling half an octavo page. (A dream of fair Hydrone. H.E.Armstrong \emph{Science Progress} 3., 1909.)} rich in magnesium chloride, carbonic acid and simple nitrogen compounds. Presumably the nascent light-sundered atoms and radicals would rapidly build up not only C.O.H. but Mg. C.O.H.N.\footnote{Willstäter contrasts the constructive or synthesizing ``life with magnesium'' with the destructive (abbauende) ``life with iron.'' Liebig's \emph{Annalen}, 350, p. 65, 1906. Also Schryver \emph{The Chemistry of Chlorophyll}, \emph{Science Progress} 3 (2) 1909, p. 425.} compounds, and a plankton heliobius would arise, and perhaps from the latter a benthos bathybius in the form of amoebulae. The latter would form calcareous shells, as do the young of \emph{Polystomella} at the present day; and the shells would become mixed up with silica skeletons as now happens.

\centerline{*\hspace{15mm}*\hspace{15mm}*\hspace{15mm}*\hspace{15mm}*}
\bigskip

In 1853 and later Pasteur was led by his researches on the behaviour of crystals and solutions of the tartaric acids and their salts towards polarized light, to formulate his theory of the molecular asymmetry of natural organic products.\footnote{Pasteur. \emph{Recherches sur la Dyssymétrie moléculaire des produits organiques naturels}. Paris, 1861. Rather rare, but the Alembic Club published a translation in 1897. See also a commentary on Pasteur's research in Prof. F. R. Japp's ``Stereochemistry and Vitalism.'' Brit. Assoc. Report, 1898, p. 813.} He pointed out that this behaviour could be explained by assuming that the atoms of the chemical molecules of optically active substances (\emph{e.g.} dextro- and laevo-tartaric acids) were arranged in a right or left spiral sense, and that therefore they rotated the plane of polarization; and that in optically inactive bodies such as racemic and meso-tartaric acids there were respectively inter- and intra-molecular blendings of right and left spiralities, and accordingly the plane of polarization was not rotated.

In 1858 Kekulé founded his ``Structur-theorie'' concerning the mode of linking of atoms, and showed it was possible to explain the constitution of organic compounds by assuming that Carbon had four units of affinity. In 1874 Le Bel and Van t'Hoff independently established the science of Stereo-chemistry which is concerned with the arrangement of atoms in space. The four directed attractive powers of Carbon were assumed to radiate from the center in more than one plane. If these powers acted along lines making equal angles with each other, the four attracted atoms or groups would be at the solid angles of a tetrahedron, the latter being regular or irregular according to whether the atoms or groups were equally or unequally attracted. If the four attracted atoms or groups are all different, the carbon atom is termed ``asymmetric,'' and any compound C R$_{1}$ R$_{2}$ R$_{3}$ R$_{4}$ could exist in two ``enantiomorphs'' (opposite forms). A continuous curve passing through the four attracted groups will form a screw-like spiral. The enantiomorphs are similar in all respects excepting that their crystals are, as with the right and left hand, non-superposable mirror-images of each other, and that their solutions are optically active in an opposite sense.

The finding of spirodiscoid structure in nummulites and in Rhizopodal protoplasm acquires significance from the point of view of the theories of Pasteur, of Kekule and of Le Bel and Van t'Hoff. Perhaps the most stupendous visible manifestation of spirality on this planet is shown by the planetary deposit of nummulites.\footnote{The prevailing spirality visible in the materials of the earth's crust is not a property of matter \emph{per se}, but a property impressed upon matter by life, for the crust is a deposit of mineralized nummulites.} Not only are the shells spiral, but they are built of spiral bricks; and now it is found that the protoplasm that built the shells and made the bricks, is itself visibly spiral. Natural organic products are optically active, \emph{i.e.} their molecules may be assumed to be built on a spiral plan, and seemingly the spirality seen under the microscope is a visible expression of the invisible spirality revealed by polarized light.

Concerning the cause of molecular asymmetry in natural organic products, Pasteur asks, ``Ces actions dissymétriques placées peut-être sous des influences cosmiques, resident elles dans la lumière, dans l'électricité, dans le magnétisme, dans la chaleur? ... Il n'est pas même possible aujourd'hui d'émettre à cet égard les moindres conjectures.'' Various lines of research appear to favour the suggestion that light has been the agency that may have brought about gyrate and active rather than racemate and inactive characters of living matter.

In 1850 Jamin\footnote{\emph{Comptes Rendus} 36, 1850, p. 696.} showed that a straight pencil of polarized rays when reflected from water took an elliptical course. Cotton\footnote{\emph{Ann. Chim. Phys.} (7) 8. p. 373, 1896.} found that circularly polarized light was affected differently and had a different effect on active and inactive salts of organic compounds (\emph{viz.} tartaric and racemic cupric ammonium salts). Byk\footnote{\emph{Zeitsch. physikal. Chemie.} 49, p. 641, 1904. Also A. W. Stewart, \emph{Science Progress} 2. 2, p. 449, 1908.} pointed out that wherever there is water, \emph{i.e.} over the greater area of the planet, there is circularly polarized light, and that this would have the effect of producing a unilateral (einseitig) fauna and flora.

Apparently sunlight falling on a universal ocean would produce a universal unilateral plankton heliobius, whence perhaps the universal unilateral benthos bathybius would arise. Accordingly, assuming the spirality of nummulites to be an expression of molecular spirality, the spirality of the nummulosphere may be due to the effects of circularly-polarized light on living matter. How reasonable to assume that a great planetary phenomenon such as the nummulosphere should be due to all-pervading cosmic influences!

If light, elliptically and circularly polarized has been the agency that has brought about asymmetry in living matter, then there would be an additional argument in favour of the theory that life originated at the surface of the ocean, and that the asymmetric bathybial life was derived from heliobial life.

\centerline{*\hspace{15mm}*\hspace{15mm}*\hspace{15mm}*\hspace{15mm}*}
\bigskip

Problems concerning the essential nature of existence are generally held to be insoluble owing to the limitations of the human intellect. The following quotation from H. Spencer expressing this view, will not, I hope, be wholly out of place in a chapter of scientific speculations on the origin of life:--

``Having, throughout life, constantly heard the charge of materialism made against those who ascribed the more involved phenomena to agencies like those which produce the simplest phenomena, most persons have acquired repugnance to such modes of interpretation. Such an attitude of mind, however, is significant not so much of reverence for the Unknown Cause, as of irreverence for those familiar forms in which the Unknown Cause is manifested to us. Men who have not risen above that vulgar conception which unites with matter the contemptuous epithets `gross' and `brute,' may naturally feel dismay at the proposal to reduce the phenomena of Life, of Mind, and of Society, to a level with those which they think so degraded. The course proposed does not imply a degradation of the so-called higher, but an elevation of the so-called lower.''

``May we not without hesitation affirm that a sincere recognition of the truth that our own and all other existence is a mystery absolutely and for ever beyond our comprehension, contains more of true religion than all the dogmatic theology ever written.'' (\emph{First Principles}, Herbert Spencer.)

Concerning matter and spirit the same philosopher writes ``The one is no less than the other to be regarded as but a sign of the Unknown Reality which underlies both.''

``The belief in an omnipresent and inscrutable Power is one which the most inexorable logic shows to be more profoundly true than any religion supposes.''
\clearpage
\section{Part 4 --- Lithomorphs}
\paragraph{}
Linnaeus wrote: ``Petrifacta non a calce, sed calx a petrifactis'' --- fossils are not produced by limestone, but limestone by fossils.

In the next two chapters certain supposed fossils will be described, which are veritably nothing but products of limestone. Some of these objects are so wonderfully fashioned, as regards shape, surface-pattern and interior ``structure,'' that it is not surprising they have long deceived palaeontologists.\footnote{At a time when there was much dispute concerning the true nature of fossils, a learned theologian uttered the warning that these bodies were not relics of formerly-living creatures, but objects fashioned by ``The Adversary'' and ``planted'' by him in the course of his numerous journeyings up and down in the earth, in order to deceive mankind. If so, the evil ``designs'' of this personage have had no small measure of success. This interesting theory, however, though partly correct, is untenable as regards the agency that fashioned the pseudomorphs, for here, apparently, the real chiseller was carbonic acid water.}

The shaping effect of chemical and physical forces acting on limestone is already well known, but the additional evidence now brought to light as to their marvellous constructive and pattern-making powers will come almost as a new revelation to science. That this statement is no exaggeration is shown by the voluminous records in palaeontological literature concerning supposed orders, families, genera and species of animals and plants that have never existed, founded as they were on pseudomorphic lumps of rock artistically fashioned by concretionary forces.

Some pseudomorphs mislead no one, or only the ignorant, but even trained observers have been deceived by others. Among the latter kind are the following:-- Stromatoporoids, \emph{Eozoön}, Spongiostromidae, Archaeocyathidae, Receptaculitidae, \emph{Loftusia}, \emph{Parkeria}, \emph{Cyclocrinus}, Syringosphaeridae, \emph{Girvanella}, \emph{Vermiporella}, \emph{Palaeoporella}, \emph{Saccammina carteri}, \emph{Mitcheldeania}, and \emph{Oldhamia}. Probably the list could be considerably extended.

Prof. H. Rauff, in a critical review of certain theories regarding the systematic position of the \emph{Receptaculitidae}, writes:\footnote{\emph{Verhand. Naturver. Preuss. Rheinl. Westph. Sitzungsb.} 1892. Vol. 49, p. 90.} ``Every attempt to prove definitely that the Receptaculitidae are Siphoneae (calcareous algae) becomes shattered on the at last exposed rocks. On the other hand we must admit that owing to a number of new elements of comparison a path (perspective) has been opened up, the exploration of which might lead to results not wholly unpromising or fruitless.'' These words by an eminent palaeontologist well express the feeling almost of despair concerning the solution of certain problems in palaeontology.

There exists, however, among the rocks a safe channel through which not merely one long-storm-buffeted ship, but a richly-freighted argosy, can be safely piloted into harbour. The ``Nummulitic Channel,'' for such is its name, has escaped discovery owing to being covered with flotsam and jetsam (osmotic patterns) carried by currents.

Leaving aside metaphor, and coming to facts, it will be found that the above-named pseudomorphs are lumps of nummulitic limestone. Training of the sense of sight will enable the observer to see the nearly, but not wholly, obliterated shell structure both in the lighter and in the darker parts of the patterns, and also continuity of structure passing right across dark and light pseudomorphic areas into the amorphous matrix of the embedded specimens.

We are dealing with concretionary patterns in ancient limestones, and happily these limestones are not homogeneous precipitates of inorganic origin, but masses of broken-down nummulite shells. Had it not been for this last fact, there would have been no clue whatever to the nature of these pseudomorphs, and it is fairly certain that disputes would have been going on for all time.

Precautions are necessary in the use of this simple and invaluable diagnostic sign. A fossil sea-urchin in the chalk often has its shell filled with nummulitic material. How are we to know that a Silurian \emph{Cyclocrinus}, for example, is not a hollow body filled in with clastic nummulitic material? The reply is that the nummulitic structure is continuous over the whole pattern, and into the matrix.

It is something of a paradox that one and the same research should lead to results apparently pointing in opposite directions. On the one hand these nummulosphere studies have led to the discovery of the organic origin of the igneous rocks, and on the other to that of the inorganic origin ot many supposed organic forms. Petrologists have concluded that the author has been deceived by mineral simulations of organic structure; but the statement that \emph{Receptaculites} was a pseudomorph drew from a palaeontologist the reply, ``You will never get me to believe that!''

Certainly patience and practice are necessary to enable the observer to detect and follow the nummulitic structure in the at first apparently structureless and confused mass of shining granules. A medium power of about 400 diameters is, I find, best for detecting successive coils of furrowed marginal cord with bases of septa passing across the ridges as if binding them together. Then there are the willow-pattern edges of the striated and pillared spiral lamina, and the disk structure; and, again, under very high powers, of 2000 to 3000 diameters, the characteristic disk structure. From much practice, I can now see more nummulitic structure in a few minutes than I could formerly see in as many hours.

The removal of a large number of forms from the organic kingdom to the ranks of ``concretions'' and ``pseudomorphs'' renders necessary an attempt to classify these groups of objects; but firstly it is desirable to give them a definite name. The supposed organic forms mentioned in the preceding sections were referred to as pseudomorphs, but when their real nature is known, many of them will cease to give the impression of resembling organic structures, and further, their kinship with objects never regarded as pseudomorphic will be recognized; moreover, science has not to do with fancied organic resemblances but with real similarities and dissimilarities. Accordingly ``pseudomorph'' is excluded.

The term ``concretion'' has strong claims, but is, I think, hardly satisfactory as a group-name for the numerous and diverse objects now to be classified. The fur deposited round some nucleus in a kettle is a concretion, and so also are the wonderful objects known as \emph{Cyclocrinus}, \emph{Parkeria} and \emph{Receptaculites}. For these and similar calcareous or originally calcareous bodies with a certain shape and structure due to the operation of purely chemico-physical agencies, the name ``lithomorph'' (``forms in stone'') is suggested.

The characters available for a classification of lithomorphs are shape and structure. The shape may be spherical or massive, the former including spherical and oval shapes, and the latter massive, hemispherical, conical and branched forms. The structure or pattern may be prevailingly concentric, or prevailingly radial, or lastly ``stromatic,'' \emph{i.e.} with ``layer upon layer'' rather than with ``layer within layer.'' By taking both shape and structure (or pattern) into account, lithomorphs may be arranged as follows:--
\begin{description}
    \item Group 1. Concentric. Structure predominantly concentric, or spiral; shape spherical or oval. Examples: \emph{Parkeria}, \emph{etc.}, \emph{Loftusia}, oolite\footnote{Oolite grains may have ``a concentric and radiate, or simply a concentric or radiate structure.'' Mem. Geol. Survey. 4. 1894. \emph{Lower Oolitic Rocks of England}, p. 6. All the figures show the completely and the predominantly concentric types. Some \emph{Parkeria}-patterns, too, might almost be classified as radial.} and pisolite, final stage of certain Fulwell concretions.

    \item Group 2. Radial. Structure predominantly radial, shape various.

Sub-group \emph{a}. Shape spherical, oval or cup-shaped; structure ``tubular''\footnote{The ``tubes'' are not hollow and never have been. There is only a tubular appearance or pattern.} or columnar, often with facetted surface: \emph{Cyclocrinus} \emph{etc.}, Receptaculitidae, Archaeocyathidae, Syringosphaeridae.

Sub-group \emph{b}. Shape nodular or doubtful; structure tufted-``tubular'' or flabelliform, sometimes facetted: \emph{Palaeoporella}, \emph{Vermiporella} \emph{etc.}, \emph{Ortonella}, \emph{Solenopora}, \emph{Mitcheldeania}, \emph{Oldhamia}.

Sub-group \emph{c}. Shape doubtful, structure vermiform-``tubular'': \emph{Girvanella}.

    \item Group 3. Stromatic. Shape often massive; with reticular pattern; usually with obscurely marked radial and tangential elements: Stromatoporoids, Spongiostromids, \emph{Eozoön}, many Fulwell concretions.
\end{description}
\paragraph{}
\emph{Remarks}. --- Of living things it is said ``crescunt et vivunt,'' and further, they reproduce themselves. Consequently there is a relationship between parent and offspring: ``lapides crescunt'' only, but for all that, some concretions are known to undergo a remarkable series of changes,\footnote{Mr. G. Abbott's mince-pie-shaped concretions in Fulwell quarry appear to begin as excentrically radiate masses of pillars, and to develop into bodies with concentric laminae. Further, there might be different stages in different parts of the same concretion, certainly a trying object to fit into any scheme of classification!} and, further, these bodies are formed in accordance with certain laws. Consequently it may be possible to arrive at a natural classification. Some petrologists speak of a ``natural history'' of igneous rocks and attempt to establish a natural system of classification.

The laws of rhythmical precipitation and arrangement of suspended particles have been much studied by Liesegang\footnote{R. Liesegang, \emph{Geologische Diffusionen}, 1914.} and Hatschek, and may throw light on concretionary processes.

One might conceive the idea of a hypothetical lithomorph-shape in the form of a permeable sphere of carbonate of lime suspended in a saturated solution of carbonate of lime in carbonic acid and water. Diminution of solvent power would lead to a precipitation of carbonate of lime, concentrically and radially, or concentrically or radially the patterns varying with the conditions.

A chronological arrangement brings out interesting features. \emph{Eozoön}\footnote{\emph{Eozoön} is doubtfully included as a lithomorph, for it is a lump of limestone with silicatic layers. The limestone itself is not lithomorphic.} lithomorphs are mostly Precambrian. Archaeocyathidae are characteristic of Cambrian rocks; Stromatoporoids of Ordovician, Silurian and Devonian, a few also (\emph{fide} Gürich and Dupont) occur in the Carboniferous; Receptaculitidae in the Silurian and Devonian chiefly; \emph{Cyclocrinus} Silurian; Spongiostromids in the Carboniferous, \emph{etc.} Further, palaeontologists find these lithomorphs to be useful indicators of horizons. The fact of their not being organisms will not destroy, though it may somewhat diminish, their value in these respects. Just as an art-expert knows the period of a work of art by its style, so a palaeontologist may learn much from the ``style'' of a lithomorph.

An attempt to arrange lithomorphs according to shape and pattern will at least afford an order of description for the present occasion, and may perhaps serve as a basis for future discussion.

The first of the two following chapters will be devoted to Stromatoporoids on account of the historical interest of these objects, and the second to a very brief account of the rest of the lithomorphs mentioned in the classification.
\clearpage
\subsection{Chapter 11}
\subsubsection{Stromatoporoids}
\paragraph{}
In some of the very ancient limestones there are curious objects varying in size from a few cubic inches to a cubic foot or more, of hemispherical, conical or massive shape, and with a layered structure. They are usually composed of carbonate of lime, but occasionally of silica.

Goldfuss, in his great folio \emph{Petrefacta Germaniae} (1826), was the first to describe and figure objects of this kind, which he found chiefly in the Devonian limestone of Gerolstein in the Eifel.

Examples from this neighbourhood often have alternating dense and porous concentric layers. (See Plate 3, copied from the original figure.) Goldfuss placed the specimens in a new genus and species, \emph{Stromatopora\footnote{\emph{I.e.} bodies with (dense and) porous layers.} concentrica}, which he ranked among the corals near \emph{Millepora}. This very pardonable mistake was the parent of a long line of errors.

Gradually as the Palaeozoic limestones of the Ordovician, Silurian and Devonian periods became better known, numerous other ``fossils,'' akin to \emph{Stromatopora}, were discovered. Succeeding writers devoted memoirs and monographs to the description of these objects, and many and diverse were the views as to their nature.

Below is an incomplete list, compiled as far as 1886 from Nicholson's monograph, giving the names of writers with their opinion concerning the systematic position of Stromatoporoids:\footnote{This name, first used by Nicholson and Murie, means ``like or akin to \emph{Stromatopora}.''}
\begin{itemize}
    \item Goldfuss, 1826. Hydrocorallinae.
    \item Steininger, 1834. Sponges.
    \item F. Roemer, 1843-4. Corals.
    \item Hall, 1847. Alcyonarians near \emph{Tubipora}.
    \item D'Orbigny, 1850-51. Sponges.
    \item The two Sandbergers, 1850-56. Polyzoa.
    \item F. Roemer, 1851-56. Polyzoa, but later tabulate corals like \emph{Favosites} and \emph{Chaetetes}.
    \item Billings, 1857. \emph{Beatricea} (a Stromatoporoid), a plant.
    \item Eichwald, 1860. Horny Sponges.
    \item Hyatt, 1865. Some Stromatoporoids regarded as Cephalopoda.
    \item Baron von Rosen, 1867. Horny Sponges.
    \item Dr. G. Lindström, 1870. Foraminifera, and, in 1873, \emph{Labechia} allied to Hydractinia.
    \item Salter, 1873. Calcareous Sponges.
    \item Nicholson, 1873-4. Calcareous Sponges.
    \item Dawson, 1875. Between Foraminifera and Sponges.
    \item Sollas, 1877. Hexactinellid Sponges, and, later, partly Siliceous Sponges, partly Hydrozoa.
    \item Carter, 1877. Hydrozoa.
    \item Nicholson, 1886 (the \emph{Monograph}). Partly Hydroida, partly Hydrocorallinae.
    \item Zittel, 1903. Hydrozoa.
    \item Geikie, 1903. Polyzoa.
    \item Steinmann, 1907. Hydrozoa.
    \item Kirkpatrick, August 1912: Sponges. September 1912: Foraminifera.
\end{itemize}
\paragraph{}
It is only necessary to refer here to the view advocated by Lindström and later by Carter, that Stromatoporoids are Hydrozoa. This opinion, adopted by Nicholson in his monograph (1886-1892), is the one held generally at the present day.

Apart from the layered structure, Stromatoporoids often differ but little in outward appearance from ordinary lumps of limestone. When, however, the specimen is polished, or, still better, when cut into thin transparent slices, a curious pattern in the form of a network of ``fibers'' becomes visible in the stony matrix, the fibers being arranged roughly in two sets, \emph{viz.}, (1) vertical or radial, and (2) horizontal or concentric.
\begin{figure}[H]
\centering
\includegraphics[height=95mm,keepaspectratio]{figures/Fig27.png}
\caption*{\centerline{\small \textsc{Figure 27 --- \emph{Stromatoporella eifeliensis}.}}

\centerline{\small A, horizontal, B, vertical section, showing network. 3x.}}
\end{figure}
\paragraph{}
Stromatoporoids present two main types of pattern. In one the fibers form a continuous network in which the radial and concentric sets are not sharply distinguishable from each other, the fibers, too, often being curved and irregular (curvilinear type of Carter): in the other type the vertical radial fibers are straight and sharply defined, and give off at fairly regular intervals horizontal radial spokes which unite with those from neighbouring vertical fibers to form series of horizontal floors perforated by triangular meshes (rectilinear type, Carter).

The fibers appear to be either solid or minutely porous and ``tubulated.''\footnote{M. Heinrich, in a recent memoir on Stromatoporoids, divides the group into solid-fibered and porous-fibered, and eliminates Labechiidae and Idiostromidae (Journ. Geol. p. 57, 1916). Transl.} The spaces between the fibers in the first group were supposed to contain polyps. Stromatoporoids were divided into two main groups, \emph{viz.}, those with a continuous network of curvilinear tubulated fibers, and those with the straight radial or vertical fibers and tiers of horizontal floors. The first group was supposed to be related to \emph{Millepora}, and the second to \emph{Hydractinia}, the zoophyte which encrusts Gastropod shells inhabited by Hermit Crabs. The basal layer of this zoophyte has fibers arranged somewhat as in the second type described above. Accordingly Nicholson considered Stromatoporoids to be Hydrozoa partly akin to \emph{Millepora}, partly to \emph{Hydractinia}. Sometimes in the second group the fibers of the horizontal floors are in the form of curved plates (\emph{Beatricea} and \emph{Labechia}), giving rise to a vesiculated appearance in sections. In the first group some of the genera are branched and not massive (\emph{Stackyodes}, \emph{Amphipora}).
\begin{figure}[H]
\centering
\includegraphics[height=95mm,keepaspectratio]{figures/Fig28.png}
\caption*{\centerline{\small \textsc{Figure 28 --- \emph{Clathrodictyon striatellum}.}}

\hspace{5mm} \small If page be held slanting and at a distance, coils of a broad spiral band of a shell will be dimly seen. 10x.}
\end{figure}
\paragraph{}
When first I examined sections of Stromatoporoids under high powers I observed series of ring-shaped bodies resembling certain kinds of sponge spicules, and thereupon concluded that the supposed fossils were sponges. Later, certain appearances led me to regard Stromatoporoids as Foraminifera united in colonies. At last the simple truth was discovered, \emph{viz.}, that these ``fossils'' were lumps of nummulitic limestone composed of nummulite shells of the ordinary type. This view has now been amply confirmed.

How has it come about that for ninety years many eminent experts have been so completely deceived? My own experiences in this research lead me to assign the cause partly to the tyranny of the fixed idea. At first sight this statement may seem paradoxical in view of the diversity of opinion concerning Stromatoporoids, which have been regarded as Protozoa, Sponges, Hydrozoa, Anthozoa, Mollusca, and even in one case, plants. The idea that Stromatoporoids were not fossils at all does not seem to have occurred to anyone. I, for my part, after many mistakes, hit upon the truth deductively. In view of the all-prevailing nummulitic nature of most of the limestones, it occurred to me that Stromatoporoids might be masses of nummulites segregated in a common nummulitic matrix. Innumerable observations have confirmed the truth of this deduction.
\begin{figure}[H]
\centering
\includegraphics[width=95mm,keepaspectratio]{figures/Fig29.png}
\caption*{\centerline{\small \textsc{Figure 29 --- \emph{Stromatopora concentrica}.}}

\centerline{\small Marked areas shown in photos on Plate 4. 18x.}}
\end{figure}
\paragraph{}
In addition to having investigated thoroughly specimens and sections of the unique Nicholson collection in the British Museum (Nat. Hist.), I have visited formations where Stromatoporoids abound, \emph{viz.}, Silurian limestones at Dudley and Much Wenlock, and Devonian limestones at Teignmouth beach and Darlington and at Gerolstein. At the Bonn Museum of Natural History I saw the type specimen of \emph{Stromatopora concentrica}.

Stromatoporoids have been mistaken for fossils on account of their Shape, Surface Markings, Layered Structure and Internal Pattern.

The Shape, which is commonly hemispherical, massive, conical or pinnacle-like, is often well-defined, resembling blocks of coral and especially \emph{Millepora}. It scarcely seems credible that some of these forms result from the operation of purely physical causes, yet such is the case. \emph{Labechia} is especially remarkable, for here there appears to be a definite under-surface marked with concentric rings and bands, and an upper tuberculated surface; also the branched forms present a very deceptive resemblance to coral-like growths.

I derived much instruction from the investigation of a fine collection of ``concretions'' exhibited in the Natural History Museum. The specimens, which came from the Permian limestone quarries of Fulwell near Sunderland, were collected by Mr. George Abbott, who has done great service to science by presenting sets of these wonderful objects to public institutions. While I was planning a visit to Fulwell, I received from Mr. Abbott an invitation, which I gratefully accepted, to go over the quarries under his guidance.

The limestone formation is about 200 feet thick and covers an area of two square miles. In many places and on a vast scale the formerly homogeneous mass of limestone has become sculptured into coralloid forms. Mr. Abbott\footnote{\emph{Discoid Limestones which simulate Organic Characters}, 1914.} has traced a definite ``evolutionary'' series of stages, from forms with pillars to solid masses with concentric bands.
\begin{figure}[H]
\centering
\includegraphics[width=95mm,keepaspectratio]{figures/Fig30.png}
\caption*{\centerline{\small \textsc{Figure 30 --- \emph{Labechia conferta}.}}

\centerline{\small (P. 5984. 264 f. N.H.M.) Showing ``vesicular'' structure. 18x.}}
\end{figure}
\paragraph{}
I was struck with the frequent resemblances between the Permian concretions and the Silurian and Devonian Stromatoporoids (Plate 5).

No one has thought of regarding the Permian bodies as fossils; there is almost as little reason for considering the earlier Palaeozoic bodies as such. Careful observation reveals the common nummulitic foundation of both.

The Surface of Stromatoporoids is often nodulated in a fairly regular way suggestive of organic growth; and further, certain meandrine markings and stellate groups of ridges and furrows resemble orifices of polyps or oscules of sponges.

The Layered Structure, to which the Stromatoporoids owe their name, is suggestive of zones of organic growth, but is due to the operation of physical causes.

The existence of the Pattern of so-called fibers, although it did not originate the fossil theory, has been the chief reason for the maintenance of the latter at the present day.

The fibers have been mistaken for the skeletal network of horny calcareous or siliceous sponges, for the skeleton of a large Foraminiferan, and for various types of coral skeleton. The fibers and clear spaces between them are simply darker and lighter or opaque and crystalline areas in a mass of nummulitic limestone. Careful and detailed observation shows the nummulitic structure throughout, and in continuity across light and dark spaces.

Concentric coils of spiral lamina crossed by radial alar prolongations, and also layers of spiral lamina are visible, especially in transverse sections of ``\emph{Actinostroma clathratum}.''

Fig. 31 shows a transverse section of one end of a shell in ``\emph{Stromatopora concentrica}''; here the striated and pillared structure of successive walls is seen, and under high powers the characteristic nummulitic disk structure. Strange to relate Nicholson referred to ``tubuli'' as possible evidence of Foraminiferal nature, but ended in comparing these appearances with the canaliculi of \emph{Millepora}. (See his \emph{Monograph}, footnote p. 66.)
\begin{figure}[H]
\centering
\includegraphics[width=95mm,keepaspectratio]{figures/Fig31.png}
\caption*{\centerline{\small \textsc{Figure 31 --- From slide of \emph{Stromatopora concentrica}.}}

\hspace{5mm} \small Part of nummulite in transverse section, showing willow pattern, the spiral lamina and pillars; 100x.}
\end{figure}
\paragraph{}
I can now see even with a hand-lens a good deal of nummulitic structure on weathered surfaces of specimens, but practice and patience are required here as well as in detecting that structure in sections.

The photographs on Plate 4 were originally taken simply to show the light and dark spaces, but with a lens it is possible to make out serial rows of disks. The pictures represent a very small part of a shell which, if only an inch in diameter, would cover about 50 square yards under a magnification of 260 diameters.

Certain Stromatoporoids are traversed by long crystal ``tubes'' or ``columns'' disposed at right angles to the concentric layers of the fossil. Nicholson, who calls these structures ``Caunoporatubes,'' writes concerning them: ``The singular fossils for which the generic names of \emph{Caunopora} and \emph{Diapora} have been proposed are known, to their cost, by all students of the Stromatoporoids. They have proved a fertile source of difference of opinion.''

According to one view, fossils with these ``tubes'' were distinct genera of Stromatoporoids. Prof. Roemer considered them to be due to the commensalism of Stromatoporoids with certain corals. According to a third opinion, the ``tubes'' were states or conditions of certain individuals. This last view, held by Nicholson himself, is the one that is nearest the truth. For the tubes are nothing but peculiar mineralizations taking place in certain masses of nummulitic limestone.

The nummulitic nature of the rock probably had much to do with the creation of this pattern. Series of ``pillars'' often blend to form clear columns in \emph{Nummulites} (Fig. 23 A). In a section of \emph{Stromatoporella bücheliensis} I can see alars and marginal cords in the Caunopora-tubes. At one time I mistook these structures for the expression of the segments of Annelids. Nicholson, again, refers to ``funnel-shaped tabulae'' in the tubes. These structures are, perhaps, series of minute nummulite septa.

The greater resistance of the tubes to weathering sometimes produces extraordinary results. A specimen of \emph{Stromatopora Hüpschii} (Nicholson's \emph{Monograph}, Plate 22, Fig. 7) somewhat resembles a mass of \emph{Tubipora musica}, the Organ-pipe Coral.

The patterns in the early Palaeozoic limestones result from the operation of chemical and physical causes, apparently the chief of these being the dissolving action of water and carbonic acid acting under certain conditions. These solvents affect all limestones, but not usually with the result of creating Stromatoporoids. Sections of chalk or of Fulwell limestone, for instance, are opaque and homogeneous. Why, then, should sections of portions of the older limestones show a mottled pattern of clear and opaque areas? The answer appears to be that the metamorphosing agencies have been acting on the older rocks for a much longer time, and, further, their action has been supplemented by crustal disturbances, and often by the neighbourhood of volcanic centers. Pressure, which, as Dr. J. Lehmann demonstrated, produces such powerful effects on metamorphic rocks, would also modify the structure of limestones.

Of late years many observations and experiments, especially by Hatschek and Liesegang, have demonstrated the tendency of particles thrown down in fluid or colloid menstrua to be \emph{rhythmically} precipitated, that is to say in concentric rings separated by clear zones.

Sometimes the patterns formed by these experimental rhythmical precipitations curiously resemble those of Stromatoporoids and kindred pseudomorphs.

If the water and carbonic acid permeating a limestone dissolve out and hold the carbonate of lime in saturated solution, any diminution of the carbonic acid would cause the lime to be precipitated in concentric zones as in oolite, or in radial-concentric patterns as in many Stromatoporoids.

In \emph{Beatricea} and \emph{Labechia} the wavy vesicular patterns give the impression of being due to forces acting in a rhythmical and undulatory fashion. In the former ``genus'' the shape is frequently that of a hollow cylinder, and the wavelets are in concentric series and convex outwards. (See Nicholson's \emph{Monograph}, Plate 8, Figs. 1-3.)

The agencies which have led to the formation of Stromatoporoids have acted not on homogeneous limestones but on hardened nummulitic deposits.

In Tertiary nummulites the pillars are more transparent than the rest of the shell, this being due apparently to differences in the organic structure. Further, in Stromatoporoid masses of nummulites, certain of the stronger portions of the shells are more opaque, especially at points where alar prolongations cross coils of marginal cord. Whether these nodal points offer greater resistance to leaching, or whether they capture more precipitated material it is difficult to say.

What with variations in composition, solvent power, temperature, direction and impinging force of leaching fluids on the one hand, and of size, thickness, and mode of arrangement of nummulite shells on the other, it is not surprising that several families and genera and numerous species of Stromatoporoids have been formed.\footnote{Personally I have the impression that the degrees of difference have been somewhat exaggerated; and there would be a tendency to do this in the endeavour to find good systematic characters. It is significant that Prof. Roemer (in 1844) ``arrived at the conclusion that almost all the species of Stromatoporoids described by former observers might be regarded as variations of a single type.'' (Nicholson, \emph{l.c.}, p. 7.)}

Summary. --- In the earlier Palaeozoic limestones or in their \emph{débris} are found lumps of rock known as Stromatoporoids, and generally supposed to be fossil Hydrozoa. The lumps have a banded structure, and when polished or cut into sections show a reticular or wavy pattern. The specimens are lumps of nummulitic limestone. Nummulitic structure can be seen, but not without difficulty, in weathered surfaces with the aid of a lens. Spiral coils of marginal cord, alar prolongations, and disk structures are visible in thin sections under higher magnification. The reticular and mottled patterns are the expression of clear and opaque areas in the rock, and are due to the operation of chemical and physical agencies. Stromatoporoids are lithomorphs.

Postscript. --- Perhaps the best and simplest method of detecting the nummulitic nature of Stromatoporoids and other lithomorphs would be to examine photographs (and negatives) made from large thin sections natural size or magnified only two or three diameters.
\clearpage
\subsection{Chapter 12}
\subsubsection{Certain other Lithomorphs}
\centerline{\emph{Parkeria} and \emph{Loftusia}}
\paragraph{}
In the Philosophical Transactions for 1869 Carpenter and Brady wrote a joint memoir entitled \emph{Description of \emph{Parkeria} and \emph{Loftusia}, two gigantic types of Arenaceous Foraminifera}, illustrated by nine plates, \emph{Parkeria} being described by Carpenter, and \emph{Loftusia} by Brady. Since the memoir was published, various other opinions have been held concerning these types, which have been regarded as porcellanous Foraminifera and as Hydrozoa.

The structure and pattern are so remarkable that it is not surprising their real nature has been misunderstood. It scarcely seems credible that the wonderful lace-like patterns seen in the sections could have been fashioned by purely physical agencies, yet such is the fact.

\bigskip
\centerline{\emph{Parkeria}}

In the Upper Cretaceous formation known as Cambridge Greensand there are found spherical stone balls about half an inch to nearly three inches in diameter, with smooth or slightly papillated surface (Plate 5A, Fig. D). The fractured surfaces of a broken ball often show no structure or very little, but polished sections or thin slices reveal a beautiful pattern of concentric circles and radial spokes in a translucent matrix (Plate 6). Between the circles and surrounding the radii there are labyrinthine networks of lace-like traceries. Sometimes in the center of the sphere there is a conical area with several horizontal platforms, or there may be \emph{débris} of sand and sponge spicules, or again no noticeable structure or material. The balls are commonly composed of carbonate of lime, the ``fibers'' being opaque or semi-crystalline, and the ``spaces'' transparent and crystalline. Sometimes both fibers and spaces may be made of phosphate of lime. Again, the fibers may be of phosphate of lime and the spaces empty, or the spaces may be filled with silica.

Dr. Carpenter concluded that these balls were specimens of a gigantic Sandy Foraminiferan, which he named \emph{Parkeria} after a distinguished colleague.

He believed that in the living animal the ``spaces'' were filled with protoplasm, and that the ``fibers'' constituted the skeleton, the latter being formed of calcareous sand particles picked up by the animal, and held together by a cement of carbonate and phosphate of lime. The conical central structure constituted the primary and initial chambers.

Carter, who took up the problem of \emph{Parkeria} in 1876,\footnote{A.M.N.H., 1877 (4), 19., p. 55.} considered the fossil to be a Hydrozoan near \emph{Hydractinia} and allied to the Stromatoporoids. For him the central structures when present were simply some foreign objects around which the Hydroid began to grow. The fibers were not built of foreign calcareous particles but were solid structures formed by the animal itself. In a succeeding paper he described the central conical construction as an area modified by the growth of a saprolegnious alga --- \emph{Millarella}. The ``lynx-eyed Carter'' certainly made some valuable and significant observations. He detected on the surface of \emph{Parkeria}, \emph{Loftusia} and of \emph{Stromatopora} groups of pores: see his Plate 8 Figs. 16, 18, 19, in Annals (4) 19. In one place\footnote{A.M.N.H., 1876 (4), 18., p. 209.} he even remarks ``I have alluded to the absence of the foraminated areas; but I think I can see one of these....'' Again he writes ``one of the chief characters of Foraminifera is their foraminated areas.''

Nicholson\footnote{A.M.N.H., 1888 (6), 1., p. 3.} adopted Carter's Hydrozoan theory. He figures the zoophyte \emph{Parkeria} growing round the supposed chambered shell of a young Ammonite (Annals \emph{l.c.} Plate 3, Fig. 6), just as Hydractinia invests a Gastropod.

Steinmann\footnote{\emph{Paläontologie}, 1907, p. 152.} and Zittel\footnote{\emph{Grundzüge}, 1910, p. 119.} placed \emph{Parkeria} among the Hydrozoa.

Numerous and very careful observations of specimens and sections have shown me beyond any doubt that specimens of \emph{Parkeria} are lithomorphic lumps of nummulitic limestone, with the nummulites almost obliterated, and with the lithomorph pattern in the form of concentric zones and radial spokes. I can actually see with the naked eye the faint circular and oval outlines of nummulites about an inch in diameter on the surface of a large specimen in the Natural History Museum. A lens (10x) shows some of the structure of the worn-down shells, \emph{viz.} edges of spiral lamina, radial alar prolongations, and traces of willow pattern. Under high powers, too, I can see series of nummulitic disk structures both in the ``fibers'' and in the ``spaces.'' \emph{Parkeria} then is a lithomorph of similar nature to an oolitic granule, but the latter is only a minute part of one shell, and the former a mass of many shells.

If I had believed in the fossil nature of \emph{Parkeria} and of Beatricea I would probably have made the former a synonym of the latter. The supposed central Ammonite or tabulated cone of \emph{Parkeria} may be compared with the hollow tabulated space in Beatricea; and the concentric wavy pattern of the latter is rather like that of the \emph{Parkeria} pattern, though less elaborate. It is significant too that A. Hyatt regarded Beatricea as a Cephalopod! Again, Carter describes a \emph{Parkeria nodosa}, and Billings a \emph{Beatricea nodulosa}.

Many circumstances conspired to deceive investigators, but especially the apparent existence of ``fibers'' and ``spaces''; and when specimens were found with empty spaces between fibers, and with spaces filled in with silica, the deception became complete.
\begin{figure}[H]
\centering
\includegraphics[width=85mm,keepaspectratio]{figures/Fig32.png}
\caption*{\centerline{\small \textsc{Figure 32 --- Fulwell concretion resembling \emph{Parkeria}.}}

\centerline{\small $\frac{2}{3}$ nat. size.}}
\end{figure}
\paragraph{}
The genesis of \emph{Parkeria} may have been somewhat as follows: a nummulitic limestone permeated by water charged with carbonic acid and super-saturated with carbonate of lime; lowering of solvent power, and deposition of the lime in concentric and radial precipitations over sundry areas; occasional dissolving out of the more soluble spaces between the less soluble fibers, the spaces being left empty or filled in with silica; and segregation of the hard spheres from the softer matrix.

In the Geological Survey Memoir, \emph{Cretaceous Rocks}, 1., 1900, \emph{Parkeria} is recorded as a Hydrozoan, and the Cambridge Greensand specimens as ``derived'' fossils. Even so, the spherical shape is due primarily rather to the concretionary nature of the objects than to the fact of their having been rolled about.

Green (? glauconitic) grains are common in \emph{Parkeria}, as also, and to a much greater extent, in the nummulite zone of the Lower Barton Beds.

There are Fulwell concretions very similar to \emph{Parkeria}, but without the finer traceries (Fig. 32). In both the Permian and Cretaceous lithomorphs the matrix is nummulitic.

\bigskip
\centerline{\emph{Loftusia}}

In 1855 W. K. Loftus, in a memoir on the geology of the Turko-Persian frontier, refers to certain spindle-shaped stony fossils about two to three inches long which he found embedded in a blue marly Eocene limestone in the Bakhtiyari mountains. Loftus believed the fossils to be gigantic Foraminifera allied to \emph{Alveolina}. In 1869 Brady\footnote{Phil. Trans. Roy. Soc., 1869, p 739.} investigated some of Mr. Loftus's specimens and concluded they were gigantic \emph{Sandy} Foraminifera comparable with the porcellanous \emph{Alveolina} on the one hand and vitreous \emph{Fusulina} on the other. On Plate 77, Fig. 1 (\emph{l.c.}), he figures several specimens embedded in a rich Foraminiferal limestone matrix.

Carter, who regarded \emph{Loftusia} as a Hydrozoan, thought the existence of the network structure of the central spaces was fatal to the Foraminiferal theory. Steinmann records \emph{Loftusia} as a Foraminiferan; and Zittel, as a Hydrozoan.

Brady compares \emph{Loftusia} to a loosely rolled scroll (of paper) drawn together at each end. The spiral space is subdivided along the length by septa passing obliquely from wall to wall, and a second set of septa pass at right angles from the first set to the main spiral wall. Accordingly there is a spiral labyrinthine pattern. Several small Foraminiferal shells seen in the sections were supposed to have been gathered up and incorporated by the animal.\footnote{G. M. Dawson (A.M.N.H. (5), 2., p. 344, 1878) describes a small species, \emph{L. columbiana} (8 x 5 mm.), from the Carboniferous limestone of Frazer River, B.C. Whether the supposed fossil is a Foraminiferan or a lithomorph I do not know.}
\begin{figure}[H]
\centering
\includegraphics[width=90mm,keepaspectratio]{figures/Fig33.png}
\caption*{\centerline{\small \textsc{Figure 33 --- \emph{Loftusia persica}.}}

\centerline{\small Half of section, 3x.}}
\end{figure}
\paragraph{}
As in the case of \emph{Parkeria}, these oval bodies are simply concretionary lumps of Foraminiferal, mainly nummulitic, limestone, with a lithomorphic pattern. It required long and patient observation to enable me to distinguish the nearly obliterated nummulitic structure but the latter exists throughout the lumps. The smaller Foraminifera are not particles collected by a gigantic species, but simply smaller species, existing in a Foraminiferal deposit composed chiefly of nummulites.

Among the Fulwell specimens exhibited in the Natural History Museum there is an oval concretion, but with large rounded ends and very probably without the pattern. Some of the Loftusias also have rounded ends.

\subsubsection{Oolite}
\paragraph{}
Limestone shows various kinds of texture, but none more remarkable than the oolitic, in which the rock resembles petrified fish roe, hence the name ``oolite'' or ``egg-stone'' (Fig. 34). Several theories have been put forward to account for this condition. Dana regarded oolite as granular coral mud, and Sorby thought that oolite grains were formed by deposition of calcite round drifting particles. Another theory attributes the formation of the granules to the presence of an alga. Dr. Rothpletz showed that certain concretionary granules in Salt Lake were formed by algae; and, indeed, there is often a curious structure in oolite termed \emph{Girvanella}, commonly supposed to be a calcareous alga. The difficulty about the reef-theory lies partly in the absence of any trace of coral structure in the grains, and partly in the non-existence of remains of real coral-reefs in the contemporaneous strata.

Without attempting in this brief note to examine critically the various theories concerning the origin of oolite rocks and oolitic structure, I shall give the results of my own observations on Jurassic oolitic rocks of the west of England. I paid five visits to oolite districts, \emph{viz.}, to Bath, Cheltenham and Portland, and collected samples from ancient churches in these localities and in London; and lastly I examined with lens and microscope sections of fresh and weathered rocks. As a result of these observations, I have made the surprising discovery that typical Jurassic oolites are nummulitic rocks formed of deposits of nummulite shells, and that the oolite grains are portions of these shells modified by concretionary action. A large nummulite shell an inch in diameter would contain or be made up of some hundreds of these concretions (Plate 13, Fig. A).

In some large sections taken from three sides of a cube of Portland oolite I can now without difficulty make out the concentric coils of furrowed cord and radiating septa and alars, sometimes with the concretions filling the chambers of the shell and modified in shape according to their position in the shell. In the marginal cord the grains run in parallel rows following the cord structure. It is clear that oolite grains are not concretions formed round sedimentary particles, but that they have been formed \emph{in situ}. The photograph (Plate 13, Fig. A) shows outlines of a shell. Even in cliché Fig. 34, obscure circular and radial shell-outlines are faintly perceptible.

Although I only discovered the fact at the very end of my investigation, I can now detect on weathered specimens of oolite the faint outlines of large nummulite shells in various aspects --- transverse, oblique or horizontal. The transverse section shows traces of willow-pattern and successive bands of spiral lamina, the parallel and divergent striae being replaced by rows of granules.
\begin{figure}[H]
\centering
\includegraphics[width=90mm,keepaspectratio]{figures/Fig34.png}
\caption*{\centerline{\small \textsc{Figure 34 --- Portland oolite.}}

\centerline{\small From photo of rough surface of piece of rock. 3x.}}
\end{figure}
\paragraph{}
A lens 3x will suffice to show the definite and by no means accidental grouping and orientation of the granules. The horizontal aspect of shells will show circular groups of granules corresponding to the tops of pillars. These observations are difficult, but I am convinced the facts are correctly stated. Large transparent sections viewed with a lens and by transmitted light will also show outlines of nummulite shells an inch or more in diameter.

I first got on to the nummulite track by discovering with a lens circular porous areas in weathered oolite from ancient abbeys (\emph{viz.} Westminster and Bath). I took the circles for small shells, but really they were septa and marginal cord or pillar and inter-pillar areas of spiral lamina of large shells. The weathering had partly dissolved out the masking concretionary pattern and left traces of original nummulitic structure.

The nucleus of each granule often consists of a little mass of nummulitic shell-material. At first I mistook sieve-like perforated disk structures of nummulites for Radiolaria. Concretions may form round any nucleus, such as a Radiolarian or a joint of a Crinoid, but in pure typical oolite I have found either no structure at all or a nucleus of nummulitic material.

The cause of this strange concretionary process in oolitic rocks has been put down either to mechanical or to organic agencies, but probably the true explanation is that of physico-chemical action in the solid rock. The sequence may have been as follows: soaking of the very porous nummulitic deposits by water and carbonic acid super-saturated with carbonate of lime derived partly from other strata, partly from the stratum about to undergo concretionary change; diminished solvent power followed by concentric deposition round the partly dissolved groups of disk-structures or in the spaces of the median plane of the nummulites.

Pisolite or Pea-Grit is usually regarded as a sedimentary formation. The concretions have a concentric structure. The \emph{Girvanella}-pattern is often present, and sometimes surrounds a Crinoid joint. I find abundance of nummulitic structure in sections of this rock.

Concretions are sometimes of large size as in the cannon-ball bed at Roker.

Lindström describes a bed of ``ball-stones'' in the upper Silurian of Gotland, some of which are over a yard in diameter. He refers to them as corals (\emph{Coenostroma discoideum}), and Nicholson calls them Stromatoporoids. Judging from sections named \emph{S. discoidea} in the Nicholson Collection they are huge concretionary masses of nummulitic limestone.

\subsubsection{\emph{Cyclocrinus}, \emph{etc.}}
\paragraph{}
In Silurian limestones both of the old and new world there are found little stalked spherical or pear-shaped bodies an inch or more in diameter and with a tessellated surface; each tile may have a pattern of spokes radiating from a central knob or circle. A vertical section shows a surface layer of small cup-like cavities each closed by its tile or lid; frequently, also, there are lance-shaped rods radiating excentrically from some point to the surface. Between the rods there are slender radial spaces ending in a ``pore'' in the floor of a surface cup. At one point on the surface of the fossil there is an appearance of an opening.

These wonderful and mysterious objects have been a continual source of bewilderment to palaeontologists.

In 1840 Eichwald described them as Crinoids. Since then they have been looked upon as Protozoa, Sponges, \emph{Receptaculites}, Anthozoa and Calcareous Algae. Dr. E. Stolley, who has done the most thorough work on these and similar Silurian forms, gives over sixty references to literature in his account of the genus.\footnote{\emph{Archiv. Anthropol. Geol. Schleswig-Holstein}, 1. p. 216. 1896.} After many failures, at last light appeared to be thrown on this obscure problem by researches in recent algology.

Some of the Siphonaceous algae are spherical or pear-shaped, with tessellated calcareous surface, and with a radial arrangement of densely packed branches. In 1893 (\emph{Ann. Jardin Botan. Buitenzorg}, vol. 11.) Solms-Laubach described a siphonaceous alga \emph{Bornetella oligospora}, with many apparent resemblances to the Silurian (or Ordovician) fossils. \emph{Bornetella} (\emph{Neomeris}) \emph{capitata} Agardh, again, is a stalked spherical form to which \emph{Cyclocrinus} bears resemblance. Accordingly recent authorities place this and kindred fossils among the calcareous algae.
\begin{figure}[H]
\centering
\includegraphics[width=95mm,keepaspectratio]{figures/Fig35.png}
\caption*{\centerline{\small \textsc{Figure 35 --- Some remarkable lithomorphs.}}

\hspace{5mm} \small \emph{a}, \emph{Cyclocrinus}, 4x; \emph{b}, section of Coelosphaeridium, 2x; \emph{c}, section of \emph{Cyclocrinus}, 2.5x; \emph{d}, lids of \emph{Cyclocrinus} 8x; \emph{e}, ditto, another pattern, 12.5x; \emph{f}, Silurian limestone of Estland, showing sections of three specimens in contact, without radials (N.H.M., Bather), nat. size; \emph{g}, \emph{h}, \emph{Ischadites murchisoni}, rough copy after Rauff; \emph{i}, \emph{Parkeria}, with conical axial body; \emph{j}, Fulwell concretion, section showing central concentric and peripheral radial-tubular structure, ½x, after G. Abbott. Figs. \emph{a}--\emph{e}, after Dr. Stolley.}
\end{figure}
\paragraph{}
Sections of \emph{Cyclocrinus} from the Ordovician of Estland, East Baltic, show these fossils to be spherical facetted lumps of nummulitic limestone, with the nummulitic structure in continuity throughout the whole mass. Further, in a large section of a boulder from the same locality (5. 3960, N.H.M., Coll. F. A. Bather) there are many \emph{Cyclocrinus} sections appearing like circles with merely a rim of cups and without radial rods.

Sometimes adjoining circles in contact are incomplete, one pressing into the other. Sections show the nummulitic structure passing continuously from the circles to the embedding matrix.

\emph{Cyclocrinus} is a concretionary lithomorph and owes its structure to the action of physical causes. A comparison with other lithomorphs such as \emph{Receptaculites}, \emph{Parkeria}, Fulwell concretions, \emph{etc.}, will, I think, explain the course of ``evolution'' of this marvellous lithomorph. In some of the Fulwell concretions, as Mr. G. Abbott has shown, there is an initial stage with pillars radiating from an excentric point. Sometimes the densely-packed radial pillars have radial spaces between them, the pillars themselves being as walls to those spaces, so that an appearance of radial tubes arises. Further in the apparent course of evolution of certain mince-pie-shaped concretions, Mr. Abbott found that the pillars \emph{disappeared} and that they were replaced by concentric laminae. In \emph{Cyclocrinus} there are densely-packed lance-shaped rods with fine radial spaces between. The ends of the radii surround the ends of the spaces, which are usually hemispherical. Sometimes the radial spaces open out gradually as in ``Coelosphaeridium,'' but in \emph{Cyclocrinus} the spaces open abruptly, by means of a circular pore, into the floor of a hemispherical cup; or rather it should be said that the radial linear spaces expand suddenly into hemispherical spaces.

The areas which cover the spaces at the surface become differentiated out as hexagonal lids, on each of which patterns arise.

The stalk or stem, leading into the interior excentric area whence the rods radiate out, is a common feature in lithomorphs.

I have spoken of cups and cavities and canals, but really the lithomorph is in its origin solid throughout, the supposed cavities being merely clear areas of calcite. It is this idea of cavities that has been the chief cause of deception. It is true that variation in hardness and solubility may lead to the clear areas being dissolved away, thus leaving real cavities, and these again may become filled with some foreign material. When the results which Nature can achieve by purely physical means are so extraordinary, it is not surprising that the literature of palaeontology is crowded with descriptions of concretions mistaken for animal and plant remains.

The discovery of the truth has been rendered possible owing to the persistence of nummulitic structure in the lithomorphs and in the matrix out of which they have become segregated by the action of concretionary forces. Further help has been afforded by a comparative study of the various types of lithomorphs and of the changes they appear to undergo.

\subsubsection{Receptaculitidae}
\paragraph{}
Prof. H. Rauff, in the preface of his memoir on the Receptaculitidae,\footnote{\emph{Untersuchungen Organisation u. systematische Stellung d. Receptaculitiden}. Abhand. Akad. Wiss. München, Bd. 17. p. 645. 1892.} writes: ``With respect to the problem of the true nature of the Receptaculitids my researches end only in the sad result that these interesting bodies will once more be cast out of `The System,' again to wander around without shelter.''
\begin{figure}[H]
\centering
\includegraphics[width=95mm,keepaspectratio]{figures/Fig36.png}
\caption*{\centerline{\small \textsc{Figure 36 --- \emph{Receptaculites reticulatus}.}}

\centerline{\small Silurian, Niagara group. (P. 6514, N.H.M.) $\frac{4}{3}$x.}}
\end{figure}
\paragraph{}
In the earlier chapters of the book of life there are mysterious signs and characters to the meaning of which there appears to be no satisfactory clue at least if we may judge from the fact that a dozen experts may have come to as many different conclusions concerning them.

I am certain that at last the true explanation of the \emph{Receptaculites} has now been found, and it is one which seems to me more interesting than any of the hypotheses hitherto put forward. Apparently the real truth has never even been suspected, because, in the absence of clear evidence, it would have seemed too improbable to have been thought of.
\begin{figure}[H]
\centering
\includegraphics[width=95mm,keepaspectratio]{figures/Fig37.png}
\caption*{\centerline{\small \textsc{Figure 37 --- Ischadites, Wenlock limestone.}}

\centerline{\small (P. 4232, N.H.M.) $\frac{4}{3}$x.}}
\end{figure}
\paragraph{}
\emph{Receptaculites} and kindred objects are typically spheroidal or pear-shaped bodies from one to several inches in diameter or length, with a facetted or geometrically-patterned surface. They have the appearance of being hollow bodies filled in with the matrix of the surrounding rock. Prof. Rauff regards cup-, dish- or plate-shaped forms as imperfect examples of completely spheroidal ones. A fracture or section shows the wall of the sphere to be built up of curiously shaped ``elements'' or blocks fitted against each other with their long axis radial.

Each ``element'' typically has six parts, \emph{viz.} (1) an outer diamond-shaped facet, beneath which lie (2-5) four tangential arms, and (6) a fifth long radial arm or column projecting inwards and expanding into a foot, which along with other ``feet'' helps to form the inner surface of the wall of the hollow sphere. Between the radials or columns there are ``spaces.'' One ot the four tangential arms is fused to the facet. The radial arm is hollow. There is an under and an upper pole marked by areas of peculiar facets. \emph{Receptaculites} are found in ancient rocks from Lower Silurian to Carboniferous. Rauff showed that their original composition was calcareous, though some may have become silicified subsequently.
\begin{figure}[H]
\centering
\includegraphics[width=95mm,keepaspectratio]{figures/Fig38.png}
\caption*{\centerline{\small \textsc{Figure 38 --- \emph{Receptaculites neptuni} 4x.}}

\centerline{\small Near Gerolstein. Author's collection.}}
\end{figure}
\paragraph{}
\emph{Receptaculites} have been regarded as fossilized fir-cones; corals; Crinoids; Foraminifera; calcareous algae (Siphoneae verticillatae) related to the Dactyloporidae, or to such Dasycladaceae as \emph{Bornetella}; primitive types of Sponges; gemmules of Freshwater Sponges; peculiar Hexactinellid Sponges; Tunicata.

Prof. Rauff himself concludes as follows (\emph{l.c.} p. 717): ``For the moment we must be content with an `Ignoramus,' and can only say that the \emph{Receptaculitidae} form a peculiar family which after their death failed to leave behind any similarly organized representatives either in the ancient or modern world.''
\begin{figure}[H]
\centering
\includegraphics[width=95mm,keepaspectratio]{figures/Fig39.png}
\caption*{\centerline{\small \textsc{Figure 39 --- \emph{R. neptuni} 3.5x.}}

\centerline{\small Author's collection. See photo from this, Plate 2 E 22, and Fig. 40.}}
\end{figure}
\paragraph{}
The results of my own research can be summed up in a few words.

I find \emph{Receptaculites} to be lithomorphs --- for the moment they may be called pseudomorphs. They are forms of limestone which have become differentiated out of the matrix in which they are embedded, by the action of chemical and physical forces. How can this be known? By the simple fact that the limestone of which these bodies are composed is \emph{nummulitic} limestone, which may occasionally be silicified. The patterns, the spaces and the interior of the globular bodies are all composed solely of nummulitic material which can be made out when very carefully examined in thin slices (Fig. 40, and photo Plate 2, E 22), or even in the mass with a hand-lens.

How can the peculiar shape and structure of \emph{Receptaculites} be accounted for?
\begin{figure}[H]
\centering
\includegraphics[height=95mm,keepaspectratio]{figures/Fig40.png}
\caption*{\centerline{\small \textsc{Figure 40 --- \emph{Receptaculites neptuni}.}}

\centerline{\small Transverse section of nummulite 250x; from section shown in Fig. 39.}}
\end{figure}
\paragraph{}
Just as a comparison of living organisms with each other helps us to arrive at a knowledge of their mutual relationships and of the manner in which they have evolved, so with lithomorphs.

In lithomorphs there appear to be two dominant types of pattern, \emph{viz.}, concentric and radial. In the various stages of development of certain lithomorphs the ``evolution'' of which is known, the pattern may be predominantly radial at one stage, and predominantly concentric at another; and, further, both types may be present at the same time in one and the same specimen.

Mr. G. Abbott after a twelve years' search found certain mince-pie-shaped or biconvex concretions \emph{in situ} in the rocks at Fulwell. He found that these bodies were formed in the first stage of pillars radiating from some excentric point to the surface. In the second stage the pillars became regularly and serially nodular, in the third the nodules joined laterally to form concentric bands with spaces still between, and lastly the spaces were filled up, and a solid mass with a finely concentric structure resulted. Sometimes concretions exhibited partly the radial, partly the concentric type. This interesting discovery teaches us there may be an ``evolution'' in the formation of lithomorphs.

Another discovery made by Mr. Abbott was that of the fusion of the pillars of pillared forms in such a way as to form tubes.

Turning now to the concretions of the palaeozoic rocks older than the Permian, we find a much more finished type of ``art.'' Were it not for the fact that \emph{Coelosphaeridium}, \emph{Cyclocrinus} and \emph{Receptaculites} resembled the Permian concretions in being wholly composed of nummulitic limestone, we could never have suspected that these bodies have evolved from lithomorphs with simple radial and concentric patterns. I must refer once more to one or two details mentioned under \emph{Cyclocrinus}. In the spherical Coelosphaeridium the radial pillars are always present, and the spaces between the pillars expand gradually up to the surface of the sphere. In \emph{Cyclocrinus} the radial pillars are frequently absent and there remains only a thin surface zone of cup-like patterns, each ``cup'' having a ``lid'' often of marvellous design and, still more strange, apparently a ``pore'' in the floor. In some examples of \emph{Cyclocrinus}, however, the radial pattern exists, or rather I will venture to say, persists. The space between the radial pillars is linear or thread-like and of uniform diameter, till it arrives at the floor of the ``cup'' into the cavity of which it suddenly expands.

Coming to the ``Receptaculitidae,'' in the \emph{Receptaculites}-type of receptaculite the supposed infilling matrix in the assumed hollow interior is simply a part of the lithomorph without visible structure.

The ``wall'' is a peripheral zone which has retained some of the radial and concentric structure as in certain examples of \emph{Cyclocrinus}.

In the \emph{Polygonosphaerites}-type of receptaculite the radial element is wanting. In the Ischadites type the wall is very thick and the central ``cavity'' relatively small. The center whence the radial pillars radiate in \emph{Receptaculites} and kindred bodies is usually excentric.

Frequently these ancient lithomorphs are stalked. The continuation of the homogeneous stalk structure into the interior point whence radii originate, may be compared with a variety of ``phragmacone'' in \emph{Parkeria} (Fig. 35 \emph{i}).

G. Abbott finds the surface of the biconvex Permian concretions rough and tuberculated in the first stage. A Permian body exhibited in the Natural History Museum has a facetted surface, but somewhat irregular, it is true.

It must be remembered that in the first instance there are no ``spaces'' in a lithomorph, but simply a solid lump of rock with a pattern. Later, cavities may arise by solution and removal of more soluble elements.

The Permian lithomorphs are to the Silurian as the works of art of the craftsman period are to those of the great period of any school of art.

The ``evolution'' of complex inorganic structures such as \emph{Cyclocrinus} suggests a comparison with processes of organic evolution. A homogeneous lump of limestone undergoes a change into a mass of excentrically radiate pillars crowded together. The central and peripheral parts become differentiated. The peripheral parts undergo further differentiation both radially and concentrically, and the outer concentric plates themselves develop radial patterns, knobs, \emph{etc.}, each variation as it arises becoming a starting point for further variations, just as in organic life.

\subsubsection{Archaeocyathinae}
\paragraph{}
In the most ancient palaeozoic limestones, \emph{viz.} the Cambrian, remarkable cup- or vase-shaped objects of an unknown nature are found. Like the Cambrian limestones themselves they are practically of world-wide distribution.

Billings in 1861 was the first to describe one of these forms which he named \emph{Archaeocyathus}.

In 1910 Dr. J. Griffith Taylor\footnote{\emph{The Archaeocyathinae of the Cambrian of South Australia}. Mem. Roy. Soc. South Australia, vol. 2., 1910.} published a beautiful memoir illustrated with ninety-four collotype and fifty text figures, in which reference is made to no less than eighty species of these ``fossils.'' Dr. Taylor's material came from a formation extending 400 miles north and south in the Adelaide district, the fossiliferous strata being about 800 feet thick. The formation has been compared with a great barrier reef. Dr. Taylor records the singular fact that with very rare exceptions there were no other fossils than Archaeocyathids.\footnote{One recalls Carpenter's description of tertiary nummulitic limestones composed of ``little else than \emph{Nummulites}.''}

The interval between the inner and outer surface of the typically stalked cup is divided by vertical radial septa and horizontal tabulae into cubical spaces. Sometimes the inner and outer wall, and the septa and tabulae are all perforated by pores.

Archaeocyathids have been regarded as Calcareous Sponges, Hexactinellid Sponges, a new class of Sponges near Calcarea (Dr. Taylor), Corals of various kinds, and as Calcareous Algae. The porous inner walls and certain other structures recall Sycon sponges. The radial septa on the other hand, suggest coral structure, Dr. Bornemann, who seemed to incline to the coral theory, thought the pores might be for the emission of whip-like filaments.

I had the good fortune to have access to some of Dr. Taylor's South Australian material. A section of a typical mass of Archaeocyathid limestone, including outer and inner walls and septa of Archaeocyathids, showed the material to be nummulitic throughout, the supposed fossils being merely concretionary or lithomorphic patterns in the common mass. Along with considerable differences, there are certain resemblances between Silurian Receptaculitid and Cambrian Archaeocyathid lithomorphs. The perforations might be compared with the patterns of circles on the cuplids of certain ``species'' of \emph{Cyclocrinus} (Fig. 35 \emph{e});\footnote{Dr. Stolley, \emph{Archiv. Anthrop. Geol. Schleswig-Holstein}, 1. pp. 196-7, figs. 14-16. 1896.} for along with differentiation in pattern there is often difference in degree of solubility, so that circles might easily become pores.

\subsubsection{Syringosphaeridae or Karakoram Stones}
\paragraph{}
In a report on the results of the Second Yarkand Mission (1879), Prof. M. Duncan described certain large oval or spherical stone balls collected from Tertiary limestones in the Karakoram Mountains. The balls are an inch or two in diameter and often have a nodulated surface. On section they reveal a radiating pipe-like pattern.

At first they were regarded as Crinoids, corals, or Foraminifera. Prof. M. Duncan placed them in ``a new order of Rhizopoda called the \emph{Syringosphaeridae}.'' I have examined specimens and sections of those objects and find them to be concretionary lumps of nummulitic limestone. The lithomorphic pattern is radial-``tubular,'' although there are no \emph{hollow} tubes. Certain other forms apparently allied Syringosphaeridae are \emph{Ellipsactinia} Steinmann, \emph{Sphaeractinia} Steinm., and Heterastridium Reuss., but I have not seen sections under the microscope and am not certain.

\subsubsection{Solenopora}
\paragraph{}
In 1878 Dybowski\footnote{\emph{Die Chaetetiden ostbaltischen Silur-Formation}, plate 2. fig. 2 a, b.} established a genus \emph{Solenopora} for certain small spheroidal fossils about a cubic inch in size occurring in the Silurian of the East Baltic. Sections showed the lumps to be formed of extremely fine apparently parallel-tubular structures. Dybowski believed the tubes contained polyps, and that the fossils were corals. Nicholson\footnote{\emph{Geol. Mag.} (3), vol. 2. p. 529, 1885, and vol. 5. p. 19, 1888.} found similar fossils in the Ordovician limestone at Craighead, Ayrshire, and regarded \emph{Solenopora} as a Hydrozoan. Soon Dybowski's coral theory became discredited, for seemingly 150 polyps would have only a square millimetre of area in which to perform their evolutions. A much more plausible theory was adopted to the effect that the fossil was a calcareous alga related to \emph{Lithothamnion}. In 1894 Dr. A. Brown\footnote{\emph{Geol. Mag.} (4.), vol. 1. p. 145. 1894.} wrote a paper on the structure and affinities of \emph{Solenopora}. His plate giving figures of sections of \emph{Lithothamnion} and \emph{Solenopora} certainly reveals striking superficial resemblances.

Dr. Rothpletz\footnote{\emph{Kalkalgen Obersilur Gottlands}, p. 7. 1913.} also has put the genus among the Coralline algae.
\begin{figure}[H]
\centering
\includegraphics[width=95mm,keepaspectratio]{figures/Fig41.png}
\caption*{\centerline{\small \textsc{Figure 41 --- \emph{Solenopora compacta} 10x.}}

\hspace{5mm} \small From Ordoxvician limestone, Craighead (R. 475, N.H.M.). Nummulitic pattern faintly visible over whole field beneath dominant patterns.}
\end{figure}
\paragraph{}
My own observations are based on sections of \emph{Solenopora compacta}, Billings, from the Ordovician limestone of Craighead, Ayrshire. At first sight nothing could seem more reasonable than the coralline-alga theory. I have seen a thick bed of corallines of Miocene age in the islet of Cima near Porto Santo, with the algal structure very well preserved; and many species of Tertiary and Cretaceous Lithothamnions are known. The ``cells'' of \emph{Solenopora} though extremely large in comparison with those of known coralline algae yet might pass for a large variety of such cells.

When I came to examine sections of the Ayrshire Solenopora critically and with extreme care, I not only found nummulitic structure throughout, but saw that the supposed cells could not possibly be of the nature of algal cells. For as Dybowski pointed out, the longitudinal outlines are very distinctly wavy. Further, under high powers it will be found that there is no structure in these outlines \emph{qua} outlines; they are simply dark zones, and nummulitic disk structures may be partly in the dark zone and partly in the adjoining more transparent areas. The supposed ends of the algal cells (``tabulae'' on the coral theory) are not infrequently arranged in fine concentric zones, though often the fine lines appear to be broken up into bars, each bar being confined to a separate tube. Accordingly in some places there is a network pattern of longitudinal and transverse lines, the former being by far the more predominant (compare Fig. 28 of \emph{Clathrodictyon}). I am convinced that the Craighead \emph{Solenopora} is a lithomorph.

As for the many other species, also H. Yabe's\footnote{\emph{Science Reports, Tohuku Imp. Univ.}, ser. 2, Bd. 1. 1912.} new Japanese genus Metasolenopora, I think it probable, judging from the photographs (\emph{vide} memoirs by Rothpletz and Yabe), that these forms also are lithomorphs.

The lithomorph pattern is so strong and the nummulitic structure so faint that much patience is needed for the detection of the latter. I am fairly certain I can make out nummulitic details both in the lower and upper half of Dr. Rothpletz' photo of \emph{Solenopora gotlandica} (Plate 1, Fig. 3, \emph{l.c.}). Fig. 2 when examined with a lens 10x shows disk-structures and a curious pattern of fine closely-set transverse lines other than the main zonal lines.

In some of the sections of Girvan limestone perhaps two genera of ``calcareous algae'' and also stromatoporoid patterns may be found in a very small area (\emph{e.g.} \emph{Girvanella}, Solenopora, and Beatricea, or Labechia-like vesicular pattern). In such a section there is nothing but a common basis of nummulitic limestone in which is woven a wonderful variety of osmotic patterns.

\subsubsection{Palaeoporella, Vermiporella, \emph{etc.}}
\paragraph{}
Dr. E. Stolley\footnote{E. Stolley, \emph{Ueber silurische Siphoneen}, Neues Jahrb., 1893, 2., p. 135.} describes certain funnel- or club-shaped calcareous bodies of Silurian age, varying from 2 to 14 mm. in length, found by him in the diluvium of East Holstein. The objects may be hollow, with a pore below and a depression at the upper end, and with a facetted surface. He regards them as calcareous algae related to Bornetella (Siphoneae Verticillatae).

Thanks to the courtesy of Dr. Stolley I have some authentic pieces of limestone rich in these objects. In my sections I can find nothing else than nummulitic structure throughout, both in clear and opaque areas. I am led, then, to regard these curious objects and markings as purely concretionary. Among this group I would place the Silurian \emph{Rhabdoporella}; and also \emph{Ortonella} described by Prof. Garwood from carboniferous limestone. I have not seen examples of these two forms.

\subsubsection{\emph{Mitcheldeania}}
\paragraph{}
This form was first described from the Carboniferous marly limestone at Mitcheldean by E. Wethered, who writes (Geol. Mag. 1886, p. 535): ``The organism consists of a series of concentrically arranged layers or laminae penetrated by systems of tubuli... The tubuli are separated by the skeleton fiber, which is itself penetrated by a minute canal system.'' Mr. Wethered refers the genus to Stromatoporoids; and records the opinions of experts who pointed out to him the resemblances to \emph{Parkeria} and \emph{Girvanella}.

I collected specimens of marly limestone from the quarry at Mitcheldean whence Mr. Wethered got his material, and have examined a microscopic slide of \emph{Mitcheldeania nicholsoni} Wethered, presented by Mr. Wethered himself.

The limestone shale is mainly nummulitic, and \emph{Mitcheldeania} is a lithomorphic pattern comparable with that of \emph{Girvanella} and \emph{Syringosphaeria}.

\emph{Mitcheldeania gregaria} Nicholson (Geol. Mag. 1888, p. 17), from Carboniferous limestone at Kershope Foot, Roxburgh, is likewise nummulitic, as is evident from samples which I obtained from the locality.

See also some interesting notes on \emph{Mitcheldeania} by Prof. Garwood, who regards this form as a calcareous alga (Geol. Mag. 1913, p. 546).

\subsubsection{\emph{Oldhamia}}
\paragraph{}
The study of a photograph published by Ch. Barrois,\footnote{Ch. Barrois, \emph{Note sur l'existence du Genre \emph{Oldhamia} dans les Pyrénées}. Annal. Soc. Geol. Nord, 15. p. 154. 1888. An account of the various theories concerning this form, and also many references to literature, are given.} of \emph{Oldhamia} from the Pyrenees, leads me to believe that the structure therein depicted is nummulitic. The furrowed and banded marginal cords and fan-like banded septa astride of them would, when compressed, tend to form serial fan-like ribbed structures.

I firmly believe that the above suggestions will furnish the true explanation of the \emph{Oldhamia} figured by Barrois, and probably of all other Oldhamias. Sollas showed that these structures in the hardened Cambrian mud were definite and solid, and not mere ripple-markings.

\subsubsection{\emph{Girvanella}, \emph{etc.}}
\paragraph{}
In 1880 Nicholson and Etheridge\footnote{\emph{Silurian Fossils of Girvan}, 1. 1878-80.} discovered in the ancient limestones of Girvan in Ayrshire curious little irregular brown or grey calcareous patches about 2 mm. in diameter, and with a very remarkable structure. Sections examined under the microscope revealed a dense labyrinthine mass of coiled solid ``tubes,'' each about 0.04 mm. (0.0016 inch) in diameter. (Photo, Plate 1, Figs. 4, 5.)

Brady believed these bodies were allied to the sandy Foraminiferan \emph{Hyperammina vagans} which consists of a plexus of coiled tubules. Nicholson partly adopted this view, but thought it well to place the fossils in a new genus and species, \emph{Girvanella problematica}. \emph{Girvanella} has since been found in many palaeozoic and mesozoic limestones, and Rhumbler places even certain living Foraminifera in the genus. In certain Silurian strata in the Baltic region, thick layers of limestones are mainly composed of masses or ``knolls'' of \emph{Girvanella} each perhaps several centimeters in diameter.

Many and varied opinions have been published in the now considerable literature on this subject. \emph{Girvanella} has been regarded as Foraminiferan, as masses of worm tubes, as a Blue-Green Alga, as a Sponge, Hydrozoan (Stromatoporoid), as a calcareous alga related to the Codiaceae. At the present time, opinion is divided between the calcareous-algal and the Foraminiferal theories.

Owing to the kindness of Prof. A. W. Gibb of the University of Aberdeen, I had the good fortune to obtain some of Dr. Nicholson's own Girvan material, from which I was allowed to cut sections. These show abundance of ``\emph{Girvanella problematica}.''

\emph{Girvanella} is beyond any shadow of doubt a pattern in nummulitic limestone. With careful observation it is possible to detect the nummulitic structure beneath and in the pattern and continuing into the surrounding clear part of the limestone. One feature can soon be made out under a power of about 400 diameters, \emph{viz.}, a finely punctate appearance due to the disk-structure, each dark point being surrounded by a circle. The successive ``disks,'' furrowed marginal cord, septa, alar prolongations, \emph{etc.}, all require great patience for their detection, but they exist.
\begin{figure}[H]
\centering
\includegraphics[width=95mm,keepaspectratio]{figures/Fig42.png}
\caption*{\centerline{\small \textsc{Figure 42 --- \emph{Girvanella problematica}.}}

\hspace{5mm} \small From Nicholson's Girvan material. A, Two marginal cords, also a pillar, 260x. B, disk structure from same section, 65x.}
\end{figure}
\paragraph{}
The sets of parallel curved tubes or bands common in \emph{Girvanella} are formed by curved, furrowed and banded marginal cord. The rows of little circles (mistaken by some observers for cells of blue-green algae) belong to bases of septa astride of the cord, or to the ends of fan-like septa. Accordingly, although \emph{Girvanella} is a concretionary structure, the pattern is not quite in the same category as those radio-concentric ones that are independent of nummulitic structure, \emph{e.g.} \emph{Parkeria}, simple oolite granules (without \emph{Girvanella} pattern), \emph{etc.}

\emph{Girvanella}, then, is a pattern in \emph{chiaroscuro} --- mainly \emph{oscuro}. The nummulitic structure perhaps influences the form of that pattern, and I doubt whether \emph{Girvanella} could form in any other limestone than a nummulitic one.

\emph{Sphaerocodium} Rothpletz is very probably a lithomorph akin to \emph{Girvanella}. The ``schlauchförmige zellen'' and supposed ``Sporangia'' must, I believe, be interpreted in a sense otherwise than that implied by these designations. In Dr. Rothpletz' photo of \emph{S. gotlandicum} (Plate 4, Fig. 1, Kalkalg Gottlands) I am fairly certain I can detect the underlying nummulitic structure.

In oolite grains there may be successive zones of ordinary oolitic pattern and of \emph{Girvanella} structure. Wethered figures a pisolite grain with \emph{Girvanella} surrounding a Crinoid joint.

\subsubsection{Spongiostromids}
\paragraph{}
Dr. G. Gtirich, having discovered a Stromatoporoid in the Carboniferous limestones near Cracow, resolved to study Carboniferous Stromatoporas of other localities, and especially those of Belgium.

He investigated at the Brussels Museum the magnificent series of large thin plaques of the Belgian rocks.

In the thin sections of the Carboniferous limestones of the Viseen of the province of Namur he observed certain obscurely defined masses with granular, nodular or vesicular structure, which he regarded as encrusting organisms. On account of the stratified and spongy reticular structure of certain typical forms, he called them Spongiostromids, and placed the group in a new order of Protozoa. His memoir\footnote{\emph{Mém. Musée Roy. Hist. Nat. Belgique}, 1906.} is illustrated with numerous large photographs on 22 double quarto plates.

In 1913 Dr. Rothpletz\footnote{Swedish Geol. Survey, 1913. \emph{Ueber die Kalkalgen, Spongiostromen... Obersilur Gottlands}, Prof. A. Rothpletz.} described a Spongiostromid from the Silurian of Gottland, and expresses the belief that these organisms are Hydrozoa.

In the summer of 1913 I visited the Brussels Museum and examined the types of Spongiostromids described by Dr. Gürich. I found in every instance that they were concretionary masses of nummulitic limestone, slightly differentiated out from the surrounding matrix as is commonly the case in concretions.

The collotype photographs, though excellent for the purpose of showing stratified vesicular and other structure, do not reveal clearly the nummulitic structure. At the same time, I myself from very careful study of these photographs and from much general experience in the matter of nummulites, can make out parts of large shells in horizontal transverse and oblique aspects and also marginal cord and disk structure in most of the plates. The latter will bear being looked at with a lens 4x. Careful note must be taken of the three degrees of magnification of the photos, \emph{viz.} 0, 5, and 20.

The ``stercomes,'' dark oval bodies, which Dr. Gürich regarded as excrementitious pellets of the supposed Rhizopod, are merely peculiar mineralizations. The occasional regular arrangement appears to me to depend partly on the construction of nummulites with their layers of spiral lamina and coils of marginal cord. I think the serial arrangement of mineral patches in certain silicated nummulitic deposits (\emph{e.g.} hornblende in Dresden syenite) may be due partly and remotely to the nummulitic character of the rock, even though other causes have been at work.

\subsubsection{Precambrian Lithomorphs?}
\paragraph{}
Mr. G. Abbott points out that \emph{Atikokania}, described by Dr. C. W. Walcott as a fossil possibly related to Sponges, is probably a concretion similar to some of the Fulwell forms with radial pattern. The discovery of general nummulitic structure would confirm Mr. Abbott's supposition.

The same test might be applied to certain other dubious Precambrian forms --- such as Cryptozoon.

\subsubsection{Supposed Fossil Micro-organisms in Jurassic and Cretaceous Rocks}
\paragraph{}
Dr. D. Ellis\footnote{\emph{Fossil Micro-organisms from the Jurassic and Cretaceous Rocks of Great Britain}. Proc. Roy. Soc. Edinburgh, 35. parts 1. and 2. p. 110, 2 plates. 1915.} has published a very interesting account of certain appearances he has found in Jurassic and Cretaceous rocks, and which he interprets as being due to fossil micro-fungi. I have examined the two plates of photographs accompanying the memoir, and I would suggest that another interpretation is possible and indeed probable. Dr. Gürich describes stercomes in the Carboniferous limestones, and Dr. Hahn algal sporangia in \emph{Eozoön}. The latter objects I know to be disk structures.

The dark lines, bands, and ampulla-like objects shown in Dr. Ellis's photographs I take to be mineral markings in nummulitic structures.

Dr. Ellis (\emph{l.c.} p. 114) describes the limestone (Froclingham ore) in which he found the supposed micro-organisms as a rock formed of oolitic grains and of non-oolitic fragments of irregular shape, both structures being embedded in calcite. He states that the non-oolitic fragments were of an animal nature, and represented the last stage of decomposition of a comparatively large animal. The photograph on Plate 1, Fig. 1 appears to me to show traces of marginal cord and nummulitic disks. In Fig. 2, also, I can see faintly defined bands, which are probably alar prolongations. I suggest, then, that the irregular fragments embedded in calcite are not decomposed parts of an animal, but simply nummulitic limestone and that the supposed fossil micro-organisms are appearances due to mineralizations of various kinds.

\emph{Summary}. --- Limestones are peculiarly liable to concretionary changes due to deposition of carbonate of lime from solution. The concretions often have a definite shape and a concentric and radial structure and pattern. The term lithomorph is suggested for objects of this kind. Many of the lithomorphs found in limestones of all ages from Tertiary to Precambrian, have such wonderful and highly differentiated patterns, that they have been mistaken for the skeletal remains of animals and plants, and it is not surprising that numerous and varied interpretations have been made concerning them. The clue that led to the discovery of the truth was the fact that most of the limestones are mainly nummulitic, that the lithomorphic patterns have arisen in this common matrix, and that with due patience and care the nummulitic structure can be traced throughout the outlines of the patterns and the matrix.
\clearpage
\section{Summary}
\paragraph{}
Vast deposits of nummulites extending across the old world from N.W. Africa to Japan, and thousands of feet thick in places, were formed during the Eocene period, \emph{i.e.}, in relatively recent times geologically speaking. This epoch has been termed ``The Nummulitic Epoch,'' and d'Archiac refers to it as the ``nummulitic enigma'' owing to the seemingly sudden apparition of these enormous deposits and to their equally sudden disappearance.

There is, however, no enigma so far as predecessors to the Eocene deposits are concerned, for the Chalk ocean was nummulitic, and also seas and oceans throughout the Mesozoic, Palaeozoic and Archaeozoic Eras, the last provisionally including the igneous rocks. In the text an attempt is made to account for the dying out of nummulites about the middle of the present era.

The marine limestones are benthoplankton deposits. They contain a varying amount of silica, magnesium, \emph{etc.}, derived from the sea by deposition or through the agency of life. Silica is separated either directly from the sea, or --- according to Sir J. Murray --- from silicate of alumina in the sea, by Diatoms, Radiolaria and Sponges. In the earliest period the universally distributed silica of the t igneous rocks was probably derived in part from universally distributed simple plankton organisms.

Dissolved silica has become diffused through the deposits of nummulites, \emph{etc.}, and has silicified them to form flint, chert, malmstone, phthanites.

In the most ancient nummulitic deposits, the action of heat has caused the silica to enter into combination with aluminium, magnesium, calcium, iron, sodium and potassium to form silicates. These crystalline masses of silicates are known as igneous rocks. The nummulitic structure is to be seen without difficulty by the trained observer.

The heating of the oldest silicated deposits of shells has been attributed to contraction of the earth (due to cooling in cold space), followed by folding and crumpling of the crust in the course of adjustment to the shrinking nucleus. According to another theory the heat existing in the deeper parts of the earth's crust and continuously and everywhere passing through to the surface is derived from radio-active sources. Uranium and thorium and their derivatives exist in the sea, and fairly abundantly in the deposits (the earth's crust) derived from the sea. The radio-active energy ultimately becomes manifested as heat. It has been pointed out that, although the amounts of uranium, \emph{etc.}, are relatively small, the materials are universally disseminated and have been continuously active for incalculable periods of time.

The softened or melted rocks become squeezed up through the overlying crust in which they form dykes and sills; or the material may pour out or burst out through the surface and form lava flows and conical heaps. If the force of eruption is very great, lumps and dust may burst through the atmosphere and escape terrestrial control. Such material in the course of its new orbit will cross from time to time the earth's orbit. If the earth happens to be there at the same moment, the fragments and particles travelling with terrific velocity will plunge into our aerial ocean and be seen as shooting stars and meteorites. I have found nummulitic structure in all the meteorites I have examined in the British Museum.

\centerline{*\hspace{15mm}*\hspace{15mm}*\hspace{15mm}*\hspace{15mm}*}
\bigskip

Emerged areas of ocean floor and also deep-seated crystalline silicated nummulitic deposits exposed by denudation or extrusion become ground down by meteorological agencies into fragments and particles to form sedimentary rocks. Nummulitic structure is commonly visible even in the smallest particles.

\centerline{*\hspace{15mm}*\hspace{15mm}*\hspace{15mm}*\hspace{15mm}*}
\bigskip

The new facts that have come to light lead to the inference that life originated at the sunlit surface of the ocean, because independent life came first and the sun is the initial quickener of that most abundant form of independent life, the holophytic or plant form; and, further, they point to the probable existence of a former universal ocean. For aeons life has been extracting solid material (the known crust of the earth) from the ocean and depositing it on the ocean floor. Therefore the ocean must once have been considerably deeper. Seeing the relatively small bulk and area of the present emerged part of the ever-undulating ocean floor, it is probable that during the earliest phases the whole floor was submerged. The fact of silicated benthoplankton being universal from pole to pole,\footnote{Igneous rocks exist not far from both poles.} favours the theory of a universal plankton origin of life rather than a local benthos one restricted to regions fringing upheaved areas.

Possibly that manifestation of the infinite and eternal energy known as life arose concomitantly with the formation of plankton colloidal particles containing a substance sensitive to sunlight, and capable of ``assimilating'' carbon, hydrogen, nitrogen, silica, \emph{etc.}, from inorganic material. In time a primitive plankton flora would arise, and also a rhizopodal (amoebula) benthos fauna\footnote{There may be phylogenetic significance in nummulites having a flagellula and an amoebula phase, a free-swimming and a creeping phase; further, the tempting antithesis ``ancestral plankton and benthos'' suggests itself. W. K. Brooks points out that most of the modern plankton had a benthos ancestry, but apparently this benthos fauna had plankton ancestors which ``discovered the bottom.''} by vertical migration to depths in which photosynthesis would cease and be replaced by the animal mode of nutrition.

\centerline{*\hspace{15mm}*\hspace{15mm}*\hspace{15mm}*\hspace{15mm}*}
\bigskip

The earth is an ocean planet, and its crust is a deposit mainly derived from the ocean through the agency of life. Leaving aside considerations of systematic biology, it may be said that the volume\footnote{Calcium carbonate of the benthos scaffolding of nummulites has been replaced by Si. Al. Mg. K. Na. Fe. molecules partly with and partly without an organic history.} of the earth's crust has been woven by two films of protoplasm ever renewed --- a plankton film and a benthos film, the scaffolding of benthos shells being the warp and the plankton-derived silica the woof.

\centerline{*\hspace{15mm}*\hspace{15mm}*\hspace{15mm}*\hspace{15mm}*}
\bigskip

So far as concerns the proof of the nummulosphere theory it is a question of the recognition --- partly with the aid of the microscope --- of nummulitic structure in pre-Eocene limestones, igneous rocks and sedimentary particles. To recognize this structure in pre-Eocene rocks, it is desirable to be familiar with its appearance in Eocene rocks. In the latter, even the experts have not detected certain important structures, \emph{viz.} the spiral disks. The individual shells in masses of nummulites tend to lose their outlines, and, owing to the porous structure, easily to become soaked and mineralized. Accordingly the nummulites older than the Eocene have hitherto escaped detection. Training of the sense of sight will reveal their existence.

\centerline{*\hspace{15mm}*\hspace{15mm}*\hspace{15mm}*\hspace{15mm}*}
\bigskip

Nummulites as viewed in thick sections of transparent crystalline rock will be seen under an unfamiliar aspect not figured in text-books, which usually depict the many-chambered spirally coiled median (splitting) plane and the willow patterns. In the crystalline rocks the more or less obliterated shells will appear in any aspect or section as complicated many-layered objects. For many months I wandered about in these ruined labyrinths without understanding the plan, which is now fairly clear to me. Portions of furrowed and banded marginal cord with bases of septa across furnish perhaps the best clue to orientation, for portions of cord concentric to the first will probably be traceable. With these clues even a beginner may hope to find his way about without much difficulty. Let a diagram of a shell an inch in diameter and lying flat be roughly sketched under a magnification of five diameters. Make a spiral of a few coils in the area of 25 square inches, the coil-outline being drawn (over a small segment) with five or six parallel lines (furrowed and banded marginal cord). Draw a narrow radial band or loop from center to first innermost coil, then one to the second and so on to the last, each being nearly over and embracing the preceding one. In the actual shell there would be radial series all ``round the clock.'' Some idea will be gained of the complicated nature of a transparent many-layered shell with concentric and radial pattern. Then, too, there are the furrowed marginal cord and the coiled disk structures. High powers reveal ever smaller disk structure. A shell an inch in area magnified 2,500 diameters would cover 4,822 square yards.

I have examined igneous rocks from many parts of the world and from all horizons and have never failed to detect nummulitic structure, and can now do so with ease and certainty in most cases.

\centerline{*\hspace{15mm}*\hspace{15mm}*\hspace{15mm}*\hspace{15mm}*}
\bigskip

During the last three years I have studied nummulites of Eocene formations, and have made some tens of thousands of careful observations on nummulitic structure in rocks other than Eocene.

The opinions of those who have never seen either a nummulite shell or sections of one under a microscope will not be of value concerning the structure of nummulites.

When I look back on the distance I have travelled, continually losing my way at first and being deceived by false appearances, I realize the necessity of warning future travellers of the need for patient observation.
\begin{figure}[H]
\centering
\includegraphics[height=110mm,keepaspectratio]{figures/Proteus-Figure.png}
\caption*{\small ``So might I

Have sight of Proteus rising from the sea.''

\vspace{1\baselineskip} % Whitespace after the title block

``Proteus gerôn halios nêmertês'' --- ``Proteus, the ancient one of the sea, whose testimony is true,'' or ``Old sea-tell-truth.''}
\end{figure}
\clearpage
\section{Appendix}
\subsection{Note A. \emph{Wernerian perversity}}
\paragraph{}
``I heard the Professor, in a field-lecture at Salisbury Crags, discoursing on a trap dyke, with amygdaloid margins and the strata indurated on each side, with volcanic rocks all around us, say that it was a fissure filled with sediment from above, adding with a sneer that there were men who maintained that it had been injected from beneath in a molten condition. When I think of this lecture, I do not wonder that I determined never to attend to geology.'' --- \emph{Life and Letters of Charles Darwin}, vol. 1. chapter 2. ``Autobiography,'' p. 41. 1887.

Seeing that Werner did not bring forward a particle of evidence in support of his theory of the aqueous origin of granite and basalt, the ultimate rejection of that theory by the scientific world is not a matter for surprise. Not only was Werner ignorant of the organic basis of these rocks, but also of the fact that they were once in a semi-molten or molten condition.

\subsection{Note B. \emph{Potstones}}
\paragraph{}
Potstones are common round Norwich, and are often seen in the village gardens. The classical quarry at Horstead, figured in Lyell's \emph{Elements}, is quite grown over, but good examples are to be found elsewhere, notably at Whittingham. These curious objects have been taken for gigantic sponges. One theorist thought they were due to lightning forming tubes of fused sand! The theory which attributes their formation to sagging-down of the heavy silica into soft areas and hollows in the chalk, seems a reasonable one. In the thick tabular masses of flint the upper surface is usually more or less flat and smooth, and the lower provided with great bosses. A potstone is a still more gravitating mass. Where vertical pipes are formed, the horizontal tabular layers are further apart, as if the vertical masses had drained off the material.

In one quarry near Norwich I dissected out a slender green glauconite pencil-thick axis or core through two potstones and a portion of a third --- through 4 feet 11 inches in all, till I came to a solid floor of flint and a layer of water. Many of the green grains are casts of Foraminifera. Mr. Leney, the curator of the Norwich Museum, showed me a potstone with a lateral branch of glauconite running from the central axis to a hole in the side of the potstone. Possibly the core results from a surface-layer of shells and clay sediment being carried down in the vortex of the sinking funnel. These surface-shells would, perhaps, still contain animal matter which is supposed to set up the changes resulting in the formation of glauconite. At the same time, a difficulty about this view is that the core is continuous even when there is an interval between the stones, which latter might be compared to huge cylindrical beads on a thread of glauconite.

\subsection{Note C. \emph{The Dolomites of South Tyrol}}
\paragraph{}
The dolomite mountains of South Tyrol are among the grandest limestone formations on the globe. Huge pillars or cimas, slender campaniles and long curved walls rise sheer for thousands of feet. These remarkable Triassic rocks have been the subject of much controversy, and a general agreement as to their real nature and mode of formation has not yet been arrived at. In 1860 Richthofen advocated the view that the ``Schlern'' dolomite --- the crystalline unstratified rock which forms the bulk of many of the dolomites --- could be accounted for on Darwin's coral-atoll theory, the almost total absence of coral being attributed to its obliteration by dolomitization.

D. Stur (1868, 1871), on the other hand, concluded that Schlern dolomite was merely part of a great series of sedimentary limestones, tuffs and clays, and that it had no definite horizon.'' There are some localities where the whole of the sedimentary strata are represented by dolomite, and other areas where the dolomite appears first in the upper horizons and rests on the lower sedimentary strata.''

Gümbel (1873), who rejected what he terms ``the very convenient reef theory,'' regarded Schlern dolomite as occupying a distinct horizon characterized by the presence of a calcareous alga \emph{Diplopora annulata} Schafhäutl, another species \emph{D. pauciforata} Gümbel being found in the older Mendola dolomite. He considered the isolated massifs to be parts of a once continuous formation broken up by volcanic upheavals, followed by ages of denudation.

He (and Lepsius in 1878) explained the great variation in thickness within small areas as being due to inequalities resulting from upheavals and heaping up of volcanic material.

Mojsisovics in his great work \emph{Die Dolomit-Riffe von Süd-Tirol und Venetian}, 1879, supported the coral-reef theory.

Suess (1888) asserts ``that the expression `reef' is justifiable, can scarcely be doubted.''

Mrs. Ogilvie Gordon (1893-5), who wrote a brilliant series of stratigraphical papers on the Triassic rocks of South Tyrol, regarded the Schlern dolomite as an ordinary marine deposit and not as coral-reefs. She showed that frequently the dolomite masses along with their accompanying sedimentary strata had been twisted round and shifted by crust movements in Tertiary times.

E. W. Skeats (1905) believed in the reef theory on account of the chemical purity of the rock and its freedom from residues. A typical Schlern dolomite contained 0.55 per cent. residue; a Mendola sample 0.93 per cent.

Dr. Rothpletz (1894 and 1899) considered the Schlern massif by no means comparable with reefs like those of the Pacific, but rather with certain submarine banks such as the North Dacia Bank or ``Coral-patch,''\footnote{J. Y. Buchanan, Proc. Roy. Soc., Edinburgh, 13. p. 431, 1885; also Langenbeck, \emph{Koralleninseln und Korallenriffe}, p. 32, 1899. Buchanan regarded the Dacia Bank as a very \emph{thick} accumulation of organic remains possibly on a volcanic rock, and this also was the view of Dr. Rothpletz; but the evidence seems inconclusive. Perhaps the bank is a volcanic rock (like the emerged Porto Santo Island not very far away) with only a \emph{thin} layer of coral. The dead coral on the top of the plateau is tinged black with manganese; this deposition would hardly have taken place if the coral grew on the summit of a high pillar of purely calcareous remains.} which rises abruptly from the abyssal floor of the Atlantic off Morocco. If the term reef were used, its significance must be altered and extended. He believed in the ``localized'' character of the dolomite of the Schlern massif, as opposed to the theory of its being part of an extensive continuous deposit.

Amidst all these conflicting opinions, I find that, as in the case of Chalk and many other limestones, an important feature has been overlooked, \emph{viz.}, the organic structure of the main mass of the rock. My sections of typical lumps of Schlern and Mendola dolomite show them to be made of nummulite shells, \emph{i.e.}, Schlern dolomite is mainly a nummulitic limestone. It is almost impossible to detect the shells in the solid rock, so greatly has the latter become crystallized; and it is difficult to make out the nummulitic structure in sections. Large and rather thick sections cut from three sides of a cube of rock show the shells in various aspects when held up to the light and examined with hand-lenses magnifying 3 to 10 diameters. With patience it is possible to discriminate spiral laminae, alars, marginal cord, \emph{etc.}, of large shells. Under higher powers also (\emph{e.g.} 300 diameters), I can now see all these structures with singular clearness. I find 2 mm. Oc. 18, --- tube drawn out, diaphragms nearly closed --- the best combination for the smallest disks. What seem like transparent luminous flakes under lower powers and a fair amount of light, will, in dim light, slowly reveal well-defined disk structures. Each disk is spirally coiled in one plane, with series of smaller spirals situated at intervals round the larger spiral and in planes vertical to that of the latter.

Transparent sections of Schlern rock viewed with a lens show a meandrine pattern formed by opaque and clear calcite, recalling some of the ``Stromatoporoid'' patterns. The Mendola rock-sections, on the other hand, are more like those of Carrara marble, \emph{i.e.}, uniformly granular and clear. The nummulitic structure is to be detected here as in the Schlern material.

The larger fossils are not common in Schlern dolomite, Mollusca, Echinoderms, Corals (rarely) and Calcareous Algae being found.

Mojsisovics (\emph{l.c.} p. 500) writes of ``der grossen armuth des ungeschichteten Dolomits an fossilresten,'' and Rothpletz (1899) of ``der eigentlich Schlerndolomit ganz fossilarm'' (Zeitsch. Deutsch. Geol. Gesellsch. Verhand. p. 105), but the rock is a mass of fossils\footnote{Prof. Skeats examined thirty-eight sections of Schlern dolomite. He found calcareous algae in three, doubtful ditto in five; coral in one, doubtful coral (meandrine pattern) in eleven; mollusc shell in one; echinoderm spine in one; no trace of fossils in seventeen (\emph{i.e.} 45 per cent.). I think it not improbable that there is nummulitic structure in all, excepting perhaps where algae are massed together. Prof. Skeats was unable to find any trace of organisms in a specimen of Mendola dolomite. (Q. J. G. S., 1905.)} throughout, though for the most part they are only to be recognized with some difficulty.

The discovery of the mainly nummulitic nature of Schlern dolomite will not finally solve the complicated problems concerning the character of the formation and its stratigraphical relations, but may throw fresh light on the subject.

Atoll-crowned Christmas Island (Indian Ocean), which has from certain points of view the aspect of a miniature terraced dolomite cima, is partly composed of Orbitoidal limestone: and, moreover, the remarkable calcareous sponge-fauna\footnote{Kirkpatrick, Proc. Roy. Soc. vol. 83, p. 124, and vol. 84, p. 579; and A.M.N.H. July 1911, p. 177.} recalls that of St. Cassian. Sollas has pointed out, too, how large a share Foraminifera take in reef-building.

Yet the huge masses of Triassic nummulitic limestone known as Schlern dolomite --- sandwiched between sedimentary strata and occurring in a district that has been subject to violent disturbances, and also to the influence of the usual unceasing agencies of denudation for immeasurable ages\footnote{Concerning denudation in the Andes, Darwin writes: ``It is not possible for the mind to comprehend, except by a slow process, any effect which is produced by a cause repeated so often that the multiplier itself conveys an idea not more definite than the savage implies when he points to the hairs of his head.'' \emph{Voyage of the Beagle}.} --- may perhaps be relics of a formerly continuous deposit on an uneven and unstable bottom, rather than reefs even in the extended sense of the term.\footnote{The poverty of the Marmolata in magnesium may be due to the accumulation of the deposit in rather deep water, the ordinary dolomite rock having accumulated in relatively shallow warm water and in the coralline zone. Dana attributed dolomitization to deposition of magnesium salts from the warm water of lagoons. Dr. A. G. Högbom believes calcareous algae contribute a considerable percentage of magnesium (Neues. Jahrb. Min. 1894, 1. p. 262).}

\subsection{Note D. \emph{Precambrian succession in Canada}}
\paragraph{}
A joint-committee of the Canadian and American Geological Surveys adopted the following order of succession for the Precambrian rocks of the Lake Superior region:--

Keweenawan

\hspace{10mm}Unconformity

Huronian Upper

\hspace{10mm}Unconformity

Huronian Middle

\hspace{10mm}Unconformity

Huronian Lower

\hspace{10mm}Unconformity

Keewatin

\hspace{10mm}Intrusive contact

Laurentian

\paragraph{}
The Report of the Committee is published in ``The Journal of Geology,'' 1905, 13. p. 89. The Laurentian and Keewatin comprise the Archaean, and the Huronian and Keweenawan the Algonkian.

The Grenville series overlying the Laurentian in Eastern Ontario and Quebec is generally regarded as Archaean, although it is uncertain to what extent it is correlated with the Keewatin. The ``basal complex'' is generally called after the locality in which it is exposed. In the Western Hebrides for instance it is called Lewisian, although it \emph{may} be of the same age as the Laurentian of Canada. ``It is not impossible that they'' (the Laurentian rocks) ``may reach the Western Hebrides, which lie in the latitude of Labrador'' (Suess, \emph{The Face of the Earth}, 4. p. 258). If so it is not surprising to find that the ocean floor abounds in ancient nummulite shells. The Laurentian and Lewisian rocks are nummulitic.

\subsection{Note E. \emph{Precambrian Fossils}}
\paragraph{}
In the Survey Memoir \emph{The Geological Structure of the North-west Highlands of Scotland}, 1907, the piped quartzites\footnote{It is very doubtful whether the ``pipes'' are due to worms. Dr. A. G. Högbom believes the pipes may have arisen from a purely mechanical cause, such as the ascent of bubbles in wet sand in certain conditions of tide and sandy shore. Bull. Geol. Inst. Upsala, 13. p. 45, 1915.} above the Torridonian and below the \emph{Olenellus} zone are reckoned as Lower Cambrian; there are, then, no known Precambrian fossils in Great Britain.

C. D. Walcott gives an account of Precambrian fossils of North America in \emph{Precambrian Fossiliferous Formations}, Bulletin Geol. Soc. America, 10. pp. 199-244, Plates 22-28, 1899; also in \emph{Precambrian Algonkian Algal Flora} (Smithsonian Miscellaneous Collections, vol. 64, p, 77, plates 4-23, 1914), and in \emph{Notes on Fossils from Limestone of Steeprock Series, Ontario, Canada}, Canadian Geol. Survey Memoir 28, p. 17, 1912. Several of the fossils described in these memoirs are probably pseudomorphs.

Dr. G. Abbott (\emph{Nature}, Dec. 31, 1914, p. 477) points out the remarkable similarity between \emph{Atikokania} from Steeprock and ``concretions'' in the Fulwell Limestones. The Steeprock limestones are at the base of the Algonkian, and Dr. Walcott believes they are marine. Probably, like the Fulwell rock, they will be nummulitic limestones.

\emph{Cryptozoon? occidentale} Dawson (Bull. Geol. Soc. America. 10. p. 233), which Dawson compared to \emph{Loftusia}, appears to be (like \emph{Loftusia} itself) a pseudomorph.

Dr. Walcott thinks the entire series of Algonkian limestones of western America may be fresh-water deposits laid down in epi-continental basins wholly or sometimes partly cut off from the ocean. It will be possible in some measure to test this theory. If, like common chalk, the rocks are nummulitic, they will be marine.

The calcareous masses of supposed Algonkian algae so closely resembling the fresh-water ``Lake Balls'' of the present day as described by Dr. Walcott, bear some resemblance also to Fulwell-quarry concretions.

\subsection{Note F. \emph{On Geological Succession}}
\paragraph{}
As examples the following may be cited:--

Goatfell in Arran. From the shore one climbs past red sandstone, then schists are reached, and finally a granite core (A. Geikie).

Again, in a journey across England from the Essex coast to the Welsh mountains a succession of diluvial, less recent, and very ancient rocks will be met with. The ``Floetz'' and ``primitive'' strata of the mountains of Thuringia, Harz and Erzgebirge described by Lehmann and Füchsel also afford classical examples.

The sea may skirt strata of any age, and the relative heights of less ancient and more ancient strata vary. A simple succession is selected in the text.

\subsection{Note G. \emph{On previous Notices of Organic Remains in Igneous Rocks}}
\paragraph{}
Among the writers who have described organisms in igneous rocks and gneisses may be mentioned Dr. G. Jenzsch, \emph{Ueber eine mikroskopische Flora und Fauna krystallinischer Massengesteine}, Leipzig, 1868. He describes organisms in the melaphyre of Zwickau, but I doubt if the highly organised fresh-water fauna and flora described by Dr. Jenzsch belonged to the rock.

Dr. Jenzsch also speaks of organisms in porphyry. He promised to figure them, but no further descriptions appeared.

Dr. Otto Hahn wrote a book entitled \emph{Die Urzelle nebst dem Beweis dass Granit, Gneiss .... Meteorstein und Meteoreisen aus Pflanzen bestehen} (1879). Dr. Hahn did good work in figuring certain appearances he saw in \emph{Eozoön}. He depicts certain curious funnel-shaped structures, which he regarded as species of algae. As a matter of fact they are portions of nummulites. Möbius, again, figured some of these objects (which he considered to be purely mineral forms) in his memoir on \emph{Eozoön} (Plate 34, fig. 44).

The scientific world can hardly be blamed for regarding Dr. Harm's work as ``a somewhat elaborate joke.'' Once he upset some fuchsin on a marble slab and proceeded to clean the latter.

``Ich rieb sie. Plötzlich was sah ich! Kelche von Meter Länge. Taf. 11. Ich nenne sie \emph{Marmora Darwini}. Meine Schreibtisch hat dieselbe Platte... Ich schrieb also eine ganze Abhandlung über die Urzelle auf der Urzelle, denn der ganze Marmor ist nichts als Pflanze. Wirklich --- nicht bios Ironie des Schicksals.'' The strange figure of the supposed plant occupies a whole plate. Again this author saw Sponges (\emph{Urania}, \emph{etc.}), Corals and Crinoids in the chondrules of meteorites, and ``vegetable cells'' in the Widmanstätten figures which appear on polished surfaces of siderites exposed to acids, and which are said to be due to inequality of action of reagents on three kinds of alloys. In all these instances there is no doubt Dr. Hahn was mistaken. In spite of all this, in \emph{Die Urzelle} he figures appearances which actually exist.

C. Montagna \emph{Nouvelle théorie du Métamorphisme des Roches}, 1869, figures Lepidodendron scales in granite of Calabria, a doubtful interpretation.

A. Sismonda (Comptes Rendus, 1865, tome 60, p. 492), describes an imprint of Equisetum in gneiss.

\subsection{Note H. \emph{On deposition of Magnesium Carbonate from Seawater}}
\paragraph{}
Sir John Murray informed me in a letter that an analysis of a large Tridacna shell from Tongatabu showed I per cent, of magnesia in the inner layer of the shell, and no less than 10 to 12 per cent, in the outer parts near the umbo. ``The Tridacna was alive when collected, and twelve sailors had dinner out of it.''

It is highly improbable that the animal secreted a shell with 1 per cent, magnesia in one part and 12 per cent, in another. It would seem as if the higher proportion at the umbo was due partly to deposition from the sea, and partly to solution of the calcium carbonate leaving a larger proportion of magnesium carbonate behind.

\subsection{Note J. \emph{On Archbishop Usher's calculation concerning the Age of the World}}
\paragraph{}
I saw one day in the parish church of the ancient Cinque Port of Sandwich the following epitaph:--
\begin{displayquote}
``'Twas on October's three and twentith day,\\The world was born as learned annals say.''
\end{displayquote}
\paragraph{}
Here followed the moral which I am sorry to say I have forgotten. I was sufficiently curious ``to look up the reference,'' and found the learned annals to be ``Annales veteris et novi testamenti'' (1650) by James Usher. The learned chronicler consulted ``sacred and exotical history,'' the astronomical calendar and the old Hebrew calendar. He therefrom computed that ``the creation of the world did fall out upon the 710 year of the Julian period, placing its beginning in the autumn.'' ``Forasmuch as the first day of the world began with the first day of the week, I have observed that the Sunday, which in the year 710 aforesaid came nearest the autumnal equinox, notwithstanding the staying of the sun in the days of Joshua, and the going back of it in the days of Ezekiah, happened upon the 23rd day of the Julian October; from thence concluded that from the evening preceding that first day of the Julian year both the first day of the creation and the first motion of time are to be deduced.''

\subsection{Note K. \emph{On the Markings of Diatoms (not referred to in text)}}
\paragraph{}
Certain considerations lead me to offer a suggestion --- probably already familiar to students of Diatoms --- concerning the valve-markings. Some of the silica of the lithosphere appears to be derived from a scum of vegetable protoplasm living at the surface of the ocean. The universality of markings in Diatoms must be due to some all-prevailing general cause, which in the case of these chlorophyll-containing plants may well be photosynthesis. This process involves the liberation of bubbles of oxygen in the region of the endochrome granules. Pfitzer and Paul Petit have classified Diatoms according to the characters of the endochrome. Petit states (in Pelletan's Diatomees, Introduction) that a constant relation exists between the disposition of the endochrome and the external form of the siliceous skeleton. As regards the deposition of silica his view is that at the moment of deduplication of valves silica is conveyed by protoplasmic currents in the form of anastomosing threads, the meshes remaining as spaces.

The spheroidal cavities in the valves rather suggest models of a set and compressed foam; and linear spaces might arise from pressure, or from fusion of spheres. If the spheroidal cavities are not bubble-moulds it is difficult to account for the deeply biconcave form of any particular strand of the protoplasmic or siliceous network. The silica could not form where the bubbles are, and when that material sets, the bubble-cavities would remain marshalled in order and fixed.

Recently N. E. Brown\footnote{Journal Quekett Microscopical Club, 1914, p. 317.} after a prolonged search has found genuine pores in the frustule of a species of Pinnularia. In this form the pinnate markings are linear cavities, and the pores form a single row of dots along the center of each line. The dots are so close together that they seem to form a continuous line (\emph{l.c.} plate 23, fig. 13).

D. D. Jackson (\emph{American Naturalist} 1905, p. 287) attributes certain of the movements of Diatoms to the escape of gases.

Referring to species of \emph{Udotea}, A. and E. S. Gepp write:\footnote{\emph{The Codiaceae of the \emph{Siboga} Expedition}, 1911, pp. 5, 6, 102, and Fig. 182.} ``The calcareous sheath of the filaments is seen to be porose.'' ``In all probability their'' (the pores) ``distribution corresponds with that of the green chromatophores inside the filaments, and they themselves mark the spots where bubbles of oxygen were evolved during the photosynthetic process of the plant.'' ``Presumably the deposition of CaCO$_{3}$ would be effected at the time of photosynthesis only and naturally could not occur at the points where bubbles of gas were clinging to the sides of the filaments.'' ``The pores which are so abundant in the calcareous sheath are minute spherical bubble-like chambers in the thickness of the calcareous layer. They are each covered by a delicate calcified pellicle in which is a minute ostiole.'' These spherical bubble-like chambers each covered with a pellicle pierced by a minute ostiole seem comparable with the bubble-like cavities in the siliceous valves of Diatoms.

\subsection{Note L. \emph{Diatomaceous Ooze}}
\paragraph{}
The Diatomaceous ooze from St. 157, 1950 fathoms, in the Southern Ocean (\emph{Challenger} Station 157), is a fine white floury deposit which crumbles at a touch. This ooze is rich in nummulite probably derived from ice-borne erratics or floating pumice. Not only have the nummulites become glassy and powdery, but parts of the anatomy, such as disks of the pillars, have acquired a superficial resemblance to \emph{Coscinodiscus}) many species of which are present in the deposit. The nummulitic structures will be found in regular series, and high powers will show the spirodiscoid structure.

\subsection{Note M. \emph{Barbados Earth}}
\paragraph{}
Barbados earth, a pliocene formation, is usually regarded as a rock composed almost exclusively of Radiolaria. I find it, however, to be a silicified earthy deposit of nummulites very rich in undissolved Radiolaria. Because Radiolarian oozes are usually found at great depths it is assumed that the Barbados deposit may perhaps have been formed at a depth of 3000 fathoms. The nummulites appear to have lived contemporaneously with the Radiolaria sunk from the surface, for the two elements are mixed together in the samples I have seen, so that it is not a case of Radiolaria sinking into a more ancient nummulitic mud. As it is highly improbable that nummulites lived below 1000 fathoms, probably the Barbados earth was not formed at the abyssal depth usually assigned to it. (See \emph{Geology of Barbados}, Part 2. Oceanic Deposits. Jukes-Browne and Harrison. Q.J.G.S. p. 170, 1892.)

\subsection{Note N. \emph{Chalk with Flint and without Flint}}
\paragraph{}
The presence or absence of layers of flint seemingly depends on two factors, \emph{viz.} the abundance or paucity of benthos and plankton siliceous organisms and the degree of permeability of the rock. A marly chalk in shallow water would probably be poor in siliceous benthoplankton and would be relatively impermeable, so that disseminated silica would not all gravitate to form a layer. The upper chalk which indicates deep-water conditions might have a rich benthos sponge fauna and an oceanic plankton one, and the pure nummulitic deposit would be very permeable. W. K. Brooks pointed out the singular poverty of the coral-reef areas in plankton, and the great richness of the oceanic areas not far away.

In Portland, Dorset, the Jurassic rocks from the point of view of chert beds are in the reverse order to the chalk.

\subsection{Note P. \emph{Origin and Metamorphism of Rocks}}
\paragraph{}
It may be well to mention that theories of aqueous and igneous \emph{origin} of rocks are distinct from aqueous, igneous and aquo-igneous theories of \emph{metamorphism} of rocks. Igneous rocks have an aqueous origin, and seemingly their metamorphism is partly due to aquo-igneous agencies.

\subsection{Note R. \emph{Affinities of Lagena and Biloculina}}
\paragraph{}
These forms may belong to a parallel series comparable with that of \emph{Cyclodypeus} --- \emph{Orbitolites} for example. The entosolenian body in \emph{L. marginata} may be the modified relic of a lost central chamber. In any case the term ``introvert'' seems unsuitable.
\clearpage
\section{Epilogue}
\paragraph{}
A plain record of facts should hardly be in need of an epilogue any more than was the play of the immortal Bottom\footnote{``No epilogue, I pray you; your play needs no excuse.''} whose theme too was of ``raging rocks and shivering shocks.''

I hope, however, it may not be altogether amiss to make mention of certain phantasies engendered in my mind in the course of the work and arising out of that work; and an epilogue seems the only place in which to bestow on such unsubstantial things a local habitation. I refer more particularly to a new version of the story of Proteus, whom some --- influenced perhaps by etymological fancies --- suppose to signify elemental and primitive substance. The legend in certain of its aspects seems to anticipate modern ideas of evolution.

A few personal notes bearing on matters in the text are also included.

\centerline{*\hspace{15mm}*\hspace{15mm}*\hspace{15mm}*\hspace{15mm}*}
\bigskip

The first link in the chain of events leading to the final overthrow of the usurping dominion of Pluto was forged, as was befitting, by a humble dependent on the bounty of Neptune, to wit, a poor fisherman of Porto Santo.

One day he caught on his briquera hooks a small lump of rock lying on the bottom of the sea. Strange to relate, this seemingly trivial capture was an event of no small importance to science. It was as if some Arabian Nights djinn had been brought up from the vasty deep, a djinn of the kind that point out paths leading to hidden treasure, but along which they themselves do not go.

The fisherman, instead of impatiently throwing the stone back into the sea, kept it for a learned friend, Senhor Noronha, who had a hobby for collecting sea-things. Some years later, Senhor Noronha gave the stone to Canon Norman, another enthusiastic seeker-after and storer of Neptune's treasures.

Canon Norman, wishing to know the nature of a mysterious calcareous crust on the stone, sent it to the British Museum (Natural History). As soon as I had examined the little patch, in order to learn the cause of it I set out for Porto Santo. After dredging for nine days, I obtained living examples of the encrusting organism, which was found to be a siliceous sponge with a supplementary calcareous skeleton. For a time the nature of the latter continued to be obscure. At this stage I examined certain ancient fossils with a superficial resemblance to the calcareous crust, and drifted on to studying some still more ancient lumps of rock named \emph{Eozoön}, which were found to possess definite organic structure.

In Porto Santo, again, I first detected organic structure in an igneous rock, \emph{viz.}, in a piece of trachyte given me by Senhor Noronha. The gift proved to be something of a talisman. For through it there was conjured up a true vision, at first dim and obscure but later clear and defined, of life, the foam-born, building up in the course of immeasurable aeons the visible frame of the earth.

\centerline{*\hspace{15mm}*\hspace{15mm}*\hspace{15mm}*\hspace{15mm}*}
\bigskip

After the discovery of the nummulitic nature of nearly the whole island of Porto Santo, of the buildings, wine-presses, soil, \emph{etc.}, the name \emph{Eozoön portosantanum} seemed a fitting one for the fossils. The friendly inhabitants had sometimes jokingly greeted me as fellow-citizen, and it occurred to me to suggest to the commune the adoption of a nummulite shell for a crest. When the igneous rocks of Madeira were likewise found to be nummulitic, \emph{Eozoön atlanticum} seemed a more fitting name.

On my return to London, I annexed in one morning for \emph{Eozoön} the volcanic rocks of the Arctic and Atlantic, and in the afternoon of the same day those of the Pacific, Indian and Antarctic oceans. The designation \emph{Eozoön orbis-terrarum} then suggested itself; and, further, a nummulite shell seemed a fit emblem for the Hague Conference (or future parliament of man), in view of the fact that mineralized masses of these shells (territories) are the chief subjects or objects of adjudication.

If \emph{Eozoön}, after taking in the world, had sighed for more worlds to conquer, its fortunes would have surpassed those of Alexander, for its desires would have been realized. When the empire of the nummulites was found to extend to space a final alteration of name to \emph{Eozoön universum} apparently became necessary. Later \emph{Eozoön} was found to be simply a mass of ordinary nummulites, and there was no longer need to invent specific names for it.

\centerline{*\hspace{15mm}*\hspace{15mm}*\hspace{15mm}*\hspace{15mm}*}
\bigskip

The inhabitants of Porto Santo often find on the northern coast of the island giant Entada beans stranded after their long voyage across the ocean.

Columbus, who married the daughter of Perestrello, governor of the island, spent several years there before he made the great voyage. It has been said that the Entadas gave Columbus the idea of the existence of land below the western horizon. The story has been denied, but yet is likely to be true. Admittedly the north shore of the little island does not front the region whence the beans actually came, but Columbus would know that these huge mysterious seeds, products of the abundance of tropical life, could hardly have come from Europe.

It is singular that the island should have been associated with two important discoveries, \emph{viz.}, of the new world and of the oceanic and organic origin of the planetary crust.

The new discovery seems to me important not only on account of its value as an instrument of research, but also because of the labours that have been devoted by men of science to problems\footnote{Palaeontological ones with a pronounced mineralogical aspect, or \emph{vice versâ}, mineralogical problems into which palaeontological considerations enter.} concerning the nature and origin of igneous rocks and meteorites. It may appear strange that success should have come to one who is not by profession either a geologist or a petrologist. There exists a certain element of ``chance'' even in scientific research: ``nor yet (is) favour to men of skill; but time and chance happeneth to them all.''

\centerline{*\hspace{15mm}*\hspace{15mm}*\hspace{15mm}*\hspace{15mm}*}
\bigskip

\centerline{\emph{The Legend of Proteus, the Ancient One of the Sea}}

Telemachus seeking news of his father Odysseus visits the court of Menelaus, who tells his guest the story of his return journey after the fall of Troy. Becalmed at Pharos, he and his men are in danger of starvation. The nymph Eidothée taking pity on him, advises him to capture and question Proteus, that ``takes all manner of shapes of things that creep upon the earth, of water likewise, and of fierce fire burning.''

Neptune's shepherd, having come ashore and counted his seals, lays himself down to rest in the midst of his flock. Menelaus and three of his men, who had been lying in ambush under the flayed skins of seals, rush forward and hold him down. When caught ``that ancient one forgot not his cunning. At first he turned into a bearded lion, and thereafter into a snake, and a pard and a huge boar; then he took the shape of running water, and of a tall and flowering tree.'' But the men, mindful of the nymph's advice, grasp him steadfastly and press him the more. At last, resuming his proper shape, he yields the desired information.

The accumulated exuviae of Proteus are as deceptive in appearance as Proteus himself (see the records of petrogeny and palaeontology), and apotheosized fragments of these remains have even been enshrined in temples.

Under close observation, the earth's crust is seen to be a mineralized deposit of the coiled shells of \emph{Nummulites}, first cousin to the houseless Amoeba Proteus.

\centerline{*\hspace{15mm}*\hspace{15mm}*\hspace{15mm}*\hspace{15mm}*}
\bigskip

\centerline{\emph{The Mysterious Birth and Transformations of Deathless Proteus}}

At one time the earth, covered with a universal mantle of waters, was without life.

But a dawn came when there arose at the edge of the azure plain a crimson and golden glory.\footnote{Presage of a chequered pilgrimage for Proteus.} The rim of the sun's orb appeared above the horizon. At that moment, a ripple on the boundless deep caught the light and seemed as a flame of fire. A cloud of transparent particles of a peculiar dust suspended in the glowing wave entangled and entrapped the rays, and suddenly assumed a bright green colour. Proteus was born.\footnote{The records relating to the manner of Proteus' birth are somewhat vague, and open to various interpretations.}

Suffused with radiant energy, his appetite (perhaps the first appreciable psychic manifestation) was insatiable and the food supply, consisting as it did of air, water and salts of the sea, inexhaustible.

With great rapidity he spread over the face of the waters in the form of floating prairies. From the sea he extracted material wherewith he built himself innumerable palaces of opal ornamented with wondrous designs formed seemingly by the windows of exhalation. On the ocean-floor he constructed on a spiral plan marble halls of many a thousand columns. These abodes when deserted became changed into adamantine crystal and sardonyx and ruby and emerald and sapphire and the mother-substance of these gems. In the course of aeons the ruins formed an immensely thick accumulation --- the known crust of the earth.

Proteus' self underwent ever higher transformations. He became a sac with an inner surface for digestion and an outer one sensitive to impressions from the surroundings. Then a series of sacs remained joined in a row, the various units becoming highly specialized, and so on. When the sea-floor emerged as land above the waters, he became adapted to living in the aerial ocean, creeping, running and flying therein.

At last behold him standing erect, endowed with great psychic storage-batteries encased in bony mail, no longer Neptune's thrall, but lord of the earth, the ocean and the air, seeking to solve the enigma of the infinite and eternal universe and of his own existence therein; and believing himself predestined, at the end of his toilsome ascent up ``the mountainous slopes of the ages,'' from the summit of the mount of blessing to behold
\begin{displayquote}
``Beyond a hundred ever rising mountain lines,\\And past the range of night and shadow,\\The high-heaven dawn of more than mortal day\\Strike on the mount of vision.''
\end{displayquote}
\clearpage
\section{Supplement}
\begin{displayquote}
``That the cell consists of more elementary units of organization is nevertheless indicated by \emph{a priori} evidence so cogent as to have driven many of the foremost leaders of biological thought into the belief that such units must exist, whether or not the microscope reveals them to view.'' --- E. B. Wilson.
\end{displayquote}
\paragraph{}
That the rind of this ocean-planet should be composed of a deposit of marine calcareous shells formed by Rhizopodal protoplasm, that the calcareous matter should be replaced by silica, that the covered-in deposits should become heated owing to mechanical chemical and radioactive causes, that the mass of silicic acid (silica), alkalies, alkaline earths, \emph{etc.}, should combine to form silicates, that the molten pent-up material should be squeezed through the overlying strata and occasionally hurled into space with explosive violence, all this is in no way contrary to what might be expected. I have now, however, to call attention to a discovery of a wholly \emph{unexpected} kind. I find that protoplasm, beneath its various and ever-changing aspects, has a definite fundamental structure of a very remarkable nature, which I have already named ``spirodiscoid.''

The discovery, if true, would be of high importance not only in itself but in its implications. For we cannot hope adequately to understand the nature of the mysterious phenomena of growth, development and heredity, of organic evolution in fact, until we know the real structure of the material basis of those phenomena. Further, a perception of the architectural design of protoplasm may lead to a knowledge of the principles in accordance with which that design has been formed.

As this discovery, strangely enough, has resulted from the observations on igneous rocks, I think it would be well to publish here a brief supplement. A fuller account with photographs will be given later when circumstances permit.

\centerline{*\hspace{15mm}*\hspace{15mm}*\hspace{15mm}*\hspace{15mm}*}
\bigskip

I had found the nummulite shell to be built of ``spirodisks,'' \emph{i.e.} disk-like spirals with a coiled series of secondary spiral disks in planes vertical to the primary, each secondary again having its series of secondary spirals, and so on to the limits of microscopic vision. Presently I found the spirodiscoid structure outlined in ``granules'' in the protoplasm of various Foraminifera and in Amoeba. An examination of plant protoplasm revealed similar spirodiscoid outlines.

Returning to the investigation of animal protoplasm, I found the spirodisks in the nuclei of white and coloured blood-cells of vertebrates, in squamous and columnar epithelium, in the nuclei of cerebro-spinal ganglion cells and of fibrillar tissues, \emph{viz.}, tendon, nerve and muscle. Accordingly I have arrived at the conclusion that protoplasm is constructed on a spirodiscoid plan.

The reader may well take up a guarded attitude when confronted with a new theory of protoplasmic structure. Even a superficial study of the vast literature relating to this subject brings out a strange fact, justifying an attitude of defense. Distinguished men of science after devoting much labour to the investigation of protoplasm have arrived at very conflicting results, so that we have many theories, \emph{viz.}, (1) granular, (2) spherular, (3) fibrillar, (4) spiro fibrillar, (5) reticular, (6) alveolar, (7) fundamentally homogeneous, the reticular and fibrillar appearances being due to artefacts formed by coagulation or precipitation. Truly the elusive Proteus ``that takes all manner of shapes'' has been at his old tactics, but I believe it has fallen to my lot to behold ``the ancient one'' in his real form. Accordingly I am constrained to bring forward yet another theory, \emph{viz.} the spirodiscoid, or rather ``spirad.'' A spirodisk is built of spirads, the latter being the smallest microscopic and the still smaller ultra microscopic, presumably spiral elements of spirodisks.

According to Altmann\footnote{\emph{Die Genese der Zelle}. Beiträge z. Physiol. (C. Ludwig). 1887. --- \emph{Die Elementarorganismen und ihre Beziehungen zu den Zellen}. 2nd Edit. 1894.} the protoplasm of a cell is composed of ``granules'' or bioblasts, which are essentially a kind of very simple organism and which undergo division, in fact \emph{omne granulum e granulo}. Daringly carrying his theory to its logical conclusion, Altmann classifies his granule organisms into ``monads'' and ``nematodes'' (thread organisms). He regards his elementary organisms as organic crystals which grow by intussusception, whereas inorganic crystals grow by apposition. According to E. B. Wilson ``Altmann's premature generalization rests upon a very insecure foundation and has been received with just skepticism.''

J. Künstler\footnote{\emph{Contributions â l'étude des Flagellés}. Bull. Soc. Zool. France. 1882.} believed protoplasm to be built of closely-packed spherules, each of these having a firm wall and fluid contents.

W. Flemming\footnote{\emph{Zellsubstanz, Kern und Zelltheilung}. 1882.} sums up the results of his investigations with the statement that besides the nucleus and occasional (etwanig) granules, only two different essential substances can be distinguished in the cell, \emph{viz.} fibrils and the material between them, the former being slightly more refractive than the latter.

V. Fayod\footnote{\emph{Ueber die wahre Structur des lebendigen Protoplasmas und der Zellmembran}. Naturwiss Rundschau 5. p. 81. 1890.} believes protoplasm to be formed of spirally-coiled hollow fibrils --- spirofibrillae, the latter often being plaited together into hollow cords or ropes --- spirosparta. The spirofibrillae are composed of tough transparent uncolourable material. The lumina of the coils are filled with granular protoplasm, the circulation of which is confined to these ``vessels.'' Fayod injected plant tissues with mercury under a pressure of 1.5 atmospheres, by fixing the material at the end of a long glass tube 1 to 2 meters long and gently filling with the metal. The mercury is supposed to fill the interior of the hollow fibrils and cords.

He gives six figures of the spirofibrillae in the cells of \emph{Fucus} and \emph{Arisarum}. He regards Altmann's granules as broken down particles of spirofibrillae. Criticizing Fayod's results Bütschli remarks, ``I need scarcely state that supported by the results of my investigations I must deny Fayed's statements altogether.'' I, for my part, think Fayod's investigations, excepting those on blood plasma, to be very interesting, and his results valuable.

Bütschli\footnote{\emph{Investigations on Microscopic Foams and on Protoplasm}. English Transl. by E. A. Minchin 1894. The work contains a good critical summary of the various theories of protoplasm, and descriptions of many beautiful experiments with artificial foams.} considers protoplasm to be a microscopic foam, the spaces or ``alveoli'' being filled not with air as in soap-and-water foam but with liquid.

The various ``reticular'' theories as, for example, that of H. M. Bernard,\footnote{\emph{Some Neglected Factors of Evolution}. 1911.} usually unite the features of the granular and fibrillar, chromatin granules being situated at the nodes of a linin-chromatin network.

E. B. Wilson\footnote{\emph{The Cell in Development and Inheritance}.} writes, ``In common with many other investigators, therefore, I believe that protoplasm may in fact be homogeneous down to the present limit of microscopical vision.''

The extraordinary divergences of opinion concerning the structure of protoplasm appear to be due to the varying appearance of this changeable and susceptible material under differing conditions; the varying technique as regards reagents, optical instruments, the quality and intensity of light, \emph{etc.}; and in addition to all these the preconceptions of the observer.

In view of the efforts that have been made by many eminent men to discover the fundamental structure of living matter, it is only after careful research that I venture to bring forward my own observations and conclusions.

To find that protoplasm is formed of granules, of fibrils straight or twisted, of spherules, of networks, of foam, or that it is homogeneous down to the limits of vision, would certainly be very interesting, especially to those engaged in the study of protoplasm, but perhaps facts of this kind would not carry us much farther. The discovery, however, that protoplasm has a visible fundamental structure apparently traceable to the operation of physical and chemical laws would be of high interest from the point of view of a firmer linking up of biology with chemistry and physics. Emil Fischer points out (Ber. 29. 1, 1906, p. 530) that views differ concerning the precise point where an association of biology and chemistry would be profitable. Perhaps one profitable meeting-point would be on the ground of the minute morphology of protoplasm, built as it is of spirads or spiral masses of asymmetric protein molecules.

The great belt of Tertiary nummulites stretching across the eastern hemisphere is not an enigma but a link in a long chain of events stretching back nearly to the dawn of life. Similarly, the planetary deposit of nummulites constituting the bulk of the earth's crust is something more than a phenomenon to be recorded in the annals of systematic palaeontology; the nummulitic spirality may be an expression of the molecular asymmetry of natural organic products.

\centerline{*\hspace{15mm}*\hspace{15mm}*\hspace{15mm}*\hspace{15mm}*}
\bigskip

I for my part find ``granules'' or rather granular appearances often present in protoplasm living and dead, stained and unstained, and I can very frequently see the granules as parts of the peculiar structures termed spirodisks; and, further, a granule above a few $\mu$, in diameter is itself often seen to be a spirodisk with a series of definitely placed smaller granules in relation with it.

Granules vary in size, and especially in their behaviour towards reagents and stains. Altmann, who did an immense deal of work on granules in cells of invertebrates, fixed his material with osmic acid, and stained with aniline dyes, such as aniline acid fuchsin. It is evident that many of the granules have an affinity for the latter, and hold on to it even after ``differentiating'' treatment with picric acid alcohol, which washes out the dye from less retentive tissues. Hence these particles came to be called ``fuchsinophile granules'' or ``Altmann's granules.'' Protoplasm and cells contain granules other than fuchsinophile, such as pigment granules, kerato-hyalin of squamous epithelium, and eosinophile granules. As already mentioned, Altmann regarded his ``granules'' as organisms. Others have held these bodies to be artefacts, or again, metabolic or secretory products.

My own opinion is that ``granules'' are the essential elements of protoplasm, that they are parts of spirodisks, and that they are themselves spirodiscoid.

Spaces between granules appear to be occupied by a more fluid and less easily stainable material. Just as the numerous cell species differ in function, so also the granules in these cells (in nucleus and cytoplasm for instance) differ in function and in chemical composition, and react differently towards stains and reagents. I believe the kerato-hyalin non-fuchsinophile granules of squamous epithelium to be morphologically the same as the fuchsinophile granules of ordinary columnar epithelium or as the granules in gland cells, or as Nissl's granules in ganglion cells.

The failure to detect the nummulites in \emph{Eozoön} and igneous rocks was simply due to the fact that it never occurred to anyone to look for these shells, and naturally the almost obliterated outlines are not to be seen unless carefully sought for. In the case of protoplasm, however, it is somewhat surprising that this closely-scrutinized material should not long ago have yielded up the open secret of its spirodiscoid structure.\footnote{Perhaps the fibrillar structure of fibrillar tissues and the mitotic phenomena of cell division were chiefly responsible for this fact.} If a number of persons are imprisoned in a labyrinth the possessor of a map will find the way out. So with the spirodiscoid plan in mind, the coils will be detected in nuclei which had previously appeared to be confused tangles or networks, and in cytoplasm which had seemed structureless or merely ``granular.'' It was my prolonged observation of sections of igneous rocks that led to the discovery of what I believe to be the fundamental structure of living matter. A layer of particles of protoplasm on the ocean floor built calcareous models of their own inherent structure; then the calcareous molecules became replaced by silica. Accordingly the silicated nummulosphere is an expression of the minute structure of living matter. I can now see granular structure in chalk, igneous rocks, and meteorites. In chalk, the calcified models of Altmann's granules make up Ehrenberg's ``granulirte scheibchen.'' In igneous rocks and meteorites the ``granules'' have become silicated. The earth's crust is made of fossilized ``granules.'' The spirodisk in animal and plant protoplasm is identical with that of fossil nummulites. In the protoplasm of cells I can sometimes see nucleolus, nucleus and cytoplasm as central, middle and faint outer zones of one spirodisk. Again what seem to be limiting membranes of nucleolus and nucleus are sometimes seen to be well-defined rims of coils. Series of outer coils cleared of their inner coils by the mechanical action of injected mercury apparently constitute the hollow spirofibrillae and spirosparta of Fayod who got nearer the truth than other observers. Altmann's granules are spirodisk elements (or spirads) of larger spirodisks, and Bütschli's foam spaces only 1 $\mu$ in diameter are the gaps between the septum-like spirads of small spirodisks. As for fibrillae, I believe these also to be spirodiscoid in character though I have not at present succeeded in detecting their intimate structure. (Observers frequently refer to the ``granular'' appearance of fibrils.) I had similar difficulties in making out the structure of the furrowed and ridged ``marginal cord'' in nummulites. In some sections of foetal dog lent me by Dr. R. J. Gladstone, the developing striped muscle appears as a richly nucleated plasmodium in the jelly of which the fibrillae mysteriously appear. All the nuclei are spirodisks, and I believe each nucleus has its attendant fascicles and perhaps loops of fibrillae.

According to Wilson (\emph{l.c.} p. 22), ``It is nevertheless certain that... the nucleus actually consists of self-propagating units of a lower order than itself, and there is some ground for regarding the cyto-microsomes in the same light.'' Wilson mentions some of the names given to these hypothetical bodies --- Physiological units (Spencer), gemmules (Darwin), pangens (De Vries), biophores (Weismann), bioblasts (Beale), \emph{etc.} To the long list I shall venture to add still another term, \emph{viz.}, ``spirads,'' with the statement that they are not hypothetical, but actual and visible.

I shall now briefly record a few preliminary observations on animal and vegetable cells. Mostly I have made use of Zeiss 2 mm. apochromatic objective, 4 and 18 oculars, drawn out tube, and dim light; but I have tried many other combinations and conditions.

Blood cells. Fresh blood was examined pure, and after treatment with water, salt, acetic acid, and aniline stains. Human colourless corpuscles, of which five kinds are now described, show the ``granular'' nucleus especially after addition of acetic acid. The spirodiscoid plan of the so-called granules often becomes clear. The two or three nucleoli often seen are secondary spirodisks. Even very small ``granules'' 2 or 3 $\mu$ in diameter will be seen as discoidal bodies with peripheral smaller granules, the latter too small to be further resolved.

Coloured blood cells. The anomalous nature of mammalian coloured blood corpuscles presents a highly interesting problem. Physiologists have been in doubt whether to regard these bodies as cells without nuclei, as cells with nuclei (Böttcher), as nuclei without cell-bodies (Huxley, Strieker), or as objects not comparable with cells.

I hope to show that an understanding of the real structure of mammalian coloured blood cells will not only afford a beautiful example of an increasingly perfect adaptation of means to ends, but will throw a new light on the meaning of the ``cell'' and its ``nucleus'' --- truly a ponderous superstructure of induction to establish on a foundation 7.5 $\mu$, in diameter (the width of the human coloured blood cell!)

Human coloured blood corpuscles are now generally regarded as structureless biconcave disks differing from the nucleated coloured blood cells of lower vertebrates in being non-nucleated. I believe this view to be erroneous, for there are no \emph{essential} differences between the coloured corpuscles of fishes, amphibia, reptiles, birds, and mammals, though there appear to be such.

A. Boettcher pointed this out fifty years ago,\footnote{Virchow's \emph{Archiv} vols. 36, 39, and Mém. Acad. Imp. Sci. St. Pétersbourg (7 ser.), 22. No 11. 2 plates, 1876.} but he was held to have been led into error\footnote{Quain's \emph{Anatomy}, ed. 9. vol. 2. p. 29. 1882.} owing to the methods he employed. I find coloured human blood corpuscles are spirodisks with very faintly but indubitably differentiated central coils. A drop of blood simply with a cover slip lowered on it and examined under 3500 diameters in a low light showed in a peripheral zone of plasma a number of cells with regular radial marginal crenations. In these blood cells I could distinguish central coils, not merely as an optical effect in a biconcave body but as parts of a structure with two coils and with series of radial spaces and lines. The marginal crenations (resembling modified ``marginal alveoli'' of Bütschli) were seen to be circular rims with spaces between; a corpuscle on edge was like a milled coin with \emph{rounded} bars for ridges. Frequently pointed crenations occur all over a cell. This condition may be due to the projection of edges of interior coils in slightly varying planes. The pointed shape may result from distortion of the minute plastic secondary disks. The whole cell is plastic and sensitive, reacting rapidly to changes in the medium. The phenomenon of crenation may arise, then, in part owing to radial secondary spirodisks being rendered prominent on account of osmotic changes in the spaces between them; and the crenations reveal the otherwise almost invisible structure of the corpuscle. Under strong crenation a third order of spiral becomes visible, even secondary coils appearing ``milled.''

In the lower vertebrates the central coils (constituting the ``nucleus'') are well-marked and still differ chemically from the outer; but comparative observations only confirm the view that all the coloured corpuscles, whether circular or oval, whether with slightly or markedly differentiated central coils are essentially identical structures.

The function of the coloured corpuscles is to carry oxygen. Therefore the more homogeneous the structure, the more evenly and easily will the oxygen-carrying haemoglobin be diffused, and the more rapidly will the vitally important\footnote{Nature shields the red-cell-forming bone marrow as effectively as the nervous marrow.} function of the cell be carried out. Hence the mammalian coloured blood cell is the most highly evolved from the point of view of utility and efficiency. In the lower vertebrates the disturbing presence of the ``nucleus'' still offers a structural and chemical barrier which is at last broken down in the mammals. The biconcavity in some of the mammalian cells may be due to the smaller size of the chemically uniform central coils, which latter easily swell out in water. Where the central coils (though small) differ chemically from the peripheral the former may be biconvex in the mass.

This gradual obliteration of the distinction between nucleus and cytoplasm following on the attainment of a chemical uniformity required for a special purpose, will I believe throw light on the meaning of the ``cell.'' For the latter is sometimes seen to be a spirodiscoid structure with central coils or ``nucleus'' differentiated \emph{chemically} from the peripheral coils or ``cytoplasm.''

Epithelium. --- Moist squamous epithelial cells examined in water show well a large discoidal spirodiscoid nucleus; also spirodiscoid arrangement is traceable in groups of granules in the cytoplasm. These cytoplasmic coils, sometimes seen as the continuation of the nuclear coils, are looser and less distinct than the latter. The successive series of ``rings'' have been mistaken for large alveoli (not those of Bütschli), for meshworks, \emph{etc.}

The spirodiscoid nucleus in cylindrical epithelium appears to stand on its ``rim,'' and the cytoplasmic coils to be squeezed up into an elongated bundle. The granules are distributed throughout the cell excepting in a clear zone at the outer end.

Among the many lines of investigation being followed at that beneficent institution, the Cancer Research Laboratories, Middlesex Hospital, is one dealing with Altmann's granules. Dr. Henry Beckton\footnote{``Absence of Altmann's Granules as a Histological test for Malignant Disease.'' Journ. Pathol. Soc. 14., p. 408, 1910. In Journ. Pathol. Soc. 13. p. 191, 1909, Dr. Beckton describes his modification of Altmann's method for fixing, staining, and differentiating. In place of osmic acid for fixing he uses formol-Miiller solution; for staining, aniline acid fuchsin; and picric acid alcohol for differentiating.} has found these bodies to be absent from cancer cells. Accordingly a very valuable test --- subject to certain reserves --- is afforded for diagnosing the benign or malignant character of any particular growth. In the hope that my present work might possibly throw some gleam of light on the great problem of malignant tumours, I applied to the laboratories for the loan of some sections. In the fine typical series entrusted to me, I can see spirodisk structure in the nuclei of the epithelial cells; and further, in a columnar-celled cancer of the colon I can make out very faintly-outlined ``granules'' forming extra-nuclear coils in the cytoplasm.

My belief is that these ``granules'' are morphologically identical with ``Altmann's,'' but that owing to degenerative (? colloid) changes they no longer respond to the Altmann reaction, no more than do the keratohyalin granules. The value of the Beckton test would be in no wise diminished by the fact of the granules being still faintly discernible, for even so, in malignant cells they have undergone peculiar changes as regards degree of visibility and reaction to stains.

Another line of research at Middlesex Hospital has been that of the influence of radium in relation to Altmann's granules. In regard to inoperable cases of malignant disease nothing seems at present to hold out so much promise as radio-therapy. Throughout (the kingdom of life from Diatoms to negroes, Nature screens living matter from the effects of over-stimulation by radiant energy.\footnote{Those who work much with X rays and radium have to adapt themselves to a new situation by interposing metal between the skin and the penetrating rays. Apparently radium may bring on a kind of cancer, possibly by causing a certain degree of degeneration in overstimulated cells.} The degenerate essential cells of malignant growths succumb more quickly to excessive radio-bombardment than do the cells of healthy tissues. Apparently radio-therapy makes use of this fact to kill or maim the essentially malignant cells while not injuring the healthy.

The finding in normal and morbid epithelial cells of structure not hitherto detected, \emph{viz.}, the spirodiscoid arrangement of the so-called Altmann's granules and the discovery of the essential nature of these granules, justify the hope that some further advance may be made. If cancer be due to a parasite --- although there is not a particle of evidence to show that it is --- it would be less difficult to detect the intruder, if the plan of construction of normal cells were known, than it would be in the absence of such knowledge.

Plant protoplasm, like that of animals, commonly has a granular appearance, and the granules are arranged in the spirodiscoid manner, \emph{i.e.}, a central nucleolus (if visible as such), and nucleus and cytoplasm are central median and peripheral coils of spirodisks.

I have seen this structure in the seed-pods of nasturtium, in the apical cells of \emph{Nitella} and \emph{Chara}, in the antheridia of \emph{Chara}, in staminal hairs of \emph{Tradescantia}, in cells of potato, in growing root-tip of bean plant, in ovules of \emph{Lilium}, \emph{etc.}

I am greatly indebted to Prof. Blackman and Mr. Paine, of the Imperial College of Science, for the loan of some beautiful sections of root-tip of bean plant\footnote{See Plate 24.} and of ovules of Lilium, the former stained with iron-haematoxylin and showing all the stages of mitosis\footnote{Mitos, a thread.} in the rapidly growing cells.

When a cell is going to divide, the chromatin of the nucleus arranges itself as a convoluted thread or spireme\footnote{Spirema, a coiled thread.} formed of a single or double row of deeply-stainable granules. The coiled thread breaks up into straight or curved rods --- the chromosomes, constant in number for each species. While these changes have been taking place in the nucleus, a small body in the cytoplasm (centrosome), if not already divided, separates into two portions, each having a corona of ``rays.'' The intercentrosome rays meet to form a ``spindle,'' with the chromosomes forming an equatorial plate at the junction of the bases of the opposing cones. The chromosomes split longitudinally into two, the halves move apart to the poles, and each set there forms a daughter nucleus of a new cell. This wonderful process has for its object the distribution of the chromatin granules of the nucleus into two halves, and some authorities (\emph{e.g.} A. Brauer) think into two equal halves. Strasburger and E. B. Wilson believe the influence emanating from the centrosome to be chemotactic. Van Beneden, the discoverer of the centrosome, attributed the migration of the chromosomes from equator to poles to the tension of spindle fibrils. Prof. S. Leduc\footnote{``La Biologie Synthetique,'' 1912.} considers the phenomena of mitosis as being due to osmotic pressure setting up currents radiating from or to centers of concentration. By dropping into a saline solution two tinted drops (``centrosomes'') less or more concentrated than the common medium, and between the two drops a third more or less concentrated and tinted one (``nucleus''), he obtained the various stages of karyokinesis in due sequence (see the astonishing pictures from photographs, \emph{l.c.} p. 125, Fig. 77 A-D). The above sketch is an outline of what takes place in all ``somatic'' cells. In germ cells there is a singular modification which consists in a ``prophetic'' reduction of the number of chromosomes to half, so that when the germ cells unite, the balance is again restored. My own observations on growing root-tip of bean show the cells with their nucleolus, nucleus and cytoplasm to be spirodisks. The spireme gives me the impression of being an uncoiled spirodisk, but I am not certain. I can only make out the spirodiscoid structure before and after the mitosis, but not while the latter is proceeding.

A nucleus stained with iron haematoxylin has the general appearance of a network with dark nodal points. Fortunately very long training in the matter of the calcified and mineralized spirodisks of the nummulites, of igneous rocks and meteorites enabled me to see the real form underlying this very deceptive appearance. Spirodisks are mortal and subject to change, and cells may be crowded with alveoli, secondary products, \emph{etc.}, therefore it is necessary to make numerous observations.

To return to the stained nuclear meshwork. Careful observation will generally reveal very minute coils at the center --- perhaps showing as a nucleolus. The coils are not quite in one plane, but oscillate slightly above and below a central median plane. It is well to bear this fact in mind in following round the curves as they get larger and broader. Between the curves will be seen regularly and serially arranged ``granules.'' When the space between successive curves becomes greater the ``granules'' enlarge, to the extent that presently they themselves can be seen to be spirally coiled. The secondary spirals are like cogs on a wheel and vary in appearance according to the aspect, whether horizontal, vertical, or oblique. When secondary coils attain a sufficient size they show their own series of secondaries.

The chromatin granules are spirads, and the object of mitosis is the proper distribution of spirads.

There is a more rare process of division without mitosis, \emph{i.e.} direct division of nucleus and cell into two. Here we must assume the spirads to be already suitably arranged for distribution into two sets.

The phenomenon of sex occurs almost throughout the kingdom of life, but in spite of much study of its manifestations in animals and plants, authorities agree that there is still an inexplicable residuum of mystery. G. C. Parker\footnote{Encyc. Brit. ed. 11. Article, ``Reproduction.''} writes, ``The essence of a sexual cell is that'' (unlike a gonidium or a spore) ``it cannot give rise by itself to a new organism.'' It has to be fertilized. ``Again, sexual cells differ in sex, but there are as yet no facts to demonstrate any essential structural difference between male and female cells. What is known about them tends to prove their structural similarity rather than their difference.'' According to E. B. Wilson,\footnote{``Recent Researches in the Determination and Heredity of Sex.'' \emph{Science}, vol. 29. p. 53, 1909.} ``The conclusion has become in a high degree probable that sex is controlled by factors internal to the germ cells.'' He asks, ``Upon what conditions within the fertilized egg does the sexual differentiation depend? In some way, we may now be reasonably sure, upon the physiological reactions of nucleus and protoplasm.'' It has occurred to me that possibly the sex difference may be due to the ``spirads'' of germ cells being enantiomorphs (opposite forms). Polariscopic and microscopic observations on the oogonia and antheridia of \emph{Chara} yielded negative and inconclusive results. I think, however, the path is worth following. Protein molecules are laevo- and dextro-gyre, and protoplasm in the mass has a gyrate structure. A glance at the figures of male cells of plants and animals shows a marked tendency to helicoid form probably partly in keeping with their motile function. Possibly natural selection would seize on a character such as protoplasmic enantiomorphism to secure the advantages that result from the blending of different strains. It is suggested that an enantio-spirad theory might help to explain the sex quality of the germ cell, and the incompleteness of the single cell.

\centerline{*\hspace{15mm}*\hspace{15mm}*\hspace{15mm}*\hspace{15mm}*}
\bigskip

Almost throughout the organic world there exists a parallel series of two kinds of ``germ'' cells, commonly but not always exhibiting visibly-contrasting features especially as regards size and mobility. The two kinds of cells may form part of one organism or of two organisms specifically identical.

When \emph{Penicillium} is put into a solution of racemic ammonium tartrate the organism selects the dextro-tartrate and leaves the laevo-tartrate. According to one theory the reaction depends on the organism and dextro-salt being molecularly adapted to each other as templet and material to be moulded, hand and glove, or lock and key.

Similarly it is suggested that possibly the attraction-characters of the two kinds of germ cells may depend on the existence in them of dominant laevo-gyre or dextro-gyre physiological units (spirads).

Apparently one objection to this supposition lies in the fact that the chromosomes of germ cells of two parent organisms often continue to be distinguishable after division of the nucleus of the fertilized ovum.

\centerline{*\hspace{15mm}*\hspace{15mm}*\hspace{15mm}*\hspace{15mm}*}
\bigskip

If the observations recorded in this supplement are correct, then biological evidence shows protoplasm to be constructed on a spiral plan. The polariscope reveals that the molecules of the proteins of which protoplasm is formed are spirally constructed, for they rotate the plane of polarization.\footnote{S. B. Schryver, ``The General Characters of the Proteins,'' 1909.}

Prof. H. E. Armstrong and E. F. Armstrong have constructed a model to illustrate the spiral arrangement in space of the groups of atoms of a polypeptide. See figure, p. 317.

Dr. P. W. Robertson, in his research on the melting points of the Anilides, p-Toluidides, and a-Naphthalides of the Normal Fatty Acids (Trans. Chem. Soc., 1908, vol. 98, p. 1033), observes that ``a maximum and minimum occur in the same series at a difference of six carbon atoms, a result which can possibly be correlated with the fact that a chain of six carbon atoms bends round on itself in space.''

It would appear that stereochemistry, physics, and biology bring evidence tending to show that the materials of which the physical basis of life is composed have a spiral plan of construction.

\centerline{*\hspace{15mm}*\hspace{15mm}*\hspace{15mm}*\hspace{15mm}*}
\bigskip

How the discovery of a peculiar kind of spirality in living matter will affect the great controversy between vitalists and mechanists it would be rash to predict. Huxley would have as soon spoken of the horologity of a clock as of the ``vital force'' of an organism. Prof. F. R. Japp believes some directive force may have come into operation when life originated. Lord Kelvin wrote, ``The only contribution of dynamics to theoretical biology is the absolute negation of automatic commencement or automatic maintenance of life.''

According to Lord Kelvin again, ``A watch spring is much farther beyond our understanding than is a gaseous nebula.'' The little coil of steel when wound up is the vehicle of the mysterious force of elasticity. Much farther still beyond our understanding is the spirally coiled spirad of a human brain cell, the vehicle, it may be, of sensation, volition and thought.

Summary. --- The ``granules'' so often present in protoplasm are very frequently seen to be arranged in a peculiar spiral fashion which I have named spirodiscoid. The granules themselves, when of sufficient size, are likewise seen to have a similar construction. The spiral plan is visible in the so-called nuclear network in cells, and less distinctly in the cytoplasm.

The illustrious Pasteur saw, through the medium of his theory of the molecular asymmetry of natural organic products, a vision of distant horizons. By the discovery of the visible spirodiscoid and spirad structure of protoplasm, and of its skeletons constituting the bulk of the known crust of the earth, it may be that one of those horizons has been reached.
\clearpage
\section{Plates and Guide Diagrams}
\centerline{Synopsis of Plates\footnote{\emph{Note}. Plates 2C, 2D, 2E and 11 were done and printed off at an early period of the work. They include several useless photographs which could not well be eliminated later.}\footnote{\emph{Note}. --- The Stavropol meteorite is a flat brick-shaped object about 5 by 4 inches in area and 2 inches thick. It was seen to fall on March 24 (o.s.), 1857, at 5 P.M. by a man working on a farm, and at a distance of about eighty yards, at Stavropol in the Caucasus. The specimen is in the Petrograd Museum. See H. Abich., Bull. Acad. Imp. Sci. St. Pétersbourg, tom. ii. 1860, p. 407. The photographs are those of a section made from a fragment in the British Museum (Nat. Hist.).}}
\bigskip
\begin{description}
    \item A. Oceanic oozes, 1, 20, 21.
    \item B. Tertiary nummulites, 2B, 15, 23.
    \item C. Limestones, flint and chert.
    \item\hspace{15mm}Chalks, 11, 13, 20, 22, 23.
    \item\hspace{15mm}Oolite, 12.
    \item\hspace{15mm}Dolomite, 12.
    \item\hspace{15mm}Flint and chert, 11, 22.
    \item D. \emph{Eozoön}, 2, 2A, 2D, 11, 13, 14, 23.
    \item E. Igneous Rocks
    \item\hspace{15mm}Volcanic, 14, 15, 21.
    \item\hspace{15mm}Plutonic, 2C, 2D, 11, 12, 15, 27, 28, 29.
    \item F. Meteorites, 2C, 2E, 14, 16, 19, 22.
    \item G. Lithomorphs, 1, 2E, 3A, 3B, 4, 5, 5A, 6.
    \item H. Miscellaneous: Mica, 17; Diamond, 20.
    \item J. Protoplasm: Bean-plant and human brain, 24.
\end{description}

\bigskip
\centerline{\emph{Note on the Guide-Diagrams Accompanying the Photographs}}

These guides have been made by tracing with a needle outlines on gelatine films placed over the photographs. The outline tracings represent only roughly and very inadequately the structure which I myself can see in the pictures. Often the pictures may at first seem to show little else than meaningless patches of light and shade, but certainly the more the photographs are studied the more they will reveal of organic structure. Under low powers (3x to 10x) the larger features of shells are to be seen, and under high powers the spirodisks. If in spite of prolonged study of the pictures and specimens errors have occasionally been made the matter is not serious, for the photographs from untouched negatives are there to help the observer to arrive at his own conclusion.

Many of the pictures show structural details best when examined with a small pocket-lens 3x. A large reading-glass is not so useful because it is often desirable to concentrate attention on a small area.
\clearpage
\pagestyle{fancy}
\fancyhf{}
\rhead{Plate 1: Diatom ooze, Red clay, \emph{Girvanella}.}
\cfoot{\thepage}
\begin{figure}[b]
\centering
\includegraphics[width=\textwidth,keepaspectratio]{figures/Plate1-Figure1.png}
\caption{\small Plate 1: Figure 1 --- Fragment of Diatom ooze, 5x (\emph{Challenger}), Southern Ocean St. 157, 1950 fms. This ooze (in the dried state) is a powdery mass of nummulites crammed with microscopic plankton skeletons, chiefly Diatoms. The faint circular or oval outlines of the nummulites are about 25 mm. (an inch) in diameter in the photo, \emph{i.e.} about 5 mm. ($\frac{1}{5}$ inch) nat. size. [Nummulitic structure can be seen better in sections of this ooze hardened in balsam.]}
\end{figure}
\clearpage
\begin{figure}[b]
\centering
\includegraphics[width=80mm,keepaspectratio]{figures/Plate1-Figure2.png}
\caption{\small Plate 1: Figure 2 --- Fragment of Red Clay, \emph{Challenger} St. 165, 2600 fms., 5x and 8x, showing faint traces of nummulites. See Plate 21, Fig. C, D.}
\end{figure}
\clearpage
\begin{figure}[b]
\centering
\includegraphics[width=80mm,keepaspectratio]{figures/Plate1-Figure3.png}
\caption{\small Plate 1: Figure 3 --- Fragment of Red Clay, \emph{Challenger} St. 165, 2600 fms., 5x and 8x, showing faint traces of nummulites. See Plate 21, Fig. C, D.}
\end{figure}
\clearpage
\begin{figure}[b]
\centering
\includegraphics[width=80mm,keepaspectratio]{figures/Plate1-Figure4.png}
\caption{\small Plate 1: Figure 4 --- ``\emph{Girvanella problematica}'' (from Nicholson's ``type''), 260x. This ``labyrinthine-tubular'' pattern forms a dark patch in the midst of clear calcite. The whole section clear and opaque consists of nummulite shell-structure throughout. Coils of band-like furrowed marginal cord are very obscurely discernible, the outermost forming a broad oval band. Here and there indications of fan-like septa are visible; also spiral disk structure.}
\end{figure}
\clearpage
\begin{figure}[b]
\centering
\includegraphics[width=80mm,keepaspectratio]{figures/Plate1-Figure5.png}
\caption{\small Plate 1: Figure 5 --- Another patch with much larger ``tubes,'' 60x. The ``\emph{Girvanella}'' here forms a small speck in the midst of a coil of a nummulite, and is just visible under a lens 10x when section is held up to light. Dark patch --- hardly in \emph{Girvanella} condition --- in lower right corner shows fairly well a minute nummulitic spiral disk.}
\end{figure}
\clearpage
\begin{figure}[b]
\centering
\includegraphics[width=100mm,keepaspectratio]{figures/Plate1-Guide.png}
\caption{\small Plate 1: Guide Diagram}
\end{figure}
\clearpage
\rhead{Plate 2 --- Specimen of \emph{Eozoön}.}
\cfoot{\thepage}
\begin{figure}[b]
\centering
\includegraphics[width=80mm,keepaspectratio]{figures/Plate2-Figure.png}
\caption{\small Plate 2: A specimen of \emph{Eozoön canadense} or the Dawn Animal. Reduced two-thirds. [By viewing the above figure at some distance, or by holding up in front of a light to eliminate the aggressive banded pattern, it is possible to see here and there faint circular and oval outlines of shells.]}
\end{figure}
\clearpage
\rhead{Plate 2A --- Nummulitic structure in \emph{Eozoön}}
\cfoot{\thepage}
\begin{figure}[b]
\centering
\includegraphics[width=80mm,keepaspectratio]{figures/Plate2A-Figures.png}
\caption{\small Plate 2A: Figures A--D --- Portions of \emph{Nummulites} in \emph{Eozoön}. Drawn from a section of \emph{Eozoön canadense}. A, C 30x; B, D 90x. A --- Center of a shell showing several planes (in horizontal aspect). B --- Fragment of marginal cord, alars and pillars. C --- Two bands of marginal cord and radial alars. D --- Portions of marginal cord.}
\end{figure}
\clearpage
\rhead{Plate 2B --- Tertiary nummulites.}
\cfoot{\thepage}
\begin{figure}[b]
\centering
\includegraphics[width=80mm,keepaspectratio]{figures/Plate2B-FigureA.png}
\caption{\small Plate 2B: Figure A --- \emph{Nummulites gizehensis}, microspheric form, perpendicular section through central median (splitting) plane 4x. [The megalospheric phase of \emph{N. gizehensis} is \emph{N. curvispira}, shown in Fig. 15, a-e.] All from Great Pyramid of Gizeh [Giza].}
\end{figure}
\clearpage
\begin{figure}[b]
\centering
\includegraphics[height=100mm,keepaspectratio]{figures/Plate2B-FigureB.png}
\caption{\small Plate 2B: Figure B --- Transverse section of a shell, willow pattern 6x.}
\end{figure}
\clearpage
\begin{figure}[b]
\centering
\includegraphics[width=\textwidth,keepaspectratio]{figures/Plate2B-FigureC.png}
\caption{\small Plate 2B: Figure C --- Megalospheric shells of a related species, central median plane 10x.}
\end{figure}
\clearpage
\rhead{Plate 2C --- Granite and meteorites.}
\cfoot{\thepage}
\begin{figure}[b]
\centering
\includegraphics[width=95mm,keepaspectratio]{figures/Plate2C-Figure1.png}
\caption{\small Plate 2C: Figure 1 --- Nummulitid pillars and chambers of Orthophragmina pratti, 110x. Figure of no value here.}
\end{figure}
\clearpage
\begin{figure}[b]
\centering
\includegraphics[width=95mm,keepaspectratio]{figures/Plate2C-Figure2.png}
\caption{\small Plate 2C: Figure 2 --- Section of Cornish Granite, 65x. Portion of a nummulite in horizontal aspect, showing region of central median plane. Section shows 3 furrowed marginal cords sloping down from west to east, and septa and alar prolongations at right angles to them. The lowest furrowed cord just above lower edge of photo shows over the middle an inverted V- or U-like septum about 20 mm. high, and the slit-like orifice of same just above cord. Note circular areas of dotted disks astride of cord, also convexity of cord from side to side. To see dotted circles or disks a lens 3x should be used, but photo will bear magnification of 10. After careful inspection, the picture (especially lower third) will be found to be crowded with details of organic structure.}
\end{figure}
\clearpage
\begin{figure}[b]
\centering
\includegraphics[width=95mm,keepaspectratio]{figures/Plate2C-Figure3.png}
\caption{\small Plate 2C: Figure 3 --- ``Wold Cottage'' meteorite. Portions of marginal cord with remains of septa astride, lying obliquely in lower half of photo. Compare with Fig. 2. 260x.}
\end{figure}
\clearpage
\begin{figure}[b]
\centering
\includegraphics[width=80mm,keepaspectratio]{figures/Plate2C-Figure4.png}
\caption{\small Plate 2C: Figure 4 --- Particle from ``Wold Cottage,'' at one time mistaken by me for Radiolarian; but is part of end of bundle of fine tubules; similar pattern later seen \emph{in situ} in Tertiary nummulite shells. 1500x.}
\end{figure}
\clearpage
\begin{figure}[b]
\centering
\includegraphics[width=80mm,keepaspectratio]{figures/Plate2C-Figure5.png}
\caption{\small Plate 2C: Figure 5 --- Supposed Radiolarian! Really a particle of nummulitic structure, neck of ``bottle'' being a crystal. 450x.}
\end{figure}
\clearpage
\begin{figure}[b]
\centering
\includegraphics[width=80mm,keepaspectratio]{figures/Plate2C-Figure6.png}
\caption{\small Plate 2C: Figure 6 --- Cornish granite. Two areas of broken down disk structure. 260x.}
\end{figure}
\clearpage
\begin{figure}[b]
\centering
\includegraphics[width=80mm,keepaspectratio]{figures/Plate2C-Figure7.png}
\caption{\small Plate 2C: Figure 7 --- Supposed Diatoms in granite! Really nummulitic particles. Fig. 7 2000x. Fig. 8 1500x.}
\end{figure}
\clearpage
\begin{figure}[b]
\centering
\includegraphics[width=80mm,keepaspectratio]{figures/Plate2C-Figure8.png}
\caption{\small Plate 2C: Figure 8 --- Supposed Diatoms in granite! Really nummulitic particles. Fig. 7 2000x. Fig. 8 1500x.}
\end{figure}
\clearpage
\begin{figure}[b]
\centering
\includegraphics[width=80mm,keepaspectratio]{figures/Plate2C-Figure9.png}
\caption{\small Plate 2C: Figure 9 --- Chantonnay meteorite. Two parallel sets of dots (upper very faint) astride furrowed cord. Highly magnified.}
\end{figure}
\clearpage
\begin{figure}[b]
\centering
\includegraphics[width=100mm,keepaspectratio]{figures/Plate2C-Guide.png}
\caption{\small Plate 2C: Guide Diagram}
\end{figure}
\clearpage
\rhead{Plate 2D --- \emph{Eozoön} and granites.}
\cfoot{\thepage}
\begin{figure}[b]
\centering
\includegraphics[width=95mm,keepaspectratio]{figures/Plate2D-Figure10.png}
\caption{\small Plate 2D: Figure 10 --- Section of \emph{Eozoön}, poor but showing in upper lighter part of photo a portion of a broad curved band of marginal cord with bases of septa across, and (under lens 3x) disk structures. 175x.}
\end{figure}
\clearpage
\begin{figure}[b]
\centering
\includegraphics[width=80mm,keepaspectratio]{figures/Plate2D-Figure11.png}
\caption{\small Plate 2D: Figure 11 --- Cornish granite. Three marginal cords passing obliquely downwards from left to right and alars at right angles. I can see better by reversing figure. The curve of a broad ridged marginal cord can be followed from middle of the then right edge near circular patch. With lens 3x circular granular disks are clearly visible lying across cord, also faintly two fan-like septa with semicircular edge in oblique projection, 65x.}
\end{figure}
\clearpage
\begin{figure}[b]
\centering
\includegraphics[width=95mm,keepaspectratio]{figures/Plate2D-Figure12.png}
\caption{\small Plate 2D: Figure 12 --- Cornish granite. Rhomboidal feldspar crystal to left of center shows $\frac{1}{8}$ inch (3 mm.) above middle of lower left border and well between two parallel dark streaks a spiral rim (1 mm. diam.) with cog-like light and dark flecks. On left are two much larger concentric rims with radial ``septa'' and ``alars.'' The upper dark streak, nearly tangential to inner of these, stabs five septa, 3x. Lens necessary.}
\end{figure}
\clearpage
\begin{figure}[b]
\centering
\includegraphics[width=95mm,keepaspectratio]{figures/Plate2D-Figure13.png}
\caption{\small Plate 2D: Figure 13 --- Cornish granite. Photo poor but showing about center of photo in oblique-vertical perspective three coils (respectively 2, 4 and 9 mm. in diameter) of central part of a shell; also radial septa; second coil like a thick-rimmed ring. 6x.}
\end{figure}
\clearpage
\begin{figure}[b]
\centering
\includegraphics[width=80mm,keepaspectratio]{figures/Plate2D-Figure14.png}
\caption{\small Plate 2D: Figure 14 --- Denmark granite. Showing two marginal cords passing obliquely from left to right and two embracing alars, also septa and disk-structure. 95x.}
\end{figure}
\clearpage
\begin{figure}[b]
\centering
\includegraphics[width=90mm,keepaspectratio]{figures/Plate2D-Figure15.png}
\caption{\small Plate 2D: Figure 15 --- Denmark granite. Two coils of marginal cord; and septa and alars. 95x.}
\end{figure}
\clearpage
\begin{figure}[b]
\centering
\includegraphics[width=95mm,keepaspectratio]{figures/Plate2D-Figure16.png}
\caption{\small Plate 2D: Figure 16 --- Surface of Denmark granite showing faint outlines of several shells. 4x.}
\end{figure}
\clearpage
\begin{figure}[b]
\centering
\includegraphics[width=100mm,keepaspectratio]{figures/Plate2D-Guide.png}
\caption{\small Plate 2D: Guide Diagram}
\end{figure}
\clearpage
\rhead{Plate 2E --- Meteorites, melted Barbados earth, \emph{Receptaculites}.}
\cfoot{\thepage}
\begin{figure}[b]
\centering
\includegraphics[width=90mm,keepaspectratio]{figures/Plate2E-Figure17.png}
\caption{\small Plate 2E: Figure 17 --- ``Wold Cottage'' meteorite. The lighter circular area about an inch in diameter against upper part of right edge of photo shows beautifully the spiro-discoid structure, \emph{i.e.} coils of miniature ``marginal cord'' with radial ``septa'' and ``alars'' in plane vertical to ``parent'' spiral. 260x.}
\end{figure}
\clearpage
\begin{figure}[b]
\centering
\includegraphics[width=80mm,keepaspectratio]{figures/Plate2E-Figure18.png}
\caption{\small Plate 2E: Figure 18 --- Barbotan meteorite. Fragment of much-blasted nummulite. Orientation better found by making left border the lower. About center is curved rim with circular lines of ridges and with bands across. Around left half are three or four septa and alars passing across a second cord near left upper corner. 65x.}
\end{figure}
\clearpage
\begin{figure}[b]
\centering
\includegraphics[width=80mm,keepaspectratio]{figures/Plate2E-Figure19.png}
\caption{\small Plate 2E: Figure 19 --- ``Wold Cottage'' meteorite. Sections of portions of three nummulites. Below and to right trans. section showing striated gothic edge of spiral lamina, also coil of marginal cord in oblique projection. In lower left quadrant a shell in perp. sect., and in upper left one in trans. sect. 6x.}
\end{figure}
\clearpage
\begin{figure}[b]
\centering
\includegraphics[width=80mm,keepaspectratio]{figures/Plate2E-Figure20.png}
\caption{\small Plate 2E: Figure 20 --- ``Wold Cottage.'' Dark patch. Poor, but showing traces of disk structure. 2500x.}
\end{figure}
\clearpage
\begin{figure}[b]
\centering
\includegraphics[width=80mm,keepaspectratio]{figures/Plate2E-Figure21.png}
\caption{\small Plate 2E: Figure 21 --- Barbados Earth boiled at temp, of 1700° C. to form a glass. Certain dark specks in the glass show distinct traces of nummulitic structure that have escaped solution. 2500x. (Barbados Earth is a deposit of siliceous nummulites rich in still undissolved Radiolaria.)}
\end{figure}
\clearpage
\begin{figure}[b]
\centering
\includegraphics[width=95mm,keepaspectratio]{figures/Plate2E-Figure22.png}
\caption{\small Plate 2E: Figure 22 --- \emph{Receptaculites neptuni}, showing continuity of nummulitic structure in light and dark areas. The photo is from section shown on p. 254, fig. 38. Light and dark areas have precisely the same nummulitic structure; also there are disks partly on light partly on dark side of sharp dividing line. 110x. Note: Recently I have found that the photograph from which Fig. 39 was taken shows a nummulite very plainly when magnified only three diameters.}
\end{figure}
\clearpage
\begin{figure}[b]
\centering
\includegraphics[width=100mm,keepaspectratio]{figures/Plate2E-Guide.png}
\caption{\small Plate 2E: Guide Diagram}
\end{figure}
\clearpage
\rhead{Plate 3A --- \emph{S. concentrica}, variety. Several specimens.}
\cfoot{\thepage}
\begin{figure}[b]
\centering
\includegraphics[width=\textwidth,keepaspectratio]{figures/Plate3-Figure1.png}
\caption{\small Plate 3A --- ``\emph{Stromatopora concentrica} Var. Colliculata.'' Specimens collected by the author at Gerolstein. One-third natural size.}
\end{figure}
\clearpage
\rhead{Plate 3B --- \emph{Stromatopora concentrica}, Goldfuss' type specimen.}
\cfoot{\thepage}
\begin{figure}[b]
\centering
\includegraphics[width=100mm,keepaspectratio]{figures/Plate3-Figure.png}
\caption{\small Plate 3B --- \emph{Stromatopora concentrica}, Goldfuss. Peroy Highley, del et lith. C. Hodges \& Son. imp}
\end{figure}
\clearpage
\rhead{Plate 4 --- \emph{S. concentrica} magnified.}
\cfoot{\thepage}
\begin{figure}[H]
\centering
\includegraphics[width=95mm,keepaspectratio]{figures/Plate4-Figure1.png}
\caption{\small Plate 4 --- \emph{Stromatopora concentrica}.}
\end{figure}
\paragraph{}
All from a slide in Nicholson's collection (P. 5869) figured in his monograph Plate 11, Fig. 16. Figs. 1, 2 are from the marked areas shown in Fig. 29 on p. 226. A lens 3x should be used. All 260x.

The whole slide shows nothing else than nummulitic shell structure throughout. The photos were originally taken to show light and dark areas with continuity of structure throughout light and dark, thereby showing the light spaces could hardly be former polyp tubes filled in with calcite. Figs. 1, 2, represent each an area of 0.5 mm. ($\frac{1}{50}$ of an inch). Owing to the remarkable character of repetition on a continually increasing or diminishing scale, nummulitic structure has somewhat the same aspect under very low or very high powers. A very small area highly magnified may reveal a surprising amount of structure. Fig. 2 shows strands of curved marginal cord passing from side to side and ``septa'' of spirodisk from above downwards. Notice in central part of lower half of Fig. 2 a well-defined broad oval ring lying sideways. In the interior of the right end of the oval are six or seven very well-defined little disks in succession round the bend. Arches, bands and disks visible over the field.
\clearpage
\begin{figure}[b]
\centering
\includegraphics[width=95mm,keepaspectratio]{figures/Plate4-Figure2.png}
\caption{\small Plate 4 --- \emph{Stromatopora concentrica}. All from a slide in Nicholson's collection (P. 5869) figured in his monograph Plate 11, Fig. 16. Figs. 1, 2 are from the marked areas shown in Fig. 29 on p. 226. A lens 3x should be used. All 260x.}
\end{figure}
\clearpage
\rhead{Plate 5 --- Permian concretions and Devonian stromatoporoids.}
\cfoot{\thepage}
\begin{figure}[b]
\centering
\includegraphics[width=\textwidth,keepaspectratio]{figures/Plate5-FigureA.png}
\caption{\small Plate 5: Figure A --- Permian concretions (Fulwell) and Devonian Stromatoporoids compared. Permian concretions in elevation and trans-section. Devonian of Buchel [Buechel]. About natural size.}
\end{figure}
\clearpage
\begin{figure}[b]
\centering
\includegraphics[width=\textwidth,keepaspectratio]{figures/Plate5-FigureB.png}
\caption{\small Plate 5: Figure B --- Permian concretions (Fulwell) and Devonian Stromatoporoids compared. Permian concretions in elevation and trans-section. Devonian of Buchel. About natural size.}
\end{figure}
\clearpage
\begin{figure}[b]
\centering
\includegraphics[width=\textwidth,keepaspectratio]{figures/Plate5-FigureC.png}
\caption{\small Plate 5: Figure C --- Permian concretions (Fulwell) and Devonian Stromatoporoids compared. The Stromatoporoid \emph{Idiostroma oculatum}. Devonian of Buchel. About natural size.}
\end{figure}
\clearpage
\begin{figure}[b]
\centering
\includegraphics[width=\textwidth,keepaspectratio]{figures/Plate5-FigureD.png}
\caption{\small Plate 5: Figure D --- Permian concretions (Fulwell) and Devonian Stromatoporoids compared. The Stromatoporoid \emph{Idiostroma oculatum}. Devonian of Buchel. About natural size.}
\end{figure}
\clearpage
\rhead{Plate 5A --- Specimens of \emph{Parkeria}, \emph{Loftusia}, \emph{Syringosphaeria}.}
\cfoot{\thepage}
\begin{figure}[b]
\centering
\includegraphics[width=95mm,keepaspectratio]{figures/Plate5A-FigureA.png}
\caption{\small Plate 5A: Figure A --- \emph{Loftusia persica}, whole and in section. All natural size.}
\end{figure}
\clearpage
\begin{figure}[b]
\centering
\includegraphics[width=95mm,keepaspectratio]{figures/Plate5A-FigureB.png}
\caption{\small Plate 5A: Figure B --- \emph{Loftusia persica}, whole and in section. All natural size.}
\end{figure}
\clearpage
\begin{figure}[b]
\centering
\includegraphics[width=75mm,keepaspectratio]{figures/Plate5A-FigureC.png}
\caption{\small Plate 5A: Figure C --- \emph{Syringosphaeria}, a Karakoram stone. All natural size.}
\end{figure}
\clearpage
\begin{figure}[b]
\centering
\includegraphics[width=75mm,keepaspectratio]{figures/Plate5A-FigureD.png}
\caption{\small Plate 5A: Figure D --- \emph{Parkeria}. All natural size.}
\end{figure}
\clearpage
\rhead{Plate 6 --- Sections of \emph{Parkeria}.}
\cfoot{\thepage}
\begin{figure}[b]
\centering
\includegraphics[width=\textwidth,keepaspectratio]{figures/Plate6-FigureA.png}
\caption{\small Plate 6: Figure A --- Sections of \emph{Parkeria}. Unpolished and polished surfaces viewed by reflected light 4x.}
\end{figure}
\clearpage
\begin{figure}[b]
\centering
\includegraphics[width=\textwidth,keepaspectratio]{figures/Plate6-FigureB.png}
\caption{\small Plate 6: Figure B --- Sections of \emph{Parkeria}. Unpolished and polished surfaces viewed by reflected light 4x.}
\end{figure}
\clearpage
\begin{figure}[b]
\centering
\includegraphics[width=\textwidth,keepaspectratio]{figures/Plate6-FigureC.png}
\caption{\small Plate 6: Figure C --- Sections of \emph{Parkeria}. Section viewed by transmitted light 5x.}
\end{figure}
\clearpage
\rhead{Plate 11 --- Chalk, flint, \emph{Eozoön}, granite, Barbados earth.}
\cfoot{\thepage}
\begin{figure}[b]
\centering
\includegraphics[width=\textwidth,keepaspectratio]{figures/Plate11-Figure39.png}
\caption{\small Plate 11: Figure 39 --- Surface of a piece of chalk from Haling, 5x. There are portions of several nummulites visible in photo. The whole surface shows nothing but nummulitic structure, and every particle is in orderly arrangement, \emph{i.e.} there is no \emph{débris}. A marginal cord with septa curves round lower left corner within radius of 18 mm. Above middle of lower border are embracing gothic arches and a septum of shell in trans. section. (Lens 3x necessary for seeing details.)}
\end{figure}
\clearpage
\begin{figure}[b]
\centering
\includegraphics[width=95mm,keepaspectratio]{figures/Plate11-Figure40.png}
\caption{\small Plate 11: Figure 40 --- Surface of flint, 8x. As in Fig. 39, but silicified. Nummulitic structure faintly visible over whole field. A marginal cord with septa curves upward obliquely in projection from left of center of lower edge of photo to right of center of upper edge.}
\end{figure}
\clearpage
\begin{figure}[b]
\centering
\includegraphics[height=110mm,keepaspectratio]{figures/Plate11-Figure41.png}
\caption{\small Plate 11: Figure 41 --- \emph{Eozoön canadense}, banded structure, 4x. Nummulitic structure is traceable both in light and dark bands, but with difficulty.}
\end{figure}
\clearpage
\begin{figure}[b]
\centering
\includegraphics[width=95mm,keepaspectratio]{figures/Plate11-Figure42.png}
\caption{\small Plate 11: Figure 42 --- From section of \emph{Eozoön}, and wrongly described in Nummulosphere Part 1 as a small nummulite shell. The dark gibbous area is really only a minute particle, 0.1 mm. in diameter, of a nummulite. The area shows disk structure especially along straight edge, and also septa astride of cord along upper left edge. 250x.}
\end{figure}
\clearpage
\begin{figure}[b]
\centering
\includegraphics[width=95mm,keepaspectratio]{figures/Plate11-Figure43.png}
\caption{\small Plate 11: Figure 43 --- Rough surface of Cornish granite showing part of a marginal cord and embracing alar; also spirodisks single and in series. 4x. (Photo seen better reversed.)}
\end{figure}
\clearpage
\begin{figure}[b]
\centering
\includegraphics[height=110mm,keepaspectratio]{figures/Plate11-Figure44.png}
\caption{\small Plate 11: Figure 44 --- Cornish granite. In upper left corner a septum in projection with ``pans-pipe'' rim, rising from banded cord; at lower left corner of photo part of another cord; disks fairly plentiful. 65x.}
\end{figure}
\clearpage
\begin{figure}[b]
\centering
\includegraphics[width=85mm,keepaspectratio]{figures/Plate11-Figure45.png}
\caption{\small Plate 11: Figure 45 --- Cornish granite. A broad band of marginal cord extending across from S.E. to N.W. corner with two septa across and in projection; disks well shown. 110x. (Low-power lens should be used.)}
\end{figure}
\clearpage
\begin{figure}[b]
\centering
\includegraphics[width=75mm,keepaspectratio]{figures/Plate11-Figure46.png}
\caption{\small Plate 11: Figure 46 --- Monte Somma bomb. Part of vitrified septum. For long a mystery to me, at first taken for Globigerina! Series of faint circles visible in projection round rim of large circle above. 110x.}
\end{figure}
\clearpage
\begin{figure}[b]
\centering
\includegraphics[width=95mm,keepaspectratio]{figures/Plate11-Figure47.png}
\caption{\small Plate 11: Figure 47 --- Cornish granite. Good spirodisk in oblique projection in lower right corner. Lower of three radiating bands is probably the edge of a layer of spiral lamina. 110x.}
\end{figure}
\clearpage
\begin{figure}[b]
\centering
\includegraphics[width=95mm,keepaspectratio]{figures/Plate11-Figure48.png}
\caption{\small Plate 11: Figure 48 --- Banded trachyte. Surface ½x.}
\end{figure}
\clearpage
\begin{figure}[b]
\centering
\includegraphics[width=95mm,keepaspectratio]{figures/Plate11-Figure49.png}
\caption{\small Plate 11: Figure 49 --- Radiolarian in Barbados earth heated to white hot slag condition. 500x.}
\end{figure}
\clearpage
\begin{figure}[b]
\centering
\includegraphics[width=95mm,keepaspectratio]{figures/Plate11-Figure50.png}
\caption{\small Plate 11: Figure 50 --- One of several dark spots in a section of glass bead made by melting Barbados earth in electric furnace. Useless figure; better seen in Plate 2. E, Fig. 21.}
\end{figure}
\clearpage
\begin{figure}[b]
\centering
\includegraphics[width=100mm,keepaspectratio]{figures/Plate11-Guide.png}
\caption{\small Plate 11: Guide Diagram}
\end{figure}
\clearpage
\rhead{Plate 12 --- Oolite, dolomite, syenite.}
\cfoot{\thepage}
\begin{figure}[b]
\centering
\includegraphics[width=105mm,keepaspectratio]{figures/Plate12-FigureA.png}
\caption{\small Plate 12: Figure A --- Section of Portland oolite, 3½x. Concentric beaded bands of marginal cord and radial alars and septa of nummulites faintly visible over the whole field, but seen best in right half of upper left quadrant. The granules follow to a considerable extent the lines of nummulitic structure. (Low power lens necessary.)}
\end{figure}
\clearpage
\begin{figure}[b]
\centering
\includegraphics[width=95mm,keepaspectratio]{figures/Plate12-FigureB.png}
\caption{\small Plate 12: Figure B --- Section of Schlern dolomite, 4x. Marginal cord and septa of a nummulite. Many hours of the closest scrutiny are necessary to trace the nummulitic structure in this apparently structureless confusion. (Owing to increased experience since Plate 12 was done, I would now have little difficulty in rendering obvious the nummulitic nature of dolomite.)}
\end{figure}
\clearpage
\begin{figure}[b]
\centering
\includegraphics[width=95mm,keepaspectratio]{figures/Plate12-FigureC.png}
\caption{\small Plate 12: Figure C --- Dresden syenite, 4x. Nummulitic structure rather muzzy and confused, but abundant. In lower third marginal cords passing down obliquely from left to right with septa across. Within center of right edge of photo, 3 mm. below lower edge of C, a disk 3 mm. across and with cogged rim, also three little dotted circles to left of disk. A disk in oblique perspective, 6 mm. diam., with left lower rim blacked out, is visible 18 mm. above center of picture.}
\end{figure}
\clearpage
\begin{figure}[b]
\centering
\includegraphics[width=100mm,keepaspectratio]{figures/Plate12-Guide.png}
\caption{\small Plate 12: Guide Diagram}
\end{figure}
\clearpage
\rhead{Plate 13 --- Totternhoe Stone, Melbourn Rock, \emph{Eozoön}.}
\cfoot{\thepage}
\begin{figure}[b]
\centering
\includegraphics[width=85mm,keepaspectratio]{figures/Plate13-FigureA.png}
\caption{\small Plate 13: Figure A --- Section of Totternhoe Stone (Lower Chalk), 110x. A curved band of marginal cord with septa extends from side to side across middle of field. (Picture too dark, and not equal to negative. It has been difficult to get good results from photos of any of the chalks.)}
\end{figure}
\clearpage
\begin{figure}[b]
\centering
\includegraphics[width=95mm,keepaspectratio]{figures/Plate13-FigureB.png}
\caption{\small Plate 13: Figure B --- Section of Melbourn Rock (Middle Chalk) to show circular and oval ``spheres,'' 65x. These ``bodies'' are simply clarified areas in nummulitic rock. The areas are usually discoidal, and not spheres either whole or in section. Nummulitic structure exists in the clear spaces, and can often be traced from light to dark areas. (See also Plate 23 Fig. H.)}
\end{figure}
\clearpage
\begin{figure}[b]
\centering
\includegraphics[width=110mm,keepaspectratio]{figures/Plate13-FigureC.png}
\caption{\small Plate 13: Figure C --- Weathered surface of a lump of \emph{Eozoön canadense}, \emph{i.e.} of a mass of very ancient nummulites. Natural size. The external form and inner structure of the nummulites in various aspects are visible to unaided eye, or under lens 3x. They are best seen in lower half of picture, where one or two might almost serve for descriptions of species.}
\end{figure}
\clearpage
\begin{figure}[b]
\centering
\includegraphics[width=100mm,keepaspectratio]{figures/Plate13-Guide.png}
\caption{\small Plate 13: Guide Diagram}
\end{figure}
\clearpage
\rhead{Plate 14 --- Snake River basalt, meteorite, \emph{Eozoön}.}
\cfoot{\thepage}
\begin{figure}[b]
\centering
\includegraphics[width=95mm,keepaspectratio]{figures/Plate14-FigureA.png}
\caption{\small Plate 14: Figure A}
\end{figure}
\clearpage
\begin{figure}[b]
\centering
\includegraphics[width=95mm,keepaspectratio]{figures/Plate14-FigureB.png}
\caption{\small Plate 14: Figure B}
\end{figure}
\clearpage
\begin{figure}[b]
\centering
\includegraphics[width=95mm,keepaspectratio]{figures/Plate14-FigureC.png}
\caption{\small Plate 14: Figure C}
\end{figure}
\clearpage
\begin{figure}[b]
\centering
\includegraphics[width=95mm,keepaspectratio]{figures/Plate14-FigureD.png}
\caption{\small Plate 14: Figure D}
\end{figure}
\clearpage
\begin{figure}[b]
\centering
\includegraphics[width=\textwidth,keepaspectratio]{figures/Plate14-FigureE.png}
\caption{\small Plate 14: Figure E}
\end{figure}
\clearpage
\begin{figure}[b]
\centering
\includegraphics[width=100mm,keepaspectratio]{figures/Plate14-Guide.png}
\caption{\small Plate 14: Guide Diagram}
\end{figure}
\clearpage
\rhead{Plate 15 --- Clee Hill diorite, Tertiary nummulite, syenite.}
\cfoot{\thepage}
\begin{figure}[b]
\centering
\includegraphics[width=95mm,keepaspectratio]{figures/Plate15-FigureA.png}
\caption{\small Plate 15: Figure A --- Section of Clee Hill diorite by transmitted light, 4x. Coils of marginal cord, with septa of a nummulite. Parallel lines of cord with bands across, and dishes between septa are visible with a lens.}
\end{figure}
\clearpage
\begin{figure}[b]
\centering
\includegraphics[width=85mm,keepaspectratio]{figures/Plate15-FigureB.png}
\caption{\small Plate 15: Figure B --- \emph{Nummulites laevigata} (from Selsey). Perpendicular section (ground down), 20x. Showing two marginal cords, also septa.}
\end{figure}
\clearpage
\begin{figure}[b]
\centering
\includegraphics[width=85mm,keepaspectratio]{figures/Plate15-FigureC.png}
\caption{\small Plate 15: Figure C --- Dresden syenite, 60x. Same view of a nummulite as in Fig. B.}
\end{figure}
\clearpage
\begin{figure}[b]
\centering
\includegraphics[width=100mm,keepaspectratio]{figures/Plate15-Guide.png}
\caption{\small Plate 15: Guide Diagram}
\end{figure}
\clearpage
\rhead{Plate 16 --- Stavropol meteorite.}
\cfoot{\thepage}
\begin{figure}[b]
\centering
\includegraphics[width=\textwidth,keepaspectratio]{figures/Plate16-FigureA.png}
\caption{\small Plate 16: Figure A --- Stavropol meteorite. Section 5x, viewed by transmitted light.}
\end{figure}
\clearpage
\begin{figure}[b]
\centering
\includegraphics[width=105mm,keepaspectratio]{figures/Plate16-FigureB.png}
\caption{\small Plate 16: Figure B --- The same, 10x. (The central squarish black patch and two large white ones easily compare with the same in Fig. A.) Concentric --- radial shell-structure and spirodiscoid structure fairly clear.}
\end{figure}
\clearpage
\begin{figure}[b]
\centering
\includegraphics[width=95mm,keepaspectratio]{figures/Plate16-FigureC.png}
\caption{\small Plate 16: Figure C --- A center of a coil, 95x. (It is not certain whether this center is that of a shell or of a spirodisk.) The furrowed band of marginal cord with coils winding at slightly different planes, with septa astride, and with smaller spirodisk structure is absolutely clear and indubitable after a little careful study with a lens.}
\end{figure}
\clearpage
\begin{figure}[b]
\centering
\includegraphics[width=95mm,keepaspectratio]{figures/Plate16-FigureD.png}
\caption{\small Plate 16: Figure D --- Center of C, 1100x.}
\end{figure}
\clearpage
\begin{figure}[b]
\centering
\includegraphics[width=100mm,keepaspectratio]{figures/Plate16-Guide.png}
\caption{\small Plate 16: Guide Diagram}
\end{figure}
\clearpage
\rhead{Plate 17 --- Snake River basalt, mica.}
\cfoot{\thepage}
\begin{figure}[H]
\centering
\includegraphics[width=95mm,keepaspectratio]{figures/Plate17-FigureA.png}
\caption{\small Plate 17: Figure A --- Section of Snake River Basalt, 10x. Showing a slightly wider area than in Plate 14 A.}
\end{figure}
\paragraph{}
Careful study will show the ridged band of a ``marginal cord'' like a spirally wound motor-car tyre, with ``septa'' across. For guide to center of coil see diagram.
\clearpage
\begin{figure}[b]
\centering
\includegraphics[width=95mm,keepaspectratio]{figures/Plate17-FigureB.png}
\caption{\small Plate 17: Figure B --- Mica 3½x. This wonderful photograph is crammed with details of nummulitic structure. The mica film was flaked off a thick block of dark amber-coloured Himalayan mica given to me by the great mica firm of Messrs. Wiggins. I have detected nummulitic structures even in the most transparent varieties of mineral by using the highest powers, but here the shell structure can be seen with a lens 3x. Spiral coils of marginal cord with septa and alars and spirodiscoid structures in all these are present over the whole field.}
\end{figure}
\clearpage
\begin{figure}[b]
\centering
\includegraphics[width=100mm,keepaspectratio]{figures/Plate17-Guide.png}
\caption{\small Plate 17: Guide Diagram}
\end{figure}
\clearpage
\rhead{Plate 18 --- Granite.}
\cfoot{\thepage}
\begin{figure}[b]
\centering
\includegraphics[width=95mm,keepaspectratio]{figures/Plate18-FigureA.png}
\caption{\small Plate 18: Figure A}
\end{figure}
\clearpage
\begin{figure}[b]
\centering
\includegraphics[width=95mm,keepaspectratio]{figures/Plate18-FigureB.png}
\caption{\small Plate 18: Figure B}
\end{figure}
\clearpage
\begin{figure}[b]
\centering
\includegraphics[width=100mm,keepaspectratio]{figures/Plate18-Guide.png}
\caption{\small Plate 18: Guide Diagram}
\end{figure}
\clearpage
\rhead{Plate 19 --- Wold Cottage and Ensisheim meteorites, syenite.}
\cfoot{\thepage}
\begin{figure}[b]
\centering
\includegraphics[width=95mm,keepaspectratio]{figures/Plate19-FigureA.png}
\caption{\small Plate 19: Figure A --- Section of ``Wold Cottage'' meteorite, 4x. To left of middle of upper half of picture is a well-marked coil of marginal cord about two inches across and in slightly oblique perspective. Right half of coil shows (like a motor-car tyre) a curve from above down and also from side to side; parallel ridges well seen, also parts of septa across breadth, and spirodisks of which the whole shell is built. Within and nearly concentric with above coil is a smaller one near center of the shell; and without and nearly concentric a much larger and less clearly denned marginal cord. The field where not too dark is crowded with nummulitic, including spirodisk structure (see, for instance, double band of disks crossing a marginal cord, just above center of lower border of photograph).}
\end{figure}
\clearpage
\begin{figure}[b]
\centering
\includegraphics[width=95mm,keepaspectratio]{figures/Plate19-FigureB.png}
\caption{\small Plate 19: Figure B --- Section of Ensisheim meteorite, 4x. Nummulitic structure is abundant, but less clear than in the ``Wold Cottage'' meteorite.}
\end{figure}
\clearpage
\begin{figure}[b]
\centering
\includegraphics[width=85mm,keepaspectratio]{figures/Plate19-FigureC.png}
\caption{\small Plate 19: Figure C --- Dresden syenite. Section showing portions of broad band-like coils of marginal cord and also septa. 95x.}
\end{figure}
\clearpage
\begin{figure}[b]
\centering
\includegraphics[width=100mm,keepaspectratio]{figures/Plate19-Guide.png}
\caption{\small Plate 19: Guide Diagram}
\end{figure}
\clearpage
\rhead{Plate 20 --- Diatom ooze, red clay, chalk, diamond.}
\cfoot{\thepage}
\begin{figure}[b]
\centering
\includegraphics[width=85mm,keepaspectratio]{figures/Plate20-FigureA.png}
\caption{\small Plate 20: Figure A --- Diatom ooze (\emph{Challenger}, St. 157 Southern Ocean, 1950 fms.). Crushed fragment in balsam, 65x. It can be said with entire certainty that the six or seven circular, oval or (U-shaped bands are parts of septa or marginal cords of nummulite shells (and not Radiolaria). The structure is obscured by the numerous Diatoms. When the eye is trained, it becomes possible to detect abundant spirodiscoid structure throughout. 65x.}
\end{figure}
\clearpage
\begin{figure}[b]
\centering
\includegraphics[width=85mm,keepaspectratio]{figures/Plate20-FigureB.png}
\caption{\small Plate 20: Figure B --- Red Clay (\emph{Challenger}, St. 165, 2600 fms.). Crushed fragment in balsam 260x. In lighter areas nummulitic structure is visible. (The poor results in this figure led to the making of the excellent sections, of which photos are shown in Place 21 Figs. C, D.)}
\end{figure}
\clearpage
\begin{figure}[b]
\centering
\includegraphics[width=95mm,keepaspectratio]{figures/Plate20-FigureC.png}
\caption{\small Plate 20: Figure C --- Section of a hard chalk from Missenden, 4x. The whole field is full of nummulitic structure, \emph{viz.}, spiral coils of cord and the septa, but it is very difficult to make out. When magnified 12 diameters the section shows Globigerina and ``spheres.''}
\end{figure}
\clearpage
\begin{figure}[b]
\centering
\includegraphics[width=95mm,keepaspectratio]{figures/Plate20-FigureD.png}
\caption{\small Plate 20: Figure D --- Small diamond from Transvaal, 4x. Showing nummulitic structure.}
\end{figure}
\clearpage
\begin{figure}[b]
\centering
\includegraphics[width=95mm,keepaspectratio]{figures/Plate20-FigureE.png}
\caption{\small Plate 20: Figure E --- The same diamond 260x. Portions of marginal cord, septa, and spirodisks are visible to the trained eye, especially above center of lower edge of picture.}
\end{figure}
\clearpage
\begin{figure}[b]
\centering
\includegraphics[width=100mm,keepaspectratio]{figures/Plate20-Guide.png}
\caption{\small Plate 20: Guide Diagram}
\end{figure}
\clearpage
\rhead{Plate 21 --- Tenerife sulphury trachyte, red clay.}
\cfoot{\thepage}
\begin{figure}[H]
\centering
\includegraphics[width=95mm,keepaspectratio]{figures/Plate21-FigureA.png}
\caption{\small Plate 21: Figure A --- Section of rotten trachyte permeated with sulphur from interior of upper crater of Tenerife, 4½x. The coils of a much-blasted nummulite in perpendicular section are visible to the trained vision. The rock section has a pale reddish tinge, the red being much more apparent in sections.}
\end{figure}
\paragraph{}
\emph{Note}. --- This plate is one of the most interesting of the set. The sections were made from red clay and rotten trachyte hardened in balsam. The trachyte would have become red clay if it had fallen into the ocean. Red Clay comes from one universal deposit of mineralized nummulites. The ``clay'' may have been erupted from a submarine deep-seated part of the deposit, or from a supra-marine or upheaved area of the deposit, or lastly from the disintegrated surface of the deposit \emph{in situ}.
\clearpage
\begin{figure}[b]
\centering
\includegraphics[height=110mm,keepaspectratio]{figures/Plate21-FigureB.png}
\caption{\small Plate 21: Figure B --- The same 450x. Every particle of the rock section shows nummulitic structure. The portion here selected shows part of a coil of marginal cord. The part figured was chosen to compare with Tschermak's figure of the section of Seres meteorite showing antler-like crystal with a wedge of olivine (biologically, part of marginal cord and septum). In addition there are septa and spirodiscoid structures.}
\end{figure}
\clearpage
\begin{figure}[b]
\centering
\includegraphics[height=110mm,keepaspectratio]{figures/Plate21-FigureC.png}
\caption{\small Plate 21: Figure C --- Section of Red Clay (\emph{Challenger}, St. 165, 2600 fms.). Showing portion of a nummulite, 10x. Best viewed by making left border the lower. Curved series of circular disks seen across the field. Patient study with lens affords a wonderful revelation.}
\end{figure}
\clearpage
\begin{figure}[b]
\centering
\includegraphics[width=75mm,keepaspectratio]{figures/Plate21-FigureD.png}
\caption{\small Plate 21: Figure D --- The same 260x. Shows coiled disk-structure, \emph{i.e.} miniature marginal cord and septa.}
\end{figure}
\clearpage
\begin{figure}[b]
\centering
\includegraphics[width=100mm,keepaspectratio]{figures/Plate21-Guide.png}
\caption{\small Plate 21: Guide Diagram}
\end{figure}
\clearpage
\rhead{Plate 22 --- Radiolarian chert, Mazapil and Jamestown siderites, chalk.}
\cfoot{\thepage}
\begin{figure}[b]
\centering
\includegraphics[height=130mm,keepaspectratio]{figures/Plate22-FigureA.png}
\caption{\small Plate 22: Figure A --- Section of Radiolarian Chert, 10x. The figure within white square is shown magnified in B.}
\end{figure}
\clearpage
\begin{figure}[b]
\centering
\includegraphics[width=95mm,keepaspectratio]{figures/Plate22-FigureB.png}
\caption{\small Plate 22: Figure B --- One of the Radiolaria, 450x. Figs. A, B, are not needed here, and would have been deleted had it not been inconvenient to do so.}
\end{figure}
\clearpage
\begin{figure}[b]
\centering
\includegraphics[width=95mm,keepaspectratio]{figures/Plate22-FigureC.png}
\caption{\small Plate 22: Figure C --- Polished surface of Mazapil meteorite, 5x. The fine dotted or granular markings distinct from lines of the section or from cracks are here and there arranged in circular groups or defined curved bands.}
\end{figure}
\clearpage
\begin{figure}[b]
\centering
\includegraphics[width=95mm,keepaspectratio]{figures/Plate22-FigureD.png}
\caption{\small Plate 22: Figure D --- The same as C, but 175x. Photographed by reflected light. Here the organic and nummulitic structure is unmistakably apparent to the trained eye. (Use of lens 3x desirable.)}
\end{figure}
\clearpage
\begin{figure}[b]
\centering
\includegraphics[width=95mm,keepaspectratio]{figures/Plate22-FigureE.png}
\caption{\small Plate 22: Figure E --- Surface of Jamestown meteorite, 5x. Showing outlines of nummulitic structure, 5x.}
\end{figure}
\clearpage
\begin{figure}[b]
\centering
\includegraphics[width=95mm,keepaspectratio]{figures/Plate22-FigureF.png}
\caption{\small Plate 22: Figure F --- Section of chalk (Upper chalk, Haling). Showing nummulitic structure, 450x. It is very difficult to secure good definition partly owing to halation. Fig. F, though poor, will well repay study. The negative was the best of many attempts.}
\end{figure}
\clearpage
\begin{figure}[b]
\centering
\includegraphics[width=100mm,keepaspectratio]{figures/Plate22-Guide.png}
\caption{\small Plate 22: Guide Diagram}
\end{figure}
\clearpage
\rhead{Plate 23 --- Tertiary nummulite, Ehrenberg's ``morpholiths,'' ``canals'' in \emph{Eozoön}, ``spheres'' in Melbourn Rock.}
\cfoot{\thepage}
\begin{figure}[b]
\centering
\includegraphics[width=95mm,keepaspectratio]{figures/Plate23-FigureA.png}
\caption{\small Plate 23: Figure A --- \emph{Nummulites laevigata}. Vertical section. Inter-pillar area of spiral lamina, showing the tubulated-striated appearance, and faint traces of spirodiscoid structure. 260x.}
\end{figure}
\clearpage
\begin{figure}[b]
\centering
\includegraphics[width=85mm,keepaspectratio]{figures/Plate23-FigureB.png}
\caption{\small Plate 23: Figure B --- \emph{N. laevigata}. Section of a pillar showing faint traces of spirodiscoid structure in the supposed glassy structureless calcite. 1500x.}
\end{figure}
\clearpage
\begin{figure}[b]
\centering
\includegraphics[width=105mm,keepaspectratio]{figures/Plate23-FigureC.png}
\caption{\small Plate 23: Figure C --- \emph{N. laevigata}. Horizontal aspect of spiral lamina showing orifices of tubules. 260x.}
\end{figure}
\clearpage
\begin{figure}[b]
\centering
\includegraphics[width=95mm,keepaspectratio]{figures/Plate23-FigureD.png}
\caption{\small Plate 23: Figure D --- As in C, but 1100x. Showing dotted granular structure and traces of spirodiscoid structure between tubules.}
\end{figure}
\clearpage
\begin{figure}[H]
\centering
\includegraphics[width=75mm,keepaspectratio]{figures/Plate23-FigureE.png}
\caption{\small Plate 23: Figure E --- Ehrenberg's granulated disklets from chalk, showing rings (the latter seen to be beaded under power of 350). 275x. Fig. E'. Other (smaller) examples of E, showing the beaded rings as solid disks (spirodisks). 2000x.}
\end{figure}
\paragraph{}
\emph{Note regarding Fig. ``E,'' as marked on plate}. --- E is the right half of the divided figure, and E' the left half (marked E on plate). See guide diagram.
\clearpage
\begin{figure}[b]
\centering
\includegraphics[width=75mm,keepaspectratio]{figures/Plate23-FigureF.png}
\caption{\small Plate 23: Figure F --- ``Morpholiths'' such as shown in Fig. 5, p. 50; really a mass of spirodisks. 2500x.}
\end{figure}
\clearpage
\begin{figure}[b]
\centering
\includegraphics[width=85mm,keepaspectratio]{figures/Plate23-FigureG.png}
\caption{\small Plate 23: Figure G --- Two ``canals'' in \emph{Eozoön}, showing granular serpentine to contain nummulitic structure continuous with similar fainter structure in the calcite. 450x.}
\end{figure}
\clearpage
\begin{figure}[b]
\centering
\includegraphics[height=110mm,keepaspectratio]{figures/Plate23-FigureH.png}
\caption{\small Plate 23: Figure H --- Melbourn Rock, showing ``spheres'' 450x. The edge of lowest sphere shows banded cord. Very careful study with lens 3x will reveal other coils and ``septa,'' \emph{i.e.} this ``sphere'' is a spirodisk.}
\end{figure}
\clearpage
\begin{figure}[b]
\centering
\includegraphics[width=100mm,keepaspectratio]{figures/Plate23-Guide.png}
\caption{\small Plate 23: Guide Diagram}
\end{figure}
\clearpage
\rhead{Plate 24 --- Spirodiscoid structure in cells of bean and in ganglion cells of human brain.}
\cfoot{\thepage}
\begin{figure}[b]
\centering
\includegraphics[height=110mm,keepaspectratio]{figures/Plate24-FigureA.png}
\caption{\small Plate 24: Figure A --- Growing root-tip of bean, stained with iron-haematoxylin. 2500x. At lower end of the great nucleus in B is a rouleau of three large spirodisks obliquely on edge.}
\end{figure}
\clearpage
\begin{figure}[b]
\centering
\includegraphics[width=75mm,keepaspectratio]{figures/Plate24-FigureB.png}
\caption{\small Plate 24: Figure B --- Growing root-tip of bean, stained with iron-haematoxylin. 2500x. At lower end of the great nucleus in B is a rouleau of three large spirodisks obliquely on edge.}
\end{figure}
\clearpage
\begin{figure}[b]
\centering
\includegraphics[width=75mm,keepaspectratio]{figures/Plate24-FigureC.png}
\caption{\small Plate 24: Figure C --- Growing root-tip of bean, stained with iron-haematoxylin. 1500x. Some of these figures show fairly well the spirodiscoid structure both in nucleus and cytoplasm.}
\end{figure}
\clearpage
\begin{figure}[b]
\centering
\includegraphics[width=75mm,keepaspectratio]{figures/Plate24-FigureD.png}
\caption{\small Plate 24: Figure D --- Growing root-tip of bean, stained with iron-haematoxylin. 1500x. Some of these figures show fairly well the spirodiscoid structure both in nucleus and cytoplasm.}
\end{figure}
\clearpage
\begin{figure}[b]
\centering
\includegraphics[height=110mm,keepaspectratio]{figures/Plate24-FigureE.png}
\caption{\small Plate 24: Figure E --- Growing root-tip of bean, stained with iron-haematoxylin. 1500x. Some of these figures show fairly well the spirodiscoid structure both in nucleus and cytoplasm.}
\end{figure}
\clearpage
\begin{figure}[b]
\centering
\includegraphics[width=95mm,keepaspectratio]{figures/Plate24-FigureF.png}
\caption{\small Plate 24: Figure F --- Growing root-tip of bean, stained with iron-haematoxylin. 1500x. Some of these figures show fairly well the spirodiscoid structure both in nucleus and cytoplasm.}
\end{figure}
\clearpage
\begin{figure}[b]
\centering
\includegraphics[width=75mm,keepaspectratio]{figures/Plate24-FigureG.png}
\caption{\small Plate 24: Figure G --- Growing root-tip of bean, stained with iron-haematoxylin. 1500. Some of these figures show fairly well the spirodiscoid structure both in nucleus and cytoplasm.}
\end{figure}
\clearpage
\begin{figure}[b]
\centering
\includegraphics[width=95mm,keepaspectratio]{figures/Plate24-FigureH.png}
\caption{\small Plate 24: Figure H --- Ganglion cells of grey matter of human brain. 1500x. The pictures are far too dark (and not so good as the negatives). The spirodisk structure is clearly visible in the preparations, for which I have to thank Dr. R. S. Trevor of the Pathological Laboratory, St. George's Hospital.}
\end{figure}
\clearpage
\begin{figure}[H]
\centering
\includegraphics[width=75mm,keepaspectratio]{figures/Plate24-FigureC.png}
\caption{\small Plate 24: Figure J --- Ganglion cells of grey matter of human brain. 1500x. The pictures are far too dark (and not so good as the negatives). The spirodisk structure is clearly visible in the preparations, for which I have to thank Dr. R. S. Trevor of the Pathological Laboratory, St. George's Hospital.}
\end{figure}
\paragraph{}
\emph{Note}. --- The above figures are best viewed under a strong light, and with a lens magnifying only 2 diameters (about 6-inch focus).

If the pictures fail to convince --- though C-G are fairly good --- it is only necessary to examine, under oil immersion, any vegetable or animal cells or tissues to learn that the physical basis of life has a spirodiscoid structure.
\clearpage
\begin{figure}[b]
\centering
\includegraphics[width=100mm,keepaspectratio]{figures/Plate24-Guide.png}
\caption{\small Plate 24: Guide Diagram}
\end{figure}
\clearpage
\end{document}

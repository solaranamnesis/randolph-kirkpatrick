\documentclass[a4paper, 12pt, oneside]{article}
\usepackage[utf8]{inputenc}
\usepackage{fouriernc}
\usepackage{csquotes}
\usepackage{booktabs}
\usepackage{url}
\usepackage{graphicx}
\setlength{\emergencystretch}{15pt}
\graphicspath{ {./figures/} }
\usepackage[figurename=]{caption}
\usepackage{fancyhdr}
\usepackage{amssymb}
\usepackage{array}
\usepackage{float}
\usepackage{imakeidx}
\usepackage{qtree}
\renewcommand{\listfigurename}{List of Plates}
\makeindex[columns=2, title=Alphabetical Index, intoc]
\begin{document}
\begin{titlepage} % Suppresses headers and footers on the title page
	\centering % Centre everything on the title page
	\scshape % Use small caps for all text on the title page

	%------------------------------------------------
	%	Title
	%------------------------------------------------
	
	\rule{\textwidth}{1.6pt}\vspace*{-\baselineskip}\vspace*{2pt} % Thick horizontal rule
	\rule{\textwidth}{0.4pt} % Thin horizontal rule
	
	\vspace{0.75\baselineskip} % Whitespace above the title
	
	{\LARGE THE NUMMULOSPHERE\\ PART 4\\ THE ORIGIN OF \\THE EARTH'S CRUST\\ AND OF METEORITES\\} % Title
	
	\vspace{0.75\baselineskip} % Whitespace below the title
	
	\rule{\textwidth}{0.4pt}\vspace*{-\baselineskip}\vspace{3.2pt} % Thin horizontal rule
	\rule{\textwidth}{1.6pt} % Thick horizontal rule
	
	\vspace{1\baselineskip} % Whitespace after the title block
	
	%------------------------------------------------
	%	Subtitle
	%------------------------------------------------
	
	{By \scshape\Large Randolph Kirkpatrick\\} % Subtitle or further description
	
	\vspace*{1\baselineskip} % Whitespace under the subtitle
	
	%------------------------------------------------
	%	Editor(s)
	%------------------------------------------------
	
	\vspace{1\baselineskip} % Whitespace before the editors

    %------------------------------------------------
	%	Cover photo
	%------------------------------------------------
	
	%\includegraphics[scale=1]{cover}
	
	%------------------------------------------------
	%	Publisher
	%------------------------------------------------
		
	\vspace*{\fill}% Whitespace under the publisher logo
	
	1$^{st}$ Edition, London 1930 % Publication year
	
	{\small William Clowes and Sons, Limited } % Publisher

	\vspace{1\baselineskip} % Whitespace under the publisher logo

    Internet Archive Online Edition  % Publication year
	
	{\small Attribution NonCommercial ShareAlike 4.0 International } % Publisher
\end{titlepage}
\setlength{\parskip}{1mm plus1mm minus1mm}
\setcounter{tocdepth}{3}
\setcounter{secnumdepth}{3}
\tableofcontents
\clearpage
\listoffigures{}
\clearpage
\section*{Introduction}
\paragraph{}
Fourteen years have passed since the publication of Part 3 of the Nummulosphere studies, but the scientific world has entirely ignored the work to its own real and serious loss. That this seemingly ungracious statement is simply true will presently become obvious when it is realized what a vast amount of untenable speculation concerning the nature and origin of the earth's crust has cumbered the literature; and, further, that more progress might have been made in the right direction. Occasionally Nature, in her irony, imposes on babes the task of adding to the wisdom of the wise, and when this happens it would be well for the wise to take heed of the message.

I have been regarded by certain official and academic orthodox authorities as a presumptuous person, a lunatic, a sensation-monger --- epithets that I accepted \emph{tout en bon gré}\footnote{Words used by the adorable Joan of Arc when giving evidence before her judges.} --- for venturing to publish my views on subjects apparently outside my special work, a branch of oceanic biology.

The humour of the situation may become apparent even to themselves, when such authorities realize --- as they will now gladly do --- that oceanic biology is at the very root of the problem of the origin of the earth's crust and of meteorites. The stone that the builders have persistently rejected will be seen to be the ``headstone of the corner'' of planetary architecture.

I would have much preferred waiting a few more years in order to bring out a complete work; but having reached a time of life when the shouldering of great enterprises is apt to prove unduly difficult and irksome, I have decided to divided up the burden and publish from time to time a series of short ``fasciculi.''\footnote{A term used by Sir A. Geikie in a letter to me.}

I think it not amiss to call attention to the financial aspect. Since its beginning, in 1908, this research has cost me much over £2,000, all paid out of a modest salary and pension, and certainly by a cheerful giver. I have been my own publisher.

Part 3, in addition to numerous text-figures, has 24 plates of photographs. Almost the only word of sympathy and encouragement from a professional source was from Sir Archibald Geikie. Acknowledging the receipt of Part 3 he writes, ``I admire the constancy and devotion with which you maintain your ideas.''

A story is told of a man who, for a wager, stood on London Bridge and offered for sale at a penny a piece a tray of golden sovereigns. He sold none, and lost his bet.

The coinage I offer --- nummulites\footnote{\emph{Nummulus}, a small coin; \emph{ite}, suffix.} --- bears the image and superscription of Neptune --- not Pluto --- and will be found current in the realm of truth.

\centerline{*\hspace{15mm}*\hspace{15mm}*\hspace{15mm}*\hspace{15mm}*}
\bigskip

I have been splendidly helped in my work. Mr. James Whitehead and his manager of works, Mr. W. W. Clark, have been especially helpful in supplying thinned slabs of rocks, often beautiful examples of craftsmanship.

Dr. V. E. Pullin, Director of the Research Laboratory at Woolwich, aided me with excellent skiagraphs of siderites and aerolites. The late Dr. Robert Knox, head of the Radio-Therapeutic Department at the Cancer Hospital, Fulham, took a great interest in my work, and enabled me to obtain a fine series of stereoscopic X-ray films. Mr. Heron-Allen and Mr. Earland have helped me considerably from time to time. Lastly I have had continual aid from my colleagues. Indeed, it is a source of satisfaction to me that I may have the good fortune to bring credit to that great institution the British Museum --- to me ever a beloved and beneficent Alma Mater.
\clearpage
\section{Chapter 1: Igneous Rocks}
\paragraph{}
I published my first pamphlet on this subject in 1913, rashly coming into the heated arena with a drawn sword --- ``An account of \emph{so-called} igneous rocks,'' and with a picture of Neptune defiantly holding aloft on his trident a lump of igneous rock. Still worse, threateningly he holds in his other hand a meteorite.

Jove, Neptune and Pluto, the sons of Saturn, settled by lot their respective spheres, but strangely it fell to \emph{my} lot to discover that Jove and Pluto had meanly defrauded their brother. For the so-called Plutonic rocks are all from Neptune's domain, and even the thunderbolt, the supposed appanage of High Jove, is really the property of Neptune. Hence my determination to restore Neptune to his just and legal rights. The igneous rocks may still be ``so-called,'' but the term plutonic is a highly illogical, misleading piece of nomenclature, and I, for one, refuse to accept and use the same, even if I continue to be \emph{contra mundum}.

``But to my task.''

Thanks chiefly to the Imperial Marble Works I have a unique collection of granites, large slabs 18 inches by 12 inches and only $\frac{1}{16}$ inch thick, polished both sides, and smaller bits $\frac{1}{50}$ inch thick and quite transparent.

Further, I have a very large collection of X-ray stereoscopic skiagraphs of granites each 12 inches by 10 inches for the Wheatstone reflecting stereoscope, and 6 inches by 3 inches for the direct vision instruments.

Also I have very numerous photographs taken both by reflected and transmitted light, some photos covering an area of six square feet --- the natural size of huge blocks of granite in quarries. All these objects convey much instruction and leave not a particle of doubt as to the organic nature of igneous rocks. But happily, the wonderful new truth can be learned by men, women and children without all this prodigious preparation. For you have only got to observe \emph{very patiently} and \emph{carefully} the polished slabs of granite so abundantly prevalent in every town, gradually and inevitably to become aware of a strange fact. Your eye and brain will presently discern and recognize obscure shapes amidst the dominating pattern of the rock minerals. To escape this pattern it is well to stand off some distance and dodge the reflections. Just as observers in aeroplanes see areas of ancient sites invisible to those on the ground, so the ancient fossils of the rocks will show better at a little distance. Medium red or grey speckled rocks perhaps show best, polished Swedish Black not all.

Take no notice of the suggestion of idle and prejudiced sitters in the seat of the scornful, who will tell you that you are deceived by optical illusions, that ``suggestion'' will reveal patterns in anything --- as indeed may happen. I who have worked for twenty years at these problems and have examined hundreds of thousands of square feet of granites, smooth and rough-hewn, have learned that the figures are due to the presence of huge discoidal biconvex nummulites from half a foot to three feet in diameter, and seemingly several inches thick. You will see concentric ovals or circles, with radial bands and loops associated with each concentric. The difficulty is at the start. When once you have gained orientation, say, with a fairly well-marked central oval or circle, then you can gradually build up the rest.

Further, if you go right up to the polished granite surface and examine it with a pocket lens, you will presently discover that the minerals are to a great extent arranged in a certain order, \emph{viz.}, of curved bands with V-shaped Turkish-saddle-like loops at intervals. How strange that this, one of the commonest forms in all nature, has never\footnote{At least there is only one figure in literature, \emph{viz.}, in Tschermak's ``Die mikroskopische Beschaffenheit der Meteoriten,'' 1885, Plate 15, fig. 4. 160x.} been detected before. I would advise you to obtain from stonemasons small sample bits of granite polished on one side --- though the unpolished and also fractured surfaces reveal plenty of organic structure also. In future fasciculi I hope to be able to give numerous figures. At present I must limit myself to that on Plate 2, a faithful picture, by a skilled and well-known artist, of the nummulite coils visible in an X-ray stereo-skiagraph of a small block of Norway Syenite or Blue Pearl, shown in Plate 1A. I had a block of Blue Pearl 9 inches square and $\frac{1}{2}$ inch thick, polished both sides, and chopped into nine equal squares. X-ray stereos were taken of each bit and of the nine together. I have a hundred films and photos of films of this lot, and some are good.

When you view of these X-ray films with the direct stereoscope you see a snowstorm about 3 inches deep. With practice, presently, you become aware of dim, ghostly forms of ringed hosepipes, ringed lifebuoys, boa-constrictors, coiled water-spouts, \emph{etc.}; gradually these may become connected into curved bands with V-shapes at intervals.

\centerline{*\hspace{15mm}*\hspace{15mm}*\hspace{15mm}*\hspace{15mm}*}
\bigskip

After prolonged observation I have concluded that these great circular disks are mineralized skeletons of Foraminifera, indeed, there is no other theory possible.

Further, I am convinced that they belong to the eleventh Family of Forams, \emph{viz.}, the Nummulitidae; and I can see no other place for them than in the genus Nummulites of Lamarck.\footnote{Prof. Cushman rejects ``Nummulites'' in favour of Brugiére's ``Camerina.'' The word nummulite, however, now belongs to common language, and is in every dictionary.} I have found not only coils, but even the so-called tubulated structure. The spiral structure, though on a huge scale, is precisely the same in pre-Eocene as in the well-known Eocene nummulites.

\centerline{*\hspace{15mm}*\hspace{15mm}*\hspace{15mm}*\hspace{15mm}*}
\bigskip

A very strange fact in Nature is the predominance of Nummulites almost from the beginning of geological time up to the Eocene, and then their disappearance. I have suggested that their extinction has been brought about by the coming into existence in Jurassic times\footnote{I have found definitely that the supposed Cambrian Globigerinas were not Foaminifera at all.} of a surface Foraminiferal Globigerinid fauna, the corpses of which have rained down and suffocated the nummulites. Possibly even the disappearance of the Ammonites may be in part due to the same cause.

\centerline{*\hspace{15mm}*\hspace{15mm}*\hspace{15mm}*\hspace{15mm}*}
\bigskip

One thousand million years ago the sun, which has been losing eight million tons a minute ever since, was somewhat hotter and heaver than now. Accordingly the earth was nearer to it and day shorter and quicker in return. Therefore there was more photosynthesis, a more abundant surface life, and more food for huge benthos nummulites. Hence the arising of a silicated nummulosphere.

\centerline{*\hspace{15mm}*\hspace{15mm}*\hspace{15mm}*\hspace{15mm}*}
\bigskip

\clearpage
\section{Chapter 2: Limestones}
\paragraph{}
I merely intend to record here that my earlier statements on Chalk and Flint and most other limestones are correct, and that these rocks are nummulitic.

Most flints are simply lumps or areas of limestone permeated by silica, and I have detected the nummulitic structure very often even in pre-Cambrian phthanites. I can see not only the remains of siliceous organisms that furnished the silica, but also the silicified nummulitic structure.

The bulk of the great limestone formations, including chalk, are not, as Prof. Tarr concludes in his valuable studies on flint, chert and chalk, formed by chemical precipitation (as happens in certain calcareous oolitic muds off Florida), but are derived from the skeletons of living organisms. Further, at one time there were no rivers to carry down silica to the sea. The silica must then have been derived mainly from plankton.

I cannot give now the numerous proofs that chalk is formed mainly of nummulites, but I am wholly certain that it is so.

\centerline{*\hspace{15mm}*\hspace{15mm}*\hspace{15mm}*\hspace{15mm}*}
\bigskip

I examined the uppermost Cretaceous or Danian limestones from Denmark and Sweden, samples of which were most kindly sent to me by Prof. Odhner of Stockholm and by Prof. Ravn of Copenhagen, in the hope of finding better preserved nummulites. I found the pre-Eocene type of nummulites. But I do not presume to say that this settles the position of these rocks in the stratigraphic series.

I have myself collected Eocene nummulites in Algeria; in N.E. Spain, where whole cities are built of them, \emph{e.g.} Gerona; also in France, Switzerland and Italy; and have obtained material from Egypt, E. Africa, India and Japan.

No wonder students spoke of the Nummulitic Era and the Nummultitic Enigma.

The solution of the enigma is as follows: The Nummulitic Era should be regarded as the dying away of the almost immortal nummulties. The universal prevalence of these fossils in time and space is due to the universality of ocean life, and living matter is built up spirally and epispirally. This planet is an ocean planet and, from its position in the solar system, a life planet; and the earth's crust is mainly formed of mineralized skeletal \emph{débris}.
\clearpage
\section{Chapter 3: Sediments}
\paragraph{}
Muds, sandstone, grits, gravels, \emph{etc.}, are, of course, simply the particles ground down from solid rocks --- igneous or siliceous. Microscopic examination will reveal that these particles contain nummulitic structure. A sand on the Sussex coast will show the structure of silicified chalk. A sand off the Cornish coast or that of Stromboli will show the same structure in the quartz or in the silicated particles.

A sand deposited in fresh water and containing fresh-water fossils --- say a Red Sandstone --- will yet show the particles of the marine nummulites. Most sediments --- muds, sands, \emph{etc.} --- even though no fossils are visible, are yet fossiliferous throughout. The igneous rocks, the limestone and silica rocks, and the sediments derived from them are one thing --- bathoplankton.\footnote{A better word than benthoplankton (used in Num. 3), from bathos, the bottom life or benthos; and plankton the surface life.}
\clearpage
\section{Chapter 4: Nummulites}
\paragraph{}
In Nummulosphere 3 I gave numerous figures of nummulites. I only intend to record here that extremely careful and prolonged study of the shell of Eocene nummulites has shown me that every particle of the shell is spiral and epi-spiral. In the so-called tubulated walls the rather disk-like spirals are closely packed; but in the usually more transparent pillars the spirals are more helicoidal and more loosely distributed.

\centerline{*\hspace{15mm}*\hspace{15mm}*\hspace{15mm}*\hspace{15mm}*}
\bigskip

I have kept live nummulitids (\emph{Polystomella}) in aquaria for months. They cling to the glass walls in multitudes, many shells surrounded by the enveloping protoplasm. I have never noticed any pseudopods emerging vertically from the living mass but only spread horizontally. Probably this will be found to be true of all glassy ``perforate'' Forams; the so-called ``pores'' are hardly for pseudopods.
\clearpage
\section{Chapter 5: The Ocean}
\paragraph{}
Homer was right when he sang ``Ocean, the parent of all.''\footnote{\emph{Iliad} 14, 246.} For apart from a negligible amount of solids from fresh water and air, all the earth's crust that we know or are ever likely to know from direct observation is derived from the ocean.

The ocean occupies about five-sevenths of the area of the globe; and if all the land were reduced to one level the ocean would cover the world to a depth of two miles.

The earth's crust being derived from the ocean, it is extremely probable that a universal ocean once existed, although we can never know whether bulges of sub-crust may not have emerged from the then deeper ocean.

Life must have originated from the ocean surface. Living matter has an affinity for carbonates of calcium and magnesium and for silica. The vast ocean prairies of the Diatoms and Radiolaria (with their symbionts) have been extracting silica from the sea for over 1,000 million years. Much surface life sank to the bottom, and sometimes adapted itself to an animal mode of nutrition.

I have often seen living and dead Forams full of Diatom frustules --- a microcosm of the earth's crust!
\clearpage
\section{Chapter 6}
\subsection{The Nummulosphere in Geology}
\paragraph{}
The two-sevenths land part of the earth's crust is formed of limestones (including silica), igneous rocks, and of the sediments derived from these two.

If the ocean were whisked off we should see, beyond the fringe of littoral sediments, on the floors of the great basins, 50 million sq. miles of organic oozes (mostly Globigerina) and in deeper areas another 50 million sq. miles of Red Clay. I find the Red Clay to be nummulitic igneous rock disintegrated \emph{in situ}, and not detritus carried by currents.

\centerline{*\hspace{15mm}*\hspace{15mm}*\hspace{15mm}*\hspace{15mm}*}
\bigskip

\subsection{Wegener's Theory of Lateral Movement of Continents}
\paragraph{}
Wegener's theory has received several severe blows lately, especially from Prof. J. W. Gregory, the late Dr. H. von Ihering (on Ocean Land-Bridges) and from Dr. Henry S. Washington in his work on the St. Paul Rocks, but I think the proof of the organic origin of igneous rocks will give this splendid edifice of speculation its \emph{coup de grâce}, for the very foundations are removed from under it. For instance, Wegener figures a section along the earth's circumference through South America and Africa (granitic Sial) in true proportions. The continents, 100 kilometers thick, are represented by two thick black lines, the molten ocean of basaltic Sima being 1,000 kilometers thick. Sial and Sima are assumed to be fundamentally different. But it is now obvious that these rocks are primarily and fundamentally the same, that is to say granite and basalt are both oceanic organic sediments --- mineralized masses of nummulites.

This primary fundamental fact has to be taken as a starting point. We know nothing of the sub-crust. Again, South America is supposed to have broken off and floated away from Africa, leaving the South Atlantic between. But I find the Red Clays of the greatest depths to be simply rotten nummulitic stuff; and, further, the great mass of Teneriffe\footnote{In 1927 I spent three months on a visit to the Canary Islands, including a fortnight near the trachytic top of the Peak, studying the rocks of the seven named islands.} shouldering its way up 24,000 feet through the Globigerina ooze is a mass of nummulites. There must be a fairly thick deposit of these shells below the present floor of the Atlantic.

So we must return to the Land Bridges and the vertical displacements of the emerged and submerged areas of the globe. These areas are fundamentally the same, differing merely in position.

Seeing that nummulitic deposits cannot arise in very deep water, we now know that the floors of the oceanic abyss were once quite shallow and probably above the surface of the sea.

\centerline{*\hspace{15mm}*\hspace{15mm}*\hspace{15mm}*\hspace{15mm}*}
\bigskip

\subsection{Palaeontology}
\paragraph{}
I think it well to call attention here to my work (Nummulosphere 3) on Stromatoporoids, Receptaculites, \emph{etc.}, \emph{etc.} I have carefully gone over it again and find it to be almost wholly correct.\footnote{One correction is required. Years ago I found that Oldhamia is not nummulitic. I had based my first opinion on poor material.} Yet since 1916 good men have been wasting their time writing monographs and papers on these supposed fossils.

These objects are not biological entities at all, but simply concretionary forms and patterns in limestones; and Stromatoporoids, at any rate, have been sufficiently figured. Beneath the surface pattern of pure and less pure areas of calcite I can see clearly the original nummulitic structure, like a palimpsest obscured by later writing.

It is not only life that fashions wonderful shapes. Non-living matter builds crystals; and Darwin speaks of concretionary action as a sort of imperfect crystallization.

It is true a veritable massacre of innocents will be involved. Perhaps rather the innocents should be regarded as Frankenstein monsters engendered out of dead material. Though it will be necessary to remove these lithomorphs from palaeontological literature and collections, they will still be useful in stratigraphy, Nature herself having certain art-periods.
\clearpage
\section{Chapter 7: Petrology}
\paragraph{}
Petrologists who have spent their lives working in laboratories at the chemical and optical properties of igneous rocks find a difficulty --- so far as my experience goes --- in accepting the idea that these rocks are the products of sunshine and sea-water; that these crystalline masses ar enot merely minerals, but mineralized masses of fossils; that the key to the problems they so often write about concerning the real nature and origin of igneous rocks must be sought, not on the land or in laboratories, but on the ocean surface. Petrologists\footnote{V. M. Goldschmidt, for instance, writes of the cooling globe separating into three fluid phases, \emph{viz.}, metall-, sulfid-, and silikat-schmelzfluss. We now know that the silkathülle is an oceanic organic deposit. (Vidensk. Skrift. 1 Math.-nat., 1923. No. 3)} write of the silica shell of the earth, and classify igneous rocks according to silica content. Certainly common sense must now teach us that the silica is derived from plankton creatures, and that this material has mineralized the benthos nummulites. In future, petrologists must take an interest in oceanic biology, for the distribution of minerals in a fossil is influenced to some extent by the structure of the fossil.
\clearpage
\section{Chapter 8: Meteorites}
\paragraph{}
Coming as they do from outer space these bodies have indeed given rise to orgies of speculation, even among the learned. I have set down a few of these theories in Nummulosphere 3.

Sir Robert Ball wrote, ``Every theory of meteorites is in itself improbable, so it seems the only course open to us is to choose the least improbable.'' Accordingly he chose the theory of terrestrial volcanic origin. Even in Nummulosphere 1 (1913) I had found that on biological grounds Sir R. Ball's theory was right. Meteorites are simply mineralized and ore-enriched lumps of terrestrial oceanic organic sediments. The fossils are portions of large nummulites. I give a figure of a piece of the Ausson meteorite that fell on Haute Garonne in 1858, the explosion being seen and heard over a large area.

Not only is the nummulitic structure visible very clearly in the X-ray stereo films, but a good deal can be made out on the surface with a lens. About ten out of twelve stony meteorites are ``chondritic,'' \emph{i.e.}, they have little spheroidal chondrules scattered through them.

I now find these bodies to be simply oolite grains primarily identical in nature with the oolite concretions of Bath Stone and of numerous other limestones throughout the ages.

My long studies of oolite rocks convince me that the oolitic granules are not formed at random, but often have gathered round epispirals that go to build the shells of large nummulites. Accordingly a pattern of curves and V's and parallel lines is often visible in sections. In some rocks the grains become permeated with iron. There are many little circular iron stains round the chondrules of the Ausson meteorite.

As for purely metal meteorites, all that I have examined are simply ore-enriched lumps of fossiliferous stuff. There is nothing very unusual in this fact. Many fossils are purely metallic.

A good skiagraph of the Carlton siderite done for me at Woolwich by Dr. Pullin shows the curved nummulitic bands and V's. But it is quite easy to detect the metal spiro-disks on the surface of most siderites, for instance, the large example just brought from S.W. Africa by Dr. L. G. Spencer.

All the speculation about the Nife (nickel iron) centrosphere of the globe may possibly be true, but we must not look to iron meteorites for evidence. For siderites are lumps of old ocean sediments. Practical miners do not work on the theory that metals get more abundant the deeper they go. The ore may give out very suddenly, leaving investors lamenting.
\clearpage
\section{Chapter 9: Biology}
\centerline{A. \emph{When life began on earth, and where}.}
\paragraph{}
Life must have begun on the surface of the ocean a very long time ago.

I have skiagraphs of Arendal granite\footnote{I have to thank Messrs. Beer, the granite merchants of Stockholm, for sending me samples of this rock.} whence emanated the pegmatite veins containing cleveite. This uranium mineral contains five times as much radio-lead as the pitchblende of Joachimsthal. According to Prof. A. Holmes\footnote{``The Age of the Earth'' (Benn Series), p. 74.} the age of the mineral must be 1,055 million years, the estimate being based on the ticking of the Uranium-Radium-Lead clock. The nummulitic structure is clearly visible to my trained sight. It is not difficult to see with a lens the spirodisk structure in the rock.

Further, I can detect, with difficulty, it is true, organic structure in nine of the siderites mentioned by Prof. Paneth\footnote{\emph{Nature}, March 29th, 1930.} and his collaborators. The ages of some of those bodies are estimated at between two and three thousand million years! These bodies must indeed come from the solar system. It would be absurd, I think, to imagine our earth, relatively the size of a sub-eletron, being hit by numerous bodies from the sidereal system.

As for life being brought here from outside, the theory is intrinsically absurd, though it has been held by distinguished men.

If I saw a heap of coal near the top of a shaft in Newcastle, I should not be ready to assume that that coal came from Timbuctoo.

Life prevails on earth from pole to pole, and from the tops of mountains to the deepest ocean abysses. Further, the whole crust is a mineralized mass of life-material. Why in the name of logic and commonsense seek elsewhere for the origin of living matter?

According to Sir. J. H. Jeans life must be a very exceptional phenomenon in the universe.

\centerline{B. \emph{Structure of living matter}.}

Not the least strange anomaly concerning these nummulosphere studies is the fact that the examination of igneous rocks should have led to the discovery of the structure of living matter. For I soon found that the peculiar spiralities and epispiralities were the exact models of what was to be found in soft protoplasm. In fact, the calcareous skeleton of a nummulite and the siliceous frustule of a diatom are not merely mineral stuff but are actually living matter with, say, 99.9 per cent. of mineral matter in them.

The skeletons are not secreted or excreted by protoplasm: they are protoplasm itself with all its marvellous structure.

The finding of this truth soon led to interesting and important discoveries.

The markings in diatom frustules have puzzled generations of skilled observers, and have not even yet been correctly explained. But a careful examination of a \emph{Coscinodiscus} --- with the spirality theory in mind\footnote{The illustrious Faraday used to say, ``Tell me what I am to see!''} --- will enable the observer to see that the polygonal facets are simply a mosaic of spirodisks (slightly helicoidal) with thick periphery and very thin center. Round the edges of the frustule are a series of lovely corkscrew spirals. In \emph{Synedra} there are two helicoidal spirals in each valve face, the bars being ``epispiral.'' I make out the raphe to be tubular with a spiral funnel at each end. (I believe this tube is for a flagellum; if so, a \emph{Synedra} or \emph{Navicula} would be four-flagellate. I watched living diatoms moving about for hours, but failed to detect flagella outside the frustule. A \emph{Coscinodiscus} might be multi-flagellate with a single radial flagellum between adjoining peripheral helices.)

A further interesting discovery was that the supposed pores in perforate glassy Forams are not pores at all, but spirodisks with very transparent centers.

It is true, as Mr. Earland kindly warned me, that in fossil and glauconitic casts of Forams one gets models of little pipes or cylinders, but this results from the centers being more permeable than the peripheries. Further, even the ``meshes'' of Radiolaria are really filled in with spirodisks.

\centerline{*\hspace{15mm}*\hspace{15mm}*\hspace{15mm}*\hspace{15mm}*}
\bigskip

Mr. Heron-Allen relates that Sir E. R. Lankester told him a story of meeting T. Schwann, the founder of the cell theory at a congress in Belgium. Schwann told him he had devoted the whole of his time for many years to an endeavour to establish the existence of any structure in protoplasm --- and that his investigations had been absolutely without result.

In Nummulosphere 3 I have alluded to the various theories concerning the structure of protoplasm, \emph{viz.} (to mention a few) granular, fibrillar, reticular, spherular, alveolar (foam), spiral (Fayod), also the view that there is no true structure at all (Wilson, Schwann), and I find much truth in all of them, especially in Fayod's theory. There is one theory, however, and I am certain it is the true one, that will reconcile all opposing and varying views; and that is the ``epispiral theory.''\footnote{Compare ``Cycle and Epicycle, Orb in Orb,'' Milton Paradise Lost Book 8.} Protoplasm is built on a spiral plan, the strands of the spirals themselves being spiral \emph{ad infinitum}.

I think this is the most important result of my twenty-two years' work. Not only is the spirality visible in protoplasm of Protozoa, but in the cells of Metazoa and in plants.

The elusive spirality and epispirality only come to view with patient observation. Every well-preserved, stained nucleus shows it, and the cytoplasm also.

I find the medullated nerves to be spiral throughout. I believe the nodes of Ranvier are little shock-absorbers or, perhaps, structures to allow of stretching when the vital stimuli traverse the cable.

Voluntary muscle fiber is Nature's supreme achievement in organized spirality. I am certain I can clearly make out the spirals in transverse sections of the fibers; further, the Bowman's rods are simply epispirals in which I can even detect epi-epispirals transverse to the longitudinal bars forming the transverse stripes.\footnote{Recently I have been much interested to hear that an Australian student is bringing out a work on the ``spiral-helicoidal'' structure of striped muscle.}

\centerline{*\hspace{15mm}*\hspace{15mm}*\hspace{15mm}*\hspace{15mm}*}
\bigskip

I have already ventured to put forward an enantiomorph theory of sex --- that mysterious phenomenon prevailing almost throughout the kingdom of life. I was led to this view by studying Pasteur's research on the tartaric acids. In the organic world the spiralities are either right or left, and it seemed to me this feature would be seized upon by Nature to bring about attractions of opposite individuals. The enantiomorph\footnote{Opposite form.} theory recalls the theory of love expounded by Aristophanes in the Symposium (Plato). Humans once had four arms and four legs and were joined back to back; they moved by spinning round. The creatures became so obstreperous that they stormed Olympus, and Jove clave them in twain with thunderbolts. Ever since each seeks its other half. In protoplasm the opposites are right and left.

\centerline{*\hspace{15mm}*\hspace{15mm}*\hspace{15mm}*\hspace{15mm}*}
\bigskip

I would call attention to a fascinating study of spiral movement in man, by A. A. Schaeffer, in \emph{Journal of Morphology}, \emph{Philadelphia}, vol. 45, 1928, and to numerous papers of his on spiral movement in Amoeba and Infusorians. It would appear that the senses of hearing, sight and balance may possibly have arisen in response to the need of curbing a spiral tendency in moving organisms.
\clearpage
\section{Plates and Guide Diagrams}
\paragraph{}
Note. --- The ocular combinations used were Zeiss 1.5 mm., oc 18; 2mm., oc 18; and 16 mm., oc 18. Unfortunately level of drawing board was not recorded, or the record lost.
\clearpage
\pagestyle{fancy}
\fancyhf{}
\rhead{Plate 1}
\cfoot{\thepage}
\begin{figure}[b]
\centering
\includegraphics[width=\textwidth,keepaspectratio]{figures/Plate1-FigureA.png}
\caption{\small Plate 1: Figure A --- Piece of Norwegian Syenite, 4 $\frac{1}{2}$ inches square by $\frac{1}{2}$ inch thick (nat. size). See drawing of X-ray skiagraph on Plate 2.}
\end{figure}
\clearpage
\begin{figure}[b]
\centering
\includegraphics[width=80mm,keepaspectratio]{figures/Plate1-FigureB.png}
\caption{\small Plate 1: Figure B --- Fragment of the ``Ausson'' meteorite (nat. size). See drawing of X-ray skiagraph on Plate 3.}
\end{figure}
\clearpage
\begin{figure}[b]
\centering
\includegraphics[width=80mm,keepaspectratio]{figures/Plate1-FigureC.png}
\caption{\small Plate 1: Figure C --- An area of B, about $\frac{1}{2}$ inch square near upper end, magnified 4 diam. to show ``chondrules.''}
\end{figure}
\clearpage
\rhead{Plate 2}
\cfoot{\thepage}
\begin{figure}[b]
\centering
\includegraphics[width=\textwidth,keepaspectratio]{figures/Plate2.png}
\caption{\small Plate 2 --- Figure drawn by Miss Gertrude M. Woodward from an X-ray stereoscopic film photo or skiagraph, of a piece of Norwegian Syenite or ``Blue Pearl'' granite (see Plate 1, fig. A). Magnified about 1.5 diam.}
\end{figure}
\clearpage
\rhead{Plate 3}
\cfoot{\thepage}
\begin{figure}[b]
\centering
\includegraphics[height=120mm,keepaspectratio]{figures/Plate3.png}
\caption{\small Plate 3 --- Figure drawn by Miss Gertrude M. Woodward from an X-ray stereoscopic film photo (skiagraph), of a piece of the Ausson meteorite (see Plate 1, fig. B) in Mr. J. R. Gregory's collection. Magnified about 2 diam.}
\end{figure}
\clearpage
\rhead{Plate 4}
\cfoot{\thepage}
\begin{description}
\item Figure A --- Central meshes of \emph{Coscinodiscus omphalantha}, showing them filled in with spiral coils.
\item Figure B --- A mesh at different focus, showing an outer coil of spiral loops; both magnified about 2,000 diam.
\item Figure C --- \emph{Synedra ulna}, showing helicoidal form of the two rows of bars in valve aspect of a frustule; magnified about 2,000.
\item Figure D --- \emph{Synedra splendens}, showing ``epi-spiral'' loops of the bars; magnified about 3,000 diam.
\item Figure E --- \emph{Planorbulina}, a Foraminiferan, fragment showing the supposed pores or perforations of this ``perforate'' Foram; magnified about 250 diam.
\item Figure F --- The same; magnified about 2,000 diam., showing the spiral infilling of the supposed pore.
\item Figure G --- One mesh in a Radiolarian skeleton, showing spiral infilling; magnified about 2,000 diam.
\end{description}
\clearpage
\begin{figure}[b]
\centering
\includegraphics[height=150mm,keepaspectratio]{figures/Plate4.png}
\caption{\small Plate 4: Figures showing spiral infillings of ``meshes'' and ``pores'' of Diatoms, Perforate Forams and Radiolaria}
\end{figure}
\clearpage
\end{document}

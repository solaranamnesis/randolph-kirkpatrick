\documentclass[a4paper, 12pt, oneside]{article}
\usepackage[T1]{fontenc}
\usepackage{aurical}
\usepackage{csquotes}
\usepackage{booktabs}
\usepackage{textalpha}
\usepackage{url}
\usepackage{graphicx}
\setlength{\emergencystretch}{15pt}
\graphicspath{ {./figures/} }
\usepackage[figurename=]{caption}
\usepackage{fancyhdr}
\usepackage{amssymb}
\usepackage{array}
\usepackage{float}
\usepackage{imakeidx}
\usepackage{qtree}
\usepackage{microtype}
\renewcommand{\listfigurename}{List of Plates}
\makeindex[columns=2, title=Alphabetical Index, intoc]
\usepackage{sectsty}
\usepackage[titles]{tocloft}

\allsectionsfont{\Fontauri}
\sectionfont{\Fontauri\Huge}
\subsectionfont{\Fontauri\LARGE}
\subsubsectionfont{\Fontauri\Large}

\begin{document}
\Fontauri
\renewcommand{\contentsname}{
\Fontauri{Index}
}

\renewcommand{\cftfigfont}{\Fontauri}
\renewcommand{\cftfigpagefont}{\Fontauri}

\renewcommand{\cftsecfont}{\Fontauri}
\renewcommand{\cftsubsecfont}{\Fontauri}
\renewcommand{\cftsubsubsecfont}{\Fontauri}

% fix toc page numbers
\let\origcftsecfont\cft
\let\origcftsecpagefont\cftsecpagefont
\let\origcftsecafterpnum\cftsecafterpnum
\renewcommand{\cftsecpagefont}{\Fontauri{\origcftsecpagefont}}
\renewcommand{\cftsecafterpnum}{\Fontauri{\origcftsecafterpnum}}
\let\origcftsubsecpagefont\cftsubsecpagefont
\let\origcftsubsecafterpnum\cftsubsecafterpnum
\renewcommand{\cftsubsecpagefont}{\Fontauri{\origcftsubsecpagefont}}
\renewcommand{\cftsubsecafterpnum}{\Fontauri{\origcftsubsecafterpnum}}
\let\origcftsubsubsecpagefont\cftsubsubsecpagefont
\let\origcftsubsubsecafterpnum\cftsubsubsecafterpnum
\renewcommand{\cftsubsubsecpagefont}{\Fontauri{\origcftsubsubsecpagefont}}
\renewcommand{\cftsubsubsecafterpnum}{\Fontauri{\origcftsubsubsecafterpnum}}

\renewcommand{\thefigure}{\Fontauri{\arabic{figure}}}
\renewcommand\thefootnote{\Fontauri{\arabic{footnote}}}

\begin{titlepage} % Suppresses headers and footers on the title page
	\centering % Centre everything on the title page
	\scshape % Use small caps for all text on the title page

	%------------------------------------------------
	%	Title
	%------------------------------------------------
	
	\rule{\textwidth}{1.6pt}\vspace*{-\baselineskip}\vspace*{2pt} % Thick horizontal rule
	\rule{\textwidth}{0.4pt} % Thin horizontal rule
	
	\vspace{0.75\baselineskip} % Whitespace above the title

        {\Huge The Nummulosphere\\ Part 1\\ An Account of \\ the Organic Origin of\\ so-called Igneous Rocks\\ and of\\ Abyssal Red Clays \\} % Title
	
	\vspace{0.75\baselineskip} % Whitespace below the title
	
	\rule{\textwidth}{0.4pt}\vspace*{-\baselineskip}\vspace{3.2pt} % Thin horizontal rule
	\rule{\textwidth}{1.6pt} % Thick horizontal rule
	
	\vspace{1\baselineskip} % Whitespace after the title block
	
	%------------------------------------------------
	%	Subtitle
	%------------------------------------------------
	
	{By \scshape\Large Randolph Kirkpatrick\\} % Subtitle or further description
	
	\vspace*{1\baselineskip} % Whitespace under the subtitle
	
	%------------------------------------------------
	%	Editor(s)
	%------------------------------------------------
	
	\vspace{1\baselineskip} % Whitespace before the editors

    %------------------------------------------------
	%	Cover photo
	%------------------------------------------------
	
	%\includegraphics[scale=1]{cover}
	
	%------------------------------------------------
	%	Publisher
	%------------------------------------------------
		
	\vspace*{\fill}% Whitespace under the publisher logo
	
	1\textsuperscript{st} Edition, London 1913 % Publication year
	
	{\small Lamley \& Co. } % Publisher

	\vspace{1\baselineskip} % Whitespace under the publisher logo

    Internet Archive Online Edition  % Publication year
	
	{\small Attribution NonCommercial ShareAlike 4.0 International } % Publisher
\end{titlepage}
\pagestyle{fancy}
\fancyhf{}
\cfoot{\Fontauri{\thepage}}
\Large
\vspace*{\fill}
\begin{quote} 
``Sic lapides ab animalibus, nec vice versa.

Sic rupes saxei non primaevi, sed temporis

filiae.'' --- \emph{Systema Naturae} (Ed. 6. [2].-9.).

Linnaeus.
\end{quote}
\vspace*{\fill}
\clearpage
\setlength{\parskip}{1mm plus1mm minus1mm}
\setcounter{tocdepth}{3}
\setcounter{secnumdepth}{3}
\tableofcontents
\clearpage
\section*{Introduction}
\paragraph{}
The discovery of the original organic nature of the igneous rocks was led up to by finding that \emph{Eozoön} was a Stromatoporoid,\footnote{\Fontauri{Στρῶμα layer, πὁρος pore, Stromatoporoids being formed of organically connected \emph{layers} of nummulites, each shell having a \emph{central perforation}, the umbilicus. The name was given to certain Palaeozoic fossils having a ``layered'' structure and peculiar pores on the surface. Many theories have been put forward to explain the nature of these layers and pores.}} \emph{i.e.} a colonial nummulite. Accordingly it is essential to give a brief preliminary account of \emph{Eozoön canadense} (see Chapter 1.).

The mere mention of the name of the Dawn Animal seems not infrequently to excite violent igneous action, a phenomenon which should not be evoked unnecessarily. With a sense of the responsibility incurred, I do not hesitate to declare that now there is convincing evidence as to the real nature of that organism around which so much controversy has raged. The modern scientific world is almost unanimous in its belief that \emph{Eozoön}\footnote{\Fontauri{\emph{Eozoön} is a word of four syllables.}} is of purely mineral origin. In a memoir on the ``Azoic System,'' two writers refer\footnote{\Fontauri{\emph{Bull. Mus. Comp. Zool.} Harvard, 1884, 7. p. 534. J. D. Whitney and M. E. Wadsworth. ``The Azoic System.''}} to the ``extraordinary delusion which has prevailed among palaeontologists with reference to the organic nature of \emph{Eozoön}.'' Such delusions, in the case of those possessed by them, are charitably assumed to be supplemented by the desire to do something sensational. The \emph{coup de grâce} is supposed to be given to the organic theory because certain writers seem to have carried their conclusions to absurd lengths. Dr. Otto Hahn, who wrote about the organisms found in meteorites and on the vegetable nature of \emph{Eozoön}, comes in for special mention. But, as Herbert Spencer observed (`First Principles,' \emph{ed.} 1884, p. 568), when he was being baited by hostile critics, there is an essential difference ``between intrinsic absurdity and extrinsic absurdity.''\footnote{\Fontauri{I have found Dr. Hahn's writings very interesting. He was nearer the truth than were his critics. He regarded \emph{Eozoön}, granite, meteorites, \emph{etc.}, as mainly composed of plants. The branching canals of \emph{Eozoön} are consigned to several genera of algae. Dr. Hahn did not allow for the changes produced by mineralisation. The oogonia of ``Chilocarpon'' shown on Pl. 10. of ``Die Urzelle'' are probably young nummulite buds, the supposed cell-nuclei being pores or septal apertures. The umbilicus is shown in each bud.}}

Then, again, these writers accuse zoologists and palaeontologists of entire ignorance as to the great variety of forms occurring in the mineral kingdom, forms ``which it only requires an imaginative temperament to endow with the attributes of organic structure.'' That may be so, but I sometimes find in petrologists a lack of knowledge of the variety of forms of organic life, and a tendency to mistake organic structures, --- as, for example, the branching canals of \emph{Eozoön} --- for purely mineral formations. They are so engrossed in the difficult study of the mineralogical characters of rocks that they seem unwilling to regard these objects from other points of view. It almost irritates them to see a zoologist prying about with a magnifying glass, apparently with nefarious designs on their property. They resent the suggestion that about half their territory should be annexed by palaeontology, and naturally they are impatient when they see some palaeontologist trespassing in their galleries and asking to look at the volcanic rocks representing the types of \emph{Eozoön atlanticum, pacificum, etc.}\footnote{\Fontauri{In presenting to the British Museum the collection of rocks I made at Madeira and Porto Santo Island, I shall be confronted with a dilemma more perplexing than one which Solomon had to face on a certain occasion. When he was called upon to settle a dispute between two women as to the ownership of a babe he had a simple problem to deal with, for the child actually belonged to one of the claimants. I am in doubt, however, how to settle the rival claims of the mineralogical and palaeontological departments. For the specimens are examples of \emph{Eozoön} (or \emph{Stromatopora}) \emph{atlanticum} in the form of lava, basalt, trachyte, granitoid rock, \emph{etc.} Perhaps the best solution will be to cut the collection into two halves.}}

Evidently, what the above quoted authors did not see is not to be allowed to exist, and anyone who dares assert that he finds evidence of the former existence of organic structure in igneous rocks will incur Athanasian penalties, and his book will be entered on the said authors' ``Index'' of ``gems.''

I believe that many conclusions of these writers are based on \emph{a priori} conceptions. They write (\emph{l.c.} p. 526) ``rocks must necessarily be azoic, when formed or originating under such conditions as were incompatible with life.'' Certainly this conclusion must be accepted. Then they go on to say, ``The original crust of the earth must have been azoic, if we adopt the views held by the large majority of geologists that our globe has cooled down from a former condition of igneous fluidity. The volcanic and eruptive rocks must necessarily be azoic because they have come from the heated interior of the globe, reaching the surface in a melted condition. We do not, however, designate the eruptive and volcanic rocks as `Azoic'; the fact that they are necessarily in this condition is assumed as something self-evident.'' On the contrary the original \emph{organic} nature of these rocks is to me self-evident, because I can see the Foraminiferal structure in them, and often very clearly indeed. There is no need, I believe, to bring the nature of the earth's interior into the question. The igneous rocks are mainly masses of crystallised silicates. Numerous fossils of later age, with skeletons originally consisting of pure calcium carbonate become masses of silica. The earliest formed Foraminiferal rocks often became changed into silica compounds. After this change had taken place, there came periods of disturbance of the earth's crust, owing to the cooling and contraction of the warm sphere in its medium of cold space. The compression and re-adjustment of the crust caused a rise in temperature sufficient to reduce to a more or less molten state the silicated limestones, but in spite of this the intractable Foraminiferal medals often retain to a considerable extent their arrangement and pattern, as can be seen by careful and patient observation.

The supposed Azoic crust of the earth is Eozoic down to its known foundations and is, as Dana stated, ``the only universal formation.'' The known part of the earth's ``rind'' is only a pellicle a few miles thick, and what is that in comparison with the diameter of the globe?

The above remarks are not written in an aggressive spirit. I believe `The Azoic System'\footnote{\Fontauri{According to my observations there is no ``Azoic'' System. I hope that in place of ``breathing out threatenings and slaughter'' against heretical ``Eozoists,'' believers in the Azoic doctrine will now become apostles of the organic theory. The term ``Eozoic'' is to be preferred to ``Archaean,'' introduced by Dana.}} to be a splendid piece of work reflecting great credit on the authors, but I am now in a position to say with certainty that they are mistaken in their opinion that \emph{Eozoön canadense} is of purely mineral origin. They and the scientific world in general are not to be blamed for their error, for correct ideas concerning the structure of \emph{Eozoön} have only recently been arrived at. I find that I myself, in spite of having had exceptional opportunities for learning the truth, have misinterpreted the character and orientation of the coiled disks of which this colonial Foraminiferan is built up, and one of the objects of this paper is to correct the mistake.

Since the personal factor has to be taken into consideration I would mention that for many years I have been in charge of one of the most important collections of Foraminifera in existence. Further, I have observed igneous phenomena of the volcanic kind in action in various parts of the world, \emph{viz.}, Italy, Sicily, Lipari Islands, Java and Teneriffe, and know something of their effects. I once spent nearly three days and a night on the summit of Stromboli, and there learned how various were the phenomena comprised under the term ``igneous action.''

It has taken over eighty years to settle the problem of the nature of Stromatoporoids. Almost by accident, I discovered the clue to the labyrinth, by noting the systems of Foraminiferal chambers coiled each round its respective umbilicus, and found that these mysterious fossils were colonial Foraminifera built up of nummulitic shells. I little realised at the time the developments that would follow, and that the clue to the nature of \emph{Eozoön} and of the igneous rocks had been found. Again, it has taken fifty years to settle the \emph{Eozoön} problem, simply because an important link in the chain of evidence, \emph{viz.}, that of the coiled Foraminiferal shell, was missing. There is no need, then, to suspect that the theory here expounded is like Jonah's gourd. It will be found that the Stromatoporoid theory is based on very solid foundations.

If the statements here put forward concerning the origin of igneous rocks are doubted, I can only repeat what Desmarest used to say to importunate Neptunians, ``Go and see.''\footnote{\Fontauri{Quoted in Lyell's `Principles of Geology,' ed. 12., vol. 1. p. 72.}} If again it should be thought that the nummulitic disk is a phantom, I ask how does it come about that I am able by its means to solve, as I believe, important problems in palaeontology. For the disks have furnished the clue to the mystery\footnote{\Fontauri{The word ``mystery'' is used, because many of the limestones formed during the successive eras have been regarded hitherto simply as limestones.}} of the Dolomitic and Oolitic Limestones. It was a specimen of \emph{Eozoön canadense} that led me on to the solution of the problem of the Jurassic Oolites.

As a result of many thousands of observations I have found that the plutonic rocks are metamorphosed nummulitic limestones silicated,\footnote{\Fontauri{A term used in a general sense to include change into silicates of various kinds, \emph{viz.} of alumina, magnesium, with soda, potash, silica, iron, \emph{etc.} ``Silicified'' implies conversion into silica.}} reduced to a more or less molten condition, and crystallized, and that volcanic rocks have had an identical origin with the plutonic. Certain Ordovician, Liassic, Permian and Jurassic limestones are also nummulitic, the Dolomitic being dolomitised, and Oolitic scarcely altered chemically.

Several writers have described traces of organic structure in igneous rocks, but they have failed to convince anyone as to the soundness of their observations, but the fact of the nummulitic nature of these rocks will soon, I believe, be generally admitted, and fortunately it is possible for anyone to look into the matter himself, the only equipment necessary being patient observation and a good lens magnifying 10 diameters.

The disks may be seen separate, or overlapping obliquely in rows, or as rouleaux in vertical section. They may be closely pressed together and flattened out like rolled oats. Each disk will show \emph{en face} the central umbilicus, a concentric and radial pattern, and frequently rows of pores. On vertical section the very minute regular network pattern, apparently formed by the edges of alar prolongations and septa, will be visible on each side of the umbilicus. I have observed all these structures in many igneous, Jurassic oolitic and Permian dolomitic rocks. Sometimes I have found it necessary to examine a fragment of rock with the closest scrutiny for hours before convincing myself that I have seen all the above-mentioned details. In the end I have come across structures that cannot possibly have arisen from a mere chance agglomeration of particles, but which must be the expression of original organic structure. The disks are usually best seen in a good side light from above. A bright light shining directly on the piece of rock that is being examined is apt to bring into relief the particles or grains which make up the Foraminiferal medals rather than the medals themselves. Impressionist pictures, when too closely viewed in a glare of light, show unduly the blobs of paint rather than the picture. A close scrutiny of the rocks is desirable, however, in all lights, direct and oblique, bright and less bright.

As regards methods, I carefully examine pieces of rock, firstly with the naked eye, then with an aplanatic lens magnifying 10 diameters, then with a Zeiss binocular magnifying 20 to 70 diameters, and lastly in thick and thin sections with high powers.

The lens is the most ready and satisfactory. For generally one is dealing with altered, ruined, and blasted structures, which are easier seen as objects of three dimensions than in thin sections.

My views on igneous and certain other rocks have been received with a good deal of scepticism, and this is not surprising. For here, almost at our doors is a new continent, the existence of which has never been suspected. I first sighted the new land in September 1912 (when I found that the Palaeozoic Stromatoporoids were colonial Foraminifera and in that \emph{Eozoön} was a Stromatoporoid). What at first appeared to be an island turns out to be a continent. It is not to be wondered at that I, who say so, am looked upon as a Munchausen, or as a person subject to delusions or entoptic troubles. For my part, I do not think it will be necessary to return to the theory that the Jurassic Oolites are made of grains of coral detritus derived from phantom coral reefs, or that certain Ordovician, Carboniferous and Permian (Dolomitic) limestones and igneous and gneissic rocks are of an indefinite nature.

I hope in course of time to issue a second part to this pamphlet, giving photographs, drawings and diagrams of nummulitic disks found in various rocks.

\centerline{*\hspace{15mm}*\hspace{15mm}*\hspace{15mm}*\hspace{15mm}*}
\bigskip

The design on the cover represents Neptune on the globe of waters. On one of the prongs of his trident is a piece of volcanic rock in the shape of a nummulitic disk, and in his hand is a meteorite. These emblems signify that Neptune's domain is enlarged not only at the expense of nether Jove, but also at that of high Jove whose supposed emblem of sovereignty --- the thunderbolt --- really belongs to the Sea God. The glow of volcanic fires results from the hammering (compression due to planetary shrinkage) of Neptune's material on the anvil of Vulcan. That it is Neptune's material that is being worked upon, is rendered clear by the image and superscription on the little medals or nummuli coined in the Sea God's mint, and easily to be deciphered by those skilled in this branch of numismatics. Neptune's bolt is poised ready to be hurled at rash and ignorant mortals of the type of the \emph{a priori} would-be refuter, daring to dispute the validity of his title-deeds.

The design which has been drawn by Mr. Highley, has been partly copied from prints in the British Museum.

\centerline{*\hspace{15mm}*\hspace{15mm}*\hspace{15mm}*\hspace{15mm}*}
\bigskip

In the course of this investigation I have received a great deal of information and many gifts of specimens from owners of quarries and of \emph{depôts} of stone, pumice and mica, who have shown me much kindness, which I very gratefully acknowledge. I wish especially to express my gratitude to Senhor A. C. Noronha, who gave me the fragments of volcanic rock that started me on the present enquiry.

I regret that I have not been able to marshal all my facts in strictly logical order, this failure being due to the continual cropping up of new discoveries after I had revised the first and second proofs of my pamphlet. Some allowance will, I trust, be made on account of the difficulties attending what may fairly be termed pioneer work in an almost unexplored territory.
\clearpage
\section{On \emph{Eozoön Canadense} Dawson}
\begin{quote} 
``A false theory, it is well known, may render us blind to facts which are opposed to our prepossessions, or may conceal from us their true import when we behold them.'' --- Lyell, `Principles of Geology,' ed. 12., vol. 2., p. 179.
\end{quote}
\paragraph{}
I do not intend to enter here into details concerning the history of the \emph{Eozoön} controversy, and shall merely refer to a few of the more important points.

The original specimens of \emph{Eozoön} were found by the officers of the Geological Survey of Canada in the Lower Laurentian Limestone rocks about 1850-1858. These specimens had a layered aspect at their weathered edges; and a vertical section revealed alternating white and greenish-yellow bands. The white bands were found to be calcareous (calcite or dolomite) and the green bands siliceous (loganite, olivine, serpentine, \emph{etc.}). It seemed unlikely that any traces whatsoever of organic structure could exist in rocks of unimaginable antiquity, vastly older than the Cambrian, and in rocks which had been exposed to terrific metamorphosing agencies. As Sir W. Dawson observed, some of the gneisses between which the limestones were imbedded had been squeezed and twisted as if they had been put through a mangle. At the same time he and Sir W. Logan were struck with the resemblance between these layered and banded lumps of rock and the Palaeozoic Stromatoporoids. Accordingly Dawson concluded they were really fossils and gave them the name \emph{Eozoön canadense}. Specimens were brought to London, and investigated by Dr. W. B. Carpenter, who endorsed the verdict of Dawson.

Carpenter regarded the greenish bands as chambers of a Foraminiferan, and the white bands as secondary skeleton. He described canals ramifying in the white bands and showed their similarity to the canals of the secondary skeleton of recent Foraminifera.

Specimens were now referred to Prof. Carl Möbius of Kiel, who had published several memoirs on Foraminifera.

He wrote a splendid monograph\footnote{\Fontauri{`Palaeontographica,' vol. 25., 1878, p. 175.}} on \emph{Eozoön} illustrated by eighteen beautiful plates, and his verdict was against the theory of organic origin. I was much struck with the fact that among the forty-seven figures illustrating the structure of \emph{Eozoön}, I could only find one (Pl. 34. Fig. 44) which showed something that might possibly be regarded as evidence of organic structure. It is not to be wondered at that the distinguished Professor of Kiel was unable to accept the ``organic'' theory. If he had seen certain rings and coils in the branching canals, probably he would have hesitated before giving his verdict in favour of the mineral theory, and if he had seen under the microscope the young Foraminiferal shells shown in the Plate at the end of this pamphlet, he would most certainly have given the whole weight of his influence in support of the theory of organic origin.

Professors Rowney and King of Galway likewise proceeded to demolish the theory of organic origin, and ransacked the mineral kingdom for minerals which appeared to show all the varieties of structure which Carpenter had found in \emph{Eozoön} and which he had brought forward as proofs of organic origin.

Later, Professors J. W. Gregory and Johnston-Lavis published a memoir on the Monte Somma bombs, which had a structure almost identical with that of \emph{Eozoön}. Here apparently was the \emph{coup de grâce} to the organic theory. Was anyone going to be so insane as to assert that fossils retaining any trace of organic structure could be hurled up as bombs from a volcano?

In the \emph{Annals}, October 1912, I stated that these bombs actually were examples of \emph{Eozoön}, and so they undoubtedly are. I was mistaken in assuming that the broken-down siliceous bands were composed of serpentine. At that time I had not read the authors' fine memoir on the Eozoonal structure of the ejected blocks of Monte Somma.\footnote{\Fontauri{The authors point out (\emph{Trans. Roy. Dublin Soc.}, 5., 1894, p. 266) that as regards the alternating silicate and calcite laminae the bombs completely resemble \emph{Eozoön}. The silicate laminae are composed typically of peridote and spinel, and do not undergo serpentinization. In the \emph{Annals Mag.}, October, 1912, p. 452, I recorded the finding of \emph{Melobesia} and of obscure traces of Polyzoa in a Monte Somma bomb. I have now confirmed this discovery. Two botanists to whom I showed a mounted fragment of the supposed alga, thought it resembled vegetable tissue. I have found traces of more than one species of Polyzoa. I remarked that the presence of these organisms seemed to point to the conclusion that Monte Somma had been under the sea, but it is not certain where the bomb came from. Professors Johnston-Lavis and Gregory state that most of the bombs are found among deposits of pumice forming a mantle round Monte Somma, especially amidst the material ejected during Phase 6 and Phase 7 (Plinian eruption A.D. 79). The bomb encrusted with organisms belongs to the Hamilton Collection in the Natural History Museum. Of course, it may have been picked up at some distance from Monte Somma, but I think not, because the encrusting organisms show traces of fusion, and must have been among hot scoriae, \emph{etc.} So it is possible that before the historic period Monte Somma may have been submerged. If so, it would not be a very exceptional circumstance. The columns of the Temple of Serapis, a few miles to the west, have been rising and falling frequently like a pressure gauge. When Vesuvius is quiet they rise, and after periods of activity they sink (Lyell, `Principles,' ed. 12., 2., p. 171). The south-west base of Vesuvius from Portici to Torre del Annunziata is by the margin of the sea.}}

Scientific men were now satisfied that \emph{Eozoön} was an object of purely mineral origin, and were probably thankful to have the matter settled. In none of the standard works (Zittel, Steinmann, Hartog, Lister, Geikie, Bütschli), is \emph{Eozoön} accepted as an organism.

Sir Archibald Geikie (`Text-Book of Geology,' ed. 4., 1903, p. 879) maintained a position of judicious reserve amidst the clash of opposing views. Möbius had stated that the canals and passages were infiltration veinings of serpentine in calcite. Geikie pointed out, however, that sometimes the canal-systems are filled in with dolomite, and therefore must have been hollow before either serpentine or dolomite were introduced; and, further, he observed that no structure precisely similar to that of some specimens of \emph{Eozoön} had been discovered. On the other hand, he demanded that evidence brought forward in support of the organic theory should be clear and indubitable.

In the course of an investigation of the \emph{Monticuliporoids}, I was led to examine the group of Palaeozoic fossils known as Stromatoporoids (layered pore animals). I found in them certain rings with signet-like swellings shaped somewhat like the siliceous rings of the siliceous sponge, \emph{Merlia normani}, and I concluded that Stromatoporoids were sponges. But presently I found similar bodies in recent Foraminifera, and soon saw many reasons for believing that these fossils belonged to that group. On examining Dr. Carpenter's sections of \emph{Eozoön}, I was amazed to find the rings there also.

I published a paper on \emph{Eozoön} and \emph{Stromatoporoids} in the \emph{Annals and Mag. Nat. Hist.}, September 1912, stating that they were Foraminifera. I found in all of them a complicated arrangement of chambers with calcareous walls perforated by tubuli and canals, and also rings and hoops similar to those of the modern Foraminifera. At this stage I held the same views of \emph{Eozoön} as Carpenter, who regarded the olivine bands as models of Foraminiferal chambers.

In October 1912 I published in the same journal a paper containing several errors,\footnote{\Fontauri{When first examining the surface of a large specimen of \emph{Eozoön canadense}, I saw branching tuft-like structures and spreading crusts, which appeared to me to be coralline algae, and to which I gave provisional names. I have found since that these supposed algae are --- possibly with one exception --- purely mineral structures. Certain opaque white incrustations with rounded edges, and apparently with a cellular structure may, however, be Melobesia-like algae.\\ \hspace*{5mm}Hitherto I have only had half a dozen cleaned specimens of \emph{Eozoön} to work with. I have asked a colleague in Ottawa to superintend the despatch to London of a quarter of a ton of Lower Laurentian Limestone, and some graphite and gneiss in the rough, and am hopeful of finding remains of Eozoic flora, if they exist.\\ \hspace*{5mm}In the \emph{Annals}, October, 1912, I figured a network of pseudopods, but I find this appearance to be due to a breaking down of \emph{Eozoön}-structure in olivine. Again, the growth of \emph{Eozoön} might be compared to a nearly sessile mushroom, but not to a fountain. Lastly, there is no evidence that the \emph{Rotaliidae} are derived from Stromatoporoids.}} but also a very important\footnote{\Fontauri{I think this adjective correctly describes the first of a series of observations that culminated in the discovery of the ``nummulosphere'' on which, strange to relate, we have been living all this time \emph{sans le savoir}.}} truth. I found that Stromatoporoids and \emph{Eozoön} were composed of layers of Foraminiferal shells organically united so as to form colonies. I thought at first that the shells of Stromatoporoids were Rotalian in character, and sometimes of large size with great coils of segments, but I have since found that the shells resemble those of \emph{Eozoön} in being typical nummulites with alar prolongations,\footnote{\Fontauri{A term designating the centripetal V-shaped prolongations of a coil embracing the previously formed coils.}} and in being on an average less than 5 millimetres (0.2  of an inch) in diameter. The large spiral and cyclical patterns seen on the polished slabs of some species of Stromatoporoids do not belong to single large shells, but to series of small ones. Similarly the long parallel bands are vertical or oblique sections of rouleaux of shells. A x10 lens will show the disks all over the surface of \emph{Stromatopora}, \emph{Actinostroma}, \emph{Beatricea}, \emph{Labechia}, \emph{etc.} The tubercles in specimens of the last-named genus are simply the effect of metamorphosing agencies, and I do not think the genus will stand. The Caunopora tubes belong to foreign organisms such as worms and corals.

On the surface of specimens of \emph{Eozoön}, if not too much crystallized, it is easy to find plenty of disks with umbilicus and radial rows of pores. Where the surface of a shell has been damaged the alae and septa can be detected.

The arrangement of disks in \emph{Stromatopora} and \emph{Eozoön} might be compared ideally with a number of vertical piles of coins, each coin having a central hole (astrorhiza or umbilicus). A Stromatoporoid approaching this ideal arrangement is found in \emph{Receptaculites} with its well-marked columns and fairly regular surface pattern.

A vertical section of \emph{Eozoön} frequently shows alternating wavy white and green bands, and curious branching canals in the white parts.

Carpenter was under the impression that a specimen of \emph{Eozoön} was a monster Foraminiferal shell with huge chambers arranged in layers separated by thick white bands of secondary skeleton permeated by branching canal-systems. He compared the enormously large ``chambers'' to those of \emph{Polytrema}, and described them as having ``grown wild,'' meaning thereby that they had escaped control of some spiral pattern and had become ``acervuline.'' He regarded \emph{Eozoön} as a member of the \emph{Nummulitidae} because he thought it possessed those types of structure, \emph{viz.} secondary skeleton and canal-system that are developed to a corresponding degree only in the higher members of that family. Brady evidently influenced by Carpenter, placed \emph{Eozoön} in a new sub-family of the \emph{Nummulitidae}, \emph{viz.}, ``sub-family 5 (?) \emph{Eozoöninae} --- test forming irregular, adherent acervuline masses.'' ``\emph{Eozoön} Dawson --- test adherent, outlines irregular; composed of segments arranged at first in more or less regular superimposed layers, subsequently acervuline; with interseptal skeleton and ramifying canals.''

It happens to be true that \emph{Eozoön} belongs to the \emph{Nummulitidae}, but neither Carpenter nor Brady bring forward any evidence in support of this theory. For certainly the green spaces are not Foraminiferal chambers in any sense, the white bands are not secondary skeleton, and the dendritic structures are not pseudopodial canals.

No wonder the theory of organic origin has not been accepted by the scientific world! I can now clearly understand why Möbius and the mineralogists summarily rejected the theory of the organic origin of \emph{Eozoön}. For it is easily conceivable that a mass of carbonate of lime of wholly inorganic origin like, for instance, a ``fur'' deposited from heated water might become dolomitised and silicated in zones, and the dolomite (or calcite) zones might become penetrated by branched veinings of silicates. Then the opponents of the organic theory might well point out the improbability of the survival of any trace of organic structure in very ancient rocks which had been subjected to terrific metamorphosing agencies. Again, when Eozoonal rocks were hurled up from the crater of a volcano, there was surely no need for further discussion. One might as well argue with an ``earth-flattist'' as with a person who stated that \emph{Eozoön} had once been a living organism. The history of this controversy, however, presents only one more instance of the danger of \emph{a priori} reasoning, for simple observation with a hand-lens shows beyond the least shadow of doubt that not only \emph{Eozoön} but the volcanic and plutonic rocks are mineralized organic structures.

A specimen of \emph{Eozoön} was once a mass of delicate hollow shells of carbonate of lime. The white and green bands in metamorphosed specimens are simply the result of a particular mode of mineralization\footnote{\Fontauri{It was only after this pamphlet was in the press that I realized that ``mineralize'' and ``crystallize,'' \emph{etc.}, are spelled with a ``z,'' and ``dike'' with an ``i'' in Murray, our final authority.}} of this uniform mass. The convincing proof is that the nummulite shells are present both in the white and in the green bands. Even where the banded structure is well marked, the white and green layers vary greatly in thickness, and often there are large non-banded areas of white or green, and sometimes of mottled green and white. Often, too, the individual shells are composed partly of olivine and partly of dolomite. Doubtless the wholly calcareous Tudor specimen of \emph{Eozoön} is simply a mass of calcareous shells. A glance with a lens would settle the point. In banded trachytes there are shells in the white felspar layers and also in the earthy-brown layers, and sometimes one can see several very fine dark bands traversing one shell. In some granites, too, I often see quartz felspar and mica in a single shell. The green bands are not Foraminiferal chambers, but silicated areas each containing several nummulite shells or portions of shells, just as the dolomitised areas may each contain several shells or parts of shells.

In 1878 Nicholson and Etheridge found certain curious labyrinthine masses of vermiculate tubes in microscopic sections of Girvan limestone. They regarded them as organisms allied to the arenaceous Foraminifera, and gave them the name \emph{Girvanella problematica}. These objects have since been found in other limestones, especially in the Jurassic Oolites. I have found that the Girvan limestones and Oolites are nummulitic rocks, and that the Girvanella structures are simply the results of partial solution and redeposition in the hollow calcareous nummulite shells with their coil-within-coil structure. Although the chambers of any particular coil extend from the periphery to the centre, it is only the peripheral part just within the apex of the V that occupies much space, the rest of the chamber-space being the thin film between the alae or arms of the V and the outer surface of the next coil embraced by those arms. Consequently there is a tendency to the formation of series of concentric tubes. Further, the series of radiating septa of all the coils lie to a considerable extent along similar vertical radial planes, a fact rendered obvious when one views a horizontal section of a nummulite through the median plane. Consequently when a shell is subject to pressure there is a tendency for their radial septa to cut out, so to speak, a radial pattern. Again, horizontal, vertical and oblique sections through rouleaux of shells would show a complicated plexus of tubes. It is not difficult to realize that when subjected to dissolving agencies single shells or groups of shells may form little isolated granules, or big pisolitic masses. One thing is certain, \emph{viz.} that the Girvan limestone of Tramitchel is a mass of nummulites, and that these shells undergo change from the effects of rainwater and carbonic acid. Further, it is possible to trace nummulitic structures in masses of Girvanella. I believe Girvanella structure will be found in nummulitic rocks from granite to the Upper Chalk and possibly later.

I now find this structure abundantly in the dolomitised silicated masses of nummulitic limestone known as \emph{Eozoön}. The supposed branching canals in the supposed secondary skeleton of \emph{Eozoön} are simply ``Girvanella'' structures. They exist also in the green silicated parts of specimens. Gradations can be traced from circular nummulite shells with radiating ``tubes,'' through closely pressed fan-shaped masses, to widely branched spread-out canal systems. It is simply a matter of breaking down of shells, of masses of tubes, and of joining up into hollow canal systems like the heading-off of river systems. Naturally the silication has been effected more readily in those parts of the dolomite bands where hollow channels have formed as the result of the peculiar mode of disintegration that takes place in the nummulite shells. The peculiar ``rings'' seen in the canals are there because they have been present in the original shells. I do not think they are purely the product of mineralization because similar structures exist in recent Foraminifera. The rings and half-rings visible in decalcified preparations of spirit specimens of \emph{Polytrema} appear to be horny thickenings in the neighbourhood of the rims of pseudopodial pores. If this be so, a series might form along vertical rows of pores in nummulites, in which there are numerous successive layers of shell-wall especially at the central parts of the shell. The existence of these bodies in \emph{Eozoön} and the later Stromatoporoids may be due to mineralization of the rings, or again, the rings may have been calcareous skeletal thickenings of pore-rims. Whatever be their nature and origin, it was the finding of them in \emph{Eozoön} that first led me to doubt, possibly on a wrong premiss, the theory of mineral origin of that structure.

I cannot find a single point of zoological difference between \emph{Eozoön canadense} Dawson and \emph{Stromatopora concentrica} Goldfuss. In both there are layers of nummulite shells of the same structure and size. As this is not a systematic treatise, I shall continue to call the Canadian Dawn Animal by its well-known name.

Although the highest type of Foraminiferal organisation is reached in the family \emph{Nummulitidae}, owing to the development of a secondary skeleton in the walls and between the septa of the shell, yet the Stromatoporoid type of nummulite is a simple one, for the thin-walled shell consists merely of a continuous and embracing spiral coil divided up into similar segments. Further, while growth in the horizontal plane produces the coil, growth in the vertical leads to the formation of a bud which simply repeats the horizontal and vertical modes of growth.

The individuality of certain types of Stromatoporoids is well-marked in the branching forms, and in \emph{Beatricea} and \emph{Receptaculites}. The shape of massive forms such as \emph{Stromatopora concentrica} and \emph{Eozoön} is not so well-defined. Similarly in corals, specimens vary greatly in the degree of definiteness of their shape. The Jurassic Oolites and Chalk are probably mainly masses of Stromatoporoids, yet it is not possible to define the shapes of specimens, nor even of reefs.

\centerline{*\hspace{15mm}*\hspace{15mm}*\hspace{15mm}*\hspace{15mm}*}
\bigskip

On p. 93 (summary) I state that \emph{Eozoön} differs from Palaeozoic forms in possessing a secondary skeleton. This statement is erroneous and should be deleted.

\centerline{*\hspace{15mm}*\hspace{15mm}*\hspace{15mm}*\hspace{15mm}*}
\bigskip

I am certain that the long and, at times, bitter controversy on the nature of \emph{Eozoön} is finally settled, and it is now possible to see that both sides were right and both wrong. The believers in the organic theory were right although they failed to bring forward satisfactory proofs,\footnote{\Fontauri{It must not be forgotten, however, that Dawson, Logan and Carpenter pointed out the external resemblance between \emph{Eozoön} and the Palaeozoic Stromatoporoids, about the organic nature of which there was no dispute.}} but the upholders of the mineral theory were fully justified in rejecting the theory of organic origin until those proofs were forthcoming.

\centerline{*\hspace{15mm}*\hspace{15mm}*\hspace{15mm}*\hspace{15mm}*}
\bigskip

\small
\emph{Miscellaneous addenda.} --- The timely discovery of the real nature of the white bands in \emph{Eozoön} necessitated the rewriting of the chapter. To fill a vacant space a few notes are added.

\emph{Note 1.} I have now found that mica is wholly composed of tremendously compressed masses of nummulites. In fairly thick laminae it is often possible to see abundantly the circular groups of scales, pores, umbilicus, \emph{etc.}, and occasionally at the edges of laminae the willow pattern and the minute reticulation formed by alae and septa. Mica is probably a remnant product, silica and other silicates having disappeared.

\emph{Note 2.} In Prof. C. Gagel's memoir on the petrology of Madeira (Zeitsch. deutsch. geolog. Gesellch. Bd. 64, Hft. 3, 1912, p. 434, Fig. 19) a micro-photograph (x50) of a section of trachydolerite is shown. The section is that of a resorbed amphibole which has become changed into magnetite dust and rhonite crystals. With the aid of a lens I can see in the lower dark part of the figure the circular outlines of masses of very small nummulite disks looming through the crosshatch, the disks being 3-6 mm. in diameter in the figure. A petrological friend showed me a crystal of hornblende breaking down. Each magnetite crystal at the periphery of the hornblende crystal formed the centre of a flake, apparently of felspar. The clear yellow central part of the large crystal also was faintly aerolated, with a dark spot in each aerola. The aerolar pattern apparently is that which has been cut out by the septa and successive coil-rims of a much-altered nummulite shell. Compare also the (?) haematite intra-cameral particles in limestone shells. Apparently ferric and ferrous granules have become ferroso-ferric magnetite-dust.

\emph{Note 3.} A specimen of Zechstein (dolomite) from Nieder-Rodenbach (in my own possession) is apparently not the original dolomite, but a dolomitic sedimentary shale.

\emph{Note 4.} Concerning R. Mallet's theory that the melting of igneous rocks is due to compression, I beg to submit a very humble personal observation. I asked a blacksmith to hammer a bar of iron on an anvil. In a few seconds the hammered end was too hot for me to touch, and in less than a minute, for the blacksmith himself to handle. The arm of the mightiest mortal blacksmith is feeble beside that of the tireless Vulcan. It seems a curious paradox that the cooling of the earth should lead to the melting of the rocks!

\emph{Note 5.} Sands and muds which I gathered at Sandwich and Cuckmere have now dried and hardened, and the nummulitic structure, difficult to detect in the moist state, is now easily to be seen.
\normalsize
\clearpage
\section{Porto Santo and Madeira}
\Large
\begin{quote} 
\hspace*{15mm}``Neptune, besides the sway\\
Of every salt flood, sand each ebbing stream,\\
Took in by lot 'twixt high and nether Jove\\
Imperial rule of all the sea-girt isles,\\
That like to rich and various gems inlay\\
The unadornéd bosom of the deep;'' \hspace*{5mm}Comus.
\end{quote}
\paragraph{}
In September 1912 I journeyed to Porto Santo via Madeira, in order to complete my investigation of that strange organism, the Sponge-alga \emph{Merlia normani}.

Going from Madeira to Porto Santo --- and I have now made the journey four times --- seems to me like making a trip from the earth to the moon. The little island, which lies about 20 miles N.E. of Madeira, is about six miles long and one to four miles broad. There are numerous little mountains or picos, of which the highest is about 1500 feet. They form two groups with a low saddle, partly occupied by a great lava flow, between.

On the S.E. shore of the island, one of the Picos (P. Baixo) forms a magnificent triangular escarpment about 900 feet high. The summit is formed by a sharply defined triangular mass of greyish-white columnar trachyte; below this are dull red and purple bands of scoriae, and still lower a great basal mass of black columnar basalt coming down to the blue water and overlying at one place a Miocene coral reef, the ensemble producing an indescribably beautiful effect in the strangely transparent air. With the exception of certain tertiary coral reefs and some post-pleistocene blown sands the whole island is volcanic. The lines of dike-like crests of the mountains, if continued, would converge towards a point situated out at sea to the north of the island, where the crater of a huge volcano must have been. The dikes and lava streams show that Porto Santo Island is the base of the southern slope of the volcano, all the rest having disappeared possibly by submergence, for, judging from the soundings, the island was probably very much larger in former times than it is now.

One day my amiable and accomplished friend Senhor A. C. Noronha, who had accompanied me to the island, showed me some fragments of banded trachyte, basalt and granitoid rocks which he had collected at spots about 1000 feet up the Pico do Facho.

I examined them carefully under my binocular microscope and found to my amazement traces of nummulitic disks in all of them. Next day I visited the place whence the fragments had come.

They had been detached from the rocks laid bare in torrent beds. When I saw the huge boulders of banded trachyte, I thought I had made a mistake about the disks, but on returning to the village, I again found sure evidence of their existence.

After all I need not have been so doubtful about my first observations, for I remembered the account which Dr. G. Lindström gave of the huge masses and balls several feet in diameter of \emph{Stromatopora discoidea} which he had found in Gothland and which he considered to be corals, but which we now know to be colonial Foraminifera closely allied to \emph{Eozoön}.

Senhor Noronha showed me the spot where he had found the highly crystalline granitoid rock, \emph{viz.}, at the place of junction between a dike and the older basalt. I spent several days collecting rocks, and found that almost the whole island from the shore to the crests of the mountains was made of metamorphosed nummulitic limestone, and that the houses, the boundary walls of the fields, the boulders on the seashore, and to some extent the soil and the shore sands\footnote{\Fontauri{The shore sands of Porto Santo are mainly calcareous, but also there are present grains of what I believe to be greenish olivine. The waves sift the darker siliceous and lighter calcareous particles and make a curious banded structure which sometimes hardens. A boulder of this finely banded material might be mistaken at first for a piece of banded volcanic rock. Of course, a glance with the lens reveals the real structure.}} were likewise composed of this material. A coarsely granular clay used for roofing small cottages also contained the disks, so that \emph{Eozoön} had at last arrived at the fate that might have befallen the dust of Imperial Caesar. Very interesting, too, were the kaolinized basaltic boulders, looking like chalk full of black pebbles, and in the final stage like pure chalk.

On my return to Madeira, I found that here also the volcanic rocks were made of changed nummulitic limestones.\footnote{\Fontauri{Accordingly, when Neptune took in by lot imperial rule of the volcanic isles, Chance sided with Justice, for volcanos and lavas are, in the main, altered Foraminiferal seafloors, to which, surely, nether Jove has no claim.}}

At a spot about 2800 feet up the Pico do Infante, and near the house of my hospitable friend the Rev. A. D. Paterson, I saw a bed of huge balls of volcanic rock, varying in diameter from one to several feet. They were so badly weathered that it was possible to peel off large flakes. The balls are composed of basalt, and have been regarded as ``bombs,'' which had been hurled up into the air and had fallen back into hot lava, thereby becoming recoated. This theory is probably incorrect, because the stones are on the mountain side, and not in a crater. Onion stones are not uncommon wherever volcanic rocks exist. The Madeira examples are in volcanic soil, \emph{i.e.} in disintegrated volcanic rock.

R. Mallet (\emph{Trans. Roy. Irish Acad.} 1837 [1839] p. 75) describes a trap rock in Galway, which fractures irregularly when struck with a hammer, but which reveals a \emph{hidden nodular structure} when blasted with gunpowder. He compares the blasting to the sculptor's chisel which (in the Greek fable) revealed the statue hidden in the marble. He notes the significant fact that certain visible crystals never went beyond the bounding surface of one nodule into another. Apparently a laminar structure must have existed in the rock before the crystals separated out. Probably the lamination in onion-stones is due to the effects of cooling.

In the case of the Madeira onion-stones percolating water and root fibrils appear to have worked along the zones of least resistance which had formed as a result of shrinkage in the cooling rock.
\clearpage
\section{On the World-wide Distribution of Nummulitic Rocks}
\paragraph{}
On my return to London I made a petrological tour of the world in the Mineral Gallery of the Natural History Museum. I found that the volcanic rocks of the volcanic islands of the Atlantic, from Iceland to St. Helena, of the Arctic (Spitsbergen), Mediterranean (Etna, \emph{etc.}), Antarctic (Ross Island, Erebus and Terror), Indian Ocean (Kerguelen and Christmas Island) and Pacific (Fiji) were composed of metamorphosed nummulitic limestones. I next proceeded to examine plutonic rocks and many of the metamorphic ones (gneisses, some schists), and found that these were of the same nature. In a piece of garnetiferous schist which I gathered in the bed of a river in the Sikhim Himalayas, the nummulitic shells have become modelled in grains of garnet. In other parts of the fragment there are ordinary crystals of garnet, and mica. The specimen formed part of a huge boulder which my kind host, Mr. John C. White, the Resident, had caused to be blown up with dynamite.

I have examined the fragment of grey basalt dredged up by the \emph{Challenger} from 1950 fathoms from Lat. 53$^{\circ}$ 55$^{\prime}$ S. and Long. 108$^{\circ}$ 33$^{\prime}$ E., also two pieces of rock obtained by the telegraph engineers of ss. \emph{Cambria} from 2,200 to 2,530 fathoms in Lat. 51$^{\circ}$ 35$^{\prime}$ N. and Long. 15$^{\circ}$ 43$^{\prime}$ W. and Lat. 51$^{\circ}$ 34$^{\prime}$ N. and Long. 16$^{\circ}$ 30$^{\prime}$ W. and find them all to be changed limestones in which the shells are clearly visible. These blocks might have been dropped from icebergs, and possibly do not belong to the localities in which they were found. At the same time they may do so, more especially the two last specimens. Lastly, I firmly believe I have found the shells in many of the stony meteorites or aerolites, and even in siderolites.

The disks are, perhaps, better seen in basalts than in the highly crystalline granites, and better in homogeneous than in vesicular lavas, in which last it is almost impossible to find them. I have not been able to detect them in certain volcanic bombs in which the lava has been reduced almost to a homogeneous paste.

The difference between the \emph{Eozoön}\footnote{\Fontauri{Probably the Eozoic limestones are all composed of the \emph{colonial} nummulite of Stromatoporoid \emph{Eozoön}.}} of the Lower Laurentian Limestones, and the \emph{Eozoön} of a granite or gneiss is due to the fact that the former was modified to a less extent and in a different manner by metamorphosing agencies than the two latter.

\centerline{*\hspace{15mm}*\hspace{15mm}*\hspace{15mm}*\hspace{15mm}*}
\bigskip

I would advise those who may wish to look into the matter for themselves to examine igneous rocks carefully and thoroughly with a good lens magnifying ten diameters. At first there will be a generalised impression, and nothing but a confused mass of minerals will be seen, but the disk structure will gradually reveal itself. Usually the central umbilicus and the circular area of the disk will first be seen, and then traces of radial and concentric lines, and possibly radial series of pores. Gradually skill in detecting these structures will come, and with it, the possibility of orientating\footnote{\Fontauri{A rock formed even of separate disks would tend to show a layered structure, but I believe the igneous rocks to be Stromatoporoids or \emph{colonial} nummulites, and therefore I use the term ``orientating.'' If these rocks are colonial Foraminifera, the stromatic arrangement must be entirely destroyed in very liquid lavas --- such, for example, as that referred to by J. D. Forbes (Occasional Papers, p. 93), which flowed down Vesuvius at a rate of a mile in 1.3 minutes, but yet the disks persist.}} the pieces of metamorphosed nummulitic rock, especially when a row of disks \emph{en face} or in rouleau is discovered. Of course it may not unreasonably be pointed out how easy it is to choose a centre and imagine a circle round it, \emph{etc.}, and that if the existence of a thing is strongly asserted, it will tend to become visible owning to the influence of suggestion. When the observer has seen the disks in many kinds of igneous rocks and gneisses, he will, I believe, refuse to accept the hypothesis that he is the victim of delusions.

I have examined many kinds of non-Foraminiferal rocks, coarse sandstones, slags, grits, pseudomorphs, concretions, \emph{etc.}, in order to note whether the disk-structure presents itself, and occasionally I have met with appearances suggestive of that structure. In the case of igneous rocks and Stromatoporoid Limestones, however, it is not merely a question of disks, but of \emph{Foraminiferal} disks often more or less definitely orientated, and individually presenting a good deal of structure whether seen \emph{en face} or in vertical or oblique section.

It is desirable to approach the subject with an open mind. A sympathetic attitude is still better, and will be rewarded with a rapid appreciation of an important truth. A disposition hostile to the new theory may, I am afraid, lead the observer to underestimate the value of even very clear indications of organic pattern.

I myself have now made numerous observations on all varieties of plutonic, volcanic and gneissic rocks and can detect the disks and a certain degree of banded structure in all, even in examples of Aberdeen and Cornish granite. Rough or polished surfaces, but best of all, surfaces roughly fractured obliquely along the original planes of bedding will show the disks, sometimes \emph{en face}, sometimes on edge or in section. When I examine a piece of grey Cornish De Lank granite about the size of my hand, and split obliquely I can after a time detect with a x10 lens a hundred subtle evidences of the indubitable fact that I am looking at a fragment of a silicated nummulitic rock. Mica, felspar or quartz may enter into the composition of the disks. The alar prolongations of the nummulitic shell are frequently modelled in series of thin layers of mica. The umbilicus is usually well differentiated, and rows of pores are often visible. It is desirable to put in a word of caution here, \emph{viz.}, not to mistake fine lines made with the ``patent-axe'' for indications of Stromatoporoid structure.

The mineralized medals or disks vary in diameter from about one-sixth to a third of an inch or more. It must be remembered that they are crushed and flattened structures pressed against each other, so that it is difficult to estimate how much of the whole original disk is seen in any particular example.

It is very difficult to detect the nummulitic structure in granites with very large crystals, or in Pegmatites. I have seen the disks, however, in a thin section of graphic granite from Harris. It was, I believe, this very section that Dr. Carpenter described as ``A New Laurentian Fossil'' (\emph{Nature}, May 4, 1876, vol. 14, p. 8). Later (\emph{l.c.} May 25, p. 68) he retracted, and explained that he had mistaken the feldspathic bands for calcareous ones. He was right after all in his first opinion. For at the junctions between the two kinds of crystals I can make out the Foraminiferal disks cut vertically and obliquely. They have wholly disappeared from the space occupied by the main width of each lamina. Carpenter writes, ``... whether the graphic granite may not be a metamorphosed form of an ancient organic structure ... is not to be decided by anyone's \emph{ipse dixit}.''

Traces of the disks are visible in pumice, asbestos, jade, meerschaum in the rough state, soapstone, serpentine, verde antico, and precious porphyry.\footnote{\Fontauri{I am much indebted to Messrs. Farmer and Brindley for specimens of the two last named rocks.}}

Messrs. O'Hara and Hoar kindly permitted me to examine on their premises numerous examples of pumice from all parts of the world. I found the disk structure best in certain hard blocks, of very little use commercially. Here the gases had burst up through the mass in such a way as to bring out the coiled pattern of the disks fairly distinctly.

I was given a piece of clay called Turkey Green, and this I found to be a kaolinized igneous product showing evident traces of the disks. Again, I found faint traces of them in a pure white siliceous ``chalk'' which had been discovered under the soil in Spain. This ``chalk,'' which contains over 80 per cent. of silica, is composed of diatoms, siliceous sponge-spicules shaped like bent rods pointed at each end, and nummulite disks. If the diatom-frustules and spicules are those of fresh-water organisms, the formation is probably an ancient lake-bottom, the disks being the detritus of nummulitic limestones carried down by rivers and deposited as mud amidst the falling diatom-frustules. In the Natural History Museum there is a similar chalk-like mass of rock from the floor of a lake in Australia. Here the diatoms are certainly fresh-water forms. The investigation of these siliceous rocks led me to examine the ``Barbados Earth,'' a marine Miocene formation very rich in Radiolaria. This ``earth'' is essentially a nummulitic rock on the surface of which Radiolaria have continually fallen in the course of its growth. Although the carbonate of lime has gone, the nummulitic structure is well-preserved.
\clearpage
\section{Geological and Geognostical Considerations}
\paragraph{}
Many men of science believe that the greater part of the land surface of the globe has been beneath the sea,\footnote{\Fontauri{Prof. J. W. Gregory (`The Making of the Earth,' p. 114) mentions Scandinavia, Finland, and certain other regions as areas which may have existed always as land. I find, however, that even the most coarsely crystalline igneous rocks from Scandinavia (\emph{e.g.} Swedish Rose Granite) retain undoubted traces of Foraminiferal structure, and therefore resemble those of other countries in being changed and upheaved seafloors.}} but the discovery of the nummulitic origin of igneous rocks tends to prove that at one time or another the whole area has been submarine.

The general distribution of igneous rocks and gneisses on or below the surface of the land, and the world-wide occurrence of volcanic islands (or heaps of silicated Foraminiferal limestone) over all the oceans, appear to point to the conclusion that there was once a nearly universally distributed crust of nummulitic limestones.\footnote{\Fontauri{Since this was printed I have found that all the deep-sea oozes have a nummulitic matrix.}}

If, over great areas of the deep oceans, the floor beneath the oozes if formed of a relatively shallow-water Foraminiferan, there would probably have been a compensating rise over other areas of the general crust of the globe. Slow secular contraction due to slow secular cooling of the globe has been going on for immeasurable ages. It is not unreasonable to assume that the crust, during the early stages after consolidation, was more uniformly level, and that great depressions and elevations below and above a mean level came later. Possibly during the earlier part of the Eozoic Era there may have been a nearly or even wholly universal ocean, relatively shallow over a great part of its area. It is, at any rate, almost certain that the ocean covered a much greater area than at present.

The Eozoic Era probably began at the moment when the planet had cooled down sufficiently to allow of the formation of the highly complex and unstable molecular combinations out of which protoplasm, the physical basis of life, is made. This substance must evidently have been formed on an immense scale. At first it must have had the plant mode of nutrition by means of solar energy, but portions of it took on the saprophytic and parasitic habit. A kind of widely spread Bathybius must have existed after all, and when that took to forming a skeleton of carbonate of lime, a colonial reef forming Foraminiferan would gradually be evolved.

It is extremely probable that \emph{Eozoön} lived in shallow water, because an organism so extensively and abundantly distributed would require a plentiful food supply such as could be found only in shallow sunlit waters. Further, the Coral reefs of later epochs exist only in relatively shallow water. Again, I have found coralline algae growing on Palaeozoic Monticuliporas associated with Stromatoporoids, and it is unlikely that the Eozoic would differ greatly in bathymetric range from the Palaeozoic forms. Accordingly there is reason for believing that \emph{Eozoön} lived in the coralline zone. In course of time thick beds of Foraminiferal limestone were formed, and these underwent metamorphosis just as has happened in the case of many fossils in later ages.

The gradual shrinkage of the earth's crust would give rise to a great elevation of temperature, sufficient to fuse to a greater or less extent the silicated nummulitic rocks.

When I discovered that the igneous rocks which form a great part of the known crust of the earth were composed of altered nummulitic limestones, it became evident that igneous action could only be accounted for on the theory of shrinkage of the earth's crust, and that theories of chemical action or of interior heat and fluidity were out of the question.

After I had formulated on biological grounds a theory of shrinkage, it was a relief to find that Lord Kelvin and Robert Mallet had already concluded that ``vulcanicity'' could be wholly explained on the hypothesis that the earth is a cooling body with a contracting crust. (See Appendix, Note 6.)

Lord Kelvin writes (`On the Secular Cooling of the Earth,' Trans. Roy. Soc. Edinburgh, 1862, 23. p. 160), ``The less hypothetical view, however, that the earth is merely a warm chemically inert body is clearly to be preferred'' --- \emph{i.e.} to theories of chemical action.

Again, R. Mallet (`Volcanic Energy. An attempt to develop its true Origin and Cosmical Relations.' Phil. Trans. 1873, p. 147, and 1875, p. 205) proves, on the supposition of a shell 800 miles thick, that the annual shrinkage due to secular cooling would amount to about 1.5 billionths of the total diameter, \emph{i.e.} to a little less than 7 inches in 5,000 years. He writes, ``Yet insignificant when thus measured as is the amount of annual contraction of our globe by its secular refrigeration, we see how important and mighty are its effects in preserving through the volcano the cosmical regimen of our world; it is another added to the many instances already known in the range of natural philosophy, in which causes so minute as for long to remain occult to us are yet, though unseen and unnoticed, essential parts of the mighty machine.'' Another hitherto unnoticed occult cause primarily contributing to the construction of the so-called igneous rocks has been the life force.

After the shrunk crust of the earth had re-adjusted itself for a time, the rocks would cool and crystallize. Where there were weak spots the crust would give way, and there would be an eruption, the force of which would often be increased by inroads of the sea.

At times, the relatively feeble force of gravity would be overcome and blocks of metamorphosed nummulitic limestones would be hurled into space with such violence that they would for a time escape control of terrestrial attraction and take on an orbit of their own, but would be captured later.\footnote{\Fontauri{During the eruption of Krakatoa a great part of the mountain was blown away, and the explosion was said to have been heard 3,000 miles away. Probably many meteorites would be ejected on such an occasion.}}

Dr. Otto Hahn,\footnote{\Fontauri{Dr. Otto Hahn imagined he had found in meteorites traces of many organisms representing the fauna and flora of cosmical bodies existing ``wissnichtwo.'' When someone succeeded in creating similar bodies in the laboratory by means of infernal boilings, Dr. Hahn's theory was supposed to have been effectually refuted.}} the learned doctor of Tübingen, doubts whether any conclusion as to the origin of meteorites --- ``ob ... kinder oder brüder der Erde'' --- can be drawn from their chemical, morphological or textural characters. If, however, it is true that some aerolites are pieces of changed limestone, then it becomes difficult to imagine any other than a terrestrial origin for such bodies, and they would naturally be regarded as ``kinder'' or offspring.\footnote{\Fontauri{Laplace's theory of the lunar origin of meteorites, which is admitted to be valid on mathematical grounds, is improbable on biological ones, for the absence of atmosphere and water, and the great oscillations of temperature at the present time render improbable the theory of the existence of organic life even in the simplest forms. We do not know enough about past conditions to be able to speculate profitably concerning them.}} Even some of the siderolites and siderites may be of terrestrial origin. For these bodies may have been formed just as iron from ironstone in a blast furnace. Some basalts (changed nummulitic limestones) are known to contain a good deal of iron, and this is only what might be expected, for all protoplasm contains this metal. Further, I often saw on the beach at Porto Santo boulders of trachyte with well-marked rusty, evidently ferruginous, bands. Igneous rocks, too, often possess magnetic properties, as, for example, those of Giant's Causeway and of Compass Hill in Cannay Island, Hebrides.

One of the difficulties in the way of the theory of terrestrial origin of meteorites lies in the presence in them of certain substances (phosphorus, nickel-iron, sulphides) which could not exist in an environment which included air and water.\footnote{\Fontauri{`An introduction to the Study of Meteorites.' British Museum (Natural History), Ed. 10. 1908, p. 33.}}

In the hottest part of a volcano there would be a process of dissociation taking place, so that compounds would be resolved into elements, some of which would again enter into combination when a lower temperature permitted it.

If products of dissociation were present in the interior of masses which were being hurled into space with terrific force and velocity, then before air or water could effect a change, the masses would be out of range and go on for ever unchanged until they collided with other bodies or entered an atmosphere. Sir Robert Ball inclines to the belief that the meteorites were projected from the earth in past ages, and the discovery of the nummulitic nature of these bodies confirms that belief. If these views are correct, the date of ejection must, of course, have been later than the period of metamorphosis of the fragment of changed limestone which constituted the particular aerolite.

\centerline{*\hspace{15mm}*\hspace{15mm}*\hspace{15mm}*\hspace{15mm}*}
\bigskip

If the plutonic rocks and such metamorphic rocks as the gneisses are all of them metamorphosed limestones, then it becomes difficult to draw any line of distinction between the two classes. Indeed, this difficulty seems to have been an old standing one in geology. I have seen for instance, a specimen from Argyllshire showing granite intruded into gneiss. The nummulitic disks are present in both portions, and can be seen easily in the gneiss, but only with difficulty in the granite. The granite was simply a part of a lower zone of much metamorphosed nummulitic limestone intruded into an upper zone of probably sedimentary changed nummulitic detritus (gneiss).

\centerline{*\hspace{15mm}*\hspace{15mm}*\hspace{15mm}*\hspace{15mm}*}
\bigskip

Although plutonic and volcanic rocks are arranged chronologically according to their position with regard to the various strata to which they are juxtaposed, here they are consigned provisionally to the one Era --- the Eozoic --- because not only is the fauna identical, but the calcareous remains of that fauna have become ``silicated.'' In `The Student's Lyell,' Ed. J. W. Judd, 1911, p. 521, it is written, ``There is no reason for doubting, however, that, if we could penetrate many thousands of feet beneath the roots of such volcanoes as Vulcano and Vesuvius, we should find the rhyolites of the one graduating through quartz-felsites into granite, and the basalts of the other passing by easy transitions through dolerites into gabbro.''

\centerline{*\hspace{15mm}*\hspace{15mm}*\hspace{15mm}*\hspace{15mm}*}
\bigskip

The metamorphosis of calcareous skeletons whereby they become silicated, silicified, or dolomitised is common throughout the ages. \emph{Eozoön} which must now be definitely accepted as a Foraminiferan, furnishes an excellent example both of change into various silicates and into dolomite. A common form of silicon compound in \emph{Eozoön} is the magnesium silicate with iron, known as olivine, this when hydrated becomes serpentine. The grains of the Cretaceous Greensands are very frequently casts of Foraminiferal shells in glauconite or hydrated silicate of iron alumina and potash.

The felspar and mica of granite are double silicates of alumina and potash. In the case of granite, the Foraminiferan is a massive colonial form, but the Greensand Foraminifera are minute separate shells, and, further, they have not been subjected to heat and pressure. Again, the originally pure carbonate of lime skeletons of the dolomites have become changed into calcium and magnesium carbonate.

The prevailing conditions at various periods have been very different. Sir R. Murchison (`Siluria,' Ed. 4., 1867, p. 489) writes: ``I could here cite the works of many eminent writers for numerous evidences of the grander intensity of causation in former epochs....''

When the oceans were formed, the pounding down of the heated waters on to the perhaps scarcely formed crust must have led to their being super-saturated with certain minerals.

In the excellent handbook, `The Science of the Sea' (prepared by the Challenger Society, and edited by G. H. Fowler), Dittmar's analysis of sea water is given. It is stated that there are only 0.432 part by weight of calcium in 1,000 parts y weight of sea water, yet marine organisms with calcareous skeletons have to extract all the necessary lime out of that sea water. Silicon is not even mentioned, probably owning to it being present in such infinitesimal quantities. It must be there, otherwise how could Diatoms which have formed oozes covering an area of ten million square miles, also Radiolaria, and Sponges, make their skeletons?

The above-mentioned facts point to the probability that one of the sources of the minerals present in metamorphosed limestones is the sea.\footnote{\Fontauri{Recently I have found several important formations to be composed of sandy or sanded nummulites, the detritus of limestone rocks. The shells are modelled in fine particles of sand from which all, or nearly all, calcareous matter has gone. Sand of this nature may have been one of the sources of the silicon of the compounds of double silicates with silica found in igneous rocks.}}

\centerline{*\hspace{15mm}*\hspace{15mm}*\hspace{15mm}*\hspace{15mm}*}
\bigskip

Light is thrown on the problem of the composition of the earth's crust by the marvellous results of stellar spectroscopy. Sir Norman Lockyer (`Inorganic Evolution,' p. 169) shows that calcium, silicium and magnesium exist in the hottest stars. He points out that the gaseous elements and the non-gaseous elements first formed, together with sodium and the above-named three elements, would tend to be the chief chemical substances on and over the surface of the planet. With regard to organic evolution he writes: ``The most easily thinkable organic evolution under these circumstances would be that of organisms built up of these chemical forms, chiefly because they would represent the more mobile or the more plastic materials;'' I once heard Sir Norman Lockyer remark that elements like calcium, silicium and magnesium ``had lost their chance.'' I understood him to mean that they had early become stereotyped into elements of relatively low atomic weight unable to undergo further ``inorganic evolution'' into more highly evolved ``elements'' with more complicated spectra. However, the time came when the relative lightness of certain compounds of the first two elements enabled them to become suitable scaffolding material for protoplasm.

\centerline{*\hspace{15mm}*\hspace{15mm}*\hspace{15mm}*\hspace{15mm}*}
\bigskip

Certain observations of Darwin on the shape of air-cells in banded trachytes are of interest from the point of view of the theory of nummulitic structure of those rocks. He writes (`Geological Observations,' p. 78): ``That some cause does produce parallel zones of less tension in volcanic rocks during their consolidation we must admit in the case of the small flattened crenulated air-cells in laminated rocks of Ascension.'' Darwin attributed the existence of zones of low tension to the stretching resulting from the movement of a slowly moving semi-solid mass. He compared the flow of the molten trachyte to that of a glacier, but Forbes' viscous theory of glaciers is no longer accepted. Darwin refers to a banded trachyte described by Scrope (Geol. Trans., 1827 (2), 2. p. 195, Pl. 24, fig. 2) which has burst upwards through an overlying mass of rock. Here I would point out with deference that it does not seem obvious why under such circumstances zones of low tension should arise from stretching. Harker (`A History of the Igneous Rocks') states that generally cooling magmas contract, and that in crystallization the net result is contraction. The banding of trachytes appears to me to result from cooling of an acid magma giving rise to vacuum planes in which small crystals form. The streaking of air-cells and the arrangement of felspar crystals lengthways would result from the flow movement. According to Darwin's theory the low-tension zones arise from stretching in the direction of flow, and the gases distributed through the mass become arranged in those zones. According to the theory of shrinkage due to cooling the magma in place of cooling like a homogeneous mass of iron contracts in visible zones, and thereby gives rise to vacuum-laminae. A ``flow'' of banded trachyte might be compared to a segment of a huge onion stone. In a large mass of this rock over four feet square, which I brought back from Porto Santo Island, I can often see four or five very fine bands passing through one nummulite shell. Lines of least resistance are present also in basic lavas, but are not visible (see p. 32).

When a piece of banded trachyte is broken with a hammer, it fractures along the plane of the dark bands and reveals a layer of nummulite shells. The flattened crenulated air-cells probably owe their peculiar character to the fact of their being included within hollow flattened coiled nummulite shells or \emph{débris} of the same. The shells can be made out in the white felspar bands wherever crystallization has not gone too far, as well as in the dark zones. I find to my surprise that it is possible to make out some structural details of the shells in thin sections of rock. Very thick sections, however, would probably show the Foraminiferal structure better.

\centerline{*\hspace{15mm}*\hspace{15mm}*\hspace{15mm}*\hspace{15mm}*}
\bigskip

The discovery of the true nature of the igneous rocks will throw light on the formation of land elevations and ocean depressions.

It seems improbable that volcanic islands are mere sporadic heaps of metamorphosed limestone. It is, I think, more likely that this rock is distributed over vast areas of the ocean floors under surface layers of the various oozes (see Chapter 11.).

\centerline{*\hspace{15mm}*\hspace{15mm}*\hspace{15mm}*\hspace{15mm}*}
\bigskip

It seems not unreasonable to assume that the shifting of huge masses of rock from the interior to form great heaps on the surface must give rise to unstable conditions of the crust.

Changed nummulitic limestones support the rest of the crust of the globe, and when large quantities of it are ejected, great hollows will be formed. It is not surprising that earthquakes are common at Messina and Catania, and in Calabria. Mighty Etna, covering a base of 100 square miles and rising 12,000 feet, must have left huge subterranean gaps. Even over mines made by human hands, houses and streets fall in, as at Droitwich for example.

One day when the earth's crust has become more stable, it will perhaps be possible to travel in these huge altered-limestone caverns ``measureless to man.''\footnote{\Fontauri{The intrepid voyagers in Jules Verne's romance, `A Journey to the Centre of the Earth,' started down a jokul in Iceland. After a wonderful journey, but before reaching their destination, they were hurled up on a raft borne on a column of hot water through the crater of Stromboli, much to the disgust and disappointment of the old German professor who had planned the trip.}}
\clearpage
\section{Petrological Considerations}
\paragraph{}
The fact of having discovered a new and fundamentally important feature in the construction of igneous rocks, will, I trust, justify me in submitting a few remarks on the Stromatoporoid theory from the petrological point of view.

I believe that in time to come, no student of petrology will omit to inform himself as to the structure of nummulites. It is not a matter of the first importance from the point of view of the present theory, whether a rock is acid, intermediate, basic or ultrabasic, or whether it is abyssal, hypabyssal or superficial. In the numerous igneous rocks of all varieties that I have examined I have found them to be disguised (? colonial) Foraminifera. Petrologists will, I am certain, appreciate the value of this discovery, for it will often aid them in understanding the construction of the masses of minerals with which they have to deal. The crystals of the silicon compounds, of which igneous rocks are mainly composed, are ``allotriomorphic,'' \emph{i.e.}, their form is moulded by their surroundings. Obviously it will be important to understand the real nature of these surroundings. Firstly, there is the general ``stromatic'' arrangement of the mass as a whole, and secondly, the individual disks have each a central umbilicus (through which, apparently, a common bud-bearing stolon passed), a coiled series of segments, and --- a by no means unimportant feature --- rows of pores. I often see a relatively large formless plug, or a crystal in the umbilicus, and little crystals in the segments of the coiled Foraminiferal disks of igneous rocks, and a banded structure may be apparent not only in the gneisses but also in granites. Probably the retention of considerable traces of Stromatoporoid structure in Plutonic rocks is due to the fact that owing to compression in a confined space there is but little tendency to displacement of individual portions of the whole mass. In volcanic rocks such as viscous lavas the Foraminiferal structure is not effaced, because the temperature has not been sufficient to reduce the semi-solid mass to a liquid state.

\centerline{*\hspace{15mm}*\hspace{15mm}*\hspace{15mm}*\hspace{15mm}*}
\bigskip

When a petrologist, with the kindest intentions, shows me a section of a dolerite in which the mineral olivine is breaking down into serpentine, and apparently giving rise to structure regarded by believers in the organic theory as Foraminiferal, and asserts that the existence of such phenomena is fatal to that theory, I can only fall back on my common-sense. These petrological facts are very interesting in their place, but of no value whatever as evidence against the nummulitic theory. For I see clearly and abundantly the Foraminiferal structure in many different kinds of igneous rocks. The nummulitic theory will be found to be thoroughly satisfactory, and the more so, the more it is looked into, and I believe it has only to be examined by competent men of science to be accepted by them. I have been told that I am making too much fuss about these insignificant Stromatoporoids about which no one knows or cares anything at all. Well, I believe that the discovery of the true nature of these organisms will prove to be of value in more than one science. The foundations of the world are built of nummulites which are probably Stromatoporoids.

\centerline{*\hspace{15mm}*\hspace{15mm}*\hspace{15mm}*\hspace{15mm}*}
\bigskip

Recently a petrologist concluded that a piece of rock we were both looking at was a sandstone. I at once denied that statement. The specimen was sectionized and found to be a finely grained basalt. I had seen the disks and knew the rock was a piece of metamorphosed nummulitic limestone. I feel justified in relating this, to show what a useful and ready diagnostic instrument will presently be available for petrologists.

\centerline{*\hspace{15mm}*\hspace{15mm}*\hspace{15mm}*\hspace{15mm}*}
\bigskip

The new theory appears to lead to the conclusion that one of the factors of magmatic differentiation consists in the removal of certain elements from one part of a Stromatoporoid scaffolding or mass of shells to another. Possibly the basalts and trachytes of Porto Santo were approximately similar in character before the outbreak of volcanic activities. The differences now existing may be due to the upper strata of silicated nummulites having been poured out suddenly, while the remaining layers became subject to more prolonged igneous action, thereby losing by fluxion their alkaline constituents. Perhaps the well-known ``Agua do Porto Santo'' owes its alkaline character to the presence of salts that had once formed part of what is now a highly acid trachyte.

Where trachytes are above basalts in the volcano, the condition might be due to fluxed elements sinking down into the lower strata and being retained there.

\centerline{*\hspace{15mm}*\hspace{15mm}*\hspace{15mm}*\hspace{15mm}*}
\bigskip

A petrologist tells me that the proofs of the mineral origin of \emph{Eozoön} are now much more convincing than formerly, but so are the proofs of organic origin. One is reminded of the competition between armour and projectiles or safe-makers and burglars. The nummulitic disk, however, is doubtless going to settle the question at last.
\clearpage
\section{Biological Considerations}
\paragraph{}
Stromatoporoids and nummulites in general are probably for the most part shallow-water organisms (see p. 40).

Sir John Murray estimates that if the land were cast into the sea and all reduced to a uniform level, an ocean 1450 fathoms deep would cover the globe. It does not follow from this that the floor of the ocean could not be formed of rocks composed of benthos organisms that had lived in relatively shallow water. As I mentioned above, it seems improbable that volcanic islands represent mere \emph{sporadic} heaps of metamorphosed nummulites. It is, I think, more likely that they are local upheavals of a widely spread formation underlying the oozes which carpet the ocean floor. The depressions resulting from the assumed sinking of the shallow water formation would be complementary to the land elevations caused by the upheaval of that formation.

Wherever there are igneous rocks (plutonic and volcanic) and certain metamorphic rocks, there are nummulitic formations, \emph{i.e.} portions of ancient seafloors.

During the Eozoic Era, the plant might --- if a little poetic licence can be allowed, and not much is asked for --- almost be regarded as a gigantic Rhizopod encrusting a foreign body. A varied fauna and flora, mainly pelagic, must have existed contemporaneously, just as in the case of a great reef at the present day. These pelagic organisms probably led to a precarious existence evading the hungry pseudopods of the Dawn Animal. Possibly the lowest Metazoa (Zoophytes, Corals, Medusae, \emph{etc.}) owe the possession of those curious organs of offence and defence --- the thread-cells --- to a process of natural selection which led to the survival of the organisms which could best escape the deadly zone of pseudopods of the all-pervading \emph{Eozoön}.\footnote{\Fontauri{The word ``Eozoön'' is not used here in a strictly systematic sense. It is not at present possible, for instance, to state definitely whether or not the Stromatoporoids of the gabbro, gneiss and limestone of the Lower Laurentian region belong to one and the same genus and species, \emph{viz.} to \emph{Eozoön canadense} Dawson, but it seems to me probable that they are identical.}}

The Parazoa (Sponges) retained to a greater or less extent the microphagic way of feeding. They began a free life, but soon settled down and failed to travel far along the path of evolution. The Metazoa, on the other hand, developed a gastric cavity whereby the cells co-operated and poured out secretions which digested the food. The highly fed and free swimming Metazoa developed organs which responded to light, sound, \emph{etc.}

Although \emph{Eozoön} represented the dominant type during the Eozoic Era, yet, from the point of view of the higher life, Destiny had done with the Dawn Animal. A stationary organism enclosed in a shell and provided only with mere prolongations of sarcode (pseudopods) was not endowed with organs which could help it to rise in the scale of being. The first essential for such an ascent was the development of a motile apparatus enabling the organism to move about as a whole, and accordingly the pseudopods were replaced by flagella. From the higher Protozoa were evolved two great groups, the Parazoa and Metazoa.

\centerline{*\hspace{15mm}*\hspace{15mm}*\hspace{15mm}*\hspace{15mm}*}
\bigskip

The neighbourhood of volcanic and coral islands appears to furnish interesting archaic types of fauna.

Several of the recent Pharetron sponges, and some species of Lingula are found in the vicinity of these islands.

That strange anachronism, the sponge-alga, \emph{Merlia normani}, which is identical in structure with some of the Silurian Monticuliporoids,\footnote{\Fontauri{Recently Dr. E. R. Cumings (\emph{Proc. Palaeont. Soc.} Indiana, 1911) has produced evidence that seems to show that certain of the Monticuliporoids are Polyzoa. I have evidence apparently proving that some of these fossils are sponges. There may be here a strange instance of homoeomorphy.}} is found off Porto Santo Island, and has probably existed there since Palaeozoic times. It could hardly have migrated from elsewhere, for Siliceous sponge larvae only exist in the free-swimming state for a few hours. It seems probable that this remarkable organism will be found off all the volcanic islands in the temperate zones of the Atlantic.
\clearpage
\section{On Dolomitic Limestones}
\paragraph{}
It may at first sight be difficult to understand what connection can exist between \emph{Eozoön} and dolomites, but I have found there is a close relationship, and that \emph{Eozoön canadense} has given the clue to the real nature and origin of the Permian dolomitic limestones.

Tyrol is the fairyland of the Alps, and a glorious country to spend a vacation in. From the summits of its cimas and campaniles one looks down sheer precipices thousands of feet deep, and on every side are pinnacles of most weird and fantastic shape.

The mountains are made of pure magnesian limestone, and very frequently no fossils are to be seen in them. I believe the most accepted theory is that the rocks are coral reefs, but that all, or nearly all, traces of coral structure have disappeared during the process of dolomitization, \emph{i.e.}, the conversion of the original pure calcium carbonate into carbonate of lime and magnesia.

I have made the strange discovery that the Austrian Dolomites and the Permian dolomitic limestones of England are purely Foraminiferal, and made up of nummulites which may be colonial. I have seen the nummulitic disks in numerous examples of these rocks. Further, a layered arrangement is frequently discernible.

The substance of the rock has become so fused and amalgamated probably partly by solution before dolomitization, and by pressure, that it is not always easy, especially in the highly crystalline rocks, to see the disks with their central umbilicus and radial and concentric pattern, but they are there, and can frequently be made out without much difficulty by means of a lens.

In the case of some dolomitic rocks, other fossils than the nummulites are present. The Foraminifera did not have the sea all to themselves, other organisms died, and sank down to the Foraminiferal reefs and became covered up. I have seen the disk structure best in a specimen of pure dolomite from Ledrotal, Tyrol, one of several examples procured for me from abroad.

This discovery of the origin of the dolomitic limestones is an especially fortunate one at the present moment. For perhaps suspicion will arise that there may be, after all, some truth in the statement that granites, basalts and lavas are likewise composed of nummulite disks. Then again, the fact that dolomites form cliffs thousands of feet thick may enable us to realise that the primitive nummulites or Stromatoporoids might --- and did --- form nearly all round the globe, beds of very porous limestone of immense thickness, which became transformed into silicates. These, owing to their position became compressed on account of the shrinkage of the planet, heated, molten or semi-molten, and finally cooled and crystallized. In spite of all this the original ``layering'' and also the form of the disks can often be clearly detected; of that I am quite certain.
\clearpage
\section{On Oolitic Limestones}
\paragraph{}
The Jurassic Oolitic Limestones have a structure resembling petrified fish-roe (oon lithos stone egg). The usually spheroidal grains vary considerably in size in different oolites.

The peculiar structure of these rocks, which form very extensive beds in the West of England, has been attributed to the wearing down of coral reefs. The fine grains are supposed to have sunk gradually in the sea. When they reached certain zones of water containing excess of carbonic acid resulting from the decay of organic matter, it is assumed that the carbonic acid slightly attacked the sediment, forming the soluble bicarbonate, which was again precipitated as calcium carbonate among the interspaces of the slowly settling mud. This course of events is found to take place in the neighbourhood of coral reefs at the present day. It is pointed out that oolitic grains form in certain mineral springs and also that some coral sands in the West Indies are oolitic. Evidently oolitic structure can arise in many ways.

Dr. Rothpletz (`American Geologist,' 10., 1892, p. 280, transl.) showed that oolitic granules are present in Salt Lake, and that they are there formed by a lime-secreting alga --- \emph{Gloeocapsa}. Further he found oolitic granules on the shores of the Red Sea, and a little way inland, an oolitic limestone cliff supposed to have been formed by blown (oolitic) sands. The loose oolitic granules often had a nucleus of sand grains. All this may be wholly correct, but if I were in the neighbourhood, I would think it worthwhile to see if the cliff were not in reality a Foraminiferal one from which the oolitic granules on the shores had originated. The nucleus of sand grains in each granule might have resulted from solution of the calcium carbonate of the pure granule and redeposition round the sand particles.

I noticed on the outer surface of certain specimens of \emph{Eozoön canadense} that the Foraminiferal disks had a somewhat ``oolitic'' appearance, and this led me to examine carefully the Jurassic oolitic limestones. I found that they were composed of nummulitic disks, and that they were probably Stromatoporoids.

At first, when a fine grained oolitic rock is viewed with a lens, only a confused generalised impression of uniformly granular structure will be obtained, but gradually the disks, with the central umbilicus and concentric arrangement of grains will be made out. The finer-grained oolites show the disk structure the most clearly and especially on weathered surfaces. It is usually possible to find fairly good disks with a coiled arrangement of grains, and here and there the outer wall of the original shell. The smaller oolitic grains when broken, often show --- not a concentric structure --- but edges of parallel series of bars.

Oolitic structure, in the case of the mid-Jurassic Oolites, has, I believe, arisen in the following manner. These limestones contain over 90 per cent. of carbonate of lime. When they have been raised above the sea, the rainwater charged with carbonic acid has percolated through, dissolved out and redeposited some of the lime in the chambers and around the partitions of the hollow shells. The soft delicate Foraminiferal disks have become more or less crushed together by the super incumbent weight. The softness and uniformity make this stone a good ``freestone'' cutting easily in all directions.

Pisolitic limestones with large spherical or oval pebbles at first sight resemble conglomerates formed of gravel carried down by a rapid current, but, like the fine-grained oolites, they have been formed \emph{in situ}. The pisolitic structure is due to the same causes as the finer oolitic, but the process of solution and re-deposition has gone on to a greater extent. Possibly the pisolitic condition results from the position of the limestone in a situation where an abundant supply of water would drain through the mass. A pisolitic pebble is made of calcite and is formed of concentric layers usually with a harder crystalline nucleus. These pebbles might be compared to the stalactites and stalagmites in a limestone cavern. They have apparently the same structure and mode of formation.

The escarpments formed by oolitic limestones, in my opinion, resemble Foraminiferal chalk cliffs. The main difference between those commonly so designated and the oolites is that the Cretaceous ones have not become oolitic, for I now find that chalk is essentially a nummulitic rock. (See Appendix.)

In addition to examining numerous samples of Oolitic rocks in museums and stone yards, I travelled to an oolite district, \emph{viz.} Cheltenham, to visit the freestone quarries. At Leckhampton Hill the Cotswolds form a bold escarpment overlooking the plain on which Cheltenham stands. I was fortunate in meeting two quarrymen, who gave me a great deal of valuable information. In a piece of freestone almost homogeneous to the inexperienced eye they could always detect ``the plane of bedding.'' The stone-cutting saw went easier along that plane than in other directions --- in spite of the name of the stone, \emph{viz.} ``freestone.'' The men told me of a quarry where the stone was so soft that it could be cut with a knife, and so close grained that unless the quarrymen marked it he would be unable afterwards to know the direction of bedding. But a little later on, the weather would bring out the lines of bedding, and if a house were built and the stones fitted without regard to these lines --- at first invisible --- an ugly effect would be produced.

Some freestones are as absorbent as sponges, and hence the houses and cottages built of this material are often very damp. It is not surprising that the primitive Stromatoporoids were saturated with silicates and the dolomites with magnesium salts.

A very free freestone, soft and easily cut in any direction, was the kind I was looking for, because it would be less modified from the original state. The quarry above referred to was a ``mine-quarry'' in the Inferior Oolite at Whittington, about six miles east of Leckhampton. The disks in samples from the Whittington Quarry are fairly clear not only \emph{en face}, but even in vertical section. A word of caution is necessary, \emph{viz.} not to mistake for disks the fragments of Crinoids present in some portions of the strata of Oolitic limestones.

Later on I visited Portland, and also the Oolite district round Bath, and examined numerous examples of freestones in quarries, stone yards, and especially in old buildings. The weathered surfaces of old abbeys and churches often show the nummulitic disks very well, because the gentle action of dissolving agencies carries away the granules which mask the true structure, and leaves the outline of the shells. I found the Foraminiferal structure well shown in some large sooty flakes given me by the masons repairing Henry $7^{th's}$ Chapel. The yellow under surface of the flakes still showed the oolitic structure masking the form of the disks.

\centerline{*\hspace{15mm}*\hspace{15mm}*\hspace{15mm}*\hspace{15mm}*}
\bigskip

Several observers, notably Nicholson and Etheridge, and Wethered, have described strange dubious organisms which they have found in Ordovician, Carboniferous and Jurassic limestones. These organisms, which have been relegated to the genera \emph{Girvanella},\footnote{\Fontauri{I am much indebted to Dr. J. S. Flett of the Scottish Geological Survey for his kindness in sending me a piece of Girvan limestone. I wished to examine a specimen of Nicholson and Etheridge's \emph{Girvanella problematica}. I found the limestone to be nummulitic and probably Stromatoporoid.}} and \emph{Mitcheldeania}, have certain characteristics in common, in that they are composed of layers of concentric flattened disks with fine radiating tubules. These genera are possibly founded on the varying appearances presented by nummulitic disks. I would suggest that the vermiculate structure often seen in \emph{Girvanella}, may in some cases be due to the formation of casts of the coiled chambers of the coils within coils of the nummulite shells. A vertical or oblique section of a pile of such shells would show a labyrinthine vermiculate structure. Sometimes, however, a crinoid stem or some other object appears to form a nucleus to a mass of coils (Wethered, ``On the Microscopic Structure of Jurassic Pisolite,'' \emph{Geol. Mag.} 1889, 6. p. 196, Pl. 6. fig. 11). Whatever be the precise explanation of the various structures found in sections of Jurassic Oolites, no one, I believe, could study a weathered surface of freestone on some ancient abbey without being soon convinced as to the nummulitic nature of that stone. Nature's ``preparations'' made in a very leisurely way, are in this case the best we could have.

\centerline{*\hspace{15mm}*\hspace{15mm}*\hspace{15mm}*\hspace{15mm}*}
\bigskip

The Blue Lias limestones of Lyme Regis are partly Foraminiferal, and contain nummulitic disks. The latter are probably Stromatoporoid. These limestones contain many fossils, especially Ammonites, which of course became covered up when they died. Owing to their softness the disks are compressed to flat flakes, and from super-incumbent pressure they fill in the outer chambers of the Ammonites. The existence of great pressure may be judged from the fact that even fairly large Ammonites often form merely thin films, which can only with great difficulty be separated out from the matrix in which they are imbedded.

\centerline{*\hspace{15mm}*\hspace{15mm}*\hspace{15mm}*\hspace{15mm}*}
\bigskip

Also I have found the shells in the grey mud coating some Ammonites I picked up at Chapman's Pool and Kimmeridge Beach.

The Upper Greensand Reigate Firestone is a calcareous nummulitic rock infiltrated with sand. The disks are clearly to be seen on the weathered surfaces of old stones from Henry $7^{th's}$ chapel.

When the Kimmeridge mud and firestone were put into acid they effervesced furiously, leaving deposits respectively of fine mud and fine sand.

\centerline{*\hspace{15mm}*\hspace{15mm}*\hspace{15mm}*\hspace{15mm}*}
\bigskip

In a specimen (A 1663) in the Natural History Museum, labelled Arenicolites, Basal Quartzite, from Assynt, Sutherland, I find traces of the Foraminiferal disks. I believe we have here a silicified Cambrian or Lower Silurian Stromatoporoid. A larger specimen in the Jermyn Street Museum, from Assynt, and labelled ``piped quartzite,'' shows fairly well that the pipes or Arenicolites are etched out columns of nummulites. (The ``Basal Quartz'' is evidently the formation C depicted in the first coloured plate in Murchison's `Siluria,' third and fourth editions.)

\centerline{*\hspace{15mm}*\hspace{15mm}*\hspace{15mm}*\hspace{15mm}*}
\bigskip

About the beginning of the Cainozoic Era there was a great development of the undoubtedly \emph{individual} type of nummulites. Over immense areas extending from Morocco and Spain to the frontiers of China nummulitic formations thousands of feet thick and entering bodily into the formation of the middle and upper parts of great mountain systems (Atlas, Pyrenees, Alps, Carpathians, Himalayas) were laid down evidently over rising areas. Portions of Eocene Sea-bottoms are found at the summits of Himalayan peaks. The nummulite shells belong to many species, and vary in size from a pin's head to a Marie biscuit.

It is not unlikely that the Stromatoporoid type persisted beyond the Mesozoic Era. Indeed, a colleague has informed me that he has seen what appears to be a recent Stromatoporoid. I think it probable that there are Stromatoporoids living now in warm waters in the neighbourhood of coral islands. These colonial Foraminifera would probably resemble corals or coralline algae.

\centerline{*\hspace{15mm}*\hspace{15mm}*\hspace{15mm}*\hspace{15mm}*}
\bigskip

\small
{Postscript.} --- Mr. E. B. Wethered has most kindly shown me his valuable collection of slides and beautiful photographs of \emph{Girvanella}. I am now absolutely convinced that the explanation given on p. 67 is the true one. I can clearly see sections of shells \emph{en face} with two or three concentric ``tubes'' with \emph{radial septa}. Vertical sections show linear rows of hollow blocks or spaces, the inner walls of coils-within-coils having been dissolved; also series or pores are visible occasionally. In an interesting paper on \emph{Girvanella}, to which Mr. Wethered referred me, F. Chapman (Australian Assoc. Adv. Sci., Adelaide, 1907, 11. p. 384) assigns the genus to the blue-green algae, and states that ``it is here shown to have no claim to be regarded as one of the foraminifera.'' \emph{Girvanella} has been looked upon as a Foraminiferan, Sponge, ``Stromatoporoid'' (Hydrozoa), calcareous alga, worm-tubes, \emph{etc.} Its characters, however, are based on the altered condition of the shells of a colonial nummulite or Stromatoporoid (in the new sense).
\normalsize
\clearpage
\section{On Pennant Grits}
\Large
\paragraph{}
Recently, when looking over a stone-yard at the foot of Leckhampton Hill, I saw among the great heaps of freestone several slabs of grey rock which I took to be igneous. The Foraminiferal disks were very distinct both \emph{en face} and in vertical section. In fact these siliceous medals were in some respects better preserved than the calcareous oolitic ones which had been formed immeasurable aeons later. On my return to London, I wrote to the manager of the stone-yard for information about these slabs, and was told they were blue Pennant, \emph{i.e.} mid-Carboniferous grits. I had seen clearly the nummulitic structure, however, and I began to think that if sedimentary grits could exhibit a structure resembling that of Foraminiferal disks, then possibly my whole theory might be wrong, and for a moment, I even thought, not without a feeling of relief at the prospect of a return to more peaceful and less polemical pursuits, of consigning my manuscripts to the waste-paper basket.

I was unaware then of a fact which I found later to be of common occurrence, \emph{viz.} that the detritus derived from igneous and certain limestone rocks is composed to a great extent of nummulitic shells and not merely of shapeless particles of sand and mud.

In the memoir of the Geological Survey on the Geology of the South Wales Coal Field, Part 8, the country round Swansea, 1907, p. 58, the following account is given of the lithological characters of the Pennant.

\begin{quote} 
``E. 3579. West Quarry, Cockett.''

``A grey micaceous sandstone. Under the microscope it consists of rather well-rounded grains of quartz, with a small amount of felspar which includes orthoclase, micro-line and acid plagioclase. Colourless mica is common in worn frayed-out tablets. Small fragments of dark shale, mostly sharp-edged and angular are frequent. There are a few scales of biotite and a little chlorite. The cementing material, which is rather abundant, looks like fine \emph{débris}. It consists of small siliceous grains and fine scaly matter, partly original argillaceous sediment, partly the product of the decomposition of felspathic \emph{débris}. Carbonates are practically absent. There are black spots of iron-oxide and opaque finely divided carbonaceous (?) matter.''
\end{quote}
\paragraph{}
This is undoubtedly an excellent description of Pennant, but no mention is made of the nummulite shells.

Mr. A. Dale, the chairman of the Stone Concessions Co., very kindly sent me a piece of one of the slabs of Pennant I had seen at Leckhampton. On re-examining it, I again became convinced that it was mainly composed of silicated nummulitic \emph{débris}. I spent the next weekend at Swansea in order to investigate the Pennant series where it is best developed, the seams attaining a total thickness of 970 yards. In this thickness are included several seams of coal each with its under-clift of clay, and also layers of conglomerate, notably the one discovered in the Tawe Valley by Sir. W. Logan. After examining a quarry in Swansea itself, I visited Inismidu about twelve miles from the town. Here I found plenty of material showing well the nummulitic structure, and furnishing evidence of metamorphic action. On some slabs there are traces of graphite, and on many pieces layers of quartz\footnote{\Fontauri{Several of my Inismidu specimens show well the transition between the opaque grey rock and the bands of quartz. In some instances, part of a Foraminiferal disk is opaque, and part crystalline. In a specimen, given to me by Mr. H. P. Wiggins, of pure vitreous crystalline quartz rock from the Himalayas, it is possible to trace the disk structure, through with difficulty. Probably many quartz rocks will be found to be composed of metamorphosed nummulites.}} wholly continuous with the bluish-grey portion of the rock.

In some of my specimens from the Ffinone quarry there are traces, not only of layers of bituminous matter, but also of fine flakes showing organic structure. The bituminous material, which probably resulted from condensation of products of volatilisation of vegetable matter, is present in fine films, or it may form little blocks in the chambers of the Foraminiferal shells. Some pieces of rock show numerous small crystals of quartz.

Pennant grits are composed mainly of nummulitic detritus derived from igneous rocks. The micaceous character of the grit is, I suppose, in itself a proof that the original mother-rock was probably igneous. The bands of quartz and the films and particles of bituminous material evidently must have resulted from metamorphic action taking place in the sedimentary deposit. The Pennant is only one of innumerable examples of the endless mutation of things, for the igneous rock whence the secondarily-metamorphosed grit was derived was once a pure nummulitic limestone.

I have detected nummulitic structure in the Millstone grits (\emph{e.g.} Farewell Rock), though only with great difficulty, apparently owing to the high degree of metamorphic change whereby the silica has become re-crystallized.
\clearpage
\section{On Slates}
\paragraph{}
After I had completed, as I supposed, my preliminary investigation on the occurrence of Stromatoporoids and nummulitic rocks throughout the Geological Eras, I became aware of the apparent existence of a great gap in the sequence of these organisms. They were extremely prevalent during the Eozoic Era, and again during the Ordovician, Silurian and later periods of the Palaeozoic Era, but I had only scanty evidence of their existence during the great Cambrian Period, in which they might have been expected to be very abundant.

I have now had the good fortune to find that the gap above alluded to is filled in by the slates, which are undoubtedly composed of the compressed silicated shells of nummulites, these being almost certainly colonial nummulites or Stromatoporoids.

Recently, when examining a piece of slate with a x10 lens, I saw the much flattened barely perceptible disks all over the surface, and the willow patterns at the edges of the slate. Naturally a very close scrutiny is necessary, and every minute detail must be carefully noted. Generally an umbilical area is seen at the centre of each circle, and not infrequently in some rough slates it is possible to make out a concentric or radial arrangement of flakes, and even rows of pores. The fine, curved, closely set lines at the edges of the slab are the edges of the coils-within-coils of the nummulitic shells. The still finer spaces between these lines are divided up by the edges of the septa, \emph{i.e.} the spaces between the main ribs of the willow-pattern area occupied by scalariform series of fine lines at right angles to those ribs, with the result that an extremely fine square-meshed network pattern is formed.

The search for the disk- and willow-pattern reminds me of the game of picture-puzzles, where one has to detect some hidden design in a maze of lines or in masses of light and dark shading.

The fact must be taken into account that the nummulites out of whose skeletons the slate is formed lived many decades of millions of years ago, and that the original soft, delicate calcareous shells have become silicated, crushed together by enormous pressure and finally rent asunder artificially in various planes by means of cleavage.

When the observer sees at a broken edge of a slate the rows of very minute square meshes, he will begin to realise that he is looking at something other than a compressed mass of particles of mud. The evidence afforded by the circles or disks might not, perhaps, be so convincing, because it is conceivable that masses of polyhedral or globular particles could become pressed into flakes.

In order to inspect a large and varied assortment of slates, I obtained permission from Messrs. Bingley and Folliot to go over their extensive slate yards, which are crowded with material from the quarries of Great Britain and Ireland, Europe, and the American continent. Here I saw rough varieties showing clearly the Foraminiferal disks both \emph{en face} and in section. I tried the beautiful experiment of cleaving a long thick slab into thin plates, but found that practice and experience are necessary. Generally, after an excellent start, and when one is beginning to feel proud of oneself, the split tails off to the surface, unless due counter-pressure is exercised with foot and knee.

A charmingly written account of slates and of some of the very interesting problems connected with them is given by Tyndall in his `Fragments of Science,' and also in `The Glaciers of the Alps' (Introduction). He relates that one day he was standing at the edge of the Penrhyn slate quarry with a friend who explained the phenomenon of cleavage by comparing the slate rock to deposited sandstones with planes of bedding. Tyndall remarks: ``But this was a mistake, and, indeed, here lies the grand difficulty of the problem,'' and he points out that the planes of cleavage often make a high angle with those of stratification. He asks: ``What is the agency that enables us to split Honister Crag or the cliffs of Snowdon into laminae from crown to base?'' Professor Sedgwick had ascribed to slaty cleavage a crystalline origin. He supposed that the particles of slate rock were acted on after their deposition by ``polar'' forces, which so arranged those particles as to produce cleavage planes. Tyndall pointed out, however, that slate was composed of heterogeneous particles, which, though very fine, were of immense size in comparison with molecules.

H. C. Sorby, D. Sharpe, and Tyndall brought forward evidence based on observation of slates and the fossils they contained, and on experiment, to show that pressure alone was sufficient to account for cleavage.

In the case of most of the slates I have examined, the Foraminiferal disks are \emph{en face} on the surface of the slab and in vertical section at the broken edges of the latter. There would be a natural tendency for the disks, when deposited as sediment, to lie horizontally like saucers thrown into the water.

At the same time, I have seen disks lying very obliquely, or even vertically cleaved, on the surface of the cleavage plane. Occasionally, too, disks \emph{en face} are seen at the broken edges.

Recently I have seen some beautiful examples of slates showing nummulitic structure, \emph{viz.} in weathered slabs on the roofs of London buildings. The nummulites are well shown in certain rather calcareous slates from Lancashire quarries. Where a slab has been lightly overlaid by its neighbour, the friable products of weathering have not been washed away by the rains. Here one can often see the umbilicus, pores, and compressed layers of bran-like scales formed by the broken-down alar prolongations. These slates effervesce considerably in acid.

Tyndall refers to the mottled pale green patches on slates, and considers them to be rolled masses of fine mud containing less iron than the darker portions. I find that these spots resemble the darker parts in being composed of nummulitic shells. At Penrhyn I saw pale ``spots'' in the shape of rings surrounding a dark grey central area, and a shell may be partly pale and partly dark. The deficiency of iron which Tyndall found may have caused not only the paler colour but also the greater smoothness. For a slight difference in composition might easily bring about a variation in the grain and texture.

Adam Sedgwick, in `The Synopsis of Classification of British Palaeozoic Rocks,' 1885, Introduction, p. 34, makes an interesting comparison between slate and granite. Of the latter he writes: ``It gives no appearance of a slaty structure; yet it cleaves (sometimes through large spaces) in one constant direction, and in nearly parallel planes, much more readily than in any other.'' I can see very distinctly in medium-grained granites the crystallized shells \emph{en face}, and here and there the willow pattern of shells in vertical section, \emph{i.e.}, both granite and slate, are composed of nummulite shells, but I believe the former rock to be a limestone metamorphosed \emph{in situ} by chemical action (decalcification and ``silication'') and by heat. The slate, on the other hand, is evidently a sedimentary formation.

I paid a weekend visit to North Wales in order to see the slate quarries at Penrhyn and Llanberis, and to examine planes of bedding and cleavage, and jointings.

I was shown over the Penrhyn quarry by Mr. R. Morris, who gave me much valuable information and many interesting specimens. All of the latter fully confirm my statement that slates are composed of compressed nummulitic shells.
\clearpage
\section{The Abyssal Red Clays}
\begin{quote}
``They mount up to the heaven, they go down again to the depths.''\footnote{\Fontauri{The reference is to the occurrence of nummulitic seafloors on the summits of mountains and in the abysses of the ocean. (The Psalmist's utterance [verses 23-30] is an inimitably \emph{naïve} description of an experience painfully familiar to land-dwellers ``that go down to the sea in ships.'' The narrative is so vivid, that possibly it may be a reminiscence of the poet's own sufferings in the exposed roadstead off Jaffa, and of his deliverance on reaching the ``desired haven'' within the rock barriers.)}} --- Psalms. 107. 26.
\end{quote}
\paragraph{}
In 1875 Huxley wrote (`On some of the Results of the Expedition of H.M.S. \emph{Challenger}'): ``I think it probable that we shall have to wait some time for a sufficient explanation of the origin of the abyssal red clay...'' I believe that the answer to this riddle of the red clay has now been found, and that a wonderful light will, in consequence, be thrown on certain great problems of geognosy, and to a lesser extent, of biology.

The occurrence of volcanic islands, \emph{i.e.} of heaps of nummulitic seafloors, scattered over all the oceans, and the wide distribution of igneous rocks, which are also nummulitic, seemed to point to the probability that vast areas of the sea-bottom were of a similar nature. It seemed hopeless, however, to expect to be able to verify an hypothesis of this kind, for deep-sea investigations showed that the floor of the ocean was carpeted with various kinds of oozes which had been accumulating for countless ages. How could instruments let down through miles of water do more than scratch the surface of such deposits?

As a forlorn hope, I proceeded to examine the Challenger collection of deep-sea deposits now stored in the Natural History Museum. Out of 580 samples I selected the abyssal Red Clays, because they would be nearly free from admixture with the calcareous skeletons of the pelagic Foraminifera which mostly become dissolved before reaching the abysses beyond 2500 fathoms where the Red Clay abounds. Further it was desirable to choose a sample remote from volcanic areas.

Accordingly I started with Sounding 199, Station 165 A; June 17, 1874; Lat. 34$^{\circ}$ 50$^{\prime}$ S. Long. 155$^{\circ}$ 28$^{\prime}$ E. (about 200 miles east of Sydney, on the voyage from Sydney to New Zealand); depth 2600 fathoms. The sample, an unusually good one, consists of thirty small lumps of hard buff-coloured clay, containing 6.54 per cent. of calcium and only 5 per cent. of Foraminifera. Full details are given in Murray and Renard's ``Challenger'' Report on the ``Deep Sea Deposits,'' 1891, p. 85.

The largest fragment, which is about the size of a haricot bean, is in the form of a little oblong slab about 15 millimetres long, 5 mm. broad, and 4 mm. thick. On one surface which clearly formed part of the ocean floor, it is not difficult to see eight or nine circular disks or parts of disks, about 3 to 4 mm. in diameter, each with a central raised or sunk circular area (umbilicus), and with traces of radial and concentric pattern and pores. Where the disks are broken obliquely the edges of coils within coils are visible, and vertical sections show the extremely minute square-meshed network pattern formed by the edges of the much-compressed alar prolongations and septa. Very careful inspection of horizontal surfaces of the disks reveals concentric (really spiral) or radiating series of fine scales each usually with a minute dot or pit at or near its centre. I have seen the same appearance in the bran-like scales of the disintegrated nummulites of weathered slates. These scales appear to be little blocks cut out by the edges of the radiating septa and of the spirally coiled segments. (The drawing of a spiral figure and of radii will best illustrate the meaning). The average size of the scales is about 15 μ, but they are usually smaller at the centre and larger at the periphery of a shell.

Sometimes very minute rows of dark crystals (insoluble in water) can be seen, appearing as if they had been formed in relation to the cavities of the shell. I have seen the same phenomenon in shells composing a piece of granitoid rock from Porto Santo. I was permitted to dissolve a small piece of another fragment in acid. There was a little effervescence and the particle became a sandy lump with very little coherence.

Several well-preserved shells of \emph{Globigerina} are partly or wholly immersed in the first fragment. All stages can be seen, \emph{viz.} from being just stuck on the surface, half in and half out, to being sunk with the upper convex outline of the shell showing through the surface, or lastly they may be wholly immersed and at different depths. These apparently trivial observations will explain the real nature of all the deep-sea oozes.

The Red Clay from Station 165 A, June 19, 1874, 2600 fathoms (about 200 miles E. of Station 165), contains 13 per cent. of recent Foraminifera. The Globigerina ooze from Station 165 B, June 21, 1874, 1975 fathoms (about 300 miles E. of Station 165 A), contains 66 per cent. of recent Foraminifera. In both these last samples there is the matrix of ancient changed nummulites more and more filled with Globigerina shells. Again a pure Red Clay from Station 264, Aug. 23, 1875, 3000 fathoms (Sandwich to Tahiti), is almost wholly composed of a compressed mass of palaeo-nummulites. Even in the purest Globigerina ooze obtained by the ``Challenger,'' \emph{viz.} that from Station 33 S, Mar. 21, 1876, 1990 fathoms (Tristan da Cunha to Ascension Island), the nummulitic character of the matrix is discernible, crammed though it is with shells of pelagic Foraminifera.

Not to crowd in too many details, I would mention that I have examined deep-sea oozes of all kinds, \emph{viz.} Globigerina, Radiolarian, Diatom, Blue Mud (Terrigenous), Red Clays from the areas where they occur in the Atlantic, Indian Ocean, and Pacific, and have found them to be composed of a fundamental matrix of palaeo-nummulitic mud crowded with the remains of various organisms.

In the case of volcanic sands there is a singular contiguity of fragments of ancient sea-bottoms which have been ejected from upheaved areas, with remains of nummulitic sea-bottoms which have been depressed so as to form troughs. The Pteropod oozes again, do not reveal any traces of their immersion in nummulitic mud, probably because the latter got washed away. Sir John Murray (`Science of the Sea,' p. 214) writes concerning certain samples such as manganese nodules, \emph{etc.}, coming up in the trawl, ``it is, however, almost certain that these various objects were embedded in clay, which passed through the meshes of the net.''

Foraminifera actually living on the ocean floor do not sink, precisely because they are living, the shells being immersed in outspread contractile protoplasm.

Several of the Diatom oozes were merely powdery, but fortunately sometimes lumps were present, and these showed the nummulitic structure. The Radiolarian oozes reveal the nummulitic structure better even than Red Clay, and this is curious, seeing that the latter is the purest form of the ancient changed limestone and moreover, the common matrix of all the oozes. Apparently the long spines of the shells act as the axial rods of a kind of \emph{béton armé}.

The floor of the ocean is composed of an altered nummulitic limestone. The consistency is that of a soft clay or mud, but the nummulitic shells still retain in a varying degree their shape and structure.

When foreign bodies arrive at the surface of the ooze they gradually sink in and become more and more crowded in the nummulitic matrix, thereby imparting special characters to it. In accordance with a logical terminology for the oozes, they would be named Globigerina palaeo-nummulitic, Radiolarian palaeo-nummulitic, \emph{etc.} Sir John Murray estimates that of pelagic deposits the nearly pure palaeo-nummulitic ooze, the Red Clay covers 51.5 millions of square miles, in the Atlantic, Indian Ocean, and Pacific (mainly). Globigerina ooze occupies 49.52 millions of square miles, Radiolarian 2.29 millions, Diatom 10.88 millions, and Pteropod ooze about 0.4 million. Terrigenous deposits (blue, red, green, volcanic and coral muds) cover an area of 28.05 millions of square miles.

We now have direct and certain evidence that the greater part of the 142 millions of square miles of ocean floor is formed by a nummulitic limestone. One of the layers of the globe is made of this material and might be termed ``nummulosphere,'' this being the base of the ``biosphere'' of Walther.

The abundant evidence of compression in the silicated nummulite shells is mainly and primarily due to their being squeezed during great earth movements. The existence of minute crystals in some of the red clays, and of admixture of volcanic products (\emph{e.g.} pumice and glassy and volcanic products in the Red Clay from St. 165, which is far removed from volcanic areas) seem to indicate the existence of igneous action, which is, I believe, usually simply a symptom of compression.

It is fairly certain that these altered nummulitic limestones never grew in the abysses where they are now found. For one thing, how could such gigantic masses of organisms get their food and their calcium carbonate? Like the reef corals they must have been relatively shallow-water organisms. The sinking down of a shallow-water formation to a depth of 20,000 feet is not more wonderful than the up-raising of a nummulitic seafloor to form a Himalayan peak.

\centerline{*\hspace{15mm}*\hspace{15mm}*\hspace{15mm}*\hspace{15mm}*}
\bigskip

The fact that the disks in nummulitic clays can retain their form is seen in the case of the Fullers' Earth formation of the Inferior Oolite. Also it is not difficult to see the disks in coloured clays used for artists' pigments, these often being the products of disintegrated igneous rocks.

Further, I can detect the disks in a sandy clay which I dug out from a railway cutting (Goldsworthy Hill, Middle Bagshot) near Woking. Accordingly the persistence of nummulitic pattern in the abyssal Red Clay is not a unique phenomenon.

\centerline{*\hspace{15mm}*\hspace{15mm}*\hspace{15mm}*\hspace{15mm}*}
\bigskip

It is singular there are so few recent species of the genus \emph{Nummulites}. Brady describes only one in his ``Challenger'' Report, \emph{viz.} \emph{N. cumingii}, a tropical form living in 10-25 fathoms. He writes, ``The genus \emph{Nummulites} exemplifies the highest type of structure attained by perforate calcareous Foraminifera.'' The nummulites, whether individual or colonial, of igneous and of many other rocks have the typical characters of the genus \emph{Nummulites}. It is strange that the highest type should have existed from the earliest period.

\centerline{*\hspace{15mm}*\hspace{15mm}*\hspace{15mm}*\hspace{15mm}*}
\bigskip

The discovery of strong evidence tending to prove that the abyssal ocean floors have gradually sunk to their present position may throw light on problems of distribution of marine organisms. Life did not originate in the great depths. Fixed organisms came to be there either by migration from lesser depths, or by being carried down passively owing to sinking of the floor on which they lived.
\clearpage
\section{The Dawn Animal in Daily Life}
\begin{quote}
``Nothing of him that doth fade\\
\hspace*{5mm}But doth suffer a sea-change\\
\hspace*{5mm}Into something rich and strange.''\\
\hspace*{25mm}\emph{Ariel in ``The Tempest.''}
\end{quote}

\begin{quote}
``Swift as the sparkle of a glancing star,\\
\hspace*{5mm}I shoot from heaven.'' --- \emph{Comus.}
\end{quote}
\paragraph{}
I was recently asked which was the commonest fossil. I had no hesitation in replying, ``\emph{Eozoön},'' and it is strange that this should be the case, seeing that the very existence of this organism has been denied.

If my discourse is diffuse, it is because the subject thereof is so universally diffused. We use the calcareous or silicated skeletons of nummulitic rocks to build houses, churches and monuments. Again, much of the road metal or Macadam is composed of the same material. Sulphur, which may have other sources than the organic one, undoubtedly is associated with \emph{Eozoön}. I have a specimen of disintegrated \emph{Eozoön}, in the form of trachyte, broken off from the interior of the upper crater of Teneriffe, which contains disks of \emph{Eozoön} and granules of pure sulphur.

Possibly even the carbon of which diamonds are made once formed part of the sarcode of \emph{Eozoön}, for these gems are found in the matrix of igneous rock.\footnote{\Fontauri{I have since examined a piece of ``Blue Ground'' diamond matrix from the Premier Diamond mine near Pretoria. I found the Stromatoporoid structure clearly visible in this serpentine rock, which forms plugs in volcanic craters.\\
\hspace*{5mm}Apparently there are no dikes, and if this be the case, it may account for the formation of the carbon crystals. For if dikes could not burst through the unyielding quartz formations surrounding the craters, there would be extra pressure in the craters themselves.}} The high temperature caused dissociation of the carbon compounds and melting of the carbon, which crystallized under immense pressure.

Mr. E. Morgan of Richmond showed me a large piece of felspathic rock full of sapphires from the Burma mines. The whole rock shows the disk structure. Sapphires and rubies are probably remnant products. The lime and silica of the lime-felspar of the matrix have disappeared in places, leaving only sesquioxide of alumina tinged with chromium or iron. Traces of Foraminiferal structure are visible in some of these gems, where the transformation and crystallisation have not been completed. Diamonds may have come from the flesh of the Dawn Animal, and certainly many other gems from its silicated skeleton.

Again mica is a silicate which formed part of a nummulitic and probably Stromatoporoid rock.

Mr. H. P. Wiggins very kindly permitted me to look round his great mica warehouse crammed with specimens from all parts of the world. It was not difficult to trace the Foraminiferal structure in some of the rough masses of this mineral.

\centerline{*\hspace{15mm}*\hspace{15mm}*\hspace{15mm}*\hspace{15mm}*}
\bigskip

A curious association of the Dawn Animal with the daily course of life is afforded by the sacred stone fixed outside the south-east corner of the Kaaba at Mecca. All travellers\footnote{\Fontauri{See R. Burton's `Pilgrimage to El Medinêh and Mecca,' vol 3., pp. 159, 210, where the views of Ali Bey, Burckhardt, and himself are given.}} agree that this stone is volcanic or that it is an aerolite. Several times daily many millions of Mohammedan profiles turn or ought to turn towards Kaaba, of which the stone constitutes the most important part. Very probably, then, a fragment of the Dawn Animal constitutes the omphalos of the Mohammedan world. Mohammed gave out that the stone was given by the angel Gabriel to Abraham (a theory I believe in almost as little as in that of the mineral origin of \emph{Eozoön}).\footnote{\Fontauri{No truly religious Mohammedan need regret that the rightly venerated symbol is in all probability a fragment of changed nummulitic limestone. The great world-poet Jelaleddin writes of the pilgrims to the Kaaba:-\\
``They hear a voice from out the Temple sound:\\
\hspace*{5mm}'Why pray ye thus, O Fools, to clay and stone?'''\\
\hspace*{10mm}``The Festival of Spring'' (W. Hastie's version).}}

\centerline{*\hspace{15mm}*\hspace{15mm}*\hspace{15mm}*\hspace{15mm}*}
\bigskip

On the occasion of a recent visit to Westminster Abbey I obtained permission to view with a lens the stone of Scone. The light was not very good, but so far as I could see, the stone is not a sandstone as stated in the Guide to the Abbey, but a mass of fine-grained igneous rock possibly of the nature of a basalt. I am fairly certain I could detect not only the disk structure, but also minute yellow and dark red crystals.

\centerline{*\hspace{15mm}*\hspace{15mm}*\hspace{15mm}*\hspace{15mm}*}
\bigskip

I have seen what appears to be an interesting example of a sedimentary nummulitic formation, \emph{viz.}, in a brick which I brought back from Babylon. The brick, which is stamped with Nebuchadnezzar's signature, is a baked calcareous mud, which effervesces in acid, leaving a muddy or fine sandy residue. The nummulite shells are clearly visible mixed up with mud and, here and there, fragments of reeds. Herodotus writes: ``As they dug the moat they made bricks of the earth that was dug out; and when they had moulded a sufficient number they baked them in kilns. Then, making use of hot asphalt for cement, and laying wattled reeds between the thirty bottom courses of bricks, they first built up the sides of the moat, and afterwards the wall itself.''

Limestone slabs with bas-reliefs which I picked up at Nineveh are certainly nummulitic and apparently Stromatoporoid, and probably come from limestone mountains of Kurdistan. It seems probable, then, that the Euphrates and Tigris carry the detritus of nummulitic rocks and deposit it as mud. Below Hit and in the neighbourhood of the Median Wall, which extended between the Euphrates and Tigris, \emph{i.e.}, about forty miles due north of Babylon, there is an ancient coastline about 200 feet above sea level.

\centerline{*\hspace{15mm}*\hspace{15mm}*\hspace{15mm}*\hspace{15mm}*}
\bigskip

Seeing that many meteorites are pieces of silicated nummulitic rock, it is probable that some of those familiar objects known as shooting stars, that can be seen on any clear night, are likewise fragments of this rock, rendered white hot in their rush through the air.

\centerline{*\hspace{15mm}*\hspace{15mm}*\hspace{15mm}*\hspace{15mm}*}
\bigskip

Many of the pigments, such as umber, sienna, and the various ochres, are clays mainly composed of silicate of alumina tinged with oxides of iron and manganese. These substances are the disintegrated products of volcanic rocks, and usually it is not difficult to make out the nummulitic structure in the crude material. Indeed, I believe I can see that structure in microscopic preparations made from some ``old masters'' sold to me be a connoisseur and dealer, and sold reluctantly when he learned that it was my fell purpose to submit the treasures to the mercy of the abhorred shears. Here and there I could see a row of scales each with one or more pores. It would not be surprising if the most diligent apprentice failed to grind paints so small that no particle should be larger than .01 to .02 millimetre (the average size of broken-down shell-particles in slates).

However that may be, there is no reason for doubting that the resurrected relics of the Dawn Animal, live again in a sense, on the canvases of the great masters. Architecture, Sculpture (in Verde Antico), Painting and Bijouterie all depend on the same source for a great deal of their material.

\centerline{*\hspace{15mm}*\hspace{15mm}*\hspace{15mm}*\hspace{15mm}*}
\bigskip

At Porto Santo Island I often saw layers of Miocene corals and shells overlaid by lava --- \emph{i.e.}, by poured-out altered Eozoic limestone, surely a very strange example of superposition of an older on a younger formation.\footnote{\Fontauri{This superposition, though strange in character, is of common occurrence, for it takes place whenever lava flows down a mountain side.}} The inscribed slabs of granite covering the remains of Man constitute, however, a much stranger case, for here there is a juxtaposition of the relics of the earliest and latest products of organic evolution, and one is brought in presence of a mystery which science is unable to fathom, the mystery, not of death, but of conscious life and thought.
\clearpage
\section{Summary}
\paragraph{}
The Stromatoporoids are colonial Foraminifera consisting of layers of coiled disks, and \emph{Eozoön canadense} is a Stromatoporoid with an extra thick layer of calcite between the layers of disks, and this extra layer is permeated by the pseudopodial canals containing hoops and coils like those of the mural tubuli of Palaeozoic Stromatoporoids and of recent Foraminifera.

At Porto Santo Island I discovered that the volcanic rocks (basalts, trachytes, lavas, granitoid rocks) were composed of Foraminiferal shells.

Later I found that volcanic rocks from all parts of the globe, and also plutonic rocks and gneisses were metamorphosed nummulitic limestones, showing more or less clearly a ``stromatic'' arrangement and also the Foraminiferal disks. Many meteorites were found to be fragments of nummulitic rock, probably hurled into space during eruptions. Lastly, abyssal Red Clays, slates, and certain Cambrian, Ordovician, Carboniferous, Permian, and Jurassic limestones are composed mainly of nummulites. A granite quarry, a volcano, an oolite quarry, a dolomite peak, or an abyssal ocean floor, all have the same foundation, \emph{viz.}, the originally calcareous porous skeletons of nummulitic Foraminifera. In the case of the granite the limestone apparently became dolomitised and silicated by being permeated with aluminium and silicon compounds probably derived mainly from sea water, partly from frustules and skeletons of organisms with siliceous skeletons, and possibly also from hot solutions of silica made from deposits of silica-forming organisms on or below which the Stromatoporoid reefs rested. The silicated limestone became heated and more or less melted owing to shrinkage of the earth's crust, and on cooling crystallized to form the various kinds of igneous rocks. The later-formed limestones underwent less and less modification.

\centerline{*\hspace{15mm}*\hspace{15mm}*\hspace{15mm}*\hspace{15mm}*}
\bigskip

Careful examination of the igneous rocks shows them to be disguised Foraminifera. Accordingly, the problem of these rocks, like that of \emph{Eozoön}, is primarily a palaeontological one. The great aphorism of Linnaeus, that the rocks are the daughters of Time, is true. The bones of the elder daughters suffered ``a sea-change'' (into silicates) and later the cemetery became devastated; but it is often possible to recognise the relics.

\centerline{*\hspace{15mm}*\hspace{15mm}*\hspace{15mm}*\hspace{15mm}*}
\bigskip

Hitherto treatises on Stromatoporoids have dealt only with Palaeozoic forms. If the nummulites of igneous rocks and of many later formations are colonial, the next monograph ought to describe the Stromatoporoids of all the eras: \emph{viz.}, Eozoic, including igneous and gneissic rocks; and Palaeozoic, Mesozoic (and ? Cainozoic and Recent). Those of the first era are usually silicated, and those of the succeeding eras, silicated, silicified, dolomitised, sanded, or unaltered.

\centerline{*\hspace{15mm}*\hspace{15mm}*\hspace{15mm}*\hspace{15mm}*}
\bigskip

A sceptical friend tells me banteringly that I see nummulites everywhere. Certainly I find that their distribution in space and time is almost incredibly extensive, but everyone can verify the fact for himself with not much expenditure of time and trouble. We can now realize that life began on a planetary scale. The lighter elements and their compounds formed a surface layer of material out of which protoplasm was built up, and a portion of the latter developed into a widely distributed lime secreting Rhizopod.

\centerline{*\hspace{15mm}*\hspace{15mm}*\hspace{15mm}*\hspace{15mm}*}
\bigskip

The Nummulitic Theory of Igneous Rocks will be of importance:- to Biology, as showing that a great part of the floor of the Eozoic ocean was formed by a shell-forming ``Bathybius''; to Geology, in proving that the igneous and gneissic rocks, and numerous sandstones and limestones of later eras are composed of skeletons of a colonial nummulite; to Geognosy and Vulcanology by throwing fresh light on the problem of the risings and sinkings of the earth's crust which give rise to land surfaces and oceans; to Petrology and Mineralogy by pointing out a profoundly important element in the construction of igneous rocks, and one which has been an ``allotriomorphic'' factor influencing the form assumed by cooling minerals; and to the science of meteorites, be revealing that many of these bodies are of organic and probably terrestrial origin.

\centerline{*\hspace{15mm}*\hspace{15mm}*\hspace{15mm}*\hspace{15mm}*}
\bigskip

The most important clue to the nature of the igneous rocks was furnished by a little particle 0.1 of a millimetre (0.004 of an inch) in diameter (Pl. 1, Fig. 1-5). Soon this particle was found to be part of an originally world-wide domain --- the Nummulitic --- which has endured almost from the remotest beginnings of time, so far as Life is concerned, even to the present.

\centerline{*\hspace{15mm}*\hspace{15mm}*\hspace{15mm}*\hspace{15mm}*}
\bigskip

Palaeo-nummulitic limestones --- which are composed probably of colonial nummulites (Stromatoporoids) form on the globe a universal crust --- the nummulosphere.

Owing to shrinkage of the planet from secular cooling, the nummulosphere has been thrown into great folds, the elevations of which constitute the land, the depressions being filled with the ocean.

The soluble carbonate of lime has partially or completely disappeared and has been replaced by often complex silicon compounds. Heat due to compression from earth-movements has melted the silicated rocks, which have crystallized on cooling (igneous rocks). Traces more or less obvious of the silicated nummulites are nearly always to be found. The surface of the nummulosphere constituting the floor of the ocean is in the form of a red clay, which is often crowded with the skeletons of pelagic organisms and with various terrigenous products. Here, again, the nummulite shells are always to be seen.

Many meteorites are fragments of the nummulosphere, probably of this planet, hurled into space during eruptions and recaptured later.

\centerline{*\hspace{15mm}*\hspace{15mm}*\hspace{15mm}*\hspace{15mm}*}
\bigskip

The principle of philosophic doubt is a very important part of the equipment of a student of science, but it is a doubt that implies sympathy with new ideas, rather than hostility. Naturally there is apt to be a little igneous action when new theories suddenly come athwart old-established convictions. In spite of the past history of the \emph{Eozoön} controversy, I am confident that the Stromatoporoid theory of \emph{Eozoön} will soon be generally accepted, and when that happens, the acceptance of the theory of the organic origin of igneous rocks will probably follow, for the latter is simply an extension of the former.

\centerline{*\hspace{15mm}*\hspace{15mm}*\hspace{15mm}*\hspace{15mm}*}
\bigskip

\small
\emph{Postscript.} --- I deliberately and aggressively use the expression ``so-called igneous rocks'' in the hope of dislodging effectually a truth-obscuring obsession, \emph{viz.}, the belief in the impossibility of the organic origin of these rocks. So far as I have observed, they are all simply ancient nummulitic limestones metamorphosed by solution, chemical change, pressure and heat. I think that a more logical and scientific designation would be ``Palaeonummulitic Rocks,'' these being classified as heretofore in accordance with their position (\emph{viz.}, abyssal, hypabyssal and superficial) and their varying degrees of acidity and basicity.
\normalsize
\clearpage
\section{Appendix}
\Large
\paragraph{}
I record here several new observations too late for insertion in the body of the text.

\centerline{*\hspace{15mm}*\hspace{15mm}*\hspace{15mm}*\hspace{15mm}*}
\bigskip

\emph{Note 1.} I find that typical red sandstones of the Old Red Sandstone, Permian, and Triassic Systems are composed of sandy nummulitic, probably Stromatoporoid, shells. The nummulites are quite easily seen \emph{en face} and in vertical section in a specimen of Old Red from Orkney (O.R.S.1, N.H.M.) and less easily in finer grained and more friable examples of Lower New Red (Birmingham, N.H.M.). All calcareous matter has disappeared.

In some red crystalline examples from Penrith, metamorphic action has converted the sandy nummulitic rock into a quartz grit, in which it is still possible, though not easy, to detect traces of Foraminiferal structure.

\centerline{*\hspace{15mm}*\hspace{15mm}*\hspace{15mm}*\hspace{15mm}*}
\bigskip

\emph{Note 2.} --- A column of Portlandian sandstone (in the Jermyn Street Museum) from a boring 1200 feet deep in Kent is nummulitic.

\centerline{*\hspace{15mm}*\hspace{15mm}*\hspace{15mm}*\hspace{15mm}*}
\bigskip

\emph{Note 3.} --- Some fragments of Upper Permian (24 G; N.H.M.) from Russia, with Equisetums are sedimentary nummulitic rocks. The gentleman who showed me the specimens offered me one penny for every marine fossil found in them. Several hundred pounds are due to me.

\centerline{*\hspace{15mm}*\hspace{15mm}*\hspace{15mm}*\hspace{15mm}*}
\bigskip

\emph{Note 4.} --- Chalk is a nummulitic and probably Stromatoporoid rock. A few minutes' careful scrutiny with a lens will reveal the outlines of the disks, which are best seen in hard chalk. Globigerina ooze is a silicated palaeonummulitic formation, now abyssal, containing recent \emph{Globigerina}, \emph{etc.} Chalk is a Cretaceous calcareous-nummulitic rock, probably formed in relatively shallow water, and containing Cretaceous \emph{Globigerina}, \emph{etc.}

\centerline{*\hspace{15mm}*\hspace{15mm}*\hspace{15mm}*\hspace{15mm}*}
\bigskip

\emph{Note 5.} --- I have found a Cainozoic sandy-nummulitic formation in the Eocene Middle Bagshot sands, \emph{viz.}, in the Bracklesham Beds, Alum Bay (shells not very distinct owing to the powdery nature of the sand); and Mid Bagshot Beds, Bournemouth (shells more distinct).

The extremely delicate circular patterns of the shells can be seen on some parts of the surface, and on other parts, the delicately lined parallel series of fine bands formed by disks in vertical section. In fact it is possible sometimes to detect the willow-pattern formed by the edges of the coils of the shells meeting at an angle.

The shells have been modelled in fine particles of sand, and careful and patient scrutiny with a x10 lens will gradually reveal all the above-mentioned structures. I sometimes find little patches of finer and whiter sand which show the Foraminiferal structure very well.

These Middle Bagshot nummulitic rocks might almost be called Eocene Red Sandstones, so closely do they resemble the Old and New Red.

Probably none of the statements made in this pamphlet will cause so much surprise among geologists as those concerning the sandy nummulitic character of these sandstones.

Venturing from the safe ground of facts easily to be verified, to the shifting sands of hypothesis, I would suggest that the Mid Bagshot sands were formed by the wearing down of limestone (? upper chalk) rocks, and that the nummulitic \emph{débris} became infiltrated and mingled with fine sand, and decalcified.

In order to examine the \emph{débris} brought down by rivers that had passed over chalk, I visited the Cuckmere River in Sussex and the Stour at Sandwich in Kent. The ooze from mud-banks at the mouth of the former, and the sand from the sandy banks of the Stour both contain sandy models of nummulites. The calcareous matter has nearly or entirely gone. The spaces which were originally the chambers of the shell contain little dark grains of ? oxide of iron, arranged in radial concentric or parallel-linear series. I learned from these specimens how a central knob or pit often seen on the bran-like scales of slates and abyssal red clays apparently was formed, namely, by pressure of an intracameral concretion on the walls of the shells. The pits, again, may be in the areas of pores. Very close and detailed observation is required to detect the nummulitic structure.

These sandy and muddy deposits now being formed, in the rivers Cuckmere and Stour, consisting as they do of sandy models of nummulitic shells (\emph{débris}) of a nummulitic (? Stromatoporoid) limestone known as Chalk, are, I believe, \emph{fundamentally} the same as those altered deposits known as gneisses and slates. The haematite (?) granules in the shells of the recent deposits apparently become diffused in the case of the variegated (poikilitic) Old and New Red and Bagshot sandstones.

An interesting feature that comes out in this investigation of certain sandstones and grits is that the detritus of nummulitic limestone or altered limestone (igneous) rocks is often in the form of nummulites and not of shapeless particles.

\centerline{*\hspace{15mm}*\hspace{15mm}*\hspace{15mm}*\hspace{15mm}*}
\bigskip

\emph{Note 6.} --- \emph{Apropos} of the theory of compression as a source of heat in volcanic phenomena, F. W. Rudler (Article on Volcanos in Encyc. Brit. Ed. 11. vol. 28. p. 191) writes: ``A grave objection, however, is the difficulty of conceiving that the heat, whether due to crushing or compression, could be concentrated locally so as to produce a sufficient elevation of temperature for melting the rocks. According to the calculations of the Rev. O. Fisher the crushing could not, under the most favourable circumstances, evolve heat enough to account for volcanic phenomena.'' It must be easy for flaws to arise in the case of difficult calculations involving questions of the earth's tonnage. In view of the fact that igneous rocks are altered limestone sea-bottoms, no other theory than that of compression seems possible. When the planet contracted the nummulosphere was in a tight place.

According to the ``compression'' theory, Stromboli must be regarded as a node situated in an area subject to continual chronic pressure for a prolonged period.

\centerline{*\hspace{15mm}*\hspace{15mm}*\hspace{15mm}*\hspace{15mm}*}
\bigskip

\emph{Note 7.} The ferrocyanide test shows that the dark red granules in the chambers of the nummulite shells in some limestones and in certain sands and muds (\emph{Note 5}, p. 99) are composed of sesquioxide of iron.

\centerline{*\hspace{15mm}*\hspace{15mm}*\hspace{15mm}*\hspace{15mm}*}
\bigskip

\emph{Note 8.} The genus \emph{Receptaculites}, which has been frequently shifted from group to group, will come to rest at last among the Stromatoporoids or colonial nummulites. The ``facets'' of the engine-turned watch-case pattern are the uppermost shells of rouleaux or columns of nummulites like those of ``piped quartz.'' The umbilicus, the radial rows of pores, the minute network pattern of alae and septa exposed at broken surfaces, are best seen in the larger slightly overlapping rows of shells at the periphery. The columns radiate out from the centre in spiral linear series, and where the spirals meet from opposite sides the shells appear to be diamond shaped. The columns are extremely slender at the centre and gradually become thicker towards the periphery. Some facets have a minute bud which would from a new storey. \emph{Receptaculites} will, I believe, throw light on budding and growth in Stromatoporoids.

\centerline{*\hspace{15mm}*\hspace{15mm}*\hspace{15mm}*\hspace{15mm}*}
\bigskip

\emph{Note 9.} Senhor A. C. Noronha tells me that Prof. C. Gagel of Berlin has informed him that the granitoid rock of Porto Santo is Essexite, and that it comes from a deep zone of the same magma that forms the superficial deposits of Porto Santo and Madeira, the Essexite having the abyssal form of crystallization. See also Professor Gagel's `Studien über den Aufbau und die Gesteine Madeiras.' Zeitsch. deutsch. geolog. Gesellsch. 1912. Band 64, Heft 3.

\centerline{*\hspace{15mm}*\hspace{15mm}*\hspace{15mm}*\hspace{15mm}*}
\bigskip

\emph{Note 10.} Corrigendum. P. 93. Summary. Delete lines 3 to 8 from top. \emph{viz.}, from ``with an extra thick layer'' to ``recent Foraminifera.''
\clearpage
\section{Explanation of Plates}
\paragraph{}
Photographs (made from untouched negatives) of very young shells found in sections of \emph{Eozoön canadense}. They should be examined in different lights, with and without the aid of a lens.

Figs. 1, 2, 3, 4. --- A nummulite shell, about 0.1 of a millimetre (0.004 of an inch) in diameter, focussed from above down at four different levels. x270.

A careful study of the actual shell shows at least four spiral coils wound round the funnel-shaped umbilicus. The large primary chamber is a little to the right and above the pore-like upper opening of the umbilicus. Each of the larger dark spots is probably a large pore; double rows of very minute dots in Fig. 2 are smaller pores. The obscure radial lines are fractures in the neighbourhood of septa. The large irregular light patch to the right of the centre shows a broken area. ``Willow pattern'' obscurely visible along broken edge in Figs. 2, 3. (See Fig. 15.)

Fig. 5. --- Lower surface of above shell. In the centre is seen a defined oval area (3 by 2 millimetres) with a faintly shaded spot in the centre of the oval. The oval area is the wide lower orifice of the funnel-shaped umbilicus, and the faint spot is the minute aperture at the upper end of the same.

Fig. 6. --- Another shell broken across and showing the ``willow-pattern'' at upper end of fractured surface. There are visible four alar prolongations on each side of the central line, apical chambers, and dark spaces and septa between the alae (see Fig. 16.).

Figs. 7, 8, 9, 10. --- Upper aspect of a still younger shell. The Nautilus-like embracing of the last coil but one by the last coil is fairly well shown at the lower part of the figures, and also (in Fig. 7) the lopping round of the four or five last-formed septa and segments. x270.

Fig. 11. --- A young shell on surface of etched out specimen. Seen by reflected light. The central dark area is the (? lower) opening of the umbilicus. x270.

Fig. 12. --- Outer weathered surface of \emph{Eozoön} showing circular outlines of larger adult nummulite shells. x3. Fairly good examples of disks can be seen (with aid of lens) on each side of oblique dark streak near upper left corner. Compare also Fig. 23.

Fig. 13. --- Under surface of shell shown in Figs. 7-10. x250.

Fig. 14. --- Canals in one of the white bands of \emph{Eozoön}, showing series of rings. x270. Compare Figs. 17-22.

Fig. 15. --- A tracing of Fig. 2. \emph{a}, Upper opening of umbilicus; \emph{b}, edge of final alar prolongation; \emph{c}, primary chamber; \emph{d}, segments or chambers of final coil; \emph{e}, larger pseudopodial pores; \emph{f} double row of minute pseudopodial pores; \emph{g}, segments of last coil but one.

Fig. 16. --- Enlarged drawing of upper part of broken edge of shell shown in Fig. 14. \emph{a}, an outer alar prolongation, enclosing \emph{b}, the next in succession; \emph{c}, edges of septa separating the alae, these portions of septa being prolongations from the partitions separating the peripheral part of the whole into segments; \emph{d}, spaces between the alae and in continuity with the chambers at the periphery of the whorl; \emph{e}, chambers at periphery of the successive whorls.

The parts are emphasised, but can be seen in photo, by using a lens.

N.B. The spiral coils and willow pattern seen respectively in horizontal and vertical sections are precisely the same as are found in typical Nummulites.

Fig. 17. --- Oval rings in canals of \emph{Eozoön canadense}.

Fig. 18. --- Ditto from larger canals in another specimen.

Fig. 19. --- Rings and half-rings from mural tubuli of the Palaeozoic Stromatoporoid \emph{Actinostroma clathratum}, Nicholson.

Fig. 20. --- The same from \emph{Stromatopora concentrica}, Goldfuss.

Fig. 21. --- The same from a recent specimen of \emph{Sporadotrema cylindricum} (Carter) from the China Sea. 

Figs. 17-21 are copied by kind permission of Messrs. Taylor and Francis from a paper by the author in \emph{Annals Mag. Nat. Hist.}, Sept. 1912. Reduced from a magnification of 1,300 diameters.

Fig. 22. --- Canals in supplementary skeleton of \emph{Eozoön canadense}, being part of photograph show in Fig. 14.

Fig. 23. --- Weathered surface of \emph{Eozoön}, showing rows of disks \emph{en face}, from an area about three-quarters of an inch (18 millimetres) square a little above and to the right of centre of lower line of figure shown in Fig. 12. The disk outlines have been somewhat emphasised to enable them to be traced in the photograph. Enlarged about 3 times from photo, \emph{i.e.} about 10 times from nature.
\clearpage
\pagestyle{fancy}
\fancyhf{}
\rhead{\Fontauri{Plate 1}}
\cfoot{\Fontauri{\thepage}}
\begin{figure}[b]
\Fontauri
\centering
\includegraphics[width=0.75\textwidth,keepaspectratio]{Fig-1.png}

Figure 1
\end{figure}
\clearpage
\begin{figure}[b]
\centering
\Fontauri
\includegraphics[width=0.75\textwidth,keepaspectratio]{Fig-2.png}

Figure 2
\end{figure}
\clearpage
\begin{figure}[b]
\centering
\Fontauri
\includegraphics[width=0.75\textwidth,keepaspectratio]{Fig-3.png}

Figure 3
\end{figure}
\clearpage
\begin{figure}[b]
\centering
\Fontauri
\includegraphics[width=0.75\textwidth,keepaspectratio]{Fig-4.png}

Figure 4
\end{figure}
\clearpage
\begin{figure}[b]
\centering
\Fontauri
\includegraphics[width=0.75\textwidth,keepaspectratio]{Fig-5.png}

Figure 5
\end{figure}
\clearpage
\begin{figure}[b]
\centering
\Fontauri
\includegraphics[width=0.75\textwidth,keepaspectratio]{Fig-6.png}

Figure 6
\end{figure}
\clearpage
\begin{figure}[b]
\centering
\Fontauri
\includegraphics[width=0.75\textwidth,keepaspectratio]{Fig-7.png}

Figure 7
\end{figure}
\clearpage
\begin{figure}[b]
\centering
\Fontauri
\includegraphics[width=0.75\textwidth,keepaspectratio]{Fig-8.png}

Figure 8
\end{figure}
\clearpage
\begin{figure}[b]
\centering
\Fontauri
\includegraphics[width=0.75\textwidth,keepaspectratio]{Fig-9.png}

Figure 9
\end{figure}
\clearpage
\begin{figure}[b]
\centering
\Fontauri
\includegraphics[width=0.75\textwidth,keepaspectratio]{Fig-10.png}

Figure 10
\end{figure}
\clearpage
\begin{figure}[b]
\centering
\Fontauri
\includegraphics[width=0.75\textwidth,keepaspectratio]{Fig-11.png}

Figure 11
\end{figure}
\clearpage
\begin{figure}[b]
\centering
\Fontauri
\includegraphics[width=\textwidth,keepaspectratio]{Fig-12.png}

Figure 12
\end{figure}
\clearpage
\begin{figure}[b]
\centering
\Fontauri
\includegraphics[width=0.75\textwidth,keepaspectratio]{Fig-13.png}

Figure 13
\end{figure}
\clearpage
\begin{figure}[b]
\centering
\Fontauri
\includegraphics[width=0.75\textwidth,keepaspectratio]{Fig-14.png}

Figure 14
\end{figure}
\clearpage
\rhead{\Fontauri{Plate 2}}
\begin{figure}[b]
\centering
\Fontauri
\includegraphics[width=0.75\textwidth,keepaspectratio]{Fig-15.png}

Figure 15
\end{figure}
\clearpage
\begin{figure}[b]
\centering
\Fontauri
\includegraphics[width=0.75\textwidth,keepaspectratio]{Fig-16.png}

Figure 16
\end{figure}
\clearpage
\begin{figure}[b]
\centering
\Fontauri
\includegraphics[width=0.75\textwidth,keepaspectratio]{Fig-17-21.png}

Figures 17-21
\end{figure}
\clearpage
\begin{figure}[b]
\centering
\Fontauri
\includegraphics[width=0.75\textwidth,keepaspectratio]{Fig-22.png}

Figure 22
\end{figure}
\clearpage
\begin{figure}[b]
\centering
\Fontauri
\includegraphics[width=0.75\textwidth,keepaspectratio]{Fig-23.png}

Figure 23
\end{figure}
\clearpage
\end{document}
